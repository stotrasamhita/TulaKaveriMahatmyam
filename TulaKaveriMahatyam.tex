प्रथमोऽध्यायः
श्रीदाल्भ्यधर्मवर्मसम्वादे नारदागस्त्याभ्याम् 
हरिश्चन्द्रम्प्रति कुरुक्षेत्रे 
तलाकावेरिस्नानविधि कथनम 

श्रीसूतः

धर्मवर्माथराजर्षीन् निचुळापुरवल्लभः।
भूय:पप्रच्छ तं नत्वा दाल्भ्यं भागवतोत्तमम्।

राजा भगवन् प्राणिनस्सर्वे केनोपायेन सर्वदा।
भवन्ति पुत्रान् सम्प्राप्य सुखिनश्चिरजीविनः।

कथम् स्यात्पापनिर्झर श्रीशे भक्तिम् कथम् भवेत्।
केन धर्मेण सन्तुष्टो भगवान् भूतभावनः।

प्रसीदतिमनुष्याणाम् भुक्तिमुक्तिफलप्रदः।
विशेषात्पापभूयिष्टे दुराचारे कलौयुगे।

पापनाशोभवेद्ब्रह्मन् महापातकिनोपि वा।
एतत्सर्वमशेषेण तव शिष्यस्य मे वद। 5 

सूत:

इति राज्ञा सुसंस्पृष्टो भगवान् भगवत्प्रियः।
बभाषे धर्मवर्माणाम् धर्मिष्टम् ब्राह्मणोत्तमः। 

दाल्भ्यः 

साधुपृष्टम् महाभाग ! भगवद्भक्तिवर्धनम्।
यत्ते मनोगतंश्रोतुं दिव्याम् विष्णुकथाम् शुभाम्।

तस्मात्ते कथयिष्यामि सर्वतत्वम् यथामति।
अस्मिन्नर्थे पुरा पृष्टो हरिश्चन्द्रेणकुम्भजः।

कुरुक्षेत्रे मुनीन्द्राणामग्रतो यदवर्णयत्।
तत्तेहम् सम्प्रवक्ष्यामि शृणुष्व वहितोन्नृप।

पुराऽयोध्यापतिश्रीमान् हरिश्चंद्र इति शृतः।
कुरुक्षेत्रम् शुभक्षेत्रम् यजनार्थमवाप्तवान्।10

तत्र स्थितान् महाभागान् शौणकादि मुनीश्वरान्।
दृष्ट्वा विनयसम्पन्नो नमश्चक्रे कुतूहलि।

तम् दृष्ट्वा नृपशार्दूलम् हरिश्चंद्रम् हरिप्रियम्।
इदमूचुरनूचानाम् मुनयस्सत्यवादिनः।

मुनयः

स्वागतम् ते महाराज!पालिता धर्मत:प्रजाः।
कश्चित्सुकुशलम्राष्ट्रे पुरेकोशे चमूगणे।

अरयश्चजिताराजन् सुहृबंधूंश्चरक्षसि
गृह्णपष्टांशमुाम्यो दु/राजानपालयेत्।

आबृह्मकल्पम् नरकान् भुक्त्वास्यात् स कुयोनिज।
तस्माद्ब्राह्मणभक्तश्च राज्यम् धर्मेणपालयन्।

जितेन्द्रियोयशस्वी च दीर्घमायुरवाप्नुयात्।
ब्राह्मण्यश्च वदान्यश्च पुण्यश्लोकाग्रणीस्सुधीः। 15

त्वम् हि दान्तश्च शान्तश्च श्लाघनीयोसि सत्तमैः।
इत्युक्त्वा मुनयस्सर्वे हरिश्चन्द्राय धीमते।

आतिथ्यम् विधिवश्चक राज्ञे धर्मप्रकारतः।
आसनञ्च ददुस्तस्मै तस्मिन् स्थित्वा च तान नृपः।

अब्रवीत्विनयादीत:कृताञ्जलिरिदम् वचः।
धन्योरम्यनुग्रहीतोस्मि पावितोस्मि मुनीश्वराः।

यदद्राक्षम तपोराशीन् भवतो लोकपूजितान्।
किञ्च काचिद्विवक्षास्ति शृत्वा मे तद्विधीयताम्।20

अत्र स्थितातांश्चवश्शृत्वा सेवार्थमहमागतः।
केनोपायेनसम्सारम तरेयम् मस्तरम् द्विजाः।

भुक्तिमुक्तिकथम्स्याताम् दीर्घायुर्वंशवृद्धयः।
कथम दृष्टोभवेच्छ्रीमान् भगवान् पुरुषोत्तमः।

तत्वमेकम् विनिश्चित्य धर्मसारम्वन्तु मे। 

मुनयः 
साधुः प्रश्नः कृतो राजन् धर्मैककृतबुद्धिना।
भवतानृपशार्दूल तत्वम्बूमस्तव प्रियम्।

अश्वमेधमहायज्ञम् ब्रह्महत्यापनोदनम्।
भुक्तिमुक्तिप्रदंश्रेष्ठम् हरिसन्तोष साधनम्।

हरिश्चन्द्रः

यूयमेव महायज्ञम् यज्ञेशप्रीणनम् शुभम्।
कारयित्वा द्यविप्रेन्द्रा: कृतार्थम् कुरुताद्य माम्।

25 एतस्मिन्नन्तरे विप्र नैमिशारण्यवासिनः।
पुरस्कृत्य तदा सूतं पौराणिकमथागमन्।

वसिष्टोवामदेवश्च जाबालिकाश्यपोभृगुः।
विश्वामित्रो महातेजा:दुर्वासास्तपसां निधिः।

हारीतोत्रिर्मङ्गण वीतिहोत्रश्च गालवः।
मार्कण्डेयोसित:कण्वो सितोयाजोपयाजको।

भरद्वाजोगौतमश्च भैल्वश्चैव पराशरः।
व्यासस्यातातपोमुद्गो मौद्गल:कवषस्तथा।

वाल्मीकिवरुणोऽगस्त्यो मतङ्गश्च महातपाः।
जातुकर्णस्सुतीक्ष्णश्च शतानन्दश्च शाश्वतः।30

 सत्यव्रतस्सत्यतप: आसुरिनारद:कविः।
दौम्योऽङ्गिगिराश्चकलिन्दो माण्डव्यो मधुहास्रयः।

गर्गोगतक्षतोहोता धूमकेतुर्जलप्लवः।
ऊर्ध्वरेतामाहातेजा शङ्गश्चलिखितस्तथा।

बोधायनोयाज्ञवल्क्यो यज्ञकेतुर्मरु:क्रतुः।
पुलस्त्य:पुलहोगौरा आश्वलायन इत्यपि।

आपस्तम्बोयज्ञराशि मरीचिर्बभृरुत्तमः।
 एतेचान्येमहात्मानो मुनयस्संश्रितव्रताः।
 
 शिष्यप्रशिष्यसहित दान्ता क्षान्ताजितेन्द्रियः।
 अब्बक्षावायुभक्षाश्च पत्रमूलफलाशनाः।35

 ग्रहस्था यतय:केचित्द्वानप्रस्थाश्रया:परे।
 लोकानुग्रहकर्तारो न्यायशास्त्र प्रवर्तकाः।
 
 वेद वेदान्तवेत्तरो हृषीकेशपदाश्रयाः।
 तुलाकावेरिमाहात्मियम् श्रोतुकामाहरिप्रिरम्।
 
 कुरुक्षेत्रम् समाजग्मुर्लोकाघक्षपयिष्णवः।
 तान् दृष्ट्वा मुनयस्सर्वे कुरुक्षेत्रे निवासिनः।
 
 अपूजयन्लोकपूज्यान् प्रत्युत्थावासनार्हणैः।
 अन्योन्यकुशलप्रश्ना स्सानन्दा, परिप्लुताः।
 
 आसनेष्वथ दत्तेषु सर्वेष्वास तापसा:।
 तान् दृष्ट्वान् मुनिशार्दूलान् हरिश्चन्द्रो जितेन्द्रियः।40

 कायं निपातयामास भूमौहाटकदण्डवत्।
 कृताञ्जलिपुटो भूत्वा चानन्दापरिप्लुतः।
 
 सगद्गदमुवाचेदम् वाक्यम् वाक्यविशारदः।
 
राजा

अद्य मे सफलञ्जन्म ह्यद्यमे सफलम कुलम्।
 पितरस्तर्पिता विप्रा अद्य मे सफलम् व्रतम्।

 आराधितो मया विष्णु:प्रपन्नोभून्मम दृवम्।
 राज्ञोधर्मस्य मूर्खस्याप्यैश्वर्योन् मूर्खचेतसः

यन्नेत्रगोचरा यूयम् पूर्वजन्म तप:फलम्।
 किमलभ्यम्प्रसन्नेषु भवत्सु त्रिजगत्सु मे।

 इत्यूचिवांसं राजानम् प्रत्यूचुर्मुनिपुङ्गवाः।
 हरिश्चन्द्र महाभाग ! हरिभक्तिपरायण।

 आसनस्थे नृपे तस्मिन् शौनकच्चौरिभक्तिमान्।
 उवाच वचनम् सर्वम् मुनीन्द्रान्वीक्ष्यसत्तमः।

 शौनक:अयं राजाध्वरं कर्तुम मुद्युक्तो हयमेदकम।
 यदृच्छयागतायूयमेनम् याजयताध्वरे।

 तद्वाक्यम् नारदागस्त्यो शृत्वा वाच्यमथोचतु:
 असौराजामहाप्राज्ञा वशीदान्तिजितेन्द्रियः।
 
 यज्ञाह एव तत्सत्यम तथापि श्रूयताम वचः।
 विश्वामित्रारुणात्पूर्वम् भीतारण्यम् पलायतः।50

  दुःखार्थस्सञ्चरन्नुर्वीम् सत्यवादी क्षुदातुरः।
 तपस्यन्तं महात्मानम् किन्दमन्दमिनाम्वरम्।
 
 दृष्ट्वापि दैवयोगेन ह्युन्मत्त इव तस्थिवान्।
 एनम् दृष्ट्वा सुदु:खार्थं ब्राह्मणो चिन्तयत् तदा।
 
 हरिश्चन्द्र इति श्रीमानयम् भुवनविशृत:।
 जितेन्द्रियोजितक्रोधो नित्यम्ब्राह्मणपूजकः।
 
 करमान्नपूज्येन्नामित्येवम् ध्यानमुपागमत।
 विश्वामित्र कृतं सर्वम् तत्क्षणादवबुद्ध्य सः।
 
 सस्वस्थान्यमितिनमिति कारुणिको मुनिः।
 अवगत्यावमानेन ब्राह्मणस्यास्य भूपतेः।55

 प्रायश्चित्तम विना विप्रानधिकारोध्वरोत्तमे।
 कायेनमनसावाचा योवमन्येत भूसुरम।
 
 सद्योनष्टायवैश्वर्य: पितृभिर्निपतेदधः।
 प्रायश्चित्तीयते मोहात् ज्ञानाच्छेत् बृह्मराक्षसः।
 
 कावेर्यान्तु तुलामासे वैशाखेमेषगेरवौ।
 नर्मदायाम् सकृत्स्नात्वा शुद्धस्याद्यङ्ग तत्पर:

 कावेरी हि तुलामासे सर्वतीर्थैरुपाश्रिता।
  स्नातानां सर्वपापनि सर्वकाम समृद्धिदा।
 
 मोक्षलक्ष्मीप्रधात्री च सर्वयज्ञफलप्रदा।
 षट्पष्टिकोटितीर्थानि द्विसप्तभुवनेषु च।60

 तानिचायान्ति कावेर्याम् स्वाघौघविनिवृत्तये।
 स्नातुम तुलागते भानौ केनवा वर्ण्यते हि सा।
 
 अयुतंशरदां शेषसहस्रवदनैर्वदन्।
 पारं नाप्नोति कावेर्या माहात्म्यस्य न संशयः।
 
 स्नात्वा तुलार्केकावेर्याम् त्रिरात्रम् निवसन्नरः।
 मुक्तपापो दिवम्याति स पापोप्यपर:किमु।
 
 तीरयोरुभयोर्यस्मात्कावेर्यास्सह्यभूधरात्।
 आसागरम् शिवक्षेत्रम् विष्णुक्षेत्राणिचान्तर।
 
 मुनीनामाश्रमा दिव्यास्सर्वसिद्धि विधायिनः।
 तप:प्रारम्भमात्रेण संपूर्णफलदान नृणाम्।65

 कावेरी महिमा केन वर्ण्यते युगकोटिभिः।
 विधिर्नारायणाच्छम्भो ऋते सत्यं जगत्त्रये,।
 
 कावीरीवीतयो नद्या आवर्तास्युर्जलाश्रयाः।
 देवतावालुकस्तस्या सर्वतीर्थफलम् भवेत्।
 
 यावन्त: पांसवोभूमर्यावन्त्यो व्योम्नितारकाः।
 यावन्त्यो वर्षधारास्स्युर् यावन्त्युान्तुनानि च।
 
 तावत्कोटिशतैस्तीर्थैः कावेरी सेव्यते तथा।
 यावन्तो बिन्दवस्तस्यु: पूरा:कावेरिपाथसः।
 
 तावत्तीर्थजलाकारा कावेरीत्यवधारय।
 एकदैव समुद्भूता:पयोधे:कतिशीकराः।70

 तत्सङ्ख्या तीर्थतोयौटु वहत्यद्धा कवेरजा।
 अनेककोटिब्रह्माण्डमध्यस्था तीर्थकोटयः।
 
 कावेर्याम् योजितायस्मात्केनवा कवेरजा।
 अस्याम तुलागते भानौ मासि यस्नाति मानसः।
 
 स मुक्तस्सर्वपापेभ्यो विष्णुलोके महीयते।
 तस्मात्त्वमपि राजेन्द्र ! स्नात्वा कवेरिपाथसि।
 
 तुलां गते दिवानाथे सर्वकामानवाप्स्यसि।
 तच्छ्रुत्वा स तयोर्वाक्यम् हरिश्चन्द्रोतिविस्मितः।
 
 उवाचवाख्यं तावेवम् शुद्धो यजन तत्परः।
 अहं दासोरिम भवाम् भवन्त:करुणाब्धयः। 75

 यथा यजेयम् यज्ञेशम हयमेदेनकर्मणा।
 तथा कुरुत माम् विप्र यूयम् निर्हेतुकाश्रया:।
 
सह्यजायाम् तुलामासे वैशाखेनर्मदाजले।
कथं स्नानम् तु कर्तव्यम् को विधिः किं फलम् भवेत्।

सर्वमानेष्वतिशयो विषुमाधवयो:कथम्।
का देवता च किम दानम् कथम् स्नानम् मुनीश्वराः।
मम सर्वम विस्तरेण ब्रूत स्नानविधिम द्विजाः।

 इति श्री दाल्भ्यधर्मवर्मसम्बादे नारदागस्त्याभ्यां हरिश्चन्द्रप्रति कुरुक्षेत्रे तुलाकावेरि स्नानविधि कथनम् नाम प्रथमोऽध्यायः।
 


अथ द्वितीयोऽध्ययः

तुलाकावेरि माहात्म्य श्रवण प्रकारः 

 
दाल्ल्य:

तच्छ्रुत्वा सत्यसन्धस्य हरिश्चन्द्रस्यभाषणम्।
मुनिमध्ये महातेजा अगस्त्यो नृपमब्रवीत।
 
अगस्त्य : साधुपृष्ठं महाभाग ! हरिश्चन्द्र नृपोत्तम।
 तुलाकावेरिमाहात्म्यमध्यायं श्लोककमेव वा।

 यच्छृणोति पठन्स्नात्वा तस्य पुण्यफलं तु क:।
 वदेत्साक्षान् महाविष्णोरितेन्योभूमिपोत्तम।
 
 पूर्वपुण्यप्रभावेन साधुसङ्गस्तवा भवेत्।
 इष्टापूर्थतपश्चर्या वेदपारायणव्रतैः।
 
 अनेकजन्म संसिद्ध भवेत्सत्सङ्गतिवम्।
 सत्सङ्गत्याभवेत्पुण्यकथाया श्रवणेमति।
 
 तत्कथाश्रवणात् पापन्त्रिविधं नश्यतिवम्।
 शुद्ध्यत्यपिमन:पुम्सो निष्पापस्य नृपोत्तम।5

हृदब्जे च मनश्शुद्धौ ध्यायेत्पादाम्बुजम् हरे:
सकृद्ध्यातोमहाराज भगवान्पुरुषोत्तमः।

 प्रसन्तोनिजसारूप्यम्कालभ्यन्ददाति वै।
 तस्माद्भप प्रकृतार्थस्त्वं सत्संगनिरतो भव।

 सह्यजायास्तु माहात्म्यम्विस्तरेण ब्रवीमि ते।

 श्रीकण्ठविश्वेश्वरसन्निभानि
 लिङ्गानि यद्रोधसि लक्ष कोट्यः।
 जलप्रवाहेचसहस्रकोट्यः 
 कवेरजायाश्शिवमूर्तयस्युः।

पयाम्सितीर्थाणि शिलाश्चदेवा
दिवौकसो वालुक तां प्रसन्नाः।
 अतो नदीसह्यगिरिप्रसूता 
सरित्सुमुख्या मनुजैरलभ्या। 10

 कैवल्यसिद्धिम्बिददाति नृणा 
मन्यै स्त्रिभिः किम्पुरुषार्थवगैः।
 
 तुलामासेतु कावेरी सर्वतीर्थाश्रितानदी।
 पञ्चपातकसम्हर्षि वाजिमेधफलप्रदा।
 
 तुलामासेतु कावेरीम्प्रशम्सन्ति महर्षयः।
 दिविदेवाश्च पितरस्साधुसाद्विति हर्षिताः।
 
 तुलामासे तु कावेर्याम्त्रयहं यस्स्नाति मानवः।
 मुक्तपापो भवम्हित्वा विष्णुलोकम्स गच्छति।
 
कुतस्स्नानस्तुकावेर्याम्माघेमासि च तौलिके।
त्रिकोटिकुलसम्युक्तो ब्रह्मलोके महीयते।

 नीरान्नम्स्यात्तुलामासि दत्तं कोटिगुणम्भवेत्।
 सत्पात्रे पुण्यकालेचाप्यनन्तम्फलZच्छति।15

पिनुद्धिश्य यद्दत्तम् श्राद्धम् पिण्डम् तिलोदकम्।
आकलू तृप्तयेभूयादिति वेदानुशासनम्।

 ब्रह्माद्यादेवतास्सर्वा मातरस्सप्तचैव हि।
 सर्वाश्चाप्सरसोगौरिवाणि वैकुण्ठवल्लभा।

 इन्द्राणीरोहिणीचैव यच्चान्या अमराङ्गणाः।
 तुलामासे तु कावेर्याम्सम्प्राप्य स्नान्तिनित्यशः।

 किमत्रबहुनोक्तेन बन्दयेमुक्तयेनृणाम।
 ब्रह्मणानिर्मिता पूर्वम्कावेरीतीर्थनायिका।

 कावेर्याम्तु नदीसङ्गे यस्स्नाति प्रातरन्वहम्।
 विशिष्यतौलिकेमासि सर्वतीथमयेजले। 20

 श्वसुरस्य स्वपित्रोश्च त्रिकोटिकुलमुद्धरेत।
 सह्यजायाम्तुलामासि स्नानमम्ब:क्षयाय च।
 
 इष्टकामाप्तयेनहाँसन्ति नेहनरात्र्यः।
 कावेरीतीरजन्मानो मृगपक्षीमहीरुहः।
 
 तद्वारिशीतवाद्यैश्च स्पृष्टा मुक्तिम्प्रयान्ति वै।
 भक्त्यापुन:किम्मनुजास्याप्लुता निर्मलाम्बसि।
 
 तस्मात्कावेरिमाहात्म्यम्साक्षाच्छेषेणवाग्विदा।
 न शक्यम् विस्तराद्वक्तुम् दिव्यवर्ष शतैरपि।
 
 अहम्सझेपतो वक्ष्ये वैभवम् सह्यजोभवम्।
 विद्वत्प्रभावस्तुलसीमाहात्म्यं ज[जोद्भवम्।25

 एकादश्याश्चमाहात्म्यम् पर्युपस्ति समुद्भवम्।
 तुळस्याप्रभावम्च सालग्रामशिलोपरि।
 
 सहस्रवदनो अनन्तश्चक्तोवक्तुम्स्वयम्प्रभुः।
 कावेर्यामहिमानत्वं प्रोच्यमानम्मयाऽधुन।
 
 शृणुष्वावहितो राजन् सहस्रारायुधप्रियम्।
 कावेर्यास्स्मरणम्चस्तम्कीर्तनन्दर्शनम् श्रवः।
 
 स्पर्शनन्तजलस्नानम्जन्मान्तरतप:पलम्।
 अनेकजन्मसम्सिद्ध पुण्यराशेर्नगस्य च।
 
 स्नातुम्मतिर्भवेद्राजन्! कावेर्याविमलेजले।
 तुलामासे च कावेर्याम्माघेवैशाख एव च।30

 यस्स्नाति तत्फलम्वक्तुंशक्त एव हरि:स्वयम्॥
 "यथा गङ्गानदीश्रेष्ठा विष्णुपादाब्जसम्भवा।
 
 यथा पुष्पेषुतुळसी व्रतेष्वेकादशीव्रतम्।
 यथा पञ्चमहायज्ञा ग्रहस्थाश्रमकर्मणि।
 
 शुद्धौयथा मनश्शुद्धियथा देवेषुमाधवः।
 यथा वर्णेषु ओङ्कारो मन्त्रेषु त्रिपथा यथा।
 
 यथा सामं च वेदेषु यथा रुद्रेषुशङ्गरः।
 अरुन्धतीब्राह्मणी नाम्ना नारीणाम्कमला यथा।
 
 यथाऽन्नदानं दानेषु ग्रहेष्विन्दुर्यथपि च।
 तथा सरित्सु कावेरी राजेन्द्र प्रवरास्स्मृता। 35

  तेजस्विनाम्यथा सूर्यो यज्ञाम्मनयज्ञवत्।
मित्रेष्वपि यथाधर्मो जपवत्तपसां नृप।

 विष्णुपूजेवपूजानाङ्गार्हस्थञ्चाश्रमेष्विव।
 यथा वर्णेषु भूदेवा: क्षदावत्सुरसा यथा।

 तथा सरित्सुराजेन्द्र कावेरीप्रवरास्स्मृता।
 ब्रह्मास्त्रञ्च यथास्त्रेषु पुण्यक्षेत्रेषु रङ्गवत।

 पावनेषु यथासेतु: सूक्तानाम्पौरुषम्यथा।
 तथा सरित्सुराजेन्द्र कावेरीप्रवरास्स्मृता।

 धेनुनाम् कामधुग्धेनुयुगेषु च यथाकृतम्।
 जाह्नव्या: कीर्तनान्मुक्तिः 

धनुष्कोट्यास्तु दर्शनात्। 40

 श्रवणादामगाधाया: कावेर्याः स्मरणम्भवेत्।
 आसेतुकैलासनगम्कर्मभूमिरियम्स्स्स्मृ ता।

 नवयोजनसाहस्रम्मध्यम्वैशाल्यमागतम्।
 दीर्घतश्च तथामध्ये सहस्र शतयोजनम्।

 एतस्मिनन्तरे राजन्! सुदेश:पुण्यकारणम्।
 पुण्यव्रताद्यालोका हि तूष्णीमन्यत्रभोगतः।

 जन्मान्तरसहस्रेषु यत्पुण्यम् समुपार्जितम्।
 तत्पुण्येनभवेजन्म नरमात्रम्तु केवलम्।

 बहुजन्मभवैपुण्यभवेदेतादृश द्विजः।
 न्यायात्स्यात्काकताळियादस्याम्भूमौ हि विप्रता।45 

दुर्लभम्मानुषञ्जन्म बहुपुण्यसमुद्भवम्।
यो न स्नाति च कावेर्याम् स एव नृपगोखरः।

विशेषेण तुलामासे यस्नायात्सह्यजाजले।
 किम् तस्य कर्माचरणैः इष्टापूर्तादिकश्च वा।

 किम् तपोभिश्च किम् वेदैःकिम् तीथैःकिम् व्रतादिभिः।
 कावेर्याम्तु सकृत्स्नात्वा साक्षान्नारायणोभवेत्।

 अन्यव्रतेषुवैकल्ये प्रायश्चित्तम् विधीयते।
 तूष्नीम्कवेरजास्नानम् सप्तजन्माघनाशनम्।

 किम् पुनर्नियमेनैव कावेर्याम्स्नातयोर्नरः।
 स सप्तकुलमुद्भुत्य हुभयत्र हरिम्ब्रजेत् 50

 ब्राह्ममुहूर्तमुत्थाय हरिध्यानपरायणः।
 अथोषसि च सम्स्नाया इन्तधावनपूर्वकम्।

 नमस्कुर्यात्तु कावेरीम्पुष्पाञ्जलि पुरस्सरम्।
 नमस्क्रुत्य च रङ्गेचम् दद्यादर्घ्यम् शु चिर्जले।

 कृत्वासन्ध्यादिकम्कर्म जप्त्वा च शृतिमातरम्।
 पौराणिकम् द्विजम् भक्त्या ह्यर्चयन् शृणुयात्कथाम्।

 सर्वेसम्मिळिता भूत्वा चैकाग्रमनसस्तथा।
 विधाय चासनंशुभ्रम्तन्मध्ये शुद्धमुन्नतम्।

 श्लक्ष्णम् मृदुमनोज्ञम्च सूक्ष्मवस्त्रोत्तरक्षतम्।
 तस्मिन्जितेन्द्रिययंशांतं जितक्रोधम्सदासुचिम्। 55

 वेदवेदाङ्गतत्वज्ञम्वेदान्तश्रवणाद्भुतम्।
 धर्मशास्त्रप्रवक्तारम् पुराणेषु महामतिम्।

सदाचारपरम्विप्रम् स्थापयित्वा वरासने।
 भूषयित्वा यथान्यायम् नववस्त्रायभूषणः।

 मनुष्यबुद्धिम्सन्त्यज्य व्यासो सौ वदति स्वयम।
 इति निश्चित्य विप्रेन्द्रम सम्प्रार्थ्य च मनोरथम।

 पुराणपठनेनास्मान् पावयाशु द्विजोत्तम।
 तुलाकावेरिमाहात्म्यम् श्रोतुकामावयम् विभो।

 इति सम्पार्थ्य तम् राजन् शृणुयाद्वै समाहितः।
 यावत्समाप्तिमानेन शृणुयात्पापनाचनम्। 60

  यत्किञ्चिन्नियमं कृत्वा स्नातव्यम् मानवैनृप।
 तूष्णीमशक्तस्सम्स्नायन् नोचेत् स्नानम्तु निष्पलम्।
 
 तस्मात्स्नानविधिं कृत्वा स्नात्वासद्गतिमाप्नुयात्।
 तैलाभ्यङ्गन्दिवास्वापम्क्षौरन्ताम्बूलभक्षणम्।
 
 दुराचाराशनञ्चैव स्त्रीसङ्गं दुष्टसङ्गतिम्।
 वृधावार्ताङ्गटेच्य्याम्निषिद्धव्यञ्जनादिकम्।
 
 प्रतिग्रहम्परान्नम्च पन्थानम्वर्जयेट्विजः।
 विशेष नियमम्वक्ष्ये तुलासम्स्थेदिवाकरे।
 
 कवेरजायाम यस्स्नायात् शृणुष्व नृपसत्तम।
 कूष्माण्डम् बृहतीभेदम्निष्पावम्च कुळुत्थकम। 65

 राजमाषम् महामाषमाढकंशाकमेव च।
 कोचातकीमलाबाञ्च शिग्रुम्च तरुणीम् तथा।
 
 भोजनङ्काम्स्यपात्रे च निशाभोजनमेव च।
 प्रात:कालेच मध्याह्नसायम्काले तथैव च।

भिस्सटाम्सूरणञ्चैव कर्कन्धूर्वारुकम्तथा।
 पारसीम्विष्णुशाकम्च बालैर्जगदम्च भोजनम्।

 तथा पर्दूषितञ्चान्नम्माहिषञ्चालिकम्पयः।
 दुष्टान्नमनिवेद्यान्नम् स्त्रीशेषम्केशदूषितम्।

 श्रा द्धशेषम् वर्जयेत्तु रुद्रशेषम् विचक्षणः।
 पुण्यम् न लभ्यते राजन् कायक्लेशम्विना भुवि। 70

  तस्मादेवम्विधिम् कुर्यात् कावेर्याम् स्नानतत्परः।
 निषिद्धभक्षणम् कृत्वा मोहात् सूकरताम् व्रजेत्।

 पूर्वोक्त व्यञ्जनत्यागा दुष्टवस्तुविवर्जनात्।
 तुलायाम्सह्यजास्नाने मुक्तिर्येव न संशयः।

 विधिहीनम् तु यत्स्नानम् केवलम् पापनाशनम्।
 देहस्तुबुद्भुधाकारो विण्मूत्रशकृतालयः।

 जागरूकोयमो राजन् रन्द्रान्वेषी च देहिनः।
 उदयास्तमया राजन्नायु:क्षयकरौरिपू।

 न कर्तव्यम् न कर्तव्यम् न कर्तव्यं वृधादिनम्।
 स्वाधीनोवर्ततेदेह इन्द्रियाणिशुभानि च। 75

  नद्यश्चसुलभा:पुण्या शरत्काल शुभावहः।
 निद्रापिशाचकालहित्वा ह्युष:कालेसमुत्थितः।
 
 कुर्यात् कवेरजातोये स्नानं नरकभेतिमान्।
 भूयोभूय:प्रवक्ष्यामि भुजमुद्यम्य दक्षिणम्।
 
 प्रवहतितटिनीनाम्मुत्तमासह्यजाता
 सकलदुरितसङ्घर्वम्स दीक्षासुदक्षा।

व्रतविधिनियमैश्च स्नान्त्य कामानराये
प्रभवतिपरलोक स्नानमात्रेण तेषाम्। 78

 इति श्रीमदाग्नेयपुराणे तुलाकावेरि माहात्म्ये तुलाकावेरि माहात्म्य श्रवणप्रकारः नाम द्वितीयोऽध्यायः

***************************
अथ त्रितीयोऽध्यायः

चंद्रकांताविद्यावळीरुपाख्यानम्।

अगस्त्य:

शृणु भूप प्रवक्ष्यामि तुलाकावेरिवैभवम्।
 तुलामासे च कावेरी स्नानात्सर्वाथदानृणाम्।

 तस्मात्सर्वप्रयत्नेन तुलासम्स्थे दिवाकरे।
 सर्वपापप्रणाशिन्यां कावेर्याम्नानमाचर।

 मातृहापितृहागोना भ्रूण:गुरुतल्पगः।
 वेदाद्ययनहीनश्च सदाचार विवर्जितः।

 सुरापी केवलान्नादि वह्नयुपास्ति विवर्जितः।
 कावेर्यान्तु सकृत्स्नात्वा ब्रह्मलोकेमहीयते।

 उष्णोदकेन वा स्नायादशक्तस्तु बहिर्जले।
 तदश क्तस्तुकावेर्या माहात्म्यम्चुणुयान्नरः। 5

  तत्राप्यशक्तोमाहात्म्यवक्तारम्पूजयेद्धनैः।
 रिक्तस्तुपूजयेत्वाक्यैच्शोभनिर्घटेर्धनैः।
 
 इनोदयेतु नारीणाम्वैदिकाचारवर्जनात्।
 हीनजातेश्चबालानाम्तूष्णीम्स्नानं प्रचक्षते।
 
यस्स्नात्वा सह्यजावारि प्रात:काले तिलाक्षतैः।
 तर्पयेपितृदेवर्षीन्दीर्घमायुरवाप्नुयात्।

 प्रीणन्तिपितरं तस्य सिद्धाश्चमुनयास्सुराः।
 यक्षा:किम्पुरुषानागा रुद्रादित्यामरुद्गणाः।

 किन्नारावसवस्साध्या सर्वेलोकाश्च तर्पिता:।
 नत्वा स्तुत्वा समुद्यन्तमादित्यम्लोकसाक्षिणम्। 10

  यथाशक्ति च माहात्म्यं शृणुयाच्छावयेदपि।
 मरुद्बुधायामाहात्म्यं लिखितम्यश्चमन्दिरे।
 
 पूजयेदन्वहम्विप्र स्वयम्वागीश्वरो भवेत्।
 विप्रायविदुषेदेयम्मात्म्यं धर्मसाधनम्।
 
 समातपापनिर्मुक्तो विष्णुलोकेमहीयते।
 शृणुराजन्प्रवक्ष्यामि तन्मासे दानमुत्तमम्।
 
 मरुद्बुधायाम्यस्स्नात्वा सन्तर्प्यपितृदेवताः।
 तत्कालीनं शुभंद्रव्यम्द्विजेन्द्राय ददाति वै।
 
 भुक्त्वेहसकलान्भोगान्ब्रह्मलोके महीयते।
 घृतेनवापि तैलेन तुलामासे निरन्तरम्।15 

 हरस्यवा हरेरग्रे दीपाराधनमाचरेत्।
 ससूर्यलोकमाप्नोति पश्चात्त्ज्ञानीह जायते।
 
 योवस्त्रदाननं कुरुते दरिद्रायद्विजातये।
 सदीर्घायुर्धनीभूत्वा सोमलोकेमहीयते।
 
 दरिद्राय च पान्थायश्रोत्रियायकुटुम्बिने।
 भूमिम्वा निलयम्वापि योदद्यात्स्नानतत्परः।

ब्रह्मलोकमसौ भुक्त्वा भूमौसाम्राट् ततो भवेत्।
 योददातिदरिद्राय धनम्वाधान्यमेव वा।

 सदीर्घायुर्धनीभूत्वा कुभेरस्यसखाभवेत्।
 य:कुर्यान्मधुधानम्तु तौलिस्थे द्विजन्मने। 20
 
  वन्द्योपि पुत्रवान्भूयाच्छ्रोत्रियो जन्मजन्मनि।
 कृषीवलायरिक्ताय दद्यादडुहोपि यः।
 
 गोलोकेसकलान्भुक्त्वा पश्चात्द्भूम्यधिपो भवेत्।
 सबालाय च विप्राय गाम्सवत्साम्ददाति यः।
 
 स पुत्रवान्भवेद्दाड्य स्तस्यवंशश्च वर्धते।
 रुणत्रयविनिर्मुक्त:पित्रुलोकमवाप्नुयात्।
 
 योदद्यात्तुलमासे महिषीं मृत्युनाशिनीम्।
 नाप मृत्युभयम्तस्य तवंश्यश्चा शतायुषः।
 
 यो बीजदानं कुरुते तुलामासे द्विजन्मने।
 भुक्त्वेह सकलान्भोगान्दीर्घायु र्बहुपुत्रवान्। 25
 
  पश्चात्त्स्वर्गम्तु सम्प्राप्य यावदिन्द्राचतुर्दश ।
 दिव्यस्त्रीभिमंक्रीडन्पश्चान्द्रामाधिपो भवेत्।
 
 य:कुर्यात्तण्डुलन्दानम्दरिद्राय द्विजन्मने।
 तस्यगेहे महाराजा ! न वसेश्च कलि:क्वचित्।
 
 योरम्भाफलदा तास्यात्स्तनपिडितः।
 तुलामासे च ताम्बूलम्योदद्यान्नाळिकेरकम्।
 
 रम्म्भाधररथा स्वादलम्पटस्यान्नसंशयः।
 कर्पूरगरुकस्तूरि गन्धन्दद्याड्विजाय यः।

सर्वाप्सर:कुचतटी चर्चापङ्किलदेहवान्।
 स्वर्गेभुक्त्वादेवभोगान्सार्वभौमोपि जायते। 30

  यो गव्यदानं कुरुते तुलासम्स्थे विवस्वति।
 पशु वृद्धिम्पुत्रवृद्धिमायुर्वृद्धिमवाप्नुयात्।

 यश्चामलकपिष्टानाम्दानंकुर्याद्विजातये।
 तत्वज्ञानमवाप्नोति सस्स्याद्भागवतोत्तमः ।

 य:कुर्यात्तिलदानम्वै पिचूनुद्धिश्य भक्तितः।
 पित्रूणाम्तत्क्षणादेव गयाश्राद्धफलम्भवेत्।

 योमञ्चमुपधानम्वा स्वास्थरंम्मुदुसम्स्थरम्।
 तीर्थार्थस्यदरिद्रस्य प्रददाति कुटुम्बिनः।

 स भूयादिव्यनारीणांकुच कुङ्कुमपङ्किल:।
 तुलामासेतु योदद्यत्चीतार्थाय तु कम्बळम्। 35

  वायुरोगादिभि:घोरैर्मुच्यते नात्रसंशयः।
 यश्छत्रदानं कुरुते न प्रासादेहि मोदते।

 अर्चनार्थं रमेशस्य मल्लिकाकुसुमानि वै।
 कैतकम्पाटलम्वापि तत्कालकुसुमानि वै।

 स भुक्त्वा सकलान्भोगान्ब्रह्मलोके महीयते।
 पच्चाद्धनी च सत्पुत्रो विष्णुभक्त:प्रजायते।

 यज्ञोपवीतयुगळम्योदद्यादग्रजन्मने।
 सद्ब्राह्मणवरिष्टस्याद्दश जन्म सु वेदवित्।

 त्रिवृन्मौञ्जी तु वेदाङ्गम्वर्णिनेयो विशाम्पते।
 वितरेत्वीतरोगस्यात्सर्वविद्या विशारदः। 40

 
यज्ञोपवीतसिद्ध्यर्थम्तूलदानम्करोति यः।
 सकुष्ठैर्नाभिभूयोत सद्ब्राह्मण्यमवाप्नुयात्।

 य:कुर्यात्विष्णुभक्ताय तुळसीदानमुत्तमम्।
 गत्वा निवार्योलोकान्स सार्वभौमोभवेद्भुवि।

 समिद्दानं काष्ठदानम्य:कुर्यात्सहविर्भुजे।
 तुलामासे तु राजेन्द्र स विद्वानग्निमान्भवेत्।

 तुलामासेतु कावेरीतीरेतु पयसाशृतम्।
 नानाशाक समायुक्तम्नानारस समन्वितम्।

 योऽन्नं ददाति राजेन्द्र स देवो नात्रसंशयः।
 योमुद्गमाषमधुतैलमरीचिरूक्ष

द्रव्यं गुडं चणक शर्करसर्पिषाम्च।
 दानंकरोति निजशक्त्यनुसारतच्चा

स्स्वर्गभाग्सकल लोकसुखालयश्च 45

 तुलामासेतु कावेर्याम्यस्स्नायाद्रङ्गसन्निधौ।
 तस्य पुण्यफलम्वक्तुं शक्तयेव हरि:स्वयम्।

 ब्राह्मणीस्वैरिणीकाचित्स्नात्वा सह्योद्भवाजले।
 स्वर्गलोकम्च सम्प्राप्य तद्भक्त्या च हरिङ्गता।

 तिस्र:कोट्यर्धकोट्यश्च तीर्थानि भुवनत्रये।
 तानिसर्वाणि राजेन्द्र पुरस्कृत्यतु जाह्नवीम्।

 केशवस्यायाज्ञया यान्ति तुलामासे मरुद्भुधाम्।
 शतवर्षन्तु गङ्गायाम्यस्स्नाति नियमस्थितः।

 सत्स्नानंकुरुते तौलौमासि सह्यसुताजले।
ब्रह्माणम्पर्यपृच्छत्तम्गङ्गालोकैकपावनी 50

 स पापानाम्महापापै लेपिता शोकसम्युता।
 स्वशुद्ध्युपायम्राजेन्द्र गङ्गैषा वचनाविधेः।

 तुलामासेतु कावेर्याम्नाति श्रीरङ्गसन्निधौ।
 कावेर्याम्मौसलस्नानम्सप्तजन्माघ नाशनम्।

 भावशुद्ध्यातु यस्स्नायात्कुलकोटि समुद्धरेत्।
 तुलामासे तु यत्स्नानम्भुक्तिमुक्त्यैकसाधनम्।

 सह्यजायास्तुमाहात्मियं कोब्रूयाद्भुवनत्रये।
 स एव वक्ताश्रोता च सहस्रवदनस्स्वयम्॥ ।

 हरिश्चन्द्र:ब्राह्मणी कीदृशी बृह्मन्! कस्य भार्या च पाम्सुला।
 कथम्जाता गतास्वर्गम्पापिष्ठा नरकातिथिः। 55

  वक्तुमर्हस्यशेषेण मुनिवर्य नमोस्तुते।
 भवादृशामुनिश्रेष्ठा लोकानुग्रहकारिणः।

 निर्हेतुकदयावन्तो ब्रूयुर्गुह्यम्च मादृशाम्।
 कृपांकुरु मयि ब्रह्मन् दासोस्मि तव केवलम्।

अगस्त्य:
साधुपृष्ठम्महाराजन्! सत्कथा कथने त्वया।
 वक्ष्ये मरुद्बुधास्तोत्रम् स्वर्गतास्वैरिणी यथा।

 आसीद्राजन् पुरीश्रेष्ठा मधुरामधुराकृतिः।
 कृतमालानदीतीरे वृषभाद्रिसमीपतः।

 रथाश्वगजसम्पूर्णा भीमप्राकारसंवृता।
ब्रह्मक्षत्रिय वैश्यश्च शूद्रैर्बहुभिरावृता 60

 हर्म्यप्रासाद सम्पूर्णा नानानिगमशोभिता।
 तस्याङ्गश्चिविजेन्द्रोभूद् वेदवेदान्तपारगः।

 दान्तक्षान्तस्सर्वसहो द्वन्द्वातीतो विमत्सरः।
 विष्णुभक्तो महायोगी नित्यञ्चातिथिपूजकः।

 वेदराशिरितिख्यातो विदुषामग्रणीनृप।
 पञ्चयज्ञपरोनित्यम्प्रातस्नान परायणः।

 तस्स्यभार्याभवच्छुद्धा साद्वीनित्यम्पतिव्रता।
 चन्द्रकान्तेति विख्याता चन्द्रकान्तमुखोज्ज्वला।

 कुम्भस्तनी नृपश्यामा तप्तहाटकवर्णिनी।
 सर्वावयवसम्युक्ता सर्वाभरणभूषिता। 65

  मत्तहम्सस्वरा राजन् मत्तमातङ्गगामिनी।
 मञ्जुवेषा मन्दहासा बिम्बोष्ठी लोकमोहना।

 स्फुरत्तिलकहाराड्या रत्नवन्मुद्रिकाङ्गुली।
 गन्धकुङ्कुमकस्तूरी चर्चाराजस्तनान्तरा।

 एवम्भूता चन्द्रकान्ता कान्ताकान्तिप्रियासती।
 पतिम्परिचचराथ परलोकेपुरीश्वरम्।

 शुश्रूषाकारिणीभर्तुः पादसम्वाहनेरता।
 आस्तिकाचित्समीपेस्या: ब्राह्मणी पतिघातिनी।

 विद्यावळीति विख्याता चटुलाकुटिलाकृतिः।
 पातिव्रत्यम् तु साद्वीनाम्विघ्नन्ती पुंश्चलीसदा। 70

  स्ववृत्तमनुरञ्जन्ती चन्द्रकान्तामुवाच ह।
विन्द्यावळीचन्द्रकान्ते महाभागे मम नेत्रस्वरूपिणी।

 रहस्यम्वद मे किञ्चित्तवाहम्प्रियकारिणी।
 अवश्योवशवर्तीवा त्वद्भर्ता तव शोभने।

 आदृवम्यौवनम्स्त्रीणाम्कालोदेशश्चतादृशः।
 स्वच्छन्दा त्वम्स्वयम्नित्यम्किमु सम्भोग लालसा।

 शृत्वाद्यसङ्गमोपाय श्चिन्त्यते दक्षया मया।
 शृत्वा तद्वचनम्घोरम्ब्राह्मणी विषसम्मितम्।

 उवाचलज्जिता किञ्चित्साद्वी पतिपरायणा।
 किमिदम्पोच्यते दुष्टम्वाक्यन्दुर्मनसा त्वया।
 अनुत्तरा यदिस्याम्ते मिथ्या दुष्ठाभवाम्यहम्। 75

  त्वत्तोभीताभवाम्यद्य भावो हि भवकारणम्।
 आरभ्यपञ्चमदिनमाषोडश दिनावधि।

 रतिकालस्तु दम्पत्यो स्समयश्शोभनावहः।
 मध्ये हि दिवसान्हित्वा निषिद्धाम् बहुशो वदन्।

 सम्भावयति माम्भद्रे शृति स्मृति विचक्षणः।
 परलोकस्तु सम्साद्य आवाभ्याम्भोगपूर्वकम्।

 निषेकदिवसे शुद्धे यदिगर्भ:प्रजायते।
 उत्पन्नश्च सुतोविद्वान् दीर्घायुस्सधनी भवेत्।

 अन्यकाले प्रजाजाता आयुरारोग्य वर्जिताः।
 दु:खदाश्च सदापित्रोरितिसन्तो वदन्ति हि। 80

  षष्ट्यष्टमी हरिदिनम्द्वादशी च चतुर्दशी।
पर्वद्वयम्च सङ्क्रान्ति श्राद्दाहो जन्मतारका।

 श्रवणम्व्रतकालश्च दिवासन्धामुखम्तथा।
 एते कालानिषिद्दास्युभद्रे मैथुनकर्मणि।

 एषु क्षौरं च वायव्यं चाभ्यंगम्दन्तधावनम्।
 आचरेद्य:पतेत्सद्य श्चतुर्वेद्यापि सोमकृत्।

 एवं गृहस्थस्य विधिर्मयाप्रोक्ताद्य सङ्ग्रहात्।
 तद्वाक्यम्ब्राह्मणी शृत्वा ताम्वशीकर्तुमब्रवीत्।

 अभद्रे शृणु मे वाक्यं शुभदम्नष्टमङ्गले।
 आत्मप्रियो हि नीरोगस्सर्वेषाम्प्राणिनामपि। 85

  एतद्वयसि सम्भोगस्त्यज्यते यौवने कथम्।
 भवेज्जर्घरितोदेहो लम्बस्तन जुगुप्सितः।

 अमैथुनो हि रोगीस्यात् त्वद्भर्तापि भुजिष्यया।
 सखीक्रीडत्यहोरात्रं न जानासि वञ्चनाम्।

 ऋजुत्वं शुवंद्धभावा च भर्तादूर्तश्चठस्तव।
 ममेष्टा सा शुभादासी ममैतद्वृत्तमब्रवीत्।

 तवस्नेहान्मयाख्यातम्सुखदु:ख समानया।
 तद्वाक्यम् पावकं शृत्वा चन्द्रकान्ताऽहताम्पुनः।

 शठोवाप्यशठोवापि दूर्तोरुघ्णोथवाऽशुचिः।
 क्रोधनोवापि मूल्वा स्त्रीणाम्भाहि दैवतम्।

 न च वैदिकसम्स्कारो नानुष्ठानम्नदैवतम्।
 नचासाम्तु हरिभक्ति: योषिताम्किमुविद्यते।

 स्वर्गेच्छूनाम्तु नारीणाम् पति:परम ईश्वरः।
शुनिस्यादन्य धर्मिष्ठा यानारी पतिनिन्दिनी।

 विद्यावल्याऽहताम्भूयो दूर्ता दूतप्रियम्वचः।
 ऊर्वशीमेनकारम्भा घृताचीपुञ्जिकस्तला।

 एतादृश्यस्त्रियो भद्रे देवदास्यो न चाभवन्।
 पुण्यज्ञा देवदास्यस्ता स्संयङ्मत्वाङ्गमदृवम्।

 इहवा परलोके वा सर्वेषाम्हिसुखम्फलम्।
 स्वर्गश्च दुर्लभोदृश्य : पुण्यमित्यपि तन्मृषा। 95

  सर्वेषामपि जंतूणां प्राप्तव्यं नरकं दृवम्।
 न स्वर्गो नरकम्सत्यम्पापिनो पापिनोपि वा।

 अहम्च सर्वम्जानामि पुण्यपापविनिश्चयम्।
 पतिम्हत्वाऽभवन् दूर्ता स्वाधीना निर्भया भुवि।

 स्वाधीनो लभते सौख्यम्स्वाधीनो लभते तपः।
 स्वाधीनोलभतेपुण्य मस्वाधीनस्य किम्सुखम्।

 एतादृशैश्च तद्वाक्यैर्निजचित्त प्रमाथनैः।
 चटुलैस्सह दृष्टान्तैर्दुर्बोधैर्नरकप्रदैः।

 बोधिता चन्द्रतकान्तापि पूर्वदुष्कर्मयन्त्रिता।
 स्त्रीचापल्यादस्थिरत्वान् मनसो मैथुनेच्छया। 100

  दृष्टसौख्यमभीप्सन्ति कुम्भीकुम्बस्तनी तु सा।
 कुलाचारम्च सद्वृत्तम्परित्यज्य पतिव्रता।

 रहस्येपाम्सुलाजाता मासम्पश्चात्त्प्रकाशिता।
 तदुर्वृत्तं स विज्ञाय तद्भाब्राह्मणोत्तमः।

 दशरात्रम्तयासार्धम्निवसन् स्व गृहेवशी।
तामिङ्गितज्ञो निश्चित्य पुंश्चलीम्दुवैचोमुखैः।

 बहिष्कृत्य गृहाद्भार्याम्त्यक्त्वा च गृहसम्पदः।
 मम स्त्रिया अभूद्वृत्तमेतादृगिति विस्मित: 

 तत्सङ्गताघ निष्पत्तिम्हरिष्यन्नज्ञतोद्भवाम्। 
 पश्चात्तप्त:पापभीरु स्सह्यजातीरवास्यभूत्। 105 

 इति चंद्रकांताविद्यावळीरुपाख्यानम् नाम तृतीयोऽध्यायः

*******************************

अथ चतुर्थोऽध्यायः 

चंद्रकांतायाः कावेरीस्नानेन पापविमोचनम् 

अगस्त्य :

चन्द्रकान्तापि तत्पुर्याम्यो न आहूयजारकान्।
 हृत्वा तेभ्योन्नवासस्रग्गन्धभूषाधनादिकम्।

 तैस्समम्बुभुजेभोगान् मोहनैर्दासमेळनैः।
 ब्राह्मणा:क्षत्रियावैश्याश्शूद्राश्चण्डालजा:खलाः।

 तया सम्भोगमासेदुर्देवामुह्यन्ति वीक्ष्ययाम्।
 अथ तत्पुरवासी तु पाण्ड्योद्रामिडसत्तमः।

 इन्द्रद्युम्न इति ख्यातस्तां नारीम्कुलदूषिणीम्।
 दूतैर्विद्रापयामास स्वराष्ट्रात्पापशङ्कया।

 चन्द्रकान्ता च तैस्साकम्निर्लज्जा च विलासिनी।
 सर्वद्रव्येहृता राज्ञा रिक्ताप्राप्य मरुद्भुधाम्। 5

 तस्यामासक्तमनस: कामभोगैश्च निर्जिताः।
 चोरयित्वा बहुद्रव्यम्तस्यैददुररिन्दम।
 

तस्यास्तु चन्द्रकान्ताया दुश्चारिण्या इतस्तत:।
 महापापान्महारोगो बभूवास्य महीपते।

 तेनरोगेण महता क्रमात्क्षयमुपागता।
 भुक्त्वामन्वन्तरम्घोरान् नरकानेकविंशतिम्।

 पुनर्दास्यास्सुताजाता स्त्वैरिणी निचुळापुरे।
 पतिद्रोहेणपापेन व्याधिघ्रन्थी प्रपीडिता।

 स कृमिग्रन्थिरभवत् कर्णानासाङ्गुळीष्वपि।
 विक्रोशन्ती दिवारात्रम्हाहेति रुदतीसदा। 10

  दह्यमाना च तापेन बाधिता कृमिकोटिभिः
   पूर्वसञ्चितपुण्येन प्राप्तातत्सह्यजातटम्।
 
 प्राप्यापश्यन्महाभागान् कावेर्याम्नानतत्परान्।
 श्रीरङ्गसन्निधौनद्या माहात्म्यंशृण्वतो द्विजान्।
 
 दैवात्तत्रागतासन्नौ तापशान्त्यर्थमम्भसि।
 तुलामासे तु कावेर्या माहात्म्यपठनेसति।
 
 शुश्राव श्लोकमेकम्तु दिव्याम्हरिकथामपि।
 तुलामासे सकृत्स्नाति य: कवेरसुताजले।
 
 प्रभाते पतितोवापि विष्णुलोके महीयते।
 एतस्मिन् समयेचायु:क्षये साम्यवपुर्धराः। 15 

 आरोपयन् विमानाग्रं चंन्द्रस्कान्ताम्यमानुगाः।
 सकृत्सह्योद्भवावारि स्नानमात्रेणचानघ।
 
 दिव्यम्विमानमारूढा देवलोकम्जगाम सा।
 मार्गेमार्गे च विश्रय यथासुखमरिन्दम।
 

अत्त्वाफलास्यान्नमहोरसायनम् पेयापेयांश्चपीत्वा रसवश्चपानकम्।
 स्तुत्वा तु सह्याद्रि कुमारिकाजलम् नत्वा च साधूंपथिनाक मापसा।
 विद्यावळी च यमचोदित सारमेय शीर्णामुहुःप्रगळिता शृरसत् स्वरूपा।
 आब्रह्मनिद्रमनुभूयदरौरवादीन् भुक्त्वाशुनी त्रिदश जन्मसु सूकरीत्रिः।
 अम्बेति काशीनृपते स्तनयाभवित्री गाङ्गेयभार्गव महासमरप्रदास्यात्।
 वेदराशिरपिवेदविद्वरस्सह्यजापयसि तौलिमासके।
 स्नानकर्मविधिवत्तदाकरोद्विष्णुभक्तिनिरतस्तपोनिधिः।
 20 कालधर्ममवाप्याथ कावेरीस्नानपुण्यतः।
 निष्पापस्सूर्यसङ्काश: शुद्धान्त:करणो भवत्।
 कोटिसूर्यप्रतीकाशम्विमान मधिरूढवान्।
 वैवस्वतपुरम्नीत सौम्याकारैर्यमानुगैः।
 पुण्यपापफलम्भोक्रुम्स्तन्मार्गे सन्ददर्शह।
 कृत्वा पापम्महाघोरम् भुञ्जानान् बहुयातनाः।
 गच्छन् गच्छन् सुखम्साधून् दृष्ट्वा विषयमागतः।
 यमदूतम्वदोवादीद्वेदराशीन् द्विजोत्तमः।
 मार्गमध्येऽद्यगन्तव्यम्द्रष्टव्य:पुरुषा मया।
 इत्युक्तो यमदूतोस्मै दर्शयामास तान् पथि।
 25

हाहाकारम्क्वचिद्देशे सोपचारम्क्वचिद्वचः।
 शृत्वा भीतश्च तन्मार्गे नानाकारान् ददर्श सः।
 भुज्यताम्भुज्यतामन्नम्स्थीयताम्स्थीयतान्तरौ।
 गम्यताम्गम्यताम्मन्दमेव मासीत्क्वचिद्वचः।
 इदं क्षीरमिदंतकं इदमाज्यमिदम्दधि।
 रसाड्यम्पानकं दिव्यं मिदं दिव्यरसायनम्॥ पूर्वं विप्रवरेदत्तम्युष्माभिरिद मोदनम्।
 एलालवङ्गकर्पूरवासितम् शीतळञ्जलम्।
 इमानिदूतकदळी पनसाम्राद्भवानि च।
 फलान्यत्वा सुखम्यात गन्धकुङ्कुमचर्चितः।
 30 दत्तम्पूर्वम्द्विजेन्द्रेभ्यो यद्यद्भक्त्या च साधवः।
 भवद्भिर्दानशीलैश्च तदिदानीमुपस्थितम्।
 नादत्तम्लभ्यते किञ्चित्साधुनापि तपस्विना।
 दानाद्धनमवाप्नोति तपसात्रिदिवं नरः।
 तस्माद्विजेभ्यो दातव्यम् स्वल्पम्वित्तानु सारतः।
 नादात्रोलभ्यते नाको नादात्रालभ्यते धनम्।
 नादात्रा लभ्यतेसौख्यम्नादातुर्मान संशुचि।
 नादात्रा लभ्यतेपुत्रा नादातुर्ज्ञानमुत्तमम्।
 नादात्रु:पितृ सन्तोषो नदाता पशु वृद्धिमान्।
 न दाता नरकङ्गच्छे न्नदाता विविधा व्यधाः।
 35 कृता:पञ्चमहायज्ञा: प्रातस्स्नानम्सदाकृतम्।
 माघेमाधवमासे च तुलामासे च कार्तिके।
 

स्नानानम्मरुद्रुधामुख्य पुण्यस्रोतोवहासु च।
 त्रिदिवम्प्राप्यते सद्भिरस्मिन् जित्वा भयंकरान्।
 हेसाघ्योऽरुन्दतीतुल्या भवत्योभर्तृवल्लभाः।
 स्वकृतम्सुकृतम्भूमौ पाथेयम्पारलौकिकम्।
 सदाचाररता यूयम्नित्यशुद्धा:पति:प्रिया:।
 वैदिकैःकर्मभि:लभ्य स्सुलभस्त्रिदिवो भवत्।
 निषिद्धेषु च रुग्नेषु रुष्टेष्वपि च भर्तृषु।
 भक्तिरूपम्च पाथेयंकृतम्प्राप्स्यथ तत्फलम्।
 40 तस्माद्धर्मादवश्यम्तु सुखम्स्वर्गम्गमिष्यथ।
 इत्येवमुपचारेण यमदूतैस्सुपूजिता:।
 दिव्यम्विमानमारूडा: प्राणिनस्त्रिदिवङ्गताः।
 क्वचिद्देशेषुपापिष्ठा रुरुदुश्चक्रशुभृशम् ।
 दम्ष्ट्राकराळवदनै रूर्ध्वकेशैर्भयंकरैः।
 मुसण्डीमुद्रगदाभिण्डिवालासि दारिभिः।
 वदद्भिच्छिन्दिभिन्धीति भर्त्सिता अर्धितामुहुः।
 क्वचिच्छर्करिले मार्गे क्वचित्सन्तप्तवालुके।
 तप्तायश्चङ्कभि:कीर्णे क्वचित्कण्ठकिते पथि।
 आकृष्यमाणावेगेन पात्यमाना अधोमुखम्।
 45 क्षुत्पिपासापीड्यमाना दह्यमानादवाग्निना।
 आयुधैर्हन्यमानाश्च च्छिन्नाभिन्नाश्चलेशश:।
 निर्जने निर्जलेघोरे मित्रबन्धु विवर्जिते।
 दम्ष्ट्राकराळवदनैश्वभिर्दष्टाश्च मर्मसु।
 

हाहेति चकृशु ?रम्यमदूतैःप्रपीडिताः।
 पापिष्टस्सततम्भूमौ परहिम्सापरायणाः।
 परनिन्दापरद्रव्य परद्रोहपरायणाः।
 निषिद्धकर्मनिरता: सदाचारविवर्जिताः।
 धिक्कुर्वन्तस्सदासाधून दाम्भिका: पतिता:खला:।
 स्नानौपासनहीनाश्च वेदचास्त्रविवर्जिताः।
 50 देवतार्चा पितृ श्राद्धवर्जिता: पितृनिन्दका:।
 परार्धानाम्निरोद्धार:वरदोषैकदृष्टयः।
 अक्रेयाणाम्च वस्तूनाम्विक्रेतार: कुबुद्धयः।
 शिश्नोदरपरानित्यम्शिवविष्णु विनिन्दकाः।
 बाधकाश्नोत्रियान् विप्रान् दरिद्रम्च विचेषतः।
 भृतकाध्यापका नित्यंभृत्यदासादि हिम्सकाः।
 दुष्पतिग्रहकर्तारो दुष्टसम्सर्गदूषिताः।
 ।
 अव्रताभारणोलब्धा मोघीकृत शुभाह्निका:।
 सम्पूर्णयवसा धेनुरकृत्वा दोहनेरताः।
 अपाका:परपाकाश्चपरान्नास्वाद लोलुपाः।
 55 तुलास्नानादि हीनाश्च त्यक्तधर्माश्चतौलिके।
 परक्षेत्रापहर्तारोऽनधिकार्युपजीविनः।
 पङ्क्तिभेदं च कुर्वाणा स्वयम्स्वाध्वन्नभोजन:।
 धिक्कुर्वन्त:पितॄन्मातृरसत्या: कूटसाक्षिणः।
 सत्पात्रमवमन्वाना असत्पात्र प्रपूजकाः।
 सन्ध्याजपविहीनाश्च देवब्राह्मण हिम्सकाः।
 

वृक्षच्छेदरतास्वार्थे यज्ञहोम विवर्जिताः।
 पुण्यगाधानिरुद्धार: पुण्यभूमिविनिन्दकाः।
 एतेचान्येबाद्यमाना: भृशमुच्चैः प्रचकृशु :।
 दीयताम्दीयतामन्नम्दीयाताम्दीयताञ्जलम्।
 60 गम्यताङ्गम्यतामन्दमित्याक्रोशो बभूवह।
 माहिम्सत क्षणम्यूयम्क्षुत्पापा सार्धितान् भृशम्।
 आकाशेभ्रामिततादेगात् पात्यमानाच्शिलातले।
 छिन्नाभिन्नश्चलवच: क्रकचैरसि मुद्गरैः।
 सक्षारलवणैस्तोयै: सिश्यमाना प्रणेम च।
 पुन:पापफलावाप्त्यैः प्रोक्षताच्शीतवारिभिः।
 सूचीभिः शङ्कुभिस्तीक्ष्णैस्तुद्यमाना प्रणेम च।
 संस्कृत्य संस्कृत्यमुहुः पूर्वपापानि मानवाः।
 आकृष्यमाणादुर्मार्गे कर्णेवक्षस्सु कुक्षिषु।
 ब्राह्मणाभूमिपावैश्या शूद्रा नार्याश्च शुकृशुः।
 65 परदोषेक्षकामन्त्रयोक्तारो द्रव्यलिप्सया।
 परान्नरसिका: पुण्यकथाश्रवण वर्जिताः।
 तुलास्नानादिहीनाश्च बाद्यमाना:प्रचकृशुः।
 नार्याश्चपतिदूषिण्य स्वतन्त्रा: पापमानसाः।
 कृष्यमाणाअयस्सूत्रै स्तप्त्तैस्यूताभगेषु च।
 वाग्दूषिण्यश्च जिह्वासु श्वश्च श्वशुरान् गुरून्।
 पापकर्मरतानित्यम्सदाचार विवर्जिताः।
 निर्दयाश्च शटा:क्रूरा: पुम्च्चल्य: पतिनिन्दिकाः।
 

तस्माद्भुञ्जन्तु पापस्य फलम्किमिहरोदिथ।
 इति निर्भर्त्सतोनार्यो रुरुदु: स्कृकचार्चिताः।
 70 हाहाकारोमहानासीदेवम्सम्यमिनी पथि।
 तत्तत्पापविशुद्ध्यर्थम्तत्तत्कर्मानुसारतः।
 बबाधिरे भृशम्पापान् यमदूताभयंकरा:।
 अहम्प्रभुरहम्दक्षो धनवानहमित्यपि।
 कस्समाप्तिकुलीनोह मितिगर्वान्धलोचनैः।
 भवद्भिर्भर्सितास्सन्त स्तत्पाप फलमाप्स्यथ।
 षडशीतिसहस्राणि योजनानामतीत्यते।
 यातनाभिर्निश्वसन्त: प्राप्यसम्यमिनीम्पुरीम्।
 पापा:पापानुसारेण निरयेषु निपातिता:।
 एते नृपा दुरात्मानो राज्यलिप्सा पराश्चठाः।
 75 द्विजद्रव्यापहर्तारो धनरूडमदोद्धता:।
 एतान् बिन्धि नृपान्दुष्टान्न्पापान् छिन्दिप्रपेषय।
 एतद्विजा विकर्मस्था सत्कर्माचारवर्जिताः।
 इमानार्याश्चठा:क्रूरा भāणामप्रियेरताः।
 अब्राह्मण्या इमे भूपा दयादान विवर्जिताः।
 आच्छिद्यकृकचैघोरैरेतांश्चूर्णय ताडय।
 एवम्सम्यमिनीमार्गे शब्द:कोलाहलो भवत्।
 एवम्मार्गे द्विजः पश्यन् पापिष्ठ:पापपीडनम्।
 वेदराशिस्तपोराशि: सद्य:कारुणिकोभवत्।
 उद्धरेयम्कथमहम्सर्वानेतांश्च पापिनः।
 80

केनोपायेनसर्वेषामुत्तारो नरकाद्भवेत्।
 इति चिन्तापरोभूत्वा पुनरित्थमचिन्तयत्।
 कावेरीस्नानजंपुण्यं यदभून्मासि तौलिके।
 त्रिरात्रम्चददाम्येषाम्नानम्च सफलम्मम।
 साधु:परोपकार्येव ससाधु:परसौख्यदः।
 ससाधुस्सदयोलोके ससाधु:परदुःखहा।
 इष्टापूर्तादिकाधर्मा व्रतमन्यत्स्वनुष्ठितम्।
 तत्परानुपकारस्य दयाहीनस्य निष्फलम्।
 तस्मात्परोपकारम्वै कुर्यात्सर्वप्रयत्नतः।
 85 परोपकारेणभवेद्धिनाक: परोपकारेण तपश्च यज्ञः।
 परोपकारेणतपश्चसन्ध्या परोपकारेण हरिश्च तुष्येत्।
 परोपकारेण च तीर्थसिद्धिः
परोपकारेण च मन्त्रसिद्धिः।
 परोपकारेण च सर्वसिद्धि:
परोपकारेण शरीरदाय॑म्।
 तस्मात्सर्वप्रयत्नेन सीदमाना इमा: प्रजाः।
 उद्धर्तव्या मया सत्यम्यातनाया: क्षितीश्वर।
 इति निश्चित्य मनसा पापिनामुद्दिधीर्षया।
 धार्मिक:करुणाविष्टो वेदराशीपोत्तम।
 हृदजे माधवम्ध्यायन् तानवेक्ष्य च पापिन:।
 विषुमासेतु कावेर्याम्नानपुण्यमहस्त्रये।
 तद्दत्तमद्यपापिभ्य: सर्वलोकाश्च साक्षिणः।
 

इत्युच्चैर्घोषयामास वाचा दत्वा पुन:पुन:।
 90 एतद्वाक्यम्मुनीन्द्रेण सुधामयमुदीरितम्।
 शृत्वाऽर्धितायमभटैः पापिनस्सुखिनो भवन्।
 यमपाशैर्विनिर्मुक्ता याम्यभाधा विवर्जिताः।
 हृष्टा:पुष्टाश्चतुष्टाश्च क्षुत्पिपासा विवर्जिताः।
 छित्वापाशान् समुच्चेलुस्तत्क्षणात्सज्जना यथा।
 प्रीताप्ते देवयानेषु समारूढाधिपम्यमः।
 मरुद्बुधाप्रभावं तु शृत्वा विस्मितमानसाः।
 तस्माद् राजन्! तुलामासे कावेरीस्नानवैभवम्।
 वक्तुम्सहस्रवदना च्छक्नोत्यन्यो न कश्चन।
 त्रिदिनस्नानत:पापा बहवस्त्रिदिवम्गताः।
 95 अथ तत्प्रेक्ष्यविप्रेन्द्रम् यमदूतास्तमबृवन्।
 तत्तत्कर्माणुसारेण तत्तत्पापफलम्जनाः।
 प्राप्नुवन्ति महादु:खम्पापंकृत्वा क्षितौद्विज।
 किमेतद्भवताऽकार्यम्य माज्ञोल्लङ्घनं कृतम्।
 मध्यस्थश्च यमच्चास्ता पापा शान्त्यर्थमुन्नयन्।
 निपातयतिचास्माभि: पापिष्ठान्नरकेषु हि।
 रौरवादीम्च्च नरकाननुभूय: क्रमेण ते।
 शुद्धाश्च कर्मशेषेण जायन्ते कष्टयोनिषु।
 पुन:कर्मानुसारेण भूमौ कुर्वन्त्यघञ्जनाः।
 एवम्परम्परादु:खंप्राप्नुवन्त्यघिनोजनाः।
 100 अज्ञानाद्दर्शितामार्गे ह्यस्माभिस्नेहकारणात्।
 

शीघ्रम्गन्तव्यमित्युक्त्वा ह्यानयम्स्तम्यमालयम्।
 स तु सम्यमिनीङ्गच्छन्नन्तरा नरकाग्निभिः।
 पापिनाम्दह्यमानानां शुश्रावकरुणाम्गिरम्।
 क्वचित्विक्रोशताम्घोरम्क्वचिश्च रुदतांभृशम्।
 क्वचिद्दाहेतिघोषोपि शतशोथ सहस्रशः।
 नानाविधेषुघोरेषु नरकेषु निपातितान्।
 शूलेषुद्रधितान्क्वापि पोथितांश्च शिलातले।
 तप्ताम्स्ताम्रकटाहेषु कृकशैश्च विधारितान्।
 जलञ्चान्नम्प्रार्थयतो मुह्यमानान्मुहुर्मुहुः।
 105 क्षिप्तान् तैलकटाहेषु तप्तसीसेषु पार्थिव।
 सक्षारलवणाम्बोभि: तप्ततैलैर्वणोक्षितान्।
 सप्राणितांश्च शीतोदैः पुनरेव प्रबाधितान्।
 निरयेषु च सर्वेषु पीड्यमानान्निपात्य च।
 भ्रामितान् घटयन्त्रेषु श्वमाम्सङ्खादत: क्वचित्।
 एतानन्याम्च्चबहुशो नारकान् रुदतोभृशम् वेदराशि: कृपाविष्टी विषीदन्निदमब्रवीत्।
 हे नारकजना: पापंकृत्वा किम्रोधिताऽधुना।
 किमर्थम्वा कृतम्पापम् क्षणिकम्देहमाचितैः।
 110 दुर्लभम्च मनुष्यत्वम्प्राप्यनाकारि शोभनम्।
 न स्नातम्तु महानद्याम्प्रातर्माघादिमान्सु च।
 वस्त्रम्वा पानकम्क्षीरम्जलं दत्तं न वा द्विजाः।
 न सन्ध्यो न जपोकारी न होमातिथिपूजनम्।
 

न कावेर्याम्सकृत्स्नातम्माघेतौले च माधवे।
 एकादश्युपवासादि व्रतम्नान्यदनुष्ठितम्।
 न पूजितोमहाविष्णु: शंकरो लोकशंकरः।
 एकं च तुळसीपत्रं नदत्तो केशवोपरि।
 न श्राद्धादिक्रियाकारि पितरस्तप्तिान हि।
 न कृता पितृशुश्रूषा न कृतम्गुरुपूजनम्।
 115 नाकारि भगवत्सेवा न कृतम्द्विजतर्पणम्।
 परोपकारोन्साकारि दरिद्राय द्विजातये।
 अपि वा वैष्णवीगाधा सर्वाघौघ विनाशिनी।
 सर्वाभीष्टप्रदानृणाम् नशृता जन्मकृन्तनी।
 इति पृष्टाद्विजेन्द्रेण नारकावेदराशिना।
 तद्वाक्पीयूषपानेन तर्पितास्तम्वचोबृवन्।
 * इति चंद्रकांतायाः कावेरीस्नानेन पापविमोचनम् नाम
चतुर्थोऽध्यायः
*******************
अथ पञ्चमोऽध्यायः वेदराशेर्नारकीनां च संवादः
नारकाः
ब्रह्मन् मत्कृतपापानां वक्तुं संख्या न विद्यते।
 तथापि पृच्छतस्तेद्य मत्पापं किंचिदुच्यते।
 वयं जननमारभ्य पापकर्मरतास्सदा।
 अरुणोदयकाले च शयितुं सुखलिप्सया।
 


सन्ध्याजप विहीनाश्च तेन दह्यामहेवयम्।
 न हुतं च हविश्चाग्नौ न च पंचमहाध्वराः।
 न कृतोतिथिसत्कारस्तेन दह्यामहेवयम्।
 अन्नार्थिनश्च मध्याह्ने ह्यागताः क्षुत्पिपासिताः नमानिताश्श्च लोभेन तेन दह्यामहेवयम् ॥ लाक्षा च लवणान्नानि तैलमाज्यं पयोदधि।
 5 तिलं वस्त्रं च विक्रीतं तेन दह्यामहेवयम् ॥ परनिंदापरद्रोहः परान्नस्त्रीधनानि च।
 आहृतानि द्विजन्माभि तेन दह्यामहेवयम् ॥ निषिद्धदिवसेष्वेव योषित्संगःकृतो सकृत्।
 एकादश्यां सदाभुक्तं तेन दह्यामहेवयम् ॥ नामायां च कृतं श्राद्धं मृताहेऽपरकक्षके।
 दुर्वार्तावादिनो नित्यं त्यक्तविष्णुकथामृताः।
 असत्संगरता नित्यं तेन दह्यामहेवयम् ॥ नास्माभिःपूजितामाता न पितादेवतागुरुः।
 न ब्राह्मणाश्चभूदेवास्ते तेन दह्यामहेवयम् ॥10 याचकाय ददामीति संपाद्याशां तदर्थिने।
 किंचिन्नदत्तं विप्रेन्द्र तेन दह्यामहेवयम् ॥ न कदाचिन्महानद्यां सुस्नातं प्रातरुत्थितैः।
 न मध्याह्ने न संक्रान्तौ तेन दह्यामहेवयम् ॥ स्वाध्वन्नं च रहोभुक्त मकृत्वा वैश्वदेविकम्।
 न भिक्षाभिक्षवेदत्ता तेन तेन दह्यामहेवयम् ॥

नाकारी तीर्थयात्रा च न विष्णूत्सवदर्शनम्।
 पुण्यकालोश्च नोवन्ध्या तेन दह्यामहेवयम् ॥ कुटुंबिने दरिद्राय श्रोत्रियाय विशेषतः।
 किंचिन्नदत्तं लोभेन तेन दह्यामहेवयम् ॥15 नागायत्री सकृज्जप्ता सर्वाघौघ विनाशिनी।
 नाध्वरश्चकृतोस्माभि तेन दह्यामहेवयम् ॥ निषिद्धेषु च वारेषु निषिद्धव्यञ्जनादिकम्।
 अकारितैलाभ्यङ्गश्च तेन दह्यामहेवयम् ॥ भुक्तं पर्युषितान्नं च श्राद्धशूद्रावशेषितम्।
 अलाभु तिन्त्रिणीयुक्तं तेन दह्यामहेवयम् ॥ मूलकन्द दधिमिश्रम् च क्षीरम् च लवणान्वितम्।
 कोशातकी च पिण्याकम् तेन दह्यामहेवयम् ॥ शिग्रुशाकश्चवार्ताक श्वेतनिष्पावमेवच।
 भक्षितम् च तथस्माभिस्ते तेन दह्यामहेवयम् ॥20 तैलाक्तेर्निशिसुप्तं च निद्रितम् च तथाहनि।
 अभ्यक्तैर्मेहितं चापि तेन दह्यामहेवयम् ॥ अकालेऽभोजविप्रेन्द्र बालवृद्धातुरान्विना।
 अनिवेदितमन्नं च तेन दह्यामहेवयम् ॥ दूषितापितरौ ब्रह्मन्वाक्पारुष्यैर्मदोत्कटैः।
 अन्नार्थेभर्त्सिताबाला तेन दह्यामहेवयम् ॥ अस्नातनारि पक्वान्नं पुनःपक्वान्नमेव च।
 एकमासमसूपान्नं तेन दह्यामहेवयम् ॥
रात्रौ दधि दिवाक्षीरम् सदापि तं यदृच्छया।
25 उदक्ययाकृतस्संगस्तेन दह्यामहेवयम् ॥ परार्थानां निरोद्धारः पिशुनादोषदृष्टयः।
 स्वार्थैक साधनानित्यं तेन दह्यामहेवयम् ॥ अहं विद्वानहंदक्षः कुलीनोधनवानहम्।
 इत्येवम् धिकृतास्संतस्तेन दह्यामहेवयम् ॥ दिवास्वापःकृतोनित्यं दिवामैथुनमेव च।
 द्यूतविद्यारता नित्यं तेन दह्यामहेवयम् ॥ निषिद्धे च दिनेप्रातरभ्यङ्गोकारि सन्ध्ययोः।
 पत्न्यासार्धं तथा भुक्तम् तेन दह्यामहेवयम् ॥ शिगू च तिन्त्रिणीयुक्तो वार्ताकं भक्षितं सितम्।
 उपोतकी च द्वादश्यां तेन दह्यामहेवयम् ॥30 तूष्णिं क्रियां समुद्दिश्य गृहीत्वा तण्डुलादिकम्।
 कृतं स्वकुक्षिभरणम् तेन दह्यामहेवयम् ॥ अभ्यस्य वेदशास्त्राणि निषिद्धाचारकर्मभिः।
 उपेक्षितास्सतां धर्म तेन दह्यामहेवयम् ॥ भृतकाद्यापनं कृत्वा कुटुम्बभरणं कृतम्।
 असत्यवादिनो नित्यं तेन दह्यामहेवयम् ॥ बहुनाकिमुहोक्तेन वर्णयामोद्ययातनाः।
 
रौरवः कलसूत्रश्च महारौरव एव च।
 कराळविकराळा च तुषाग्निःकृकचस्तथा।
 कुम्भीपाकस्तप्तजल स्सन्तापस्तप्तवालुकः।
 35
तप्तसीसस्तप्ततैल स्तप्तायः पिण्डभक्षणम्।
 विण्मूत्रभक्षणं घोरं रेतःपानं भयंकरम्।
 श्वमांसभोजनंचैव स्वमांसाशनमेव च।
 रक्तपानंवसाभक्ष स्सर्पवृश्चिकपीडनम्।
 घोरंक्षारोदकंचैव घोरावैतरणीनदी।
 तप्तशैल निपातश्चमहामुद्गरमर्धनम्।
 तप्तपाषाणपेषश्च वह्निज्वाला प्रवेशनम्।
 अङ्गारभक्षणंघोर मङ्गारशयनं तथा।
 नखसूचीव्रणं चैव शिश्नशङ्ख निपीडनम्।
 एषाघोरतरापापा ह्यधोमुख निपातिता।
 कृकचेर्दारिता दष्टाक्रिमिकीटादि वृश्चिकैः।
 ब्राह्मणीयं पुराब्रह्मन् पतिता कुलदूषणी।
 उज्जयिन्याः पुरोदेशे सुघोषागार वासिनी।
 दुराचारा सदाक्रुद्धा बहुपुत्रातिगर्विता।
 श्वश्रुश्वशुरयोर्नित्यमप्रिया स्वोदरंभरिः।
 सपत्नीपुत्र हन्त्री च तद्धनाहरणेच्छया।
 नित्यं निद्रालुरसा भर्तृभर्त्सनकारिणी।
 पतिशुश्रूषयाहीना गुरुदूषणकारिणी।
 सज्जनाप्रियकर्ती च कुलघ्नीपितृधिक् कृता।
 भर्त्सयन्ती दरिद्रांश्च भिर्नन्नार्थिनश्शठा।
 स्वेरिणीप्रियकींच साद्वीनां दूषणेरता।
 50 भाच भर्त्तिताकिंचिन्मरिष्यामीति वादिनी।
 
रहस्यं द्रव्यधात्रीच स्वतन्त्राणां च योषिताम्।
 देवद्विजगुरुद्रोहकारिणी स्वैरिणीप्रिया।
 अत्युत्कटैःपापफलै श्वासकासादिभिर्गदैः।
 मृतासम्यमिनीमार्गे मर्दितामुसलादिभिः।
 पूर्वोक्तान्नारकानेता ननुभूयार्बुदत्रयम्।
 कोलाहले महाघोरे पातितादंश वृश्चिकैः।
 अधोमुखीलम्बमाना पीड्यते विड्मुजाशना।
 अन्तिकेस्याश्च दृश्येते उभौ च पुरुषाधमौ।
 हाहाकारं प्रकुर्विते क्रोशन्तोमूर्छयान्वितौ।
 55 अस्माद्दुःखाद्दुःकतर मुभाभ्यामनुभूयते।
 तस्मान् मुहूर्तं तिष्ठाद्य त्वत्संदर्शन निर्वृताः।
 वयं जीवामहे ब्रह्मन् साक्षात्त्वं सुखदो हरिः।
 इति तदर्शनवशात् अन्योन्यमभिभाषणात्।
 निष्पापाजातपुण्याश्च हरिरित्यक्षरद्वयम्।
 उच्चार्य नरकेभ्यस्ते विधूताखिलयातनाः।
 तान् दृष्ट्वोच्चलितान्विप्रो नरकेभ्यस्सुविस्मितः।
 हरिशब्दप्रभावं च दृष्ट्वा दद्यौपुनस्त्विदम्।
 केन पुण्यप्रभावेन निरयस्था इमे जनाः।
 उत्तेरुर्नरकेभ्यश्च विचित्रमिहदृश्यते।
 प्रारब्धकर्माननुभूयलोके नशक्तुमुद्धर्तुमघौघिनस्तान्।
 न यज्ञदानैर्न तपोव्रताद्यैर्भोगेन वा नश्यति कर्मबंधः।
 दभैर्विस्तार्यतां भूमिस्तपोभिः खिद्यतां जगत्।
 
दानाम्भोभवताद्वार्धिर् नभुक्तम् न्क्षीयतेह्यघम्।
 यस्त्वकामनयाकुर्याद्वैदिकं पापमेव वा।
 भोक्तव्यं तत्फलंनूनं तत्संकल्पोझीबंधदः।
 तस्मात्तमंतरा विष्णुं स्वतन्त्रम् त्रिजगत्पतिम्।
 कोवा नाशयतेबन्धुं बुद्धं प्रामादिकं च वा।
 तादृशं कीर्तनविष्णोरेभिर् कारिमुक्तिदम्।
 अमीषां केनपुण्येन निरयोत्तरणं त्वभूत्।
 इति चिंताकुलोविप्रः पुनानमुपागमत्।
 अहोज्ञातमोज्ञातमहोज्ञातमितिस्मयन् ।
 ननभेस्मत्तवस्मर्तो वष्णुभक्ति जडीकृतः ।
 अनुवादप्रसंगेन हरेरेशामभूत्सखम् ।
 हरिस्मरणमात्रेण नरकस्था इमे जनाः ।
 आसन्निस्ष्कल्मषास्सर्वे महापातकिनोपि च ।
 इमां रुंदंतियाम्याश्चेत् किं कुर्वति जना ।
 अमी।
 मृत्युश्च लोकसंहारि रुश्टशेद्धंति नारकान् ।
 तस्मादेशां क्षणार्धं किंच वैकुण्ठसिद्धये ।
 अनेषां नरकस्थानां हरिशब्दमशृण्वताम् ।
 उच्चकैश्श्रावयेष्याहं हरिनामामृतानि तु ।
 स्तेनस्सुरापो मित्रद्रुग्ब्रह्महागुरुतल्पगः ।
 स्त्रिइराज पित्रुगोहन्ता ये च पातकिनोपरे ।
 दग्धपापस्स्युरुच्चार्य हरिरित्यक्षरद्वयम्।
 ये नस्नांति तुलामासे कार्तिके निर्मलेजले।
 
दग्धपापस्स्युरुच्चार्य हरिरित्यक्षरद्वयम्।
 स्नानं नकुर्युर्यमासे कार्तिके निर्मलेजले।
 दग्धपापस्स्युरुच्चार्य हरिरित्यक्षरद्वयम्।
 स्नानं नकुरुतेमासे माधवे माधवप्रिये दग्धपापस्स्युरुच्चार्य हरिरित्यक्षरद्वयम्।
 पंचयज्ञविहीनाश्च प्रातस्नान विवर्जिताः।
 अपुत्रिणोप्यनाचारो नषिद्धव्यंजनोंदनाः।
 परार्थघातकाःपापा परनिंदासु तत्पराः।
 दग्धपापस्स्युरुच्चार्य हरिरित्यक्षरद्वयम्।
 असद्वारिता विष्णुकथाश्रवण वर्जिताः ।
 सालग्रामार्चनाहीना स्सदाशूद्रान्नभॊजनाः ।
 दग्धपापस्स्युरुच्चार्य हरिरित्यक्षरद्वयम्।
 अमहाळयकर्तारोष्यमाश्राद्ध विवर्जिताः ।
 गुरुद्रोहः कृतघ्नाश्च मातृपितृविनिंदकाः।
 अदत्वा भोजनोसत्यास्सैष्यान् अन्नानभोजनाः।
 येकचित् पापकर्तारः पंक्तिभेदरताश्च ये।
 ते सर्वे मुक्तिमायान्तिर् हरिस्मरणमात्रतः ।
 इमे च कृतनिर्वेशा जन्मकोट्यंहसामपि।
 येस्तुवन्त्यनुवादेन नामस्वस्त्ययनं हरेः।
 श्रीमान् भक्तपराधीनो भगवान् हरिरीश्वरः।
 दयाळुस्स्मृतिमात्रेण तेषां दुःखं व्यपोह्यति।
 संसारार्णवमन्नानां महापापकृतामपि।
 
कोरक्षिता विना विष्णुं चतुर्दशजगत्स्वपि।
 इति निश्चित्यमनसा हर्षगद्गदयागिरा।
 नृत्यन् हसन्नुत्पतंश्च धन्वन्वासो हरि जगौ।
 स्वामिन्नन्तपुरुषोत्तम पुष्कराक्ष
भक्तार्तिभंजन हरे भगवन् मुरारे।
 श्रीमन्नृसिंहनरलोकविडम्बनात्त
लीलावतार तरणींदु विलोचनेश।
 लक्ष्मीमहीकरसरोरुह लालिताङ्गि
पद्माप्रमेय मधुभंजनमञ्जुवेष।
 सन्मङ्गळप्रद विभो भजतां तवाyि
पंकेरुहं स्मरणमात्र भवार्तिनाश।
 सत्येशवंद्य नवनीरदनीलवर्ण
स्वर्णाम्बरोज्वल सुवर्णरथाऽमरेश।
 नारायणाच्युतविभो नवनीतचोर
श्रीकेशवश्रुतिशिरोभि रदृश्यमूर्ते।
 श्रीकृष्णविष्णोरघुनाथजिष्णो
जाण्ववियोष्टांशु कलावतम्स।
 विश्वेशविश्वंबरविश्वमूर्ते
विश्वातिशायिप्रभ शाश्वताङ्ग।
 नीरूपनित्याव्यय निर्विकार
नागेशभोगेशयलोकबंधो।
 त्रैगुण्यमूर्तिपरिवर्तितलोकलील
प्राक्पुण्यनम्र सुरसंघ मुनीन्द्रवन्ध।
 इन्द्रानुजेंद्रदनुजेन्द्र कुमारसांद्र
स्तोत्रस्तुतस्तुतिनिधे मयिदेहि दक्षाम्।
 इति हरिगुणवर्णना कथायाश्श्रवण
निरस्तसमस्त कर्मबंधाः।
 द्विजवरमभिनन्द्यनारकास्ते
सपदिहरि त्वगमन्विमानसंस्थाः।
 एतादृशोयं महिमावेदराशेविजस्य वै।
 तुलास्नानेनसिद्धां हि नारकोत्तारणादिकम्।
 इति वेदराशेर्नारकीनां च संवादः नाम
पञ्चमोऽध्यायः
**********************
अथ षष्टोऽध्यायः श्रीविष्णुकवच माहात्म्येन वेदराशेर्मुक्तिः
श्रीविष्णुकवचं अगस्त्य:महानदीमहापुण्या ब्रह्मणादर्शिता पुरा।
 जनानां मृत्युबद्धानाम्महापातकिनामपि।
 अवैश्वदेवकार्याणां दुरुचारान्नभोजनाम्।
 अवेदानां कृतघ्नानां दुराचारान्तरात्मनाम्।
 उत्तारणार्थं विधिना कायक्लेशं विना क्षितौ।
 कावेरीपुण्यसलिला सर्वाभीष्टप्रदायिनी।
 
सर्वतीर्थेषु यत्पुण्यम्सर्वतीर्थेषु यत्फलम्।
 तत्फलम्सकृदोस्नाति तुलाकावेरिमज्जनात्।
 अन्नदानम्तु दुर्भिक्षे य:करोतिदिनेदिने।
 5 तत्फलम्सकृदोस्नाति तुलाकावेरिमज्जनात्।
 श्रोत्रियाय दरिद्राय यो ददाति धनादिकम्।
 तत्फलम्सकृदोस्नाति तुलाकावेरिमज्जनात्।
 प्रोषितायदरिद्राय स्वाश्रयं प्रददाति यः।
 अतिथिम्पूजयेद्यस्तु वैश्वदेवान्त आगतम्।
 तत्फलम्सकृदो स्नाति तुलाकावेरिमज्जनात्।
 नित्यकर्मण्यशक्तो यो निरग्निर्वेदवर्जितः।
 तुलास्नानेन शुद्धस्यात् अन्यथा नरकम्व्रजेत्।
 प्रातस्स्नात्वा तु यो विप्राः सुजपेद्वेदमातरम्।
 कन्यागते रवौयस्तु पक्षश्राद्धम्समाचरेत्।
 10 द्वादश्यां वा व्यतीपाते रिक्तायाम्च ददाति यः।
 तत्फलम्सकृदोस्नाति तुलाकावेरिमज्जनात्।
 एषा दक्षिणगङ्गेति कावेरीविधिनोदिता।
 महापातकयुक्ताना मुपपातकिनामपि।
 सर्वेषामुपकाराय दक्षिणां दिशमागता।
 शतसम्वत्सरस्नानात् गङ्गायाम्यत्फलम्भवेत्।
 सह्यजायाम्सकृत्स्नात्वा तुलायाम्तत्समाप्नुयात्।
 तिस्र:कोट्यार्धकोटी च तीर्थानिभुवनत्रये।
 तानिसर्वाणि राजेन्द्र पुरस्कृत्य च जाह्नवीम्।
 
केशवस्याज्ञया यान्ति तुलामासे मरुत्रुधाम्।
 15 तुलाकावेरिमाहात्म्य श्रवणात्स्नानदानतः।
 सर्वपापविशुद्धास्यु वैकुण्ठपदभाजनः।
 तुलास्नानाद्वेदाशिाह्मणो वेदवित्तमः।
 स्वैरिणीस्सङ्गजात्पापान मुक्तोभूत्वा पवित्रतः।
 दृष्ट्वा तु नारकान्मार्गे सम्यमिन्याम्प्रविश्य च।
 दत्वा तेभ्योपि वैकुण्ठम्जित्वा यममगाद्धरिम्।
 हरिश्चन्द्र:जित्वा यमम्कथम्विप्रो देवैरपिदुरासदम्।
 वैकुण्ठमगमत्पश्चात् तन्मे ब्रूहि सुनिश्चितम्।
 अगस्त्य:एतमाकर्ण्यवृत्तात्तान्तं शस्तादण्डधरो यमः।
 उक्रोचभुजमुत्युग्रो दहन्निवजगत्त्रयम्।
 20 ग्रसन्निव जगत्सर्वम्भक्षयन्निवपर्वतान्।
 पिबन्निव च पाथोदिम्हस्तम्हस्तेनपीडयन्।
 उच्छासन्निहवद्धोरम् दन्तान्कटकटापयन्।
 धून्वन्शीर्षन्महाभीमै राजयक्ष्मादिभिर्वृतः।
 घोरम्महिषमारूडो निगात्कालदण्डभृत्।
 दिशोदश च राजेन्द्र प्रतिघोषणपूरिताः।
 सर्वेचक्षुभिरालोका विपरीतानवग्रहा:।
 समुद्राभिन्नमर्यादा निष्कम्पमभवज्जगत्।
 किमिदम्कुत आयाति युगान्तो वा समागतः।
 

यमौ भातिलुलायस्थो बडभाग्निरिवज्वलन्।
 25 इति सर्वे द्रवन्तिस्म दृष्ट्वा दण्डधरम्प्रभुम्।
 सस्तेन्यम्महिषस्थं तं कालदण्डधरम्प्रभुम्।
 दूरेऽनुमायचास्तारम्किञ्चिद्भीतस्सभूसुरः।
 नमोनारायणायेति तदाऽघोषयदुच्छकैः।
 निजरक्षा चकाराथ स विष्णुकवचेन वै।
 हरिश्चन्द्र:ब्रह्मन् श्रीविष्णुकवचम्कीदृशं किं प्रम्प्रभावकम्।
 केनोक्तम्क ऋषिच्छन्दो दैवतम्कीदृशम्मुने।
 अगस्त्य:हरिश्चन्द्र प्रवक्ष्यामि शृणुष्वावहितोऽधुना।
 श्रीविष्णुकवचंदिव्यं रहस्यम्सर्वगोपितम्।
 सृष्ट्यादौ कमलस्थाय ब्रह्मणे हरिणोदितम्।
 कारुण्येन मम प्रोक्तम्ब्रह्मणा क्षीरसागरे।
 30 गोपनीयम्प्रयत्नेन भवता च जयप्रदम्।
 इदम्श्रीविष्णुकवच मनुष्टुप्छन्दसायुतम्।
 ऋषिश्च हरिश्चन्द्र श्रीनारायण दैवतम्।
 केशवादि द्वादशाङ्ग करन्यास समन्वितम्।
 शान्ताकारमिति ध्यात्वा जप्त्वाभीष्टमवाप्नुयात्।
 पूर्वतो माम्हरि:पातु पश्चाश्चक्री च दक्षिणे।
 कृष्ण उत्तरत:पातु श्रीविष्णुश्चसर्वतः।
 ऊर्ध्वम्मे नन्दकीपातु अधस्ताच्छाभृत्सदा।
 
पादौ पातु सरोजानि जो पातु जनार्दनः।
 जानुनीये जगन्नाथ: ऊरूपातु त्रिविक्रमः।
 35 गुह्यम्पातु हृषीकेश: पृष्टम्पातुममाव्ययः।
 पातुनाभिम्ममानन्त: कुक्षि राक्षसमर्धनः।
 दामोदरो मे हृदयम्वक्ष:पातु नृ केसरी।
 करौ मे काळियरातिर्भुजौ भक्तारिभञ्जनः।
 कण्ठङ्कालाम्बुध श्याम: स्कन्धौ मे कम्समर्धनः।
 नारायणोव्यान्नासाम्मे कर्णौक्षिप्र प्रभञ्जन:।
 कपोलौ पातु वैकुण्ठो जिह्वम्पातु दयानिधिः।
 आस्यम्दशास्यहन्ताव्या न्नेत्रे मे पद्मलोचनः।
 भृवौ मे पातुभूमीशो ललाटम्मे सदाच्युतः।
 मुखम्मे पातुगोविन्दः शिरोगरुडवाहनः।
 40 मां शेषशायी सर्वेभ्योव्याधिभ्यो भक्तवत्सलः।
 पिशाचाग्निजलेभ्यो मामापझ्यो वटु वामनः।
 सर्वेभ्यो दुरितेभ्यश्च पातुमाम्पुरुषोत्तमः।
 इदं श्रीविष्णुकवचं सर्वमङ्गळदायकम्।
 सर्वरोगप्रशमनम्सर्वशत्रु विनाशनम्।
 एवम्जजाप तत्काले ह्यात्मरक्षाकरम्परम्।
 त्रिसन्ध्यम्य:पठेच्छुद्धस्सर्वत्र विजयीभवेत्।
 वेदराशिम्ततो दृष्ट्वा विप्रम्वैष्णववल्लभम्।
 ते याम्या:प्राद्रवन् सर्वे सिम्हत्रस्था मृगा इव।
 सचित्रगुप्तम्तम्सौरिम् विना व्याध्यादयोनुगाः।
 45
पश्चाद्भागमपश्यन्तो दह्यमाना द्विजौजसा।
 तम्द्रुष्ट्वा ब्रह्मणश्रेष्ठम् विष्णुभक्तम् जितेन्द्रियम्।
 यमस्सौम्य:प्रसन्नत्मा स्यन्दनस्थो ह्यदृश्यत।
 नरकान् रहितान् पापैः दावतो वीक्ष्यतानुगान्।
 वैष्णवीम्वीक्ष्य विप्रेन्द्रम्विस्मित: प्रशशंस सः।
 सोवतीर्यरथा स्तस्माद्बहुमानपुरस्सरम्।
 उवाच ब्राह्मणं नत्वा स्वागतम्ते द्विजोत्तम।
 इदमर्ध्यमिदम्पाद्यमिद माचमनीयकम्।
 इत्यातिथ्यम्यम:कृत्वा सददर्श कुतूहली।
 धन्योस्म्यनुग्रहीतोस्मि जन्म मे पावितम्मुने।
 50 भवान् पादारविन्दाभ्याम् यन्मे पुरमिहागतः।
 तव पुण्यप्रभावेन सदाचारोद्भवेन वै।
 जितोहम्हरिभक्त्या च मैत्रेणेक्षस्व चक्षुषा।
 तत्तत्पापफलम्भोक्तुम्नरकेषु निपातिता।
 दह्यमानादिवारात्रम्पापिष्ठा:पालितास्त्वया।
 ब्रह्मन्नविस्मय:कार्यो भगवत्यम्बुजेक्षणे।
 स्तेनस्सुरापोमित्रदृग् ब्रह्महा गुरुतल्पगः।
 नित्यनैमित्तिक त्यागी येच पातकिनोपरे।
 सर्वेषामप्यघवतामियमेव तु निष्कृति:।
 सङ्कीर्तनम्हरे ह्मन् लोपान्मोहाश्च दम्बतः।
 55 त्वयाभागवतश्रेष्ठ वरदु:खा सहिष्णुना।
 भक्त्यासङ्कीर्त्यकृष्णाख्या ममीषां रक्षसां रक्षणं कृतम्।
 
इदम्रहस्यं कृष्णस्य कीर्तनम्दिव्यमौषधम्।
 ममाधिकारोव्यर्थस्या न्न प्रकाश्यमिदम्त्वया।
 इत्युक्त्वाब्राह्मणम्सौरि निश्वसन् चिन्तयाकुल:।
 स्वाधिकारच्युतिम्वीक्ष्य पुनस्तत्रेदमब्रवीत्।
 ज्ञानतोऽज्ञानतोवापि पापंकृत्वा नराधमा:।
 अवचेनहरि स्मृत्वा निष्पापा यान्तिमाधवम्।
 तस्मात्कलाव विश्वासं कृत्वासङ्कीर्तने हरेः।
 पतन्तु प्रायश:पापाद्दीपहस्ता यथाबिले।
 60 तस्य प्रभावमज्ञात्वा मम लोकं प्रयान्तु च।
 अद्यप्रभृति चा श्रद्धा भूयात्सङ्कीर्तने हरेः।
 भूयात्क्वचित्क्वचित्भक्ति: केशवे तदनुग्रहात्।
 श्रीकृष्णविष्णोभगवन्मुरारे
सङ्कीर्तयन्तीति त एव सन्तः।
 न माम्च पश्यन्ति ममापि
किंकरान् बिभेम्यहम्चक्रधरस्य कीर्तनात्।
 इदानीम्तु वया दृष्टम्मम रूपम्भयङ्गरम्।
 पापिष्ठानामिदं रूपं प्रसन्नम्स्स्याद्भवादृशाम्।
 इदम्विमानमारुह्य सूर्यकोटिसमप्रभम्।
 गच्छलोकान्यथेष्ठम्वै विष्णुतेजो निवारितः।
 येन स्नान्ति तुलामासे निर्मले सह्यजाजले।
 तन्माहात्म्यम्न शृण्वन्ति सर्वाभीष्टप्रदायकम्।
 65 येनदास्यन्तिसत्पात्रे पुण्यक्षेत्रेविशेषत:।
 
कीर्तनम्येनर्वन्ति कृष्णस्यकमलापतेः।
 ते प्रयान्तु च मल्लोकम्तेषामुग्रो भवाम्यहम्।
 गच्छलोकान्यथेष्ठम्त्वम्विष्णुभक्ति द्विजोत्तम।
 इत्थम्सम्पूजितो विप्रा स्सौरिणा हरिभीरुणा।
 विमानस्थोर्चितोदेवै स्तूयमानोऽगमद्धरिम्।
 यमस्चाश्वस्य सैन्यम्स्वम्कथयन्नस्य वैभवम्।
 हरिकीर्तनमाहात्म्यम्तुला माहात्म्यमेव च।
 कावेरीमहिमानन्द स्वगृहम्प्राप पार्थिव।
 तस्मात्तुलार्के सम्प्राप्ते कावेर्या:पुण्यवारिणी।
 70 स्नानन्दानम्कुरुष्व त्वम्माहात्म्यश्रवणम्तथा।
 वित्तशाठ्यमकुर्वाण स्तुलाम्प्राप्ते दिवाकरे।
 स्नानं कृत्वा द्विजेन्द्रेभ्य: कुरु दानम्महामते।
 एनपुण्यप्रभावेन सर्वाभीष्ठान्यवाप्नु हि।
 एतत्ते सर्वमाख्यातम्यस्मान्त्वम्परिपृष्टवान्।
 इदम्पापग्नमाख्यानम्वेदराशि प्रभावजम्।
 पुत्रायुरारोग्यकरम्सर्व सम्पत्प्रदायकम्।
 य:पठेच्छावयेद्वापि सर्वपापैः प्रमुच्य ते।
 इति श्रीविष्णुकवच माहात्म्येन वेदराशेर्मुक्तिः नाम
षष्टोऽध्यायः।

*******************
लोकगुरुं गुरुभिस्सहपूर्वैःकूरकुलोत्तमदासमुदारम्।
 धीनग पत्यभिरामवरेशं धीप्रशयान गुरुं च भजेऽहम्।
 
अथ सप्तमोऽध्यायः
सुंदोपसुंदवृत्तान्तः हरिश्चंद्रःसमस्तधर्मज्ञमरुद्धायाः
पयस्यहोमासि निमज्य तौलिके।
 उपेयिवान् स्नानफलं महीतले
धनात्मजारोग्य रिपुक्षयादिकः।
 कथं विधेयश्च मरुद्धाप्लवः कदा हि
काले कमलेक्षणो हरिः।
 उपासितव्यः कुसुमैश्चकैर्मुनेः विशेषतो
वर्णय मे नमोस्तु ते।

अगस्त्यः
सुसादुपृष्टो भवताहमद्य भो कवेरकन्या महिमा तवोच्यते।
 कवेरकन्या महिमाश्रवे मतिः पुरातनैः
पुण्यफलैर् भवेन्नृणाम्।
 पुराऽर्जुनो महाभाग द्रौपदी भ्रातृभिस्सह।
 उद्वाह्य तीर्थयात्रार्थं प्रतिशृत्याग्रज प्रति।
 केनचित्कारणेनैव कावेरी तीरमाप्तवान्।
 वैशाखे तत्र च स्नात्वा मौनयुक्तो मुनीश्वरैः।
 5 सह्यजायास्तु माहात्म्यं शृत्वा माहात्म्यपाठकम्।
 प्रतिपूज्य द्विजं शांतम् भूषा स्रग्वस्त्रगोधनैः।
 
तत्र संनिहितं विष्णुं रङ्गेशं च प्रणम्य च।
 पश्चाद्वारवतीमेत्य सुभद्रोत्सवनोत्सुकः।
 प्रार्थयंती स्वयं भक्त्या उद्वाह्य भगिनी हरेः।
 विजित्य समरेशत्रूण दुर्योधनमुखान् बहून्।
 कावेरी स्नानमाहात्म्याद् भ्रातृभीराज्यमाप्तवान्।
 हरिश्चंद्रःकिमर्थं फाल्गुणोब्रह्मन् तीर्थयात्रा परोभवत्।
 उपेयेमे कथं शौरेः भगिनीमजयद्विषः।
 
अगस्त्यः
पुरा पांचाल तनयं पांचाप्युद्वाह्य पाण्डवाः।
 वंचिताः पंचबाणेन तयासाकं च रेमिरे।
 10 स्त्रियो हि मूलं नरकस्यपुंसाम् स्त्रियो
हि मूलं निधनस्य पुंसाम्।
 स्त्रियो हि मूलं व्यसनस्यपुंसाम् स्त्रियो
हि मूलं कलहस्यपुंसाम्।
 अंतःक्रुद्धाप्रसन्नास्या अगाध हृदयस्त्रियः।
 न शौचं नापिचाचारो नच सज्जन संगतिः।
 नबिभ्यतिस्त्रियोमृत्यो रुदरंभरयः परम्।
 अन्नदेचच्छलात्पत्यौ क्षणस्नेहं प्रकुर्वते।
 रुग्णेभर्तरि रिक्ते च शपंति स्वार्थसाधनाः।
 नापवादाद्भयं स्त्रीणां न मृत्योर्नरकादपि।
 15 नेक्षते वा पितृकुलं न भर्तृकुलमेव च।
 
न लज्जा न च वाक्स्तोत्रं नित्यमोदन तत्पराः।
 स्त्रीणामष्टगुणकामस्त्वहोरात्रं रिरत्सवः।
 नेच्छंति मैथुनं पत्यौ स्वप्रागल्भ्यादि शंकया।
 वृषस्यंत्योपि भर्तृण स्वान् स्वयंनायांति दुर्धियः।
 अयं स्वभावो साध्वीनां विशेषेण कलौयुगे।
 स्त्रीणां प्रसविनीनां तु क्रौर्यस्याट्विगुणंधवे।
 युवतीनां वितंतूनामपुत्राणां तु किं पुनः।
 वंध्या युवत्यो विधवा यस्मिन् ग्रामे वसंति हि।
 कलिश्च व्याधयोमृत्युस्तत्र सन्निहितस्सदा।
 20 उत्तमाश्च स्त्रियालोके शुचयश्च शुचिस्मिताः।
 अपापाःपुण्यसंकल्पा श्शिश्रूषंत्यः पतिंगुरून्।
 भूषयंत्यःकुलं पित्रोः पुत्रिण्यश्रुवते दिवम्।
 रोगिणं निर्धनं रुष्टं नान्यदं वा जुगुप्सितम्।
 उन्मत्तं वा जडंमूर्ख साध्व्यास्तं मन्यते हरिम्।
 इदं च श्रूयतां भद्रे हे वत्स वचनं मम।
 यदर्थमागतोद्याह मेवं वच्मि त्वदग्रतः।
 सुंद इत्युपसुंदश्च भ्रातरौ दैत्य वल्लभौ।
 संचरंतौ स्नेहयुक्तौ ददृशा ते स्त्रियं पथि।
 तिलोत्तमेति विख्याता ह्यप्सराश्चारुहासिनी।
 25 उत्पलाक्षी सुबिंबोष्ठी सर्वाभरण भूषिता।
 तां दृष्ट्वा त्वग्रज स्सुंदस्स्मरभाण प्रपीडितः।
 गतस्नेहो गतवीळः क्षीभोभ्रातरमब्रवीत्।
 
प्रजापती तवायाति शुश्रूषा क्रियतां त्वया।
 तच्छृत्वाऽग्रजं भ्रातौ तां दृष्ट्वा मोहिता हसन्।
 स्नुषेयम्स्यात्तवभ्रातः शुश्रूषार्थमुपागता।
 पातित्यं ते भवेदार्य गृह्णीयाच्छेदिमां स्त्रियम्।
 शृत्वैव मनुजेनोक्तं वचनं कुपितोग्रजः।
 गदयाप्राह तन्मूर्ध्नि तस्याथो वज्रकल्पया।
 उपसुंदश्च संक्रुद्धो गदया प्राजहार तम्।
 30 गदाभ्यां वज्रकल्पाभ्यां तत्क्षणान्निहतावुभौ।
 अन्योन्यं ताडयित्वा तौ दैतेया सा तिलोत्तमा।
 देवकार्यं च निर्वयंमंदहास दिवंगता।
 एवं स्त्रियस्समुत्पन्ना मरणाय जनन्यपि।
 तस्माद्रौपदि साध्वी त्वं शुश्रूषस्व पतीन् शुचीन्।
 इह भुक्त्वाऽखिलान्भोगानंते प्राप्स्यसि सद्गतिम्।
 यूयं च पाण्डवास्सर्वे स्नेहयुक्ता परस्परम्।
 अस्यां रमध्वं पंचापि यथास्नेहो विवर्धते।
 गार्हस्त्यधर्मनिरताः क्षांता दांता द्विजप्रियाः।
 कृष्णभक्तिरता नित्यं क्षत्रधर्म करिष्यथ।
 35 किं तपोभिश्च किं शास्त्रैः किमिष्टापूर्तकर्मभिः।
 तौलौ कवेरजा सिक्तो ब्रह्महापि विशुद्ध्यति।
 किमुतश्रद्धयास्नाति तुलामासे यथाविधि।
 वंशद्वयं समुत्तार्य त्रिसप्तकुलमुद्धरेत्।
 चक्षुश्श्रोत्रांघ्रि देहानां दायंकाले नृपात्मजाः।
 
माघे च माधवे मासि तौलिकेमासि कार्तिके।
 सह्यजायां सकृत्स्नात्वा सर्वपापात् प्रमुच्यते।
 अगस्त्यःधर्मसारं रहस्यं तं नारदप्रोक्तमुत्तमम्।
 शृत्वा धर्मात्मजाहृष्टः पूजयामास नारदम्।
 पूजयित्वा यथान्याय मर्घ्यपाद्यादि लक्षणैः।
 मरुद्धाया माहात्म्यं शृत्वा हर्षाश्रु लोचनः।
 40 धर्मराजःकस्मिन् काले तु कावेर्यां तुलास्नानं प्रशस्यते।
 किं फलं किं विधेयस्यात् तत्र का देवत्या स्मृता।
 किंच गार्हस्त्य धर्मांश्च वक्तुमर्हस्यशेषतः।
 
नारदः-संयक् पृष्टोस्म्यहं भद्र धर्मसारेद्यधर्मज।
 वक्ष्ये मरुद्वृधास्तोत्रं शृणुनान्यमना नृप।
 भावशुद्ध्यातु यत्स्नानं कुलकोटिसमुद्धरेत्।
 तुला मासे तु यत्स्नानं भुक्तिमुक्त्यैक साधनम्।
 सह्यजायास्तु माहात्म्यं को ब्रूयाद् भुवनत्रये।
 स एव वक्ताश्रोता सहस्रवदनस्स्वयम्।
 45 तुलामासे सकृत्स्नात्वा कावेर्यां रङ्गसन्निधौ।
 मल्लिकाकुसुमैश्शुभ्र स्तत्कालीनैश्च पङ्कजैः।
 अर्चयित्वाऽच्युतं विष्णुं पायसान्नं निवेदयेत्।
 निवेदयित्वा श्रवणे द्वादश्यां पायसं हरेः।
 भोजयेद्ब्राह्मणान् भक्त्या सार्वभौमो भवेद्धवि।
 
द्वादशाब्दं कुरुक्षेत्रे भोजयेद्यद्विजां शुचीन्।
 द्वादश्यां तत्फलं याति पायसान्न प्रदानतः।
 द्वादश्यां पायसं प्राश्य यानारी पुत्रगर्धिनी।
 दीर्घायुः पुत्रिणी सास्याद्दीर्घमाङ्गल्यमश्नुते।
 सज्ञानाविधवाचस्यादंते विष्णुपदं व्रजेत्।
 50 अमायां वा संक्रमेषु कन्यामास्युपरागयोः।
 अश्राद्धकृद् द्विजो राजन्! सद्यश्चण्डाळतांव्रजेत्।
 अन्नाभावे द्विजाभावे प्रवासे सूतके पथि।
 आमश्राद्धं द्विजःकुर्याच्द्रःकुर्यात् सदैववत्।
 अर्धभारस्तदर्थं वा तण्डुलस्तु विधीयते।
 पणद्वयमभावेपि तदर्धवेति केचन।
 महालयेप्यमायां च संक्रमेपि च पाण्डव।
 रिक्तो वा व्याधितोवापि स्वर्गेच्छु श्राद्धमाचरेत्।
 सर्वाशक्तु तुलामासे द्वादश्यां द्वादशद्विजान्।
 55 भोजयेत् पायसान्नेन सर्वश्राद्धफलं लभेत्।
 तदर्थं वा दरिद्रोपि पितृभक्तःप्रभोजयेत्।
 दध्यन्नदानाद्वैशाखे माघे वस्त्रप्रदानतः।
 कार्तिके च्छत्रदानाच्च दीपाराधनतो हरेः।
 मार्गशीषेधुषःकाले मुद्ान्नस्य निवेदनात्।
 चैत्रे पानकदानाच्च पितॄणामतृणोभवेत्।
 लुब्धावानास्तिकः कुष्ठी दरिद्रो नैव जायते।
 मागङ्गायमुनारेवा मासेतुर्माच पुष्करम्।
 
माब्धिगङ्गासंगमश्च माप्रभासो हिरण्वति।
 न द्वारका नचाऽयोध्या बदरीमाच नैमिशम्।
 60 मागौतमी मामथुरा मागोकर्णं च कौरवम्।
 तुला मासे सह्यजायां यदिस्नानं सकृद्भवेत्।
 नूनं क्लेशं विनामुक्ति श्शक्त्या साधयितुं नृप।
 यदिनायात्सह्यजायां रङ्गेशं सेवितेपि च।
 तस्य पुण्यफलं वक्तुं शक्त एव हरिस्स्वयम्।
 माभैष्ट यूयं हे पपाः कथंस्यात् पापनिष्कृतिः।
 नरकेषु पतिष्याम इति चिंतापि मास्तु नः।
 स्नानंकुरुत मत्तीर्थे उत्तरेयमहं हि वः।
 कल्लोलबाहुमुद्धृत्य गर्जयेव मरुद्वदा।
 बहुनात्रकिमुक्तेन शृणुष्व नृपसत्तम।
 कावेरीजलसंसिक्तस्तिर्यक्व दिवमश्नुते।
 भुक्तिमुक्तिफलावाप्तिः किंपुनश्श्रद्धयाकृते।
 पुरा मरुद्वृदावाति कणस्पृष्टश्चसूकरः।
 नष्टपापःक्षणादेव देवलोकमगाहत।
 युधिष्ठिरःसूकरःकीदृशो विद्वन् कुत्रत्य साजाजले।
 कथं सिक्तो दिवंयात एतद्विस्तार्य मे वद।
 
नारदःयथाधिकारं स्नानस्य फलमाप्नोत्यसंशयः।

औदासिन्येन नास्तिक्याद्यत्स्नानं पापनाशनम्।

मा कुरुष्वच संदेहं कर्मकृत् स्वस्तिमान् भवेत्।
 पुरा बाह्रीकदेशेभूद् बहुदान्य इति शृतः।
 70 अग्रहारो महान् पुण्यो वेदघोषसमन्वितः।
 वेदशास्त्रार्थतत्वज्ञैः पंचयज्ञपरायणैः।
 अग्रहारोत्तमे तस्मिन् श्रुतिस्मृतिविचक्षणः।
 ब्रह्मशर्मेति विख्यातो बहुपाठी बहुप्रजः।
 आश्वलायनगोत्रीयश्श्रौतस्मात विचक्षणः।
 सर्वशास्त्रार्थतत्वज्ञः सर्वकर्म विशारदः।
 कृषीवल कृर्हेतोष्पट्ट्सहस्रसुवर्णवान्।
 तक्रादीनां च विक्रेता वस्तूनां वाससामपि।
 अक्रेयाणां च वस्तूनां विक्रेता धनसंग्रही।
 परान्नभोक्ता स्वगृहे न तु भुङ्क्ते कदाचन।
 75 ज्योतिर्विद्रामयाजी च निरग्निस्नानवर्जितः।
 पवित्रकूर्चदर्भादीन् समिधो गृह्यगच्छति ग्रामाद्रामांतरंगच्छन् श्राद्धभोजन तत्परः।
 असौ श्राद्धक्रियावेदी बहुपाठीति वेदवित्।
 सर्वे च वृणते श्राद्धे प्रार्थयते तमन्वहम्।
 अनाहूतस्स्वयं गत्वा भुङ्क्ते स्वातन्त्र्यमावह।
 ग्रामाद्रामान्तरंगच्छेच्छ्राद्ध हेतोरसौ वृतः।
 सभोक्तासूकरस्याच्च नाश्नंति पितरोस्य च।
 प्रायश्चित्तं न कुरुते श्रुतिस्मृत्यर्थवेद्यपि।
 कुत्रश्राद्धं कुत्र होमः कुत्र वा दीयतेधनम्।
 80
इति ब्रमतिदेशेषु न कदाचिद्गृहे वसेत्।
 गृहे वा भुंजतोबाले स्मृष्टान्नं भर्त्सयत्यसौ।
 केवलान्नं स्वयं भुङ्क्ते सदास्वोदरऊरकः।
 तस्य नारीसतीपूता भर्तृशुश्रूषणेरता।
 प्रोषितेऽल्पाशना नित्यं भर्त्तितापि न कुष्यति।
 तस्मात्प्रहरभीत्या सा संगृह्णत्यर्थमन्वहम्।
 निपुणागृहिणी स्वार्थे भोधयामास तं पतिम्।
 कायस्तु क्षणिकस्स्वामिन् जलबुद्भद संनिभः।
 अद्यश्वो वा परश्वो वा मृत्युर्वै प्राणिनां ध्रुवः।
 धर्मःकार्यो धृडेदेहे ह्यन्यथा नरकोधृवः।
 85 क्लेशदैः किं धनैरेभी राजतस्करबंधुभिः।
 किमर्थं भुज्यते भर्तः परान्नं नरकप्रदम्।
 पत्यप्रिया हता नारी ह्यभैक्षाशीव्रती तथा।
 अब्राह्मण्यो हतोराजा हतोविप्रो निरग्निकः।
 तस्मादौपासनं कृत्वा भोक्तव्यं भवने त्वया।
 इमे बाला ह्यनाथाश्च उपलालस वर्जिताः।
 अहं जीवस्स्मृतप्राया त्वद्धीना विधवा यथा।
 सर्वेषामाश्रमाणां च गार्हस्थ्यं श्रेष्ठमुच्यते।
 यदानुकूला नारीस्स्यात् तदा लक्ष्मीर्वसेद्गृहे।
 प्रतिकूलायदा नारी तदा त्याज्यान्यथा पतेत्।
 90 शूद्रप्राय इमे पुत्रा वेदशास्त्र विवर्जिताः।
 अनाचाराः कृतघ्नास्स्युश्शिक्षणीया इमे सुताः।
 
इदं गृहमहं भार्या इमे पुत्रा इदंधनम्।
 त्वदंते विनिवर्तते धर्म एकस्सुहृत्तव।
 तस्माद्धर्ममुपेक्षस्व नापेक्षस्व गृहादिकम्।
 यथा गृहीत्वा पाथेयं याति गेहस्थलांतरम्।
 पारलौकिक पाथेयं कर्तव्यं पुण्यमुत्तमम्।
 यमलोकस्य पाथेयमन्नं विप्रमुखे कुरु।
 सह्यजायां कुरु स्नानं तुलां प्राप्ते दिवाकरे।
 षट्पष्टिकोटितीर्थानि द्विसप्त भुवनेषु च।
 ९5 केशवस्याज्ञयायांति तुलामासे मरुद्धाम्।
 तस्मान्नाथ कुरु स्नानं कावेर्यां तौलिभास्करे।
 सह्यजायास्तु माहात्म्यं सर्वलोकप्रदं नृणाम्।
 अस्मिन् वै भारतेवर्षे कर्मभूमौ विशेषतः।
 तुलाकवेरजास्नानं सर्वकामाभिवर्षणम्।
 
इत्याग्नेयपुराणे तुलाकावेरि माहात्म्ये सुंदोपसुंदवृत्तान्तः नाम सप्तमोऽध्यायः।

****************************
अथ अष्टमोऽध्यायः सुशीलायाः भर्तारंप्रति तुलाकावेरि माहात्म्य कथनम् कावेरी तीरजस्नानः कावेरीसलिलाप्लुताः।
 कावेरीशीतवातायै स्स्पृष्टा यांति हरेःपदम्।
 तस्मात् कावेरि माहात्म्यं कोब्रूयाद्भुवनत्रये।
 तस्मात्तस्यां कुरु स्नानं सर्वकामार्थसिद्धये।
 
धर्मसारं प्रवक्ष्यामि मन्नाथाभीष्ठदं शृणु।
 धनं देहि गृहंदेही अन्नं देहीकुटुंबिने।
 देहीति पात्र आयाते देहं वा देहि सीदति।
 जन्मांतरकृतैःपुण्यैर्हिता नारी भवेत्सती।
 नास्ति नास्तीति योब्रूयादोदनार्थीनि लोभतः।
 अस्त्यस्तीति ततो दुःखं इति देवा वदंति तम्।
 5 त्वत्पूर्वजन्मपुण्येन मत्पुण्यैःपूर्वसंचितैः।
 
अनुकूला भवेयंते यथेष्ठं धर्ममाचर।
 कःकस्यबांधवोलोके कःकस्य सुखमादिशेत्।
 कःकस्य पोषको ब्रह्मन्! कःकस्य मरणे सुहृत्।
 धर्म एव सदारक्षेदिहलोके परत्र च।
 वयंचाघं न भोक्तारो यूयं दुष्कृतभोजिनः।
 किंचिन्न दत्वा विप्राय पश्चाद्रिक्तो भविष्यसि।
 नेत्रयोश्श्रोत्रयोर्बाह्वोः पादयोश्च तथा तनोः।
 सामीचीन्ये वर्तमाने कुरु त्वं धर्मसंग्रहम्।
 इत्युक्तः प्रियाभर्ता निजसौख्येन हेतुना।
 10 हसन्मन्दं प्रियामाह स्वप्रियार्थं सुभाषिणीम्।
 सर्वज्ञातं हि कल्याणि दैवं हि बलवत्तरम्।
 ममैतावत्प्रियो मौढ्याद्वासरावितधीकृताः।
 इतःपरं भवेद्धर्मो नरको न भवेन्मम।
 तद्ब्रह्येकं विनिश्चित्य येनश्रेयोहमाप्नुयाम्।
 न स्नातं नार्चितोवह्निर् नबिल्वैरर्चितश्शिवः।
 
न स्नातं च तुलामासे कावेर्यां कामसिद्धये।
 नदत्तमन्नं विप्रेभ्यो न शुरुता वा हरेः कथा।
 मया जन्म कृतं व्यर्थं शिश्नोदरपरेण तु।
 सुशीला
प्रातरुत्थाय संस्नाय प्रातस्संध्यामुपास्य च।
 15 कुर्वन् पंचमहायज्ञानिष्टापूर्तादिकान् कुरु।
 अश्वत्थस्थापनं चापि कुरु वाऽश्वत्थसेवनम्।
 सु संस्थाप्याश्वत्थवृक्षम् द्विरेकं वाभिवर्धयेत्।
 वर्धते तद्गृहेपुत्रा वर्धं ते च धनादयः।
 सप्तप्रदक्षिणंयेन क्रियतोबोधिनं प्रति।
 प्रदक्षिणीकृताभूमि स्तेनसाब्धिवनाचला।
 त्रिमूर्तिरूपिणं वृक्ष मश्वत्थं व्रणमेत्तु यः।
 त्रिमूर्तीनां स देवानां नमस्कारःकृतो भवेत्।
 धूपैर्दीपैश्च सद्धैस्स्रक्ताम्बूल फलादिभिः।
 पूजयित्वा नमेद्वृक्ष मश्वत्थं सधनीभवेत्।
 न मध्याह्ने नसायान्ने नापराह्ने न रात्रिषु।
 अश्वत्थसेवनकार्यम् न शुक्राङ्गारवारयोः।
 20 किं च वक्ष्ये रहस्यं ते श्रुतं पौराणिकान् मया।
 श्राद्धहास्सु च सर्वेषु स्थिरवारे द्विजश्शुचिः।
 प्रदक्षिण नमस्कारान् कृत्वाश्वत्थ तरुं प्रति।
 पुंसूक्तं पित्रुसूक्तं च तद्विष्णोरितिमन्त्रकम्।
 जपण् सिंचेजलैश्शुद्धैमौनी भूत्वा पितृण स्मरन्।
 
गयाश्राद्धंकृतम् तेन पितॄणामनृणोभवेत्।
 प्रातरालिंग्य मध्याह्ने मुच्यते व्याधिमृत्युभिः।
 भानुवारेभिषिंचेद्यस्सोक्षिबाधाद् विमुच्यते।
 आरात्त इति मन्त्रेण अश्वत्थेव च इत्यपि।
 अश्वोरूपमिति द्वाभ्यां कुर्यादालिंगनं तरोः।
 गुरौ स्नात्वाभिषेक्ता यो लभेत् कन्यांसुधीरशुभाम्।
 आसत्येनेत्युपासीत मंदेभीष्टमवाप्नुयात्।
 एवं कुर्वन् सदाचार शिशवपूजारत स्सदा।
 सदाशिवशिवेत्युक्त्वा सर्वान्कामानवाप्नुयात्।
 नारदःब्रह्मशर्माथसद्बुद्ध्या भार्ययो बोधितो द्विजः।
 तथैव चक्रे सत्कर्म जरठोविजितेंद्रियः।
 
अश्वत्थ स्थापनं कृत्वा वेदिं कृत्वा मणिःप्रभाम्।
30 निषेवे रङ्गवल्याद्यैर्वर्धयन् शोभनार्हणैः।
 प्रातस्नायी बोधिसेवी जलसेचन तत्परः।
 तस्य भार्यासतीशुद्धा साध्वीभर्तृहितेरता।
 शुश्रूषयामास पतिमिंद्राणीव पुरंदरम्।
 पित्राग्निदेवकार्येषु जागरूकासदा शुचिः।
 कुपितेमंदहासास्याद् भर्त्सिताचेद् भयंगता।
 संतुष्टे च समीपस्थाभुक्ते भुवयनंतरम्।
 सुप्तेह्युत्थाय पूर्वाह्ने हरिमत्वा नमेत्पतिम्।
 प्रणमेद्भास्करंपत्युः आयुरारोग्यसिद्धये।
 
देवालये मठे गत्वा इतिहासं सभान्तरे।
 35 सौमङ्गल्यं प्रार्थयंती नमस्कुर्याद्विजान् सती।
 तुलामासे च कार्तिख्यां माघेमासी च माधवे।
 सह्यगोत्रोद्भवातोये स्नायात्कौण्डिन्यगोत्रजा।
 यदिष्टं भोजनेभर्तु स्तत्संपाद्या हरेत्स्वयम्।
 भुङ्केत्युक्तासतीभुङ्क्ते समाहूतासमीपगा।
 उक्तापि विप्रियंकिंचिद् भर्तुर्नोत्तरमादिशत्।
 शयानस्यसतीभर्तुःपादसंवहनेरता।
 सुमनाश्शयनं गच्छेत् भूषितागंधचर्चिता।
 भीता निद्रोति सा पश्चान्निद्राळौ निजभर्तरि।
 भर्तुःपूर्व समुत्थाय गृहशुद्धिं करोत्यलम्।
 40 प्रातस्स्नानरताभर्तु रायुष्यार्थपतिव्रता।
 आर्तास्वगृहप्रांते त्रिदिनं मौनमास्थिता।
 बुभुक्षिताप्यनशना भूत्वा खर्वेण वारिणा।
 निर्व]वं त्रिदिवसं स्नात्वा स्नानदिनेशुचिः।
 हरिद्राकुङ्कुमालेपैर्लेपिता शुष्कवस्त्रका।
 स्मरंती पतिपादानं स्वपदांगुष्ठवीक्षणा।
 इतरान् सा न पश्यंती स्मरंती पतिमेवतु भर्तारं भगवन्नेवेति ध्यायंती सततं शुचिः।
 नमस्कृत्य प्रतिदिनं गुरून् वृद्धान् सतो द्विजान्।
 मध्याह्नकालेह्यतिथीन् भिक्षुकान् सत्करोति सा 45 देवार्थं वैश्वदेवानां पचंतीशुद्धमानसा।
 
उपकुर्याद्दरिद्रेभ्यो धनाद्यैः पत्यनुज्ञया।
 एवंभूतामाहाभागा मध्याह्नेऽतिथिभिक्षुकान्।
 भोजयित्वासतीन् भर्तुः पर्यवेषयदेकदा।
 एतस्मिन्नंतरे दूता वैवस्वत नियोजिताः।
 ब्रह्मशर्मन् गृहप्राप्ता एतद्रहणतत्पराः।
 पातिव्रत्य प्रभावेन सुशीलायाःप्रधाविताः।
 संदह्यमानास्तेजोभिर्यमं प्राहुस्समीपगाः।
 गन्तुं न शक्त्यास्तद्नेहं दावितास्मद्गृहतिकात्।
 सुशीला तेजसादग्धा विद्रुतास्सदिशोदश।
 50 एतद्वाक्यं यमश्शृत्वा चित्रगुप्तमथाब्रवीत्।
 एतद्वाक्यं यमश्शृत्वा चित्रगुप्तमथाब्रवीत्।
 क्षीणायुर्ब्रह्मशर्माद्यपापैर् दुर्भोजनादिभिः।
 किंकरा विनिवृत्ताश्च तद्भार्यातेजसाहताः।
 चित्रगुप्त जवाद्गत्वा ब्रह्मशर्माणमानय।
 इत्युक्तश्चित्रगुप्तोपि ब्रह्मशर्मगृहं गतः।
 भीतोंतिकेऽवतस्थेस्य सुशीलाया असन्निधौ।
 
सुशर्मा
स्वाद्विदंव्यंजनभद्रे दीयतां मे निशा भुजे।
 इति द्विजवचश्शृत्वा चित्रगुप्तो जहाससः।
 सुशीला तद् ध्वनिं शृत्वा नापश्यत्किंचिदंतिके।
 किमिदं को इहायातः केवलं श्रूयतेध्वनिः।
 55 इति विस्मयमापन्ना चिंतयंत्येवमाजगौ।
 
पूजितश्चेन्मयादेवः पातिव्रत्यं ममास्तिचेत्।
 यद्यहं शुद्धभावास्यां स्फुरतां शब्दकारणम्।
 यच्चकिंचिजगस्त्यस्मिन् दृश्यते श्रूयतेपि वा।
 अंतर्बहिश्च तत्सर्वं व्यप्यनारायणस्थितः।
 इत्येषा चेत्स्मृतिस्सत्या स्फुरतां शब्दकारणम्।
 भव दैवतं नार्या इति सत्यं वचस्सताम्।
 यद्यातिथ्यम् सदाकुर्याम् स्फुरतां शब्दकारणम्।
 कावेर्यां तु तुलामासे स्नानं संयगनुष्ठितम्।
 पूजिताश्च म यासंत स्फुरतां शब्दकारणम्।
 नारायणं परंब्रह्म तत्वं नारायणःपरः।
 नारायण परोज्योतिरात्मा नारायणःपरः।
 इत्येषाचेच्छृतिस्सत्या स्फुरताम् शब्दकारणम्।
 इत्थं सुशीला शपथात्माकं
वचस्सचित्रगुप्तोपि निशंय विस्मितः।
 सत्यात्सुबुद्धेरिव पाशयंत्रितो
ह्यदृश्यतीस्याः पुरतश्शुचिस्स्मृतः।
 तं दीप्तमन्तं मधुराकृतिं
शुभं विलोक्यदिव्याभरणैरलङ्कृतम्।
 वैश्वानरो वा हरिदश्व एव वा
हरोहरिर्वा बहुधेत्यचिंतयत्।
 चिंतयंयथनिश्चित्य पातिव्रत्य प्रभावतः।
 चित्रगुप्त इति क्षिप्रं भानयन कारणात्।
 
आगतोयमिति ज्ञात्वा प्रणनाम पतिप्रिया।
 प्रणमन्तीं च तां कान्तां सुशीलां कांचनप्रभाम्।
 65 दृष्ट्वाथ चित्रगुप्तो सा वाशीर्वचन मुक्त्वान्।
 तद्वाक्यं मङ्गळं शृत्वा सौमंगल्यं ममागतम्।
 इति हृष्टात्वर्घ्यपाद्यैः पूज्यन्तं समपूजयत्।
 अवोचच्च वचोदेव तव सत्यवचस्सदा।
 सौमङ्गल्यं त्वयादत्तं परिपालय मेधुना।
 इत्युक्तो व्रीळितःकिंचित् चित्रगुप्तोब्रवीच्च ताम्।
 त्वद्भर्तानयनायाहमागतो भवनं तव।
 **नमन्ती भवतीं दृष्ट्वा ह्याशीरूपं वचोऽबृवम्।
 अभिवादयतोज्येष्ठोनाशीर्वादं ब्रवीति यः।
 तस्यायुः क्षयते सद्य आशिषायोजयेच्चतम्।
 70 हे भद्रेऽनुपनीतं तु प्रणतं श्रित्रियोगुरुः।
 आयुष्मान् भव मेदावीत्याशिषं समुदाहरेत्।
 वर्णिनं वा गृहस्थं वा प्रणमन्तं त्रिवार्षिकम्।
 आयुष्मान् भव सौम्येति नामशर्मप्लुतं वदेत्।
 कन्यकां प्रणतां विप्र स्संयग्भर्तृमती भव।
 ऊढां नमन्ति प्रवदेत् भव दीर्घसुमङ्गला।
 प्रणतां विधवाब्रूयात्संयज्ज्ञानवती भव।
 इत्येवमाशिषोभेदा एतानुक्त्वा सुखीभवेत्।
 ** तस्मान्नमंतीं त्वां दृष्ट्वा ह्यहमाषिषमब्रुवम्।
 सौमङ्गल्यं मयादत्तं दीर्घायुस्ते पतिर्भवेत्।
 75
ततोपायं करिष्यामि सत्यमेव वचो मम।
 तुलामासे त्वयास्नातं कावेर्यामेकवत्सरम्।
 दीर्घायुराप्तये पत्युस्तत्फलं दिशमाशुचः।
 दर्शयित्वायमायाहं प्रत्यानेष्ये पतिं च ते।
 सुशीलानिष्कारणं किमर्थं त्वमाह सीर्हि तदुच्यताम्।
 चित्रगुप्त उवाचैनामिति पृष्टस्सुशीलया।
 चित्रगुप्तःपापान्नष्टायुषं भद्रे त्वत्पतिं नेतुमागतः।
 मरिष्यतेऽद्य ते भर्ता क्षणादेव न संशयः।
 एतन्मरणमज्ञात्वा वर्तमान इव द्विजः।
 रात्रौ मे दीयतामिष्टम् वक्तित्वामितिचाहसम्।
 इमे पुत्रा मयापोष्य इदं कार्य शुभंगृहम्।
 भार्येयं रक्षणीया मे कर्मकृत्वापि निंदितम्।
 मयि स्थिते गृहे तूष्णीममीषाम् का गतिर्भवेत्।
 संसारपाशबंधेन ब्रामितस्सर्वतो दिशः।
 संचारक्षमकालेय स्सदाचार विवर्जितः।
 इंद्रियेष्वथ नष्टेषु व्याधिग्रस्तः प्ररोदति।
 रुदन्तं पुत्रपौत्रेति भार्या रक्षेति वा दिनम्।
 जडःक्रोशति तूष्णीं तमिति निंदति बांधवाः।
 ततः क्रोशति मूढात्मा न कृतंपारलौकिकम्।
 अणुमात्रं च सुकृतं कुत्र गन्तव्यमित्यहो।
 85
त्यक्त्वा तु ममतां धीरैः कर्तव्यो धर्मसंग्रहः।
 कावेरी स्नानदानेन त्वया दत्तेन भामिनि।
 मदनुग्रहतो भद्रे चिरायुभविता पतिः।
 रक्षां गम्याममात्रं त्वं प्रत्यानेष्ये तव प्रियम्।
 इति सुशीलायाः भर्तारंप्रति तुलाकावेरि माहात्म्य
कथनम् नाम अष्टमोऽध्यायः
*************************
अथ नवमोऽध्यायः
ब्रह्मशर्मोपाख्यानम् अगस्त्यःतुलाकावेरि माहात्म्यं सर्वकामफलप्रदम्।
 अपमृत्युहरंनृणां सर्वाभीष्टप्रदायकम्।
 सर्वपापहरं पुण्यं शृणुनान्यमनानृप।
 कावेरीमहिमा केन वर्ण्यते तौलिमासिके।
 यत्स्नानफलदानेन मृतो विप्रो यमान्तिके।
 ब्रह्मशर्मस्वभवनं पुनःप्राप महीपते।
 हरिश्चंद्रःगत्वा विप्रो यमाभ्याशं स्वगृहं पुनरन्वगात्।
 तत्कथां ब्रूहि विस्तार्य ब्रह्मन् मे कृपयाधुना।
 
नारदः
अतिश्लाघ्यं वृत्तमेतत् सुशीलाचित्रगुप्तयोः।
 ब्रह्मशर्माच तद्भर्तान् ह्यज्ञात्वाह प्रियांद्विजः।
 5
अन्नं मे दीयतां शीघ्रं गन्तव्यं कृषिवृद्धये।
 इति पृष्ट्वा तया दत्तमन्नं प्राश्नात्त्वरायुतः।
 कण्ठाशननिरोधस्तु व्याजीकृत्य क्षणेन सः।
 चित्रगुप्तो गृहीत्वा तं प्रायात्संयमिनींप्रति।
 अथोत्तमर्गौ मार्गेतं रुंधासौ चर्मकारकौ।
 दुरात्मन् कुत्रयासित्वमद्यास्मद्वशमागतः।
 पादरक्षां गृहीत्वानौ पूर्वं द्रव्यं स दत्तवान्।
 तां पादरक्षां देह्यद्य नोचेद्तुं न शक्यते।
 चित्रगुप्तस्तुतावाहा तन्मूल्यं धीयते मया।
 विप्रोयं गच्छतुस्वार्थं ग्त्विा गच्छतायुवाम्।
 10 तमूचतुःपुनर्वाक्यं चर्माका नयान्वितम्।
 क्षुद्वारणाय विक्रीतं पादरक्षाद्वयं पुरा।
 तत्काले च ननौदत्तं द्रव्यं पृष्टश्शपत्यसौ।
 आवां जघन्यजातीत्वात् तूष्णिमास्व गृहाबहिः।
 मास्तु द्रव्यमिदानीं नौ त्वग्देये पादुकासमा।
 गोनेद्विजघ्नेस्त्रीने च निष्कृतावस्ति सद्गतिः।
 ऋणग्रस्तस्य लोकेस्मिन् प्रायश्चित्तं न विद्यते।
 प्रत्यर्पणेच द्रव्यस्य प्रायश्चित्तमितिस्मृतम्।
 अन्यथा नरकानेव व्रजेदाभूतसंप्लवम्।
 इदानीं पुत्र तां गत्वा भोक्ष्यावोस्यगृहेधनम्।
 15 इत्युक्तवंतौ तावाशु प्रहर्तुं प्रोद्यतौ द्विजम्।
 एतस्मिन् समये काचित् कुमारौब्रह्मचारिणौ।
 
तत्रागत्यच संवादं शृत्वाऽब्रूतां वचश्शुभम्।
 पूर्वोपकर्तुरस्यौवां कुर्वःप्रत्युपकारकम्।
 अनेन युवयोर्देयं किंदास्यावस्तदुच्यताम्।
 चर्मकारौपादरक्षां गृत्वैष द्रव्यं नौ न प्रदत्तवान्।
 तन्मूल्यं मास्त्विदानी नौ युवाभ्यां चर्मधीयताम्।
 इति चर्मकृतोर्वाक्यं शृत्वा ब्रूतस्तु वर्णिनौ।
 न दीयते पादरक्षे युवयोस्त्वक्प्रदीयते।
 विप्रोगच्छतुधर्मात्मा विमुक्तऋणबंधनात्।
 20 धीयतां तर्हिचेत्युक्तौ निकृत्योर्वोस्वययोस्तु तौ।
 तावत्प्रददतुश्चर्म वर्णिनौ चर्मकारयोः।
 तच्चर्मतावद्गृह्णीतां कृत्वा तूलिकयासमम्।
 त्वचं प्रदाया विप्रौ तौ विप्र गच्छेति जग्मतुः।
 तौ दृष्ट्वा विस्मितौ विप्रो बभाषे वर्णिनौवचः।
 कौ युवांब्राह्मणश्रेष्ठौ कुत्रत्यौ कथमागतौ।
 अव्याजेन हि वात्सल्यान्मयिस्नेहो निपातितः।
 दयाळ्वो युवयोःको वा ह्युपकाराः कृतःपुरा।
 कदासीत्संगतिः कस्माद् भवतोर्मे तदुच्यताम्।
 साधूनां समचित्तानां सर्वत्र करुणात्मनाम्।
 पापौघादर्शनाद्यांति यथाहि गरुडेक्षणात्।
 25 वर्णिनौत्वयासंवर्धितौ पूर्वमावामश्वत्थ भूरुहौ।

प्रत्यहं जलसेकेन रक्षितौ सेवितौ द्विज।
 प्रार्थयन् पापनाशं च जलसेचनतत्परः।
 स प्रदक्षिणन्नित्यमकरोत् शौरिवासरे।
 तदिदानीमनुस्स्मृत्य भवद्रक्षणकारणात्।
 व्यतरावस्स्वयंच्छित्वा त्वचं विप्रोत्तमर्णयोः।
 वर्तावहेत्वग्विहीनौ दर्शयंतौ त्वदर्चनाम्।
 पुनरागंयसौ भूमौ द्रक्ष्यसि त्वं नदीतटे।
 इत्युक्त्वांतर्दधाते तौ चित्रगुप्तोब्रवीविजम्।
 30 दृष्टपुण्यप्रभावोयं माहात्म्यं बोधिवृक्षयोः।
 त्वयाकृतानिपापानि नष्टाश्वत्थस्य सेवनात्।
 कावेरीस्नानपुण्येन संजातायुरभूद्भवान्।
 इतःपरं क्षितौ गत्वा कर्तव्यं पुण्यमुत्तमम्।
 इदानीं यमस्तुथ्यार्थं महत्तायुष्यसिद्धये।
 उपादेक्ष्यांयह स्तोत्रम् आयुष्यम्हदि धारय।
 नारदःइत्युक्त्वा चित्रगुप्तासौ शुभं वैवस्वतस्तवम्।
 शुद्धायाचमतराजन् द्विजायोपदिशेदश।
 उपदिशेस्तवं दिव्यं पुनर्विप्रमथाब्रवीत्।
 चित्रगुप्तःमयानितोयमंसत्वं पार्श्वे तस्मैनमस्कुरु।
 पतित्वा दण्डवद्भूमौ दृष्टमात्रे तु भूसुर।
 35 प्रसादयैनं स्तोत्रेण तुष्टोदास्यति ते वयः।

तदाशीर्वादमात्रेण दीर्घायुष्यं भविष्यति।
 नारदःचित्रगुप्तो वदन्नेवं तं निनाय यमालयम्।
 अवादीच्चस्तवं विप्रो नीतस्संयमिनीपुरीम्।
 
*वैवस्वतस्तवम् सभास्थितं धर्मराजम् दूरादेवददर्शसः।
 प्रत्युप्तनवरत्नाङ्गं स्फुरत्सिंहासनेस्थितम्।
 भासमानं किरीटेन भानुभासामहाप्रभम्।
 लसंतं कटिसूत्रेण केयूरकटकान्वितम्।
 मऽम्जुमाणिक्यरत्नाड्यं मुक्ताहारमनोहरम्।
 तप्तहाटकवर्णाभ्यां कुण्डलाभ्यां लसच्छृतिम्।
 40 दुकूलवदनंदेवं दिव्यरूपं दयानिधिम्।
 यज्ञसूत्रोत्तरीयाभ्यां लसद्वक्षस्थलोज्ज्वलम्।
 मृत्युकालान्तकाद्यैश्च शांताकारैस्समावृतम्।
 धर्माधर्मविनिश्चेतृम्मुनिभिःपरिवारितम्।
 स्फुरत्कङ्कणहस्ताभ्यां दिव्यस्त्रीभ्यां समन्वितम्।
 किंचिच्चलदुरोहारं चामराभ्यां सुवीजितम्।
 ब्राह्मणैःपूर्वपुण्याढ्यैर्नृपैर्वैश्यैस्समावृतम्।
 महाराजं यमं दृष्ट्वा किंचिद्भीतो महीसुरः।
 पतित्वा दण्डवद्भूमौ तुष्टावायुर्विवृद्धये।
 
ब्राह्मणः
धर्मराज नमस्तेस्तु साक्षाद्धर्मस्वरूपिणे।
45
धर्मिष्ठशान्तरूपाय सत्यरूप नमोनमः।
 यमाय मृत्यवे तुभ्यं कालायच नमोनमः।
 सूर्यपुत्रनमस्तेस्तु सर्वभूतक्षयाय ते।
 साधूनां पित्तुल्याय वचनामृतदायिने।
 कटकाङ्गदकेयूर हारनूपुरधारिणे।
 दूरादेवनतो दृष्ट्वा दुराचार्याद्यविप्लुतान्।
 अर्चते गंधपुष्पाद्यैः प्रत्युत्थापनार्हणैः।
 मध्यस्थाय नमस्तुभ्यं तत्वज्ञाय नमोनमः।
 ददते निजसर्वस्वं साधूनां समदर्शिनाम्।
 देवदेवनमस्तुभ्यं वेदवेदान्तवेदिने।
 50 सुकृतं स्वकृतंपूर्वं जातं पाथेयमद्यवः।
 स्वर्लोकान् गच्छतक्षिप्रं इति संप्रियवादिने।
 नमस्ते पित्रुरूपाय भक्तानामभयंकर।
 कमलाकान्तभक्ताय कमलोदरचक्षुषे।
 कमलोद्भवभव्याय कमलाभासकत्विषे।
 नमो लावण्यनिधये कारुण्यवचनालय।
 पापिनां घोररूपाय गर्जते दुर्जनाग्रतः।
 दंष्ट्राकराळबृकुटी भीषणानन ते नमः।
 ऊर्ध्वरोंणेमहारोंणेदीर्घरोंणे नमोनमः।
 घण्टारवमहाचंड कालदण्डायदण्डिने।
 55 दण्ड्यान् दण्डयते नित्यमुग्रदण्डाय ते नमः।
 कोदण्डकालदंडासि परश्वथवरायुधान्,।
 
धारिणेमारिणेलोकान् पुण्यराशिस्वरूपिणे।
 ग्रहणेसर्वलोकानां जागरूकाधिकारिणे।
 दिव्यज्ञानप्रशस्ताय समस्ताङ्गाय ते नमः।
 दिव्यरूप नमस्तुभ्यं समस्ताकृतये नमः।
 चतुर्हस्तायसाधूनां चतुरास्याय ते न्मः।
 समस्तलोकवंद्याय समस्तस्तोत्र रूपिणे।
 कालांबुधमहानील महामाहिषवाहन।
 कल्पानलमहाकील ज्वलल्लोचन ते नमः।
 60 भयंकरायपापानामभयायसुधर्मिणाम्।
 द्रवत्सुधांशु संपूर्णचंद्रास्याय नमोनमः।
 प्रळयांबुधनिर्घोष भीषयित्रेम दर्शिने।
 गोभ्यां कारुण्यपूर्णाभ्यां पश्यते सुकृतात्मने।
 तत्वाय तत्वरूपाय तत्वदृष्टे नमोनमः।
 अतीवगर्जत्प्रळयांबुदध्वनि
प्रमूर्छिताशेष दिगंतराय।
 छिंदीति भिंढीतिचूर्णयेति वक्रेसतो
वीक्ष्य विभो नमोनमः।
 कराळकायातिकठोरवाचा पापिष्ठसंघभृशभीषयो।
 ज्वलद्युगांतोद्यत बाडभाग्नि
___ समूर्ध्वरोंणे परपीडिकानाम्।
 साधुमित्राय ते शश्वत् मित्रपुत्राय ते नमः।
 शांतगात्रायशान्तानां मेरुगोत्र निमौजसे।
 65
पूजामात्रातितुष्ठाय भक्ताभीष्ठप्रदायिने।
 शिपिविष्टनमस्तेस्तु नमस्ते परमेष्ठिने।
 नाचाप्रीणयते साधून पूज्यान् पूजयते नमः।
 प्रधात्रे साधुसंलोकमपहāसतांभयम्।
 धर्मशास्त्रस्वरूपाय न्यायशास्त्रार्थचक्षुषे।
 वृकोदर नमस्तुभ्यं यमुनासोदराय च।
 यमायधर्मराजाय मृत्यवेचांतकाय च।
 वैवस्वतायकालाय सर्वभूतक्षयाय च।
 औदुंबरायबुध्नाय नीलायपरमेष्ठिने।
 वृत्तोदरायचित्राय चित्रगुप्ताय ते नमः।
 70 भूयोभूयोनमस्तुभ्यं भक्तरक्षणते नमः।
 नारदःब्रमशर्मातिधर्मिष्ठ स्तोत्रवर्मातिरक्षितः।
 नर्मदं मंदहासेन धर्मराजं ननामसः।
 इति स्तुत्वा यमायाथ प्रणनाम पुनःपुनः।
 इति विप्रस्तुतश्शांतासंतुष्टोमित्रनंदनः।
 प्राजहर्षभृशं विप्रं प्रजानां परिपालकः।
 दुंदुभिस्स्वननिर्घोष स्वनेनाघोषयन् स्थितम्।
 आयुष्मान्भव सौम्येति ह्याशीर्वादमुदाहरत्।
 निषिद्य तं चित्रगुप्तं धर्मराजो महीपते।
 बभाषे ब्रह्मशर्माणं हर्षेणतरणेस्सुतः।
 75 ईषदुन्मयमानस्सन् स्तुतिप्रीतो दयानिधिः।

स्तोत्रेनानेनसंतुष्ठो दीर्घमायुर्ददामि ते।
 शीघ्रं गच्छस्व भवनं पापे बुद्धिं च मा कृतः।
 इदं पापहरंस्तोत्रं पुण्यं मृत्युहरं शुभम्।
 गुह्याद्गुह्यतमं स्तोत्रमिदमायुष्करंपरम्।
 अनेनस्तोत्रवर्येण यो मां स्तौति दिनेदिने।
 सर्वरोगविनिर्मुक्तो दीर्घमायुष्यमश्नुते।
 * किंच वक्ष्ये रहस्यं ते प्रभावो वर्ण्यतेस्तुतेः।
 जन्मइँ मम नक्षत्रे योऽनेनस्तौति मां शुचिः।
 सस्यान्नीरोगदीर्घायुस्तत्कुलेनापमृत्यवः।
 80 इदंस्तोत्रं पठेच्छुद्ध श्रावयेद्वाद्विजं द्विजः।
 प्राप्नुयात् सशतायूंषी किं पुनर्भोगसंपदः।
 ब्रह्मशर्माकेनपुण्यप्रभावेन दीर्घायुष्यम् ममागतम्।
 केनपापफलेनाद्य नष्टान्यायूंषी मे प्रभो।
 त्वयात्सविधितं नास्ति सर्वं विस्तारतो वद।
 
यमः
पूर्वजन्मनिरोगस्थं विप्रं रिक्तंविदेशगम्।
 स्वगृहे प्रश्रयं दत्वा त्वमरक्षस्सदौषधैः।
 तेन पुण्यमहिनाद्य दीर्घायुष्यं तवागतम्।
 परान्नादनदोषेण निरग्नित्वाददात्मनः।
 अभक्ष्यभक्षणाच्चैव त्वाल्पायुर्द्विजोऽभवः।
 85 तस्मादाचारवान् भूया नास्तिधर्मवतोभयम्।

नळःपूर्व महाराजा मामेतत्स्तवस्तुवन्।
 त्रैलोक्यविजयं प्राप दीर्घायुरभवच्छ सः।
 पापःपापगतिं यांति संतस्सद्गतिमाप्नुयुः।
 *अष्टवर्षाभवेत् कन्या नववर्षातु रोहिणी।
 दशवर्षाभवेद्गौरी ह्यत ऊर्ध्वं रजस्वलाम्।
 ये गुग्गुलप्रदा विप्रा ये दीपार्थं हरेघृतम्।
 ये तण्डुलप्रदाविष्णोः निवेद्यार्थं द्विजन्मने।
 90 ये वैश्वदेविके विप्रे ते नरास्स्वर्गमाप्नुय्युः।
 तुलायां कार्तिके माघे वैशाखे स्नानतत्पराः।
 येचास्तिकाःपुराणेषु ते नरास्स्वर्गगामिनः।
 येशास्त्रज्ञानवक्तारं पौराणिकमपि द्विजम्।
 सत्कुर्वति धनैर्वस्त्रैः ते नरास्स्वर्गगामिनः।
 धूपदीपादिपात्राणां येकुर्यादानमुत्तमम्।
 वस्त्रदानं च विष्णवर्थे ते नरास्स्वर्गभाजनः।
 शृणु विप्र प्रवक्ष्यामि पापिष्ठानां गतिं पराम्।
 नास्तिकाश्च पुरणेषु येपराधन् विघातिनः।
 ये घण्टानादतुळसी धूपदीपादिवर्जितान्।
 विष्णुपूजां प्रकुर्वंती ते यांति नरकन् बहून्।
 एतादृशानि पापानि विप्र संति बहूनि च।
 शास्त्रं विचार्यकर्तव्यमन्यथा नरकं व्रजेत्।
 विप्र गच्छ गृहंशीघ्रं दुःखितास्तव बांधवाः।
 इत्युक्त्वाचोदितो विप्रो यमं नत्वा विनीतवत्।
 
शीघेर्दूतैर्गृहं नीत स्सुप्तवप्रत्यपद्यत।
 सुशीला च पतिं दृष्ट्वा मृतं पुनरिहागतम्।
 सस्वजे प्रमदागाढं प्रमोदाश्रु परिप्लुता।
 प्रणय तं निरीक्षंतिमुहुस्नेहसमन्विता।
 यमलोकस्यवृत्तान्त मपृच्छत्सकुतूहलात्।
 ब्राह्मणोपि तया पृष्टो ब्रह्मशर्मासुशीलया।
 वैवस्वतपुरी वृत्तं निजवृत्तं च सोब्रवीत्।
 105 तच्छ्रुत्वा यमवृत्तान्तमित्थंपत्याप्रभाषितम्।
 ऋणानुबंधमाश्वत्थं विस्मिता प्रशशंससा।
 तदारभ्यस्वयं साध्वी भव्याभर्तृहितेरता।
 अश्वत्थसेवामकरोत् कावेर्यास्नानतत्परा।
 शुश्रूषमाणभर्तारं सुशीलाशुद्धमानसा।
 पूर्वकर्माणुसारेण कालधर्मामुपागता।
 साध्वी साभर्तृशुश्रूषा सोपानसुपथागता।
 विमानस्थास्तुदेवैश्च ब्रह्मलोकं जगामह।
 इदं सुशीलोपाख्यानं पावनं परमाद्भुतम्।
 यःपठेच्छृणुयाद्वापि सजीवेच्छरतां शतम्।
 इति ब्रह्मशर्मोपाख्यानम् नाम नवमोऽध्यायः
**************************
अथ दशमोऽध्यायः
वराहवृत्तान्तः ब्रह्मशर्माऽथ भार्यायां प्राप्तोयां ब्रह्मणःपदम्।

किंचित्कालं सुदुःखार्तो निवसन् गृहमत्यजत्।
 चोरा आगत्य तद्गहे गृहीत्वा च धनं ययुः।
 क्षुत्पिपासा परिश्रांतो बृशं दारिद्यपीडितः।
 न शर्म लेभे कुत्रापि राजयक्ष्मप्रपीडितः।
 पुनर्दुर्भोजनं कुर्वन् पुत्रैश्च परिवर्जितः।
 सासूयैः बंधुभिस्त्यक्तः पूर्वं निरुपकारतः।
 शप्तश्च परिहासोक्त्या लज्झितोव्यचरन् महीम्।
 क्वचित्पऍषितं भुङ्क्ते क्वचिदुच्छिष्टभोजनः।
 क्वचित्स्त्रीशूद्रशेषं वा नियमाचार वर्जितः।
 5 स्नानहीनस्सदाभुङ्क्ते हाहेतिप्रलपन्मुहुः।
 पूर्वकर्ममहापापा बद्धोदेवेनचोदितः।
 सुशीलाबोधितं तत्वं स्तवं मृत्युहरं शुभम्।
 विसस्मारदुराचारो दुष्टसंघेनमूडधीः।
 एवं परिब्रमन् भूयो निरग्निाह्मणाधमः।
 मध्याह्ने कस्यचिद्नेह माप्तवान् गृहमेधिनः।
 सगृहस्था वैश्वदेवं कृत्वा वीधिं विलोकयन्।
 एनं दृष्ट्वाऽतिथिप्राप्तं भोजयामासवेश्मनि।
 स भुक्त्वासुखमासीनो ब्रह्मशर्माद्विजालये।
 आत्मवैदुष्यसिद्ध्यर्थं स्वाध्यायमकरोत्तदा।
 10 स गृहस्थापि तं वीक्ष्य वृद्धं शांतेषु निष्ठितम्।
 स्वगृहे स्थापयामास वैदुष्यादृद्धभावतः।
 इति द्विजालयावासि गृहस्थे सूतकस्पृशी।
 
रहोन्यत्रौदनं भुक्त्वा देवतार्चामथाकरोत्।
 अजीर्णेनैवदोषेण मृतोसौ नरकान् बहून्।
 भुक्त्वामन्वन्तराणिविर्नरकं मलभोजनम्।
 खण्डितोवृश्चिकाद्यैश्च तत्र मन्वन्तरत्रयम्।
 देवतार्चनतः पूर्वं भुक्त्या सौ यवनोभवत्।
 तुरुष्काःपंचजन्मानि पुष्णोभूद्वह्निहीनकः।
 श्रीमुष्णोऽरण्यदेशेथ सूकरोभून्मलाशनः।
 15 अथ वृद्धाचलावासि द्विज आत्रेयगोत्रजः।
 पद्मपत्रविशालाक्षः पद्मगर्भ इति शृतः।
 पंचाक्षरीपरोनित्यं पंचयज्ञपरायणः।
 दान्तो द्वंद्वसहोयोगी षट्कर्मनिरतास्सदा।
 स्नानद्वयरतो नित्यं जपस्वाध्यायशोभनः।
 वैष्णवेषु पुराणेषु सक्तोरामायणादिषु।
 नित्यम् रामसहस्रनामपठनात् पूतश्चषट्कर्म विद्वर्ण्यग्र्यः प्रवरोद्वितीय भवनप्रोक्त व्रताचारवित्।
 अन्यैर्ब्रह्मविदांवरैश्शिवशिवेत्युच्चारयद्भिस्समं स्नातुं सह्यानगात्म जांबूनि तुलाद्भवाजले।
 पद्मगर्भोर्धरात्रेतु द्विजैस्सार्धं ययौ पथि।
 20 एतस्मिन् समयेमेधैर्महावातप्रचोदितैः।
 गर्जद्भिस्सहविद्युद्भिशिस्सर्वास्समावृताः।
 महांधकारसंदिग्धं पन्थानं प्राप्य निर्जनम्।
 पार्षादीन् तस्स विप्रोथ विप्रैस्सममथाब्रमत्।
 
तत्र घेणी महाघोरः कश्चिद्गह्वरमाश्रितः।
 सिक्तां भूमिं समाघ्राय पोत्रेणाधारयन् महीम् मुस्तकान् भक्षयन् मत्तस्समस्तान् भीषयन् नरान्।
 धून्वन् देहंतूत्पतंश्च प्रोल्लिखन् स रसातलम्।
 नासाग्रे घर्घरारावैर्देवान् मार्ग उपागमत्।
 अग्रतो वीक्ष्यतान्वेगा दनुदुद्राव सूकरः।
 25 कुर्वंश्च घर्घरारावं सकलामहिपोपमः।
 पांथास्संदिग्धमार्गस्ते प्राद्र्वन् पृथगर्धिताः।
 उत्तरां दिशमाजग्मुः सर्वेप्राणपरीप्सया।
 पद्मगर्भो द्रवन् वेगादक्षिणां दिशमास्थितः।
 दूरंजगाम सभयो दीर्घोच्छ्वास प्रपीडितः।
 सर्वान् विहाय तं दंष्ट्री प्रादृवत्पद्मगर्भकम्।
 
आयांतं सूकरं दृष्ट्वा पुनर्विप्रोत्यधावत।
 कावेरकन्यतीरान्तं विप्रोगात्तधनुर्युतः।
 सूकरोपि महावेगात् संहर्तुं विप्रमुद्यतः।
 पश्चादेवान्वधावंत्सह्यजा तीरमाप्तवान्।
 30 मरुद्धायास्तत्रैका जन्मवारिणिवारिणी।
 कचभारं तनोतिस्म दासीस्नात्वा ऽरुणोदये।
 तद्वारि कणसंस्पृष्टस्तब्धरोमा यदृच्छया।
 तत्क्षणादेवशुद्धात्मा ध्वस्तपाप निबंधनः।
 अश्वत्थसेवनात् पूर्वं शिवपूजाप्रभावतः।
 हित्वा तु सौकरी योनी देवद्दिविसंस्थितः।
 
स्रग्गंधवस्त्राभरणो दिव्यस्त्रीकुचमर्धितः।
 पश्यतां सर्वभूतानां विमानस्थाविलोकितः।
 स्नानार्थमागतास्तत्र सह्यजातटवासिनः।
 दिव्यं दुर्योनितोमुक्तं व्योम्निवीक्ष्य च विस्मिताः।
 35 नारदःमुक्तसूकरजन्मानं पुरुषं तं विमानगम्।
 पद्मगर्भो भयान्मुक्तो वीक्ष्य वाचमुवाचह।
 को भवान् देवसंकाशस्तेजस्वी दिव्यदेहवान्।
 केन पापेन संप्राप्तं सूकरत्वं जुगुप्सितम्।
 इदानीं द्रावयन्मत्वं विद्रुतस्सूकराकृतिः।
 केनपुण्यप्रभावेन मुक्तस्सूकरजन्मतः।
 इत्युक्तो विप्रवर्येण स दिव्यपुरुषोवदत्।
 कवेरजाप्रभावं च दृष्ट्वा विस्मितमानसः।
 मुने सर्वं प्रवक्ष्यामि यद्दौरात्म्यं पुराभवत्।
 बहुधान्य इति ग्रामे ब्रह्मशर्मेत्यहं द्विजः।
 40 यवनः पंचजन्मानि तथोष्ट्राह्यभवं मुने।
 अद्य मे सौकरीयोनिं प्राप्तस्यासीच्छतं नमाः।
 शिवपूजा प्रभावेन तव संदर्शनं ह्यभूत्।
 पुरा मरुद्धास्नाता या आर्धान् केश स्रवज्जलैः।
 स्पृष्टोहं सह्यजातीरे वायुवेगाद्यदृच्छया।
 भवत्संहारणव्याजात् प्राप्तोहं सह्यजा तटम्।
 कृतार्थोस्म्यनयादास्या तुलायां स्नानशीलया।
 
त
त्र
भवतानष्टपापोहं यन्मे सूकरतागता।
 भवदर्शनमात्रेण पापभोगोविनाशितः।
 अनुजानीहि मां विप्र दयाशील नमोस्तुते।
 इति स्तुत्वा स विप्रेन्द्र परिक्रम्यप्रणय च।
 प्रशस्यदासी बहुधा स्वर्गलोकमवाप्तवान्।
 विमानस्तेगते तस्मिन् द्योतितास्तु दिशोदश।
 तत्रत्या गणिकाविप्राः कावेरी प्रशशंसिरे।
 तुलामासस्यमाहात्म्यम् कावेर्यास्नानजं फलम् दृष्ट्वाथ ब्राह्मणास्सर्वे सन्नुस्सह्योद्भवाजले।
 ।
 स्नात्वाथ पद्मगर्भस्तु प्रातस्सह्योद्भवाजले।
 50 मन्त्रवद्विधिपूर्वं तु संतर्घ्य पितृदेवताः।
 जजाप परमंजप्यं गायत्रीं वेदमातरम्।
 तं जपन्तं महाभागं परिवार्यद्विजातयः।
 तुलाकावेरिमाहात्म्य श्रवणेच्छुव ऊचिरे।
 
ब्राह्मणाः
भगवन् सर्वधर्मज्ञ सर्वशास्त्रविशारद।
 इहस्मासुकृपादृष्टिं कुरुष्व करुणानिधे।
 मरुद्धाया माहात्म्यं पठन्नस्मानिहोद्धर।
 भवादृशानां साधूनां योगमोक्षाय कल्पते।
 तस्मात्स्नानविधिं ब्रूहि यथापापक्षयो भवेत्।
 किंदेयंकोविधिःकावादेदेवता नियमःकथम्।
 55 केवलं गणिकाकेशसिक्तस्सूकरः।
 
मुक्तो दुर्योनितस्स्वर्गं विमानारूढ आप्तवान्।
 तस्माद्विस्तार्य कावेर्याः माहात्म्यंवक्तुमर्हसि।
 इति तैः प्रार्थितोविप्रैः पद्मगर्भोमुनीश्वरः।
 शृण्वतां सर्वपापघ्नीं विचित्रां सुमनोहराम्।
 तुलामाहात्म्यं तां तव प्रोक्तवान् जनसंनिधौ।
 तच्छृत्वा नृपतेसर्वे तुलास्नानपरास्सदा।
 भक्त्याप्रणम्य तं विप्रं प्रशशंसुः कवेरजाम्।
 अगस्त्यःसत्कथाश्रवणम् नृणां वेदेष्वनधिकारिणाम्।
 वर्णाश्रमस्थितानां च स्त्रीणांचैव विशेषतः।
 60 सर्वेषां प्राणिनांचैव दीपोयं ज्ञानसिद्धिदः।
 शृता हि सत्कथानित्यं सर्वपापप्रणाशिनी।
 नियतात्मा मुहूर्तं तु तदर्धवान्वहं नृप।
 शृणुयात्परयाभक्त्या दुर्गतिर्नास्ति तस्य वै।
 सत्कथाश्रवणं नामकीर्तनं तु हरे वि।
 द्वयमेवमनुष्याणां संसारव्याधिभेषजम्।
 बालोयुवावृद्धो दरिद्रश्श्रोत्रियोपि वा।
 पुराणज्ञस्सदावंद्यः पूज्यश्च सुकृतात्मभिः।
 नीचबुद्धिं न कुर्वीत पुराणज्ञे न कादाचन।
 योदत्तेऽपुनरावृत्तिं कोन्यत्तस्स्मात्परोगुरुः।
 जपकोटिसहस्रेषु भूत्वा मृत्वाऽवसीदताम्।
 65 व्यासासनं समारूढो यथापौराणिकीद्विजः।

आसमाप्तेःपुराणस्य नमस्कुर्यान्न कस्यचित्।
 नदुर्जनसमाकीर्णे नशूद्रश्वापदावृते।
 ।
 देशेनद्यूतसदने वदेत् पुण्यकथांसुधीः।
 सद्रामेसुजनाकीर्णे सत्क्षेत्रे देवतालये।
 पुण्येवाथ नदीतीरे वदेत् पुण्यकथांसुधीः।
 श्रद्धाभक्ति समायुक्ता नान्यकार्येषुलालसाः।
 वाग्यताश्शुचयोव्यग्राश्रोतारः पुण्यभागिनः।
 यथावेदेषु पुंसूक्तं यथागीता च भारते।
 मन्त्रेषु त्रिपदादेवेव्रतेष्वैकादशीव्रतम्।
 70 स्नानेषु च तुलास्नानं कावेर्यां रङ्गसन्निधौ।
 देवालयमठोद्याना न्यग्रहारान् करोति यः।
 राज्यरक्षा स्मृतिप्रोक्तं तुलानानं ततोधिकम्।
 एवं स्नात्वा तुलामासे मासांते भोजयेद्विजान्।
 द्वादश्यां पायसान्नेन सहस्रद्विजभोजनम्।
 माघे च रथसप्तम्यां वैशाखे पूर्णिमादिने।
 अष्टोत्तरसहस्रं वा शतमष्टोत्तरं तु वा।
 इत्थमावृश्चिकं स्नात्वा शृत्वा माहात्म्यमुत्तमम्।
 यथा च तुष्येत्तद्वक्ता मासांते पूजयेत्तथा।
 द्वादश्यां तु तुलामासे स्नात्वा सह्योद्भवेजले।
 75 वंद्यापि पुत्रान् लभते पायसान्नप्रदानतः।
 सुज्ञानाविधवापिस्यादंते विष्णुपदं व्रजेत्।
 पुराणं ये तु संपूज्य तांबूलाद्यैरुपायनैः।
 
शृण्वंति च कथां भक्त्या न दरिद्रा न पापिनः।
 कथायांवत्म्र्जानायां येगच्छंत्यन्यतो नराः।
 भोगांतरे प्रणश्यति तेषांदाराश्चसंपदः।
 सोष्णीषमस्तका ये वै कथांशृण्वंति पावनीम्।
 ते बलाकाःप्रजायंते पापिनोमनुजाधमाः।
 तांबूलं भक्षयंतो ये श्वविष्ठोभोजिनश्च ते।
 ये च तुङ्गासनारूढास्ते भवंतीहवायसाः।
 80 येचवीरासनारूढास्ते वै ह्यर्जुनपादपाः।
 असंप्रणम्य शृण्वंतो विषवृक्षा भवंति ते।
 तथा शयानाश्शृण्वंतो भवंत्यजगराहिते।
 समासनस्थ शृण्वन् यो गुरुतल्पगतो भवेत्।
 यो निंदति पुराणझं कथा वा पापहारिणीम्।
 ते वै जन्मशतंमा श्शुनकास्संभवंति ते।
 कथायां वर्तमानायां येवदंत्य सदुस्तरम्।
 ते गर्दभाः जायंते कृकला सास्ततःपरम्।
 ये कदापि न शृण्वंति भवंतिवनसूकराः।
 ये कुर्वंति कथाविघ्नं भवंतिग्रामसूकराः।
 85 ये कथामनुमोदंते कीर्त्यमानां नरोत्तमाः।
 अशृण्वंतोपि ते यांति शाश्वतं परमंपदम्।
 ये श्रावयंति च कथां तिष्ठति ब्रह्मणःपदे।
 आसनार्थं प्रयच्छंति पुराणस्यतु योइ नराः।
 कंबळानि च वासांसि मञ्चं फलकमेव च।
 
इष्टलोकान् समासाद्य ब्रह्मलोकं प्रयांति ते।
 पुराणस्यतु येसूत्रं वसनं तां तवं नवम्।
 भोगिनोज्ञानसंपन्नास्ते भवंति भवेभवे।
 मरुद्धायामाहात्म्यं यथाविधि च पठ्यते।
 भवद्भिश्श्रूयतांसंयगिति विप्रःप्रपाठसः।
 90 तुलामासे सकृत्स्नाति यः कवेरसुताजले।
 प्रभाते सकृदेवापि विष्णुलोके महीयते।
 शृत्वा च श्लोकमेकं तु दिव्यां हरिकथामपि।
 महापापयुतोवापि मुच्यते नात्र संशयः।
 तूष्णीं कवेरजास्नानं सप्तजन्माघनाशनम्।
 किंपुनर्नियमेनैव कावेर्यां स्नाति यो नरः।
 
स सप्तकुलमुद्धृत्य ह्युभयत्र हरिव्रजेत्।
 वित्तशाठ्यमकुर्वतो यथाशक्ति धनादिकैः।
 तुलामाहात्म्यवक्तारं पूजयित्वा यथाविधि।
 कावेर्यां स्नानतो दानान्माहात्म्यश्वनादपि।
 ९5 मुक्तपापा दिवं यांति क्षिताविष्ठानवाप्नुयुः।
 आयुष्कामी तथायुष्यं तुलाकावेरिमज्जनात्।
 तस्माच्छृणुध्वं यूयं च सर्वकामाप्तये द्विजाः।
 
इति वराहवृत्तान्तः नाम दशमोऽध्यायः
***********************
अथ एकादशोऽध्यायः
पद्मगर्भ दिव्यपुरुषयोः संवादः।
 अगस्त्यःपद्मगर्भोद्विजवरस्समीपस्थान् द्विजान् प्रति।
 मरुद्धायामाहात्म्यं यदूचेतद्धिवर्ण्यते।
 नारदःधर्मसूनो महाप्राज्ञः शृणु मे वचनं हितम्।
 तुलांप्राप्तेदिवानाथे कावेरीस्नानतत्पराः।
 सर्वपापविनिर्मुक्ता यांति वैकुण्ठमुत्तमम्।
 तुलामासे विशेषेण स्नात्वा कावेरिपाथसि।
 तस्यास्तीर्ते महापुण्ये नानारस समन्वितम्।
 नानाशाकसमायुक्तं घृतक्षीरादिसंयुतम्।
 शर्करामधुसंपूर्णम् स्वाधुपायससंयुतम्।
 सह्योद्भवाजलक्लिन्नं पक्वमन्नं ददाति यः।
 5 शतवंशान् समुद्धृत्य इष्टान् कामानवाप्य च।
 भुक्त्वा तु सकलान् भोगान् शिवलोके महीयते।
 स्नात्वा तुलार्के विप्रायश्रोत्रियाय कुटुंबिने।
 दरिद्राय ददातीह सतांबूलं सदक्षिणाम्।
 वस्त्रद्वयं सुपुण्योसौ देवैरपि सुपूजितः।
 स्नात्वा तुलार्के कावेर्यां शाकमूलफलादिकम्।
 तण्डुलाडकमाषांच तांबूलं च तिलंमधु।
 दधि तक्रं घृतं क्षीरम् रजतस्वर्णवस्त्रकम्।
 
यो दद्याट्विजवर्याय दरिद्राय दिनेदिने।
 सर्वदा सफलं तस्य सर्वयज्ञफलं तथा।
 सर्वतीर्थफलं चापि लभ्यतेनात्रसंशयः।
 बहुनात्रकिमुक्तेन तुलामासे कवेरजाम्।
 प्रशंसंति हि देवाश्च स्नानादेव हि मोक्षदाम्।
 कावेरीस्नानविमुखा ये वर्तते तुलारवौ।
 जीवच्चवाश्च ते विप्रास्तेषां जन्म स निष्फलम्।
 नारदः।
 इत्येतान् धर्मपुंजाश्च पद्मगर्भोमुनीश्वरः।
 तुलाकावेरि माहात्म्यं समीपस्थानुवाचह।
 एतस्मिन्नंतरेग्राहः क्रूरःकश्चिज्जलेशयः।
 स्नास्यं तं भूसुरं कश्चिजिघृक्षुः पार्श्ववर्त्यभूत्।
 मरुद्धायामाहात्म्ये तत्र पौराणिकोदिते।
 श्लोकमेकं प्रशुश्राव पूर्वपुण्यस्ययोगतः।
 प्रातःकाले तुलामासे यस्स्नायात्सह्यजाजले।
 सर्वकर्मविनिर्मुक्तो यातिविष्णोःपरंपदम्।
 इत्येतद्वचनं शृत्वा सद्यस्संक्षीणबंधनः।
 त्यक्त्वा ग्राहत्वमत्युग्रं देववद्दिविसुस्थितः।
 20 विमानस्थमिमं दृष्ट्वा गंधस्रग्भूषणान्वितम्।
 उवाच विस्मितो विप्रः पद्मगर्भस्स धर्मज।
 पद्मगर्भःकस्त्वं विमानमारूढो वर्ततेऽप्सरसांपतिः।
 
देवः किमोपदेवोवा कुत्रवासः कुतोजनिः।
 पद्मगर्भ वचश्शृत्वा सदिव्यपुरुषो वदत्।
 अहं पांचालदेशीयो ब्राह्मणः कङ्कगोत्रजः।
 सूर्यबंधुरितिख्यातः शुरुतिस्मृति विचक्षणः।
 निर्दयानिष्ठुरःकोपी शठोनास्तिकभाषणः।
 उत्कलाख्याग्रहारेहं न्यवसं धनसंग्रही।
 याचकार्थनिरोद्धाच पापोब्राह्मणकण्टकः।
 25 नर्तकानां च दसीनां गायकानामदांधनम्।
 तुलास्नान विहीनश्च नित्यकर्मविवर्जितः।
 शौचाचारपरिभ्रष्टश्शिश्नोदर परायणः।
 यज्ञार्थं याचमानस्तु कश्चिद्विप्रो द्विजालये।
 आजगामदरिद्रश्च नाम्ना कण्व इति श्रुतिः।
 तस्यार्थयाचनिहता मयाग्रामैकलोभिना।
 तेन पापेननष्टायुर् नरकं रौरवंगतः।
 पश्चाद्राहोहमभवं सर्वग्राहीजलेचरः।
 विप्रं जिघृक्षुणाह्यत्र पठतस्तेमुखांबुजात्।
 निर्गतश्श्लोकएकस्तु सुधारूपो मयाशृतः।
 30 श्लोकश्रवणमात्रेण ध्वस्तपापोस्मिकेवलम्।
 उत्तारितोस्मि नक्रत्वात् त्वयाकारुण्यमूर्तिना।
 संतस्समदृशोलोके निर्हेतुकदयाळवः।
 उत्तारयंति संसारादर्शनादेव पापिनः।
 धन्योस्म्यनुग्रहीतोस्मि रक्षमां शरणागतम्।
 
केनपुण्येन मे ब्रह्मन् केवलं पापजन्मनः।
 तव संदर्शनंजातम् माहात्म्यश्रवणं तथा।
 कुतूहलेन पृच्छामि संदेहं छिंदिते नमः।
 दिव्यपुंसोवचश्शृत्वा विस्मितो ब्राह्मणैस्सह।
 ध्यात्वा मुहूर्तं धर्मात्मा तमुवाचवचोर्थवत्।
 35 पद्मगर्भःग्राहत्वोत्तारकं पुण्यम् तव वच्मिपुरातनम्।
 स्नात्वा मरुद्धायां तु सह्याद्रौ ब्राह्मणःपुरा।
 अरुणोत् सह्यजास्तोत्रं पौराणिक मुखोद्गतम्।
 वङ्गदेशाधिपस्तत्र देवाद्राजा समागतः।
 मृगयासक्तधीविप्रान् ददर्शस्नानतत्परान्।
 मरुद्धायामाहात्म्यं शृत्वा पौराणिकीद्विजात्।
 सह्यजायां स्वयं स्नात्वा विप्रेभ्यो व्यतरद्धनम्।
 त्वं च तत्र गतोदृष्ट्वा राजानं दानतत्परः।
 साधुवद्रव्यसिद्ध्यर्थं स्नात्वा सह्योद्भवा जले।
 पचनार्थं परेषांच द्विजांनत्वा सदोगतः।
 40 यदृच्छा सह्यजास्नानात् विप्रसेवाप्रभावतः।
 तेन पुण्यप्रभावेन सत्संगतिरभूत्तव।
 कावेरीस्तवकश्लोक सत्संगत्या त्वमाश्रुतः।
 विप्र ग्रहणव्याजेन श्लोकार्धश्रवणं कृतम्।
 एतत्पुण्येन नष्ठाघादिवं प्राप्तोसि पुण्यकृत्।
 किं पुनश्श्रद्धयास्नायात् यःप्रातस्सह्यजाजले।
 
भुक्तिमुक्तिकरस्ते हि नात्र कार्यविचारणा।
 नारदःइति दिव्यपुमान् शृत्वा विप्रवाक्यं सविस्मयः।
 परिक्रम्य प्रणंयाथ विमानस्थोययादिवम्।
 एतत्ते सर्वमाख्यातं यस्मांस्त्वं परिपृष्ठवान्।
 45 य इदं पुण्यमाख्यानं ब्रह्मशर्मविमोक्षणम्।
 सूर्यबंधोर्मोक्षणं च ह्यायुष्यारोग्यदं वरम्।
 श्रावयंश्च पठन् वापि सोन्नदोषैर्न लिप्यते।
 पद्मगर्भःतस्माच्छृणुत हे विप्रा! स्संतुभद्राणिवो भृशम्।
 तुलागर्जति यज्ञेभ्य स्तुलदानाच्च गर्जति।
 तुलागर्जति तीव्राश्च तपसश्चमहीसुराः।
 ब्रह्मवर्ते कुरुक्षेत्रे पुष्करेच पृथूदके।
 अविमुक्ते प्रयागे च गङ्गासागरसंगमे।
 यत्फलं तत्फलंप्रोक्तं तुलाकावेरिमज्जनात्।
 येषां स्वर्गेचिरंवाञ्छा भूमिलोकेपि वा शुभे।
 यत्रकुत्रापि कावेर्यां स्नातव्यं तौलिभास्करे।
 50 आयुरारोग्यसंपत्सु रूपयौवनयोस्तथा।
 येषां मनोरथस्तेस्तु स्नातव्यं तौलिभास्करे।
 भयंयेषां हि नरकाद्दारिद्र्यात् सर्वकण्टकात्।
 स्नातव्यंतैस्तुकावेर्यां तुलाम्प्राप्ते दिवाकरे।
 पापदारिद्र्य दौर्भाग्यपंकप्रक्षाळनाय वै।
 
तुलास्नानाच्च कावेर्यां नास्त्युपायांतरं नृणाम्।
 तिस्रःकोट्यार्धकोटीच तीर्थानि भुवनत्रये।
 कवेर कन्यामायांति तुलायां रविसंक्रम।
 नारदःपद्मगर्भस्समीपस्थान् द्विजानन्यान्समीक्ष्यच।
 इत्येवं च तुलामास माहात्म्यं प्रोक्तवान् मुनिः।
 55 ततस्ते ब्राह्मणाश्चान्ये जनास्सर्वामहीपते।
 आसन् मरुद्धास्नानतत्पराः पुण्यकारिणः।
 तस्मात् त्वमपि राजेंद्र! तुलास्नानं कुरुस्वयम्।
 एतत्ते सर्वमाख्यातम् यस्मांत्वं परिपृष्टवान्।
 इति पद्मगर्भ दिव्यपुरुषयोः संवादः नाम
एकादशोऽध्यायः
***************************
अथ द्वादशोऽध्यायः
मण्डूकोपाख्यानम् कार्तिकस्यतु माहात्म्यं सावधानमनाश्शृणु।
 किं तस्यमहिमालोके वर्ण्यस्ते शत वत्सरैः।
 स एव भगवान् ब्रह्म तत्प्रभावंस्मवेत्यसौ।
 धात्रीच्छाया जलेकश्चित् कार्तिके डूंडुभार्धितः।
 प्लवोजन्मांतरज्ञानी तमुत्तार्य दिवं ययौ।
 युधिष्टिरःमण्डूको ज्ञानहीनस्तु क्षुद्रजातिर्जलाश्रितः।
 
कार्तिके जलसंबंधात् धात्रीच्छायांसमाश्रितः।
 सत्कर्मलभ्यं त्रिदिवं कथमाव सुकर्मवत्।
 भावहीनस्यसंसिद्धिर् नस्यादिति भुदाविदुः।
 बहुशास्त्रविरोधेन बुद्धिर्मोहयतीव मे।
 5 चित्तशुद्धिर्यथा मे स्यात्तत्वं ब्रूहि मुने मम।
 त्वदन्यःको वदेत्तत्वं शिष्योवामादृशोस्तिकः।
 नारदःभवानेवसुधीलॊके तत्वज्ञानविशारदः।
 यश्श्रद्दालुःपुनःप्रश्ने तत्वज्ञानबुभुत्सया।
 जन्मान्तरतपोयोगात् पुण्यतीर्थ निषेवणात्।
 सत्पात्रद्रव्यदानाच्च तत्वज्ञाने मतिर्भवेत्।
 संगस्सर्वात्मनात्याज्यस्सचेत् त्यक्तुं नशक्यते।
 सद्भिरेवसहासीत संतस्संगस्य भेषजम्।
 भवान् सत्पुरुषोलोके कृतार्थस्स्वर्गभाजनम्।
 सुकृतेस्त्वत्पितेवाहो यज्जातुस्साधुसंगतः।
 10 सत्संगमेन भवतः सत्कथाश्रवणे मतिः।
 संसारामयशान्त्यर्थमपि सत्संग औषधम्।
 सत्संगभाग्य पुण्येन भेको डुण्डुभभक्षितः।
 कार्तिकस्स्नानशुद्ध्या च त्रिदिवं प्रतिपेदिवान्।
 पुनर्भूमौ द्विजोभूत्वा वैष्णवो विष्णुमापच।
 पुरा विराटदेशेभू दग्रहारोऽति धार्मिकः।
 वास्तुहोम इति ख्यातो वास्तुहोमपवित्रतः।
 
वेदवित्प्रमुखैस्सद्भिः ब्राह्मणैर्वैष्णवोत्तमैः।
 श्रुतिस्मृतिपरैश्शांतै स्सर्वभूतदयान्वितैः।
 शत्रुमित्रसमैर्ब्रह्मनिष्ठैर् वेदान्तपारगैः।
 15 जितेंद्रियैर्जितक्रोधैः जितकामैर्महात्मभिः।
 इतिहासपुराणज्ञैःसदाचारविभूषणैः।
 विष्णुपूजापरैर्विष्णुपुराण पठनोत्सुकैः।
 परिपूर्णोत्सवैःप्रख्यसरसि स्नान तत्परैः।
 तद्रामस्थाद्विजास्सर्वे कार्तिके त्वरुणोदये।
 आबालवृद्धवनिताजेपुस्स्नात्वा सरोजले।
 कार्तिकेस्नानमाहात्म्यं धात्रीच्छायां समाश्रिताः।
 अशृण्वन् विष्णुमभ्यर्च्य उद्वास्यश्रुतिमातरम्।
 केनचित्तत्रपयसा पायसेन बुभुक्षुणा।
 अर्धितोनुदूतो भीत्या निपपातद्विजाग्रतः।
 20 उत्प्लुत्योत्प्लुत्य सहसा स्पृष्ट्वा पादं द्विजस्यच।
 दुःखेन द१रोवेगात् सरसीजलमाप्तवान्।
 तत्रस्थो डुण्डुभःकश्चिजिघांसुःपादमग्रहीत्।
 दष्टोडुण्डुभदन्तेन चुक्रोशभृशमातुरः।
 निमज्जन्नुत्पतंश्चापि ग्रस्तोसौ भुजगास्यगः।
 श्रांतःखिन्नतरश्शत्रु मन्दमन्दमुवाच ह।
 किमर्थं बाधसे मां त्वं हीनजन्मन् भुजङ्गम।
 कष्टं त्वजन्मवृत्तान्तं तमविज्ञायमदोत्कटः।
 यस्तुस्वार्थपरोमूढः परिज्ञानेस्वजन्मनः।
 
इतरं बाधते साधू कोन्यस्तस्माच्छ्सस्मृतः।
 25 सर्वप्राणिषु सा|स्मे साधुवृत्ते डाकृतेः।
 यदृच्छालाभपूर्णस्य हिंसाते कुलनाशिनी।
 शीतोष्णसुखदुःखेषु समंशान्तमलंपटम्।
 यदृच्छालाभ तुष्टं यो बाधतेसत्वधःपतेत्।
 त्वं सर्वप्राणिनां मूढः केवलं पापरूपधृत्।
 नरका न विचार्यैव धिक् त्वां मां बोधसेऽधम।
 अहं पुरातनेदेहे पुण्यपाप विमूढधीः।
 ।
 ईदृशस्स्यां कुजन्मा त्वं मा माहिंसितुमर्हसि।
 बलवानिति योमूढो धनवान्निपुणस्त्विति।
 दुरात्मायंनुदेत्साधुं यमोदृष्ट्वाह सत्यमुम्।
 30 पुण्यपाप परिज्ञानं भवेन् मानुषजन्मनि।
 अधो निपतनंपापादन्यजन्मसु सर्वदा।
 देहं दुःखालयं निंद्यं तच्च विण्मूत्रपूरितम्।
 धृत्वायो मन्यते नित्यं सजीवच्छव उच्यते।
 एतादृशमविज्ञाय केवलं स्वोदरंभरिः।
 अनुभूयोहदुःखानि स्वमांसं प्रेत्यखादिति।
 अहं जानामि ते मंदजन्म वृत्तं पुरातनम्।
 
अन्यत्र याहि मां मुक्त्वा तव श्रेयो भविष्यति।
 परोपकारशीलोय स्सर्वभूतदयापरः।
 वर्तते स तपस्विस्या स यज्वा सतु मुक्तिभाक्।
 35 सर्वतीर्थावगाहीयो योगिवा दीक्षितोवशी।
 
निर्दयस्साधुहिंसश्चे केवलं नरकालयः।
 किंपुनस्त्वद्विधो जंतुर् भवेत्तस्माद्दयान्वितः।
 इति मण्डूकवचनं शृत्वासद्यः कुतूहली।
 उन्मुच्यवेगातेंकं तं पुनःपुनरुदैक्षत।
 अपसृत्यततोऽन्यत्र विस्मितोऽतिभयान्वितः।
 अवददुःखतो भेकं क्षमस्वेति मुहुर्मुहुः।
 पप्रच्छचपुनर्हर्षा द्विस्मयादपि डुण्डुभः।
 हिंसाकारं परित्यज्य पश्चात्तप्तोदयान्वितः।
 नजानेऽहं विमूढात्मा पुण्यं वा पापमेव वा।
 40 हीनजन्मेतिवासाधो केवलं स्वोदरंभरिः।
 तस्माद्वदस्व हे साधो नमस्तेह्युपकारिणे।
 त्वद्विज्ञानं विचिंत्याद्य विस्मयोजनितो मम।
 भवान् ज्ञातिपूर्वज्ञस्तववृत्तान्तमीदृशम्।
 कोहं कुतस्त्यःकिंजन्मा कस्यपुत्रस्सखे वद।
 त्वं पुण्यकृन्नसंदेहो हीनजन्मनितेस्तधीः।
 नारदःस इत्थंभुजगेनोक्तं शृत्वा वाक्यं तु दर्दुरः।
 भयं त्यक्त्वा प्रहस्त्येतद्वाक्यमाह महीपते।
 मण्डूकःवचो विश्वसनीयं ते न हि कैतवमाश्रितः।
 मूल् वा परहिंसोवा न संभाष्यःकदाचन।
 45 डुंडुभः
एवमेव न संदेहो दुष्टवाक्यम् न विश्वसेत्।
 विश्वसेत् कृष्णसर्प वा स्त्रियंमूर्ख न विश्वसेत्।
 सत्संगेन विशुद्धात्मा नाहमद्यतु तादृशः।
 तस्माद्विश्वस्य वक्तव्यं शफे सत्येन ते सखे।
 त्वं भकेजन्मकेनाद्य पुण्येन ज्ञानवानभूः।
 कथमेतादृशं जन्म पूर्वजन्मनि को भवेत्।
 केन पापेन भेकस्स्याहं केन भुजङ्गमः।
 अहं कथं क्रूरजन्मा भवानेतादृशः क्षतौ।
 सर्वं निदर्शशेषेण यदि जानासि मे वद।
 दर्दुरःअहं कांचीपुरे विप्रः पूर्वजन्मनि पण्डितः।
 चतुर्वेदधरश्शास्त्री श्रौतस्मार्त विचक्षणः।
 50 दुराचारःपरान्नादी वाग्मीजेताच सर्वशः।
 श्रुतिस्मृतिपरोनित्यं दांबिको दुर्जनप्रियः।
 उपन्यासे महाकोपि सर्वशास्त्र विशारदः।
 उपाद्यायश्च सर्वेषां कर्तास्मादिकर्मिणाम्।
 विप्रश्रेष्ठ इति श्राद्धे नित्यं भोक्ता प्रशंसितः।
 कर्मांते दक्षिणांदत्तां गृहीत्वाऽऽचार्यगौरवात्।
 स्वल्पंदत्वापरेभ्यस्तु लोभीद्रव्यार्जनोन्मुखः।
 ममैवदीयतांसर्वमिति विघ्नकरास्सदा।
 अग्निहीनो दुराचारः प्रत्यक्षाचारभावनः।
 स्त्रीसंगमंतो नित्यं प्रातस्नानविवर्जितः।
 55
नित्यं चतुर्विधान्नादी ह्यातिथ्य विधिवर्जितः।
 पुराणवक्ता सर्वत्र प्रातःकाले मठादिषु।
 न कर्ताच स्वयं धर्मान् परेषांतु प्रबोधकः।
 यत्किंचिद्दीयतेद्रव्यं धर्मश्रवण तत्परैः।
 अहमेव गृहीत्वा तत्स्वल्पं वापि न दद्मि च।
 अहं वाचालको विद्वान् सभायां बहुमानितः।
 तृणीकुर्यामहं सर्वान् विद्यागर्वेण मोहितः।
 कदाचिन्माघमासेतु पयोष्णीसलिले शुभे।
 ब्राह्मणाःक्षत्रियावैश्याश्शूद्रा स्सन्नुस्तथास्त्रियः।
 बालवृद्धायुवानश्च यतयोब्रह्मचारिणः।
 60 कदाचिन् माघमाहात्म्यं परितव्यमितीव माम्।
 ज्ञानवृद्धवयोवृद्धस्संप्रशस्य पुनःपुनः।
 तेःप्रार्थितोऽपठं माघमाहात्म्यं पापनाशनम्।
 व्रतांत आंदोलिकास्थं कृत्वा मां स पूजयन्।
 धनरत्नांबरैर्दिव्यै राशीर्वादपुरस्सरम्।
 तत्सर्वं च गृहीत्वाहं परिज्ञातापि लोभधीः।
 किंचिन्न दत्तवान् विप्रेर्नास्तिक्याद्धनसंग्रही।
 लोकभीत्याकृतंकर्म न भावेन न मन्त्रतः।
 जनस्यागमनात्पूर्व मुषस्युत्थाय वेगतः।
 शुक्लवस्त्रावृतोमौनी तिष्ठेयं जापकोयथा।
 65 संसारे बहुचिंता मे जायते जपकालके।
 न ध्यायामि हरिध्येयं योगींद्रैःपुरुषोत्तमम्।
 
वैश्वदेवोपि नकृतः कदाचिन्नार्चितोऽतिथिः।
 कदाचिन्नार्चितोविष्णुः कोमलैस्तुळसीदकैः।
 सदापि श्रोत्रियद्वेष्ठी निंदिताधर्मबोधकान्।
 माघस्नानसमाप्ते तु नच कंचिदभोजयम्।
 तद्गृहे तु मयाभुक्तम् मृष्टान्नम् तु चतुर्विधम्।
 अत्यंतपुण्यकालेषु नश्राद्धं लोभिनाकृतम्।
 दानं वा न कृतम् स्वल्पमेको वा भोजितो द्विजः।
 सर्वेपि माघमासांते मामभीष्टैरपूजयन्।
 विप्रेणकेनचिद्दत्तं नवरत्नाङ्गुळीयकम्।
 ७0 केनचित्कटिसूत्रम् च कर्णाभरणमुत्तमम्।
 दत्तम् दुकूलयुगळं केनचिन्माघमज्जिना।
 तथा वस्त्राण्यनेकानि प्राप्तं निष्कशतत्रयम्।
 दत्तं दुकूलं भार्यायै पुत्रायाप्यङ्गुळीयकम्।
 श्वश्रुवोश्च चेलयुगळम् स्यालस्यापि च कुण्डले।
 नववस्त्राणि पुत्रीणाम् मयादत्तानि दुण्डुभ!।
 कश्चिद्दरिद्रश्शीतातं एकं वस्त्रं मयाचत।
 द्रव्यम् दत्वा गृहीतव्यम् इत्युक्तं चाटुवादिना।
 एतस्मिन् समयेऽजीर्णे कदन्नेत्वर्धरात्रके।
 तप्येऽतिसारतोल्पेयुर् बंधुविश्लेष दुःखितः।
 ७5 पुत्रो ममावदत्तत्र किमर्थं रुद्यते त्वया।
 बंधून्नस्मर हे तात किमेभिस्ते प्रयोजनम्।
 एते ते शत्रवस्सर्वे केवलं धनभागिनः।
 
विसृजस्नेहजं मोहं भार्याद्यैः किंकरिष्यसि।
 परलोकस्य पाथेयं दानं कुरु हरिप्रियम्।
 पठ गीताशनैःपुण्यामेकं वा श्लोकमुच्चर।
 यथेष्ठं कुरु भोदान मस्मान् नस्मर दुर्मते।
 जीविता च त्वयापुण्यं किंचिन्नाकारि शोभनम्।
 संसार भ्रांतितो हंत सत्कर्मापि त्वयोज्झितम्।
 एको न भोजितो विप्र श्रोत्रियो नाप्युपस्कृतः।
 80 विद्वन्मानी हरिभक्त्या त्वं नाभ्यर्चितवान् बत।
 द्रव्यार्जनतया तूष्णीं पुराणपठनं कृतम्।
 परिज्ञातापि शास्त्रार्थान् तत्कियां नाकरो पुरा।
 संसारपाशबद्धत्वात् तत्वं न ज्ञातवानपि।
 कस्त्वं कोहंतु कुत्रत्यः किमर्थं बांधवाधमाः।
 किमर्थं स्नेहपाशेन नरकान्यासि यंत्रितः।
 पुराकृतं पापफलं त्वयादृवं सुरेश्वरेणाप्य
सुभोज्यमेव हि।
 यत्त्वं विमूढः परलोकचिंतने कालो
हि लोके दुरितक्रमाबत।
 नूनं हि वैष्णवी माया सर्वेषां हि दुरत्यया।
 भवानपि विमूढात्मा पुण्यपापविचिंतने।
 ममभार्यापि तरुणी रूपेणाप्सरसा समा।
 ममाहसाध्वी दुःखार्ता सर्वावयवसुंदरी।
 अग्रे निष्कसहस्रं च दधती पतिदेवता।
 
इदंद्रव्यं गृहीत्वा त्वं यथेष्ठं विप्रसात्कुरु।
 दरिद्रे वैष्णवे विप्रे धनं दापय मुक्तये।
 समस्त पापशान्त्यर्थं गां ददस्व स दक्षिणाम्।
 वैकुण्ठा) हरेःप्रीत्यै सालग्रामं प्रदेहि च।
 भूदानं कुरु विप्रेभ्यो गृहं वा यद्यदिच्छसि।
 श्मशानसदृशं गेहं किमर्थं त्वमपेक्षसे।
 वयं पांथाश्च पांथस्त्वं दैवात्संमिळितावयम्।
 ९0 ऋणानुरूपसंबंधाद् धर्ममित्रं तवाधुना।
 भयंकरायमभटा दुस्सहानरकाग्नयः।
 नानुभुक्तं तृणमपि त्वयैवेत्युदितं पुरा।
 सर्वं च विस्मृतं किं वा व्यक्ताहि नरकालयाः।
 अर्थगृहे निवर्तते श्मशाने मित्रबांधवाः।
 सुकृतं दुष्कृतं चैव गच्छंत मम गच्छतः।
 तस्मात्सर्वात्मना भर्तःकुरु धर्मान्यथाबलम्।
 इह चिंतां त्यज स्वामिन् दुःखदां नरकप्रदाम्।
 केवलं शत्रवस्सर्वे पुत्रदारागृहादयः।
 तुष्यंति फलदातारं कुप्यंति च विपर्यये।
 ९5 धर्मं च रेधसंसक्तः पुत्रादौ देहपाटवे।
 यथा यथा स्त्री पुत्रा दावासक्तस्स्या त्सदुर्मतिः।
 तथा था हि नष्टस्यात् तक्रजंबूफलादिवत्।
 गानासक्तो यथानूनं नश्येत् कृष्णमृगो जडः।
 याथा गौ निर्जले कूपे तृणार्थे मूडधीःपतेत्।
 
व्याधिग्रस्थोयथाऽपथ्यं भक्षयेज्जडधीः क्षुधा।
 तथा नश्येद्गृहासक्तः कृत्वाकृत्यजडश्चरन्।
 अनर्थदेषु व्यापार यो नरः कर्तुमिच्छति।
 स भूमौ निहतश्शेते कीलोत्पाटीववानरः।
 100 इत्यहं बोधितोवापि पुत्राद्यैर्धर्मसंततिम्।
 जीवेयमिति दुर्बुद्धिः दुःखादश्रूण्यवर्तयम्।
 कथं जीवेत्सुतश्श्रीमान् स्त्रीवाभव्या कथं भवेत्।
 चिंताविष्ट इति स्नेहान्मरणं नान्वचिंतयम्।
 एतस्मिन्नंतरेक्रूराः छिंदिभिंधीतिवादिनः।
 भयंकरा यमभटाः कालमेघसमप्रभाः।
 ताम्रोर्ध्वकेशबूकुटी दंष्ट्रिणो विकटाश्श्वभिः।
 मुद्रासिगदाशूल परश्वथ धनुर्धराः।
 पाशैर्लोहमयैर्वेगाद् धृडंबद्धा ह्यधोमुखम्।
 जिह्वां पाशेनकषतः प्राहरंतःपदेपदे।
 105 शुनकै दयंतश्च चूर्णयंतश्शिलोत्करैः।
 तक्षतःकृकचास्याद्यैर् वर्षतोंगारमुल्बणम्।
 तत्तज्जन्म महादुःखं नरकानुभवोपमम्।
 अनुभूय ततःपुण्यलेशेन ब्रह्मबंधुताम्।
 प्राप्य कार्तिकमासेतु धात्रीमूले नदीतटे।
 पठन् कार्तिकमाहात्म्यं दशविप्रा सभोजयम्।
 नास्तिकोलोकभीतस्सन् न तु श्श्रद्धा पुरस्सरम्।
 दत्वा तु दक्षिणां तेभ्यो वैष्णवाय द्विजन्मने।
 
दत्तं च गण्डकीचक्रं भक्त्याद्वादश निष्ककैः।
 तेनान्नदान पुण्येन पौर्विकेणाद्य कार्तिके।
 आसीदामलकच्छाया जलस्पर्शाद्यदृच्छया।
 विप्रपादाब्जसंस्पर्शात् पूर्वज्ञानं समागतम्।
 सालग्रामशिलादानात् दग्धपापोहमद्य वै।
 स्वर्गलोके चिरं स्थित्वा पुनर्जातः क्षितौद्विजः।
 विष्णुभक्तो महायोगी तुलायां साजाप्लुतः।
 विष्णुलोकं गमिष्यामि सुखीभव भुजङ्गम।
 त्वं तु पूर्वं धनी वैश्यो वसन् कांचीपुरेशुभे।
 पयोष्ण्यां कार्तिके स्नात्वा मासमेकं कथांशुभाम्।
 मत्तश्शृण्वन् व्रतांते तु दक्षिणां मे न दत्तवान्।
 जात्यालुब्धो धनाड्योपि विप्रं नैकभोजयः।
 यस्माल्लब्ध्वा शुभं ज्ञानं भुक्तिमुक्तिप्रदं द्विजम्।
 स्यात्तु लोकद्वयेभ्रष्टो नास्तिको गुर्वपूजकः।
 अत्यासक्तो गृहेलोभान्नदाता न व्रतीयदि।
 ब्राह्मणः करकासक्ता यथानश्येत्तथा पतेत्।
 इतःपरं पुळिंदस्त्वंभूत्वा पंचसु जन्मसु।
 
अंतिमे जन्मनिक्रोधाद् गजनाभी दृतोजवात्।
 सोमवारेतु कार्तिक्यां धात्रीमूले शिवाग्रतः।
 कृष्णाजले पतन् रात्रौ व्याकुलो मरणंगतः।
 शृण्वन् शिवशिवेत्युच्छैः तत्र संकीर्तनं शुभम्।
 ध्वस्तपापःक्षणादेव शिवलोकं गमिष्यसि।
 
पयोष्ण्यां कार्तिकेस्नानात् सत्कथाश्रवणात्तथा।
 बहुजन्मविलंब्यापि भवेत्तवगतिश्शुभा।
 इतःपरं न कस्यापि द्रोहमाचरदुर्मते।
 एकजन्मकृतं पापं तव जन्मांतरेऽभवत्।
 तूष्णींकालंप्रतीक्षस्व भुत्त्वा यदृच्चिकागतम्।
 कर्मणा मनसावाचा नकुर्यात्परपीडनम्।
 ऋणादानं परान्नं च प्रारब्धं पापमुच्यते।
 प्रारब्धकर्मणांभोगादेव नाशो न निष्कृतिः।
 तस्माद्भवेत्सदाचार स्सदासर्वत्र शांतधीः।
 यत्किंचिल्लाभसंतोषी सुखीलोकद्वये भवेत्।
 अद्यमद्वर्णितं वृत्तं शुतं डुडुभ दुर्मते।
 सर्वात्मनातु कार्षीः परहिंसां तु दुःखदाम्।
 त्वदुष्टाभिस्सुदष्टस्य मरणाभिमुखस्य च।
 उद्गच्छंतीव मे प्राणास्तवाग्रे पश्यतोधुना।
 नारदःइत्थं वदन्नेव शरीरमाशुः त्यक्त्वा
सदुर्योनिजमात्त पुण्यः।
 दिव्यांगवान् दिव्यविमानसंस्थो
दिव्यांबरस्रक्सु विभूषणाड्यः।
 दिव्यांगनोरस्थलदिव्यचर्चा
सुखाप्लुतस्स्वर्ग मवाप भेकः।
 एतत्ते सर्वमाख्यातं कार्तिकस्यतु वैभवम्।
 
को वा कार्तिकमाहात्म्यं भ्रूयाद्वा शृणुयादपि।
 यदिदैवाद्भवेत् स्नानं कार्तिके सह्यजाजले।
 तस्यापि वैभवं वक्तुं नाहं शक्तः कदाचन।
 कार्तिके सोमवारस्तु बहुपुण्यफलप्रदः।
 यस्यश्रवणमात्रेण सर्वपापैःप्रमुच्यते।
 
इति मण्डूकोपाख्यानम् नाम द्वादशोऽध्यायः
******************
अथ त्रयोदशोऽध्यायः
कार्तीकसोमवारव्रतप्रशंसा युधिष्टिर:सर्वेषामेव वाराणां सोमवारस्यकार्तिके।
 प्राशस्त्वं केनसंप्राप्तं तद्वृतेतु फलं किमु।
 नारदःसोमवारव्रस्तं वक्ष्ये कार्तिके नृपतच्छृणु।
 व्रतानामपिसर्वेषां पितृप्रीतिकरं शुभम्।
 यथासर्वेषु मासेषु कार्तिके श्रेष्ठ उच्यते।
 सोमवारोपि सर्वेषुवारेषु श्रेष्ठ उच्यय्ते।
 त्रिंशदिनानिचोर्जेस्मिन् पितृप्रीतिकराणि च।
 त्रिंशदिनेषु यो विष्णोर् दीपाराधनमाचरेत्।
 पितरस्तत्क्षणादेव स्वर्गेस्युर् निरयेस्थिता।
 विशेषात् सोमवारेस्मिन् घृतेन शिवसन्निधौ।
 5 दीपमारोपयन् भक्त्या कुलकोटि समुद्धरेत्।
 
कार्तिके सोमवारे शिवपूजां समाचरेत्।
 ब्रह्महत्यादि पापेभ्यस्तत्क्षणादेव मुच्यते।
 कार्तिके शुक्लपक्षे तु शोभने नवमीदिने।
 उद्बभूवनिशानाथस्तदा कृतयुगोद्भवः।
 तस्मिन् शिवार्चनं कुर्यात् सर्वकामार्थसिद्धये।
 यो ज्ञानी वृक्षच्छायायां तत्पत्रैर्विष्णुमर्चयेत्।
 दुराचारोपि दुष्टात्मा तत्वज्ञानमवाप्नुयात्।
 योऽर्चयेत् विष्णुमीशं वा मुनिपुष्पैस्तु कार्तिके।
 धात्रीमूलेद्विजानष्टौ भोजयेद्धवान् भवेत्।
 सहस्रवर्तिभिर्दीपैः यःकुर्यात्पूजनं हरेः।
 कार्तिके कृत्तिकायोगे स ब्रह्मज्ञानवान् भवेत्।
 बालो वा दुर्बलावृद्धो भुंजीत निशि सद्विजः।
 यत्किंचिद्दक्षीनां दत्वा स्वर्गलोके महीयते।
 इति कार्तीकसोमवारव्रतप्रशंसा नाम
त्रयोदशोऽध्यायः।

*********
**********
चतुर्दशोऽध्यायः तुलाकावेरि स्नानविधिप्रकारः (आह्निकम्)
नारदः
अथ स्नानविधिं वक्ष्ये शृणुष्वावहितोनृप।
 विधिस्स्नानेन मुक्तिस्स्याद्विपरीतेऽघनाशनम्।
 तुलासंक्रांतिदिनतः पूर्वस्मिन् दिवसे शुचिः।
 
प्रातस्स्नात्वा जपं कृत्वा शुद्धांतःकरणो नृप।
 दुष्टसंगति दुर्भाषा दुष्टबुद्धिश्च वर्जयन्।
 भोजयित्वा यथाशक्ति ब्राह्मणान् प्रार्थयेदिति।
 तुलासंक्रममारभ्य स्नास्येहं सह्यजाजले।
 निर्विघ्नं मे व्रतं भूयादनेन प्रियतां हरिः।
 भुक्तवत्सु द्विजेष्वित्थं प्रार्थियित्वाह्यनुग्रहम्।
 दत्वाच दक्षिणां तेभ्यो धर्मवक्तारमर्चयेत्।
 5 रात्रौ कृताग्निहोत्रस्स्या देकभुक्तो जितेंद्रियः।
 सुप्त्वोत्थाय पुनःप्रातराचम्याथ हरि स्मरेत्।
 ग्रामात्तु नैऋतेदेशे मलमूत्रे विसर्जयेत्।
 ग्रामाद्रतरं गत्वा तृणैराच्छाद्य मेदिनीम्।
 शिरःप्रावृत्यवस्त्रेण पिदायास्यं करेण च।
 नमार्गेनोषरे नद्यां नदेवद्विजसन्निधौ।
 नकेदारे नसेतोच नाशुचौ नचभूरुहे।
 कुर्यान् मूत्रपुरीषेतु यदिच्छेच्छेय आत्मनः।
 गृहीतशिश्नश्चोत्थाय शौचंकुर्याद्यथा विधि।
 वल्मीकोषर मार्गान् खुसंचितान्यां शुभांमृदम्।
 10 गृह्णीयादुद्धृतां वापि जलेऽरत्निप्रमाणके।
 गुदेशोडश लिङ्गेत्रिस्तथा वामकरे दश।
 उभयोस्सप्तपर्णास्तु पादयोस्सप्तमृत्तिकाः।
 एकं लिङ्गेस्त्रीरंज़्योः मूत्रोत्सर्गे विधीयते।
 मृदामलकमात्र्यातु सर्वशौचं विधीयते।
 
आद्यंतयोश्शोचयेत शौचस्थलमहर्निशम्।
 नोचेत्पितॄण भक्षयति स्वयं चाशुचितां व्रजेत्।
 इत्थं शौचगृहस्थानां वर्णिनां द्विगुणं भवेत्।
 त्रिगुणम् च वनस्थस्य यतीनां तु चतुर्गुणम्।
 स्त्रीशूद्राणां नसंख्याता स्सर्वेषां च मनश्शुचिः।
 15 कुर्याद्वादशगण्डूषान् मलमूत्रेतु षट्चरेत्।
 प्रदेयेत् कर सव्यं च जलेचामणिबंधकम्।
 तत आदाय हस्तेन प्राङ्मुखस्सलिलं पिबेत्।
 वेदत्रयमथोध्यात्वा माषदनजलं पिबेत्।
 गोकर्णवत्करं कृत्वा गृहीत्वा देवतीर्थतः।
 पिबेच्च ब्रह्मतीर्थेन रविं वीक्ष्य जलं शुचि।
 पवित्रं धारयन् विप्रश्शुद्धाचमनमाचरेत्।
 ब्रह्मग्रन्धिःकर्मकाले भुक्ति कालेतु वर्तुलम्।
 चतुर्दभैःपवित्रं तु ब्राह्मणस्यान्य वर्णके।
 एकैकन्यूनमुद्धिष्टम् दन्तकाष्ठेश्शुभैस्समैः।
 20 क्षीरवृक्षोद्भवैःकुर्याद् यद्वा कण्ठकवृक्षजैः।
 प्रदक्षिणःक्रमेणैव मन्दं यत्नेन कारयेत्।
 ऐंद्रेशानमुखःकुर्यादितरत्रतु नारकी।
 नगण्डूषा नदन्तशुद्धी नदीदेवालयांतिके।
 शिचिर्द्वादशगण्डूषैः निषिद्धदिवसेषुच।
 पुनराचंयसंकल्प्य ततस्स्नायाद्यथा विधि।
 वर्षमासर्भ वारादीनुक्त्वा स्मृत्वा च माधवम्।
 
नमस्कुर्यां नदीपूर्वं देवान् विप्रान् गुरूनपि।
 कावेरी च नमस्कृत्य प्रार्थयेद्वरमीप्सितम्।
 "मरुद्धे महाभागे महापुण्ये मनोहरे।
 सर्वाभीष्टप्रदे देवी स्नास्यतां पुण्यवर्धिनि।
 सर्वपापक्षयकरी मम पापं विनाशय।
 25 कवेरकन्ये कावेरी समुद्रमहिषी प्रिये।
 प्रदेहि भुक्तिमुक्ति त्वं सर्वतीर्थस्वरूपिणी।
 सिंधुवर्य दयासिंधो मामुद्धर भवांबुधेः।
 श्रियं देहि सुतं देहि स्त्रियं देहि ददस्वगाः।
 आयुष्यं देहिचारोग्यं ऋणमुक्तं कुरुष्व माम्।
 त्वयिस्नास्येऽद्य कावेरी सर्वपापविशुद्धये।
 अर्घ्यं गृहाण मे देवी यथोक्तफलदा भव।
 " इत्यर्थ्य सरितेदद्याद्दण्डवत्प्रणमेत्ततः।
 इत्यभिष्ट्रय कावेरी पुष्पैरभ्यर्च्य चांभसि।
 जपेन्वै वारुणं सूक्तं ततस्स्नायाद्धरि स्मरन्।
 जप्त्वाच विष्णुगायत्री मनुमष्टाक्षरम् तु वा।
 कावेर्यान्तु सकृत्स्नातो ब्रह्महत्यां व्यपोह्यति।
 पुरोवर्ति शिखातोयैः पितृणां तृप्तिरक्षया।
 स्नानेन देहात्स्रवत्तोयं न कराभ्यां प्रमार्जयेत्।
 संतमे तत उत्तीर्य पितृदेवमुनीन् द्विजः।
 गणनाथं नमस्कुर्यात् निर्विघ्न स्नानसिद्धये।
 ओंकारेव्याहृतीयुक्तां गायत्रीं शिरसा सह।
 
AO
जलमादाय वै जप्त्वा तत ऊर्ध्वं विनिक्षिपेत्।
 दशजन्मार्जितं पापं तत्क्षणादेव नश्यति।
 इत्थं संध्यामुपास्त्याथ शुद्धे संप्रोक्षितेजले।
 35 दर्भास्तृतेवशी तिष्ठन् गायत्रीजपमाचरेत्।
 जकारो जन्मविच्छेदः पकारःपापनाशनः।
 तस्माज्जप इति प्रोक्तो जन्मपाप विनाशनः।
 गायं तं त्रायतेतस्माद् गायत्रीत्युच्यते बुधैः।
 ध्यात्वार्कमण्डलेदेवीं हृदये वा जपेद्बुधः।
 उपचारैष्योडशभिर्मानसैः परिपूजयेत्।
 आद्योंकारगृहस्थस्य वर्णिनश्चतथेतरौ।
 वानप्रस्थयतीनां तु षडोम्कारःप्रकीर्तिताः।
 उत्तानितांगुळीहस्त आस्यांते प्रातरिष्यते।
 स्कंदपृशौ करौकृत्वा मध्याह्नेजप इष्यते।
 40 नाभौकृत्वाकरौ सायमेवं जपविनिर्णयः।
 नीरंद्रांगुळिकौकृत्वा किंचिन्नम्राङ्गुळीकरौ।
 जपं कुर्वीत रंध्रेतु प्रसरन् राक्षसो हरेत्।
 उदये प्राङ्मुखस्तिष्ठन् संख्यायैव जपेद्बुधः।
 असंख्यातो पवित्रश्च जपो निष्फलतामियात्।
 पर्वभिर्गणयेत्संख्यां नाक्षमालादिभिर्नृप।
 गायत्र्या वेदमूलत्वात् वेदःपर्वसु गीयते।
 आरभ्यानामिका मध्यपर्व प्रादक्षिणक्रमात्।
 तर्जनीमूलपर्यन्तं संख्याय त्रिपदां जपेत्।
 
मध्यपर्वद्वयं मेरुस्तमेरुं नातिलंघयेत्।
45 व्यस्तपादां तु गायत्रीं योजपेत् स हरिप्रियः।
 जपेत्समस्त पादांयस्सभवेद्ब्रह्मराक्षसः।
 हृदयेऽष्टदळंपद्मं कर्णिकामध्यशोभितम्।
 तस्योपरिष्टाद्रुचिरं नानारत्न विभूषितम्।
 विद्रुमस्फटिकाकार शतस्तंभसमन्वितम्।
 सूर्यकोटिप्रतीकाशं विमानं सु विराजितम्।
 कल्पवृक्षसुमोपेतं मल्लिकाकुसुमैर्वृतम्।
 एतद्विमानमध्यस्थ स्वर्णपद्मोपरिस्थिते।
 सिंहासने समासीनं सूर्यकोटिसमप्रभम्।
 विष्णुं चतुर्भुजं लक्ष्मीभूमिभ्यामुपशोभितम्।
 50 ध्यात्वैवं हृदिवा सूर्यमण्डलेमातरं जपेत्।
 स्नानं विनातु गायत्री योजपेद्रोगवर्जितः।
 पश्चाद्रिक्तो भवेत्सद्यः क्षयरोगीह जायते।
 तस्मात्स्नात्वा यथाशक्ति जपेदाभास्करोदयम्।
 अभीष्टदं नमस्कुर्याद्भास्करं भवभंजनम्।
 अभिवादयेत् सूर्य महस्मीच भो इति।
 दिशश्चतस्रः प्राच्याद्यैः प्रणवादि नमोंतकैः।
 चतुर्थ्यतैनमस्कुर्या दूर्ध्वधस्संज्ञिकादिशौ।
 55 अंतरिक्षं च भूमिं च दण्डवत्प्रणमेद्विजः।
 आत्माभिवादनं कृत्वा ततो ध्यायेजनार्दनम्।
 एकाहं जपहीनस्स्यात् संध्याहीनो दिनत्रयम्।
 
द्वादशाहमनग्नीस्तु शूद्र एव न संशयः।
 वेदेषु पौरुषं सूक्तं तथागीताच भारते।
 मन्त्रेषु त्रिपदादेवी व्रतेष्वेकादशीव्रतम्।
 स्नानेषु च तुलास्नानं कावेर्यां रङ्गसन्निधौ।
 पंचोपायामुमुक्षूणां पंचपातकनाशनाः।
 दर्शिताब्रह्मनाद्येते कारुण्येन कलौनृणाम्।
 पितृगृहीत्वा त्रिपदां गृह्णीयाच्च गुरोश्रुतिम्।
 शौचाचार समायुक्तः परंब्रह्मेधिगच्छति।
 शरीरमाद्यंखलुधर्मसाधनं विवाह
__ आद्यं खलुधर्मसाधनं।
 त्रय्यंतसारोऽमृतलोकसाधनं तस्मात्त्रयं
कार्यमिदं विबुद्ध्य।
 60 संध्ययोर्दीपमालाच यद्गृहे सस्वयं हरिः।
 यत्र पर्दूषितांबोभिः पच्यतेऽन्नादिकं नृप।
 अतिपक्कमपक्वान्नं यत्रान्नं केशदूषितम्।
 नार्यान्नं भिक्षवेयत्र नातिथेश्च प्रदीयते।
 होम धूमेन वेदश्च भार्याभर्ताऽथवाऽशुचिः।
 तत्रास्माकं सुखावास इति पापो वदंति हि।
 श्रोत्रंश्रुतेनैव न कुण्डलेन
दानेनपाणिर् न तु कङ्कणेन।
 विभातिकायःकरुणापराणां परोपकारेण न चंदनेन।
 भूभारस्स्वार्थदेहस्तु परार्थोदिव्य उच्यते।
 
भर्तरिप्रोषिते होमं कुर्यात्पत्नीस्वयंशुचिः।
65 ऋत्विगाद्याश्च वा कुर्युस्त्रियाःपुंसोथवाग्रतः।
 असमक्षेतु दंपत्योः कृतोहोमस्तु निष्फलः।
 समित्प्रादेशकेध्मं स्याद् द्विगुणः परिधिस्स्मृतः।
 अङ्गुष्ठमात्रकास्साग्रा स्सत्वचः कीटवर्जिताः।
 आधेयास्समिधस्साम्य मेतासां होमकर्मणि।
 तुळसीकाष्ठसज्वाले योजुहोति हविर्भुज।
 तस्य पापानि नश्येयुरग्निहोत्र फलं भवेत्।
 पालाशाश्वत्थखदिरा बिल्वा औदुंबराःकुशाः।
 अपामार्गाश्च दूर्वाश्च समिधोऽष्टविधास्स्मृताः।
 समित्सकण्टकात्याज्या शमीं खादिरमन्तरा।
 ७0 अग्निं प्रज्वलयेद्विद्वान् धमन्यैवच नेतरैः।
 धमन्यभावे काष्ठादीन् मध्ये कृत्वाधमेन्नृप।
 होतव्यमग्निज्वालायां साज्वाला रसनास्स्मृता।
 सर्वकर्मसु पूर्वस्यां पितृकर्मसु दक्षिणे।
 यत्रकाष्ठम् तु तत्पादः नासाधूमावृतं स्थलम् ।
 अल्पज्वाला भवेन्नेत्रं भस्मदेशस्तु शीर्षकम्।
 अल्पज्वाला काष्ठभस्मधूमेन जुहुयाद्बुधः।
 व्याधिर्धर्मक्षयंक्षयो निंदा दारिद्र्यं च फलं क्रमात्।
 आरंभे चाप्युपस्थाने होतव्यं च समिवयम्।
 देवतर्पणहोमौ च बलिकार्यं च भोजनम्।
 ७5 मार्जनं देवताधं च कुर्वीरन् देवतीर्थतः।

योऽनर्चिषी जुहुत्यग्नावंगारेवाथ भस्मनि।
 मन्दाग्निरामयावी च दरिद्रश्च सजायते।
 हविर्वादशपर्वस्थमुख्यंचोत्तानपाणिना।
 हुत्वाहुती च तत्पुण्यं होताहोमफलं लभेत्।
 समिधात्म समारूढो द्विकालमाहुतस्तथा।
 धार्यमाणश्चतूरात्र माहुताग्निस्तुलौकिकः।
 आप्रातश्चोत्तमस्स्मार्तो गौणश्चासंगवंभवेत्।
 सूर्यास्तमयमारभ्य आप्रदोषं तथोत्तमम्।
 आकाल्याडोमकालस्स्यादिति केचिद्वदंति हि।
 80 यो भुङ्क्ते तु द्विजपशुर् होमं हित्वा स मूडधीः।
 त्रैलोक्य पापं भुङ्क्तेसौ विष्ठाशीस्यात्ससूकरः।
 तस्मात्कण्ठगतःप्राणः पापभीतोऽग्निमान् भवेत्।
 षष्टिप्रस्थमिदं धान्यं त्रिप्रस्थमिदं घृतम्।
 औपासनाग्नौ नष्टे तु वत्सरे च विधिर्भवेत्।
 ऊखायां च शरावे च महाग्नावश्म मस्तके।
 शूद्राग्नौ सूतकाग्नौ न जुहुयाद्विधवानले।
 प्रातस्स्नात्वा च जप्त्वा च गायत्री भास्करोदये।
 यो जुहोति हविर्वह्नौ स एव श्रुतिपारगाः।
 मन्त्र तन्त्रविपर्यास कालन्यूनातिरेकजान्।
 85 "हंतु दोषांत्समस्तांश्च तुळसीकाष्ठपावकः।
 तुलसीकाष्ठवह्नौ यः पार्वणम् होममाचरेत्।
 सहस्र वार्षिकंश्राद्धम् गयायां तु कृतम् भवेत्।
 
योऽग्रहारेधूपयति तुळसी काष्ठवह्निना।
 तुळसीकाष्ठसंभूत दीपाराधनमाचरेत्।
 यो हरेरेकवारं वा स वैकुण्ठे महीयते।
 एकं वा तुळसीकाष्ठं चितामध्येस्थितं यदि।
 तेन दग्धोपिपापिष्ठो वैकुंठं शुचिमाप्नुयात्।
 तुलसीकाष्ठवह्नौ यः पाकं कृत्वा निवेदयेत्।
 हरये मार्गशीर्षे तु मुद्ान्नमरुणोदये।
 ९0 तस्य पुण्यफलं वक्तुं नाहं शक्नोमिनेतरे।
 तुलामासेतु द्वादश्यां तुलसीकाष्ठवह्निना।
 निवेदनं तु पक्वस्य सहस्रद्विजभोजनम्।
 माघे च रथसप्तम्यां द्वादश्यां वा निवेदनात्।
 सहस्रवार्षिकीपूजा दिनेनैकेन लभ्यते।
 " तुलाकार्तिकमाघाश्च वैशाखो वह्निसेवनम्।
 पंचैते पुण्यसंकल्पाः पंचपातकनाशनम्।
 एषुस्नातुमशक्तश्चेत् तौलिकेस्नानमाचरेत्।
 सर्वशक्तो द्विजोयस्तु चण्डालःकोटिजन्मसु।
 शक्तोग्निहोत्रं कुर्वीत रिक्तस्स्मार्ताग्निमान् भवेत्।
 95 तुळसीकाष्ठकं शकलं न्यस्य प्रज्वाल्य बर्हिषम्।
 स्थातन् वह्निमुपासित्वा श्रौतवह्नि फलं भवेत्।
 नृपैरुपद्रुतो रुग्णो न स्नायी नाग्निमान्यदि।
 नतुपापी सविज्ञेयस्तस्य नास्ति च निष्कृतिः।
 ब्राह्मेचोत्थाय पादौ च प्रक्षाळ्याचम्य सुस्थितः।
 
प्राङ्मुखो वाग्यतश्शुद्धो हरिस्स्मरणपूर्वकम्।
 गङ्गां मरुद्धां माघम् वैशाखं तुळसी शुभाम्।
 अष्टोत्तरशतक्षेत्रम् स्मृत्वा संकीर्तयेद्धरिम्।
 तत्तत्फलमवाप्नोति नात्रकार्यविचारणा।
 ये पिबंतिसदास्वाधु रामायण कथामृतम्।
 100 माहात्म्यं सह्यजायाश्च तेषां मार्गोंदिवाकरः।
 येपिबंति सकृत्तोयं कावेर्यां रङ्गसन्निधौ।
 सकृन्नद्यां तटाके वा वैशाखस्नानमुत्तमम्।
 कृत्वार्चयंति ये विष्णुं तेषां मार्गोदिवाकरः।
 मृतिकालेतुतुलसी सालग्रामशिलाजलम्।
 यःप्राश्यधारयन्मूर्ध्नि तस्य तेषां मार्गोदिवाकरः।
 देवालयमठोद्याना न्यग्रहारान् करोति यः।
 राज्यरक्षा स्मृतिप्रोक्तां तेषां मार्गोंदिवाकरः।
 यस्स्नायात्सह्यजातोये यावदामरणांतिकम्।
 तुलामासेऽथवास्नायात् तेषां मार्गोदिवाकरः।
 105 तदशक्तश्चमाहात्म्यं कावेर्यास्संशृणोति यः।
 तदशक्तस्स्मरेत्प्रातः तेषां मार्गोदिवाकरः।
 तुलामासेतु कावेर्यां यस्स्नात्वा मल्लिकासुमैः।
 अर्चयेच्चिवलिङ्गम् तु तेषां मार्गोंदिवाकरः।
 तुलामासे पठेद्यो वै श्रावयेद्वापि भारतम्।
 रामायणम् वा सायाह्ने तेषां मार्गोंदिवाकरः।
 तद्भानोर्मण्डलेज्वाला पाटले चक्रपाणिकम्।
 
स्मरन् वह्नि दयायुक्तं वामभागेस्वधान्वितम्।
 मन्त्रं चत्वारिशृङ्गादि मन्यं वेदोक्तमुच्चरेत्।
 एवम् हुत्वा हविश्शुद्ध मुपतिष्ठे तनुस्थितः।
 110 "नमो भगवते तुभ्यम् स्वाहादेवीप्रियंकर।
 दयानिधे ममस्वामिन् घोरं पापं विनाशय।
 त्वं हि रक्षा करो नित्यं देवानाम् हव्यवाहन।
 ततो नमस्करोमि त्वां त्वदाधारमिदं जगत्।
 प्रतिकूलकळत्रस्य गृहस्थस्य विधिर्यथा।
 ब्रह्मजन्म तथा नश्ये दनिवेद्यान्न भोजनात्।
 तथा साग्नेः प्रणश्यति ब्राह्मणस्याघसंचयाः।
 जिह्वां कुकिंच शिश्नं च निघृह्णीयाच्च यो द्विजः।
 सौम्याकृतिं यमं पश्येदन्यथा घोररूपिणम्।
 नूनं नरकभीतो यःप्रातस्नानं समाचरेत्।
 द्विकालं वह्निपूजां च साधूनामुपकारकः।
 115 अस्यकर्मण्यशक्तश्चेत् नत्यजेदग्न्युपासनम्।
 ब्रह्मलोकं नयातीति वदंति ब्रह्मवादिनः।
 वैश्वदेव महायज्ञान् स्थालीपाकद्वयं तथा।
 औपासनाग्नौ कुर्वाणो ब्रह्मभूयायकल्पते।
 पंचकोटितपःपुण्यं कौशिकोमुनिसत्तमः।
 आगताय वसिष्ठाय स्वाश्रमे प्रददौपुरा।
 वसिष्ठस्वाश्रमस्थायी विश्वामित्राय साग्नये।
 सायमाहुतिजं पुण्यम् प्रादात्तस्मै ततोधिकम् ।
 119
इति तुलाकावेरि स्नानविधिप्रकारः नाम
चतुर्दशोऽध्यायः
***********************
अथ पंचदशोऽध्यायः
संतपनाख्य ब्रह्मणोपाख्यानम्।
 अस्मिन्नर्थे प्रवक्ष्यामि संवादं पापनाशनम्।
 धर्मलस्य च धर्मर्धे श्चित्रगुप्तयमस्य च।
 आसीत्संतापनोनाम दयावान् ब्राह्मणोत्तमः।
 सभार्यासर्ववित्प्राज्ञ स्सर्वशास्त्रविशारदः।
 सालग्रामार्चकोधीमान् सायंप्रात ताग्निकः।
 अथ तस्य च कालेन द्वौपुत्रौ समयाजिताम्।
 चक्रे संतापनस्साग्निर् नामधेये च पुत्रयोः।
 धर्मलस्यात्कनिष्ठस्तु ज्येष्ठोधर्मर्धिरित्यपि।
 धर्मलोधर्मबुद्धिश्च वेदवेदांतपारगौ।
 तपस्यंतौ जितक्रोधौ ब्रह्मचर्यपरायणौ।
 5 प्राप्तकालेऽथपित्रा तौ कृतोद्वाहौ बभूवतुः।
 पुत्रौ विवा ह्य विधिवत् ऋषिप्रोवाच ता पिता।
 पुत्रौयुवामजस्राग्नी भवतं सर्वधात्मवत्।
 औपासनं गृहे यस्य मृत्युर्दैत्याश्च बिभ्यति।
 तस्मादग्निं पुरस्कृत्य तिष्ठतंचानिशं सुतौ।
 संतापनस्सुतावित्त मुक्त्वाकालं विलोकयन्।
 अवातिष्ठच्चिरंकालं सभार्यास्सहुताग्निकः।
 
आजगामाथविप्रस्य विमानं सुरपूजितम्।
 विमानं तत्समारुह्य विसृज्येदं कळेबरम्।
 
ययौ तन्महिषीसाद्वी पुत्रस्नेहं विसृज्य च।
 10 तेनैव भासहिता जगामत्रिदशालयम्।
 धर्मर्द्धिर्धर्मवीद्धांतस्सपत्नीकस्समाहितः।
 कृत्वापित्रोःक्रियामू| यथार्हसाग्निमात्रकम्।
 वर्तमानो निजेधर्मे तपस्वीस्वाश्रमेऽवसत्।
 धर्मलस्तु समस्तानि दानानि शतशोददत्।
 आगतानति धीनस्यान् भोजयामास षड्सैः।
 पितृभ्यामार्जितंद्रव्यं सत्वास्ते प्रतिपादयन्।
 दानं ददौ स विप्रेभ्यो गोसहस्रं दिनेदिने।
 पृथिव्यां यानितीर्थानि गङ्गाद्यास्सरितश्चयाः।
 15 निषेवितानि पुण्यानि पुण्यक्षेत्रोद्भवानिच।
 देवतायतनं चक्रे गोपुराणिविशेषतः।
 तटाकोद्यानकारीच गोब्राह्मणहितेरतः।
 सदाध्यानपरोविप्रो वेदाभ्यासरतस्सदा।
 एवं धर्मेणयुक्तस्सन् धर्मलस्त्व करोड्ययम्।
 एकदा ब्रातरिगते तीर्थयात्रापरायणे।
 पित्र्येच दिवसेप्राप्ते पैतृकं कर्तुमुद्यतः।
 मातृगेहस्थितां भार्यांमुक्त्वाबालेति मूडधीः।
 अनाहूय पितृश्राद्धे स्वयमेव पपाचसः।
 इतःपूर्वेऽकृताग्निश्च ब्राह्मणानपृणोत्तदा।
 20
हित्वा तु पार्वणींकर्म हीनाग्निःकेवलं पितॄण्।
 वृतेभ्योब्राह्मणेभ्यस्तु स्वधाकल्पं ददौ स्वयम्।
 स्वधाकल्पानितान्येव राक्षसेभ्योऽभवंस्तदा।
 आगताःपितरश्चास्य ह्यन्तर्धानगतास्तदा।
 अभोज्यंक व्यमस्माभि रग्निहीनं तु राक्षसम्।
 इत्यस्य पितरःक्रुद्धा श्शप्तुंचारेभिरे तदा।
 निस्सन्तानो धनींनोज्ञाहीनोभवत्त्वयम्।
 इत्थं शप्त्वा पितृगणाः प्रययुर् नृपसत्तम।
 धर्मर्द्धिरागतस्साग्नि स्सकळत्रोबहुप्रजः।
 एतस्मिन्नन्तरे विद्वान् चित्रगुप्तोस्तुलेभकः।
 25 सयाम्यां पटमादाय ह्यपठच्छमनाग्रतः।
 तथा धर्मलधर्मोः प्रेषयामासधर्मराट्।
 यानद्वयं च भृत्यान् स्वान्तावानयत चेत्यशात्।
 चित्रगुप्त उवाचैनं यानायोग्यो निरग्निकः।
 स्त्रीनं स्वमातृहन्तारं दृष्ट्वा चाहुत पादकम्।
 सूर्यावलोकनं कुर्यात् स्पृष्ट्वा सायान्नदीजले।
 न द्रष्टव्यो नसंभाष्य स्तावक्तेरपि किंकरैः।
 चित्रगुप्तवचश्शृत्वा धर्मराजोऽतिकोपनः।
 श्वानौ द्वौ श्यामशबळौ वैवस्वतकुलोद्भवौ।
 अन्यान्दंष्ट्राकराळास्स्यात्
कृष्णाङ्गानग्निलोचनान्।
 30 तं नेतुं प्रेषयामास यमो वैवस्वतस्स्वयम्।
 
शुनकस्सत्वरङ्गत्वा तं कण्ठे जगृहुःकिल।
 धर्मर्द्धये तयोरेकं विमानं प्राहिणोद्यमः।
 देहमुत्सृज्य धर्मर्धिविमानमधिरूढवान्।
 तावेककाले संप्राप्तौ तदा संयमिनीपुरीम्।
 आयान्तं वीक्ष्य धर्मर्द्धिमुत्थायानम्यधर्मराट्।
 साग्नये च ददौ तस्मै प्रशस्यनिजमानसम्।
 तुष्टोस्मि धर्शनात्तेऽद्य भजलोकन् यथाप्रियम्।
 इत्येवमुक्त्वा साग्निं च धर्मलंत्वग्निहीनकम्।
 घोरासिपत्रैःकृकचैः धारयामास धर्मराट।
 35 दुःखितं भ्रातरं दृष्ट्वा धर्मर्धिर्यम मब्रवीत्।
 मम भ्राता महाभाग दानधर्मपरायणः।
 तीर्थयात्रा परोदान्तः सर्वदाऽतिथिपूजकः।
 तस्येयमीदृशीबाधा किंनिमित्तं वदस्व मे।
 यमःद्विजस्य सर्वगात्रेषु ब्रह्मविष्णुशिवादयः।
 रूमकूटेषु मुनयो वदने हव्यवाहनः।
 तस्मादग्निमुखा विप्रा स्तेषां मुख्यं मुखं विदुः।
 मुखहीनो यथानैव भाति तद्वदनग्निकः।
 तस्मादग्निमुखो नित्यं धर्मं कुर्वीत मानवः।
 यवीयांस्तु तव भ्राता त्वद्व न्नहुत पावकः।
 40 अत एवास्य निरय स्त्वं दिवंगच्छमाशुचः।
 धर्मर्धिः -
ददामि तस्मै धर्मार्धमेनंमुंचेत्युवाच तम्।
 इति तद्वचनं शृत्वा पुनःप्राह द्विजं यमः।
 औपासन विहीनेतु यद्दत्तं निष्फलं भवेत्।
 अन्नपानादि दानेषु नैव पात्रविचारणा।
 अन्नस्य क्षुदितं पात्रं प्राणिमात्रं तु केवलं।
 तस्माच्चिरार्जितं पुण्यम् नऽपात्रे भस्मसात्कुरु।
 इत्युक्त्वा तं विसृज्यैव धर्मलं रौरवेऽक्षिपत्।
 मुनयःकदामुक्तिरनग्नेस्स्यात् रौरवात् किं सु जीवनम्।
 यमःपद्मम दमित्येवं संख्यायां परिवर्तते।
 45 तावन्तमग्निहीनस्ते निमज्जंत्येव रौरवे।
 पिबत्युच्छिष्टसलिल मूर्ध्वपादाऽधोमुखाः।
 विशिष्टब्राह्मणा नित्यं तेषां तृप्त्यर्थमेव च।
 उच्छिष्टसलिलं दद्युभॊजनान्ते समन्त्रकम्।
 मन्त्रः- रौरवेऽपुण्यनिलये पद्मार्बुधनिवासिनाम्।
 अर्धिनामुदकं दत मक्षय्यमुपतिष्ठतु।
 तस्मात्सर्वप्रयत्नेन ब्राह्मणोह्यग्निमान् भवेत्।
 एवं स्नाता तुलामासे समाप्तौ भोजयेद्विजान्।
 अष्टोत्तरसहस्रं तु शतमष्टोत्तरं तु वा।
 अष्टाविंशवादश वा यथाशक्ति प्रभोजयेत्।
 50 मरुद्धया माहात्म्यं पठेद्वा शृणुयादपि।
 
इत्थमावृश्चिकं स्नात्वा सर्वाभीष्टमवाप्नुयात्।
 तुलाविशुद्धद्वादश्यां पायसैभॊजयेद् द्विजान्।
 द्वादश्याम् पायसं दत्वा कुलकोटींसमुद्धरेत्।
 रंभाताम्बूल वस्त्रान्न स्वर्णधान्य तिलं मधु।
 तैलं क्षीरं घृतं तकं फली दानमाचरेत्।
 इति संतपनाख्य ब्रह्मणोपाख्यानम् नाम
पंचदशोऽध्यायः
*********************
अथ षोडशोऽध्यायः श्रीरङ्गराजाऽष्टोत्तरशतनामरत्नस्तोत्रमहामन्त्र
-शंतनुराजवृत्तान्तः।
 नारदःतवापिराजन्! प्रपितामहःपुरा दान्तः
प्रशान्तःखलु शन्तनुर्नृपः।
 स्नात्वा तुलामासि मरुद्धाया
मुदावहद्यो जनगंधिकन्यकाम्।
 युधिष्टिरःपुरा भीष्मपिता गङ्गाभार्योभूदितिशुश्रुम।
 कथं विद्वन् तुलामासि स्नात्वा मे प्रपितामहः।
 उद्वाह्यरेमे तां दूरं गंधिनी कन्यकां वद।
 नारदःचरन् कदाचिन् मृगयारतोवने
कळिंदजा रोधसि कांचनप्रभाम्।
 उद्वाह्यरेमे बहुदूरगन्धिनी
कन्यामहौ ते प्रपितामहश्शृणु।
 तद्गन्धाकृष्टचेतास्तु तां कन्यां वीक्ष्यसोब्रवीत्।
 वत्से किमर्थं वससि कस्तवाधरपानकृत्।
 साबालिकावाचमुवाच पार्थिवम्
त्वहं त्वनूडा पितृवाक्यकारिणी।
 सुतास्मिकैवर्तपतेस्तुनाविका
तयोक्तमित्थं वचनं निशम्यसः।
5 आलिंगनालंकरणेन बालिके
स्तनांकुराभ्याममरं कुरुष्व माम्।
 नाहं स्वतन्त्रेति वचोब्रवीच्च सा
गत्वा नृपो दाशपतेरथांतिकम्।
 मे दीयतां ते तनयेतिपृष्टवान्
कृत्वांजलिं प्राह नृपं स हृष्टधीः।
 ते दीयते मे तनया नृपासनं
ह्यस्यास्सुतस्यैवच दीयते यदि।
 शृत्वा स तद्वाक्यमतीवनिन्दितं
विचिंत्यवंशस्य च हीनतां तथा।
 _ विसृज्य गङ्गातनयं मम प्रियं
दौहित्रकोस्यार्हति किं नृपासनम्।
 इत्थं स्मरन् प्राप्य पुरं स शन्तनुर्न
भोज्यभोगे न च राज्यपालने।
 न नित्यकर्माण्यपि साधुपूजने मनोऽकरोन्मन्मथ संज्वरार्धितः।
 एकदा स तु मध्याह्ने ह्यागतं धौम्यभूसुरम्।
 नाहयामासपूजाहँ सोपिज्ञात्वात्मतेजसा।
 10 शन्तनुं वीक्ष्य राजानं बभाषेथस्मयन्निव।
 किमर्थं नावदस्त्वंमामागतं क्षितिपालक।
 दृष्ट्वापि तूष्णीं जडवद्वर्तसे मामपूजयन्।
 बोधितोपि ततस्तेन केवलं प्रणनामसः।
 दौम्यस्तु कृपयामूर्ध्नि प्रोक्षयामासचाम्बुभिः।
 प्रोक्षितस्तं निरीक्ष्याथ वेगादुत्थाय सांजलिः।
 निपत्यदण्डवद्भूमौ विधिना तमपूजयत्।
 कृतार्थोस्मि मुने बृह्मन् तव संदर्शनादहम्।
 इत्युक्तवन्तं सद्धर्मबोधनेच्छुरथाब्रवीत्।
 ज्ञानदृष्ट्यात्वहंजाने सर्वं तद्धृदयंगतम्।
 15 त्वं दाशकन्याविरहेणदुःखितस्तत्प्राप्त्युपायं भवतोब्रवीम्यहम्।
 स्नात्वा तुलामासि मरुद्धाजले स्तुत्वा च रङ्गेशमनन्तशायिनम्।
 स्तोत्रेणचानेन समाप्नुहीप्सितं हरेश्चतुर्द्वद्वशताग्र्यनामभिः।
 
मरुद्धायास्सदृशी नचापगा ननामरत्नस्तवसन्निभस्तवः।
 कुरुष्वशुद्धाचमनं नृपस्तवं तवोपदेक्ष्ये सकलेप्सितप्रदम्।
 तथैव कृत्वोपदिशेतिसोब्रवीद्दौम्योपिराज्ञेहुपदिष्टवान्
गुरुः।
 अस्यश्रीरङ्गराजाऽष्टोत्तरशतनामरत्नस्तोत्र
महामन्त्रस्य।

वेदव्यासो भगवान् ऋषीः- अनुष्टुप्छंदः।
 भगवान् श्री महाविष्णुर्देवता।
 श्रीरङ्गशायेतिबीजम्।
 श्रीकान्त इति शक्तिः।
 श्रीप्रद इति कीलकम्।
 मम समस्तपापक्षयार्थे रङ्गराज प्रसाद सिद्ध्यर्थे विनियोगः।
 दौम्यःश्रीरङ्गशायीश्रीकान्तश्श्रीपदाश्रितवत्सलः।
 अनन्तोमाधवोजेतो जगन्नाथो जगद्गुरुः।
 सुरवर्यस्सुराराध्यः सुरराजानुजःप्रभुः।
 हरिहतारिर्विश्वेशश्शाश्वतश्शंभुरव्ययः।
 20 भक्तार्तिभंजनो वाग्मी नीरोविख्यातकीर्तिमान्।
 भास्करश्शास्त्रतत्वज्ञो दैत्यशान्ताऽमरेश्वरः।
 
नारायणो नरहरिर् नीरजाक्षो नरप्रियः।
 बृह्मण्योबृह्मकृद्ब्रह्मा बृह्माङ्गो बृह्मपूजितः।
 कृष्णःकृतज्ञो गोविन्दो हृषीकेशोऽघनाशनः।
 विष्णुर्जिष्णुर्जितारातिस्सज्जनः प्रिय ईश्वरः।
 त्रिविक्रमस्त्रिलोकेशस्त्रय्यर्थ स्त्रिगुणात्मकः।
 काकुत्स्थःकमलाकान्तः काळीयोरगमर्दनः।
 कालांबुदश्यामळांगः केशवःक्लेशनाशनः।
 केशिप्रभंजनःकान्तो नन्दसूनुररिदमः।
 25 रुक्मिणीवल्लभश्शौरिर्बलभद्रोबलानुजः।
 दामोदरोहृषीकेशो वामनोमधुसूदनः।
 पूतःपुण्यजनध्वंसी पुण्यश्लोकशिखामणिः।
 आदिमूर्तिर्दयामूर्तिश्शांतमूर्तिरमूर्तिमान्।
 परंब्रह्मापरंधामः पावनःपवनोविभुः।
 चन्द्रश्छंदोमयोरामस्संसारांबुधतारकः।
 आदितेयोऽच्युतोभानुः शंकरश्शिवऊर्जितः।
 महेश्वरोमहायोगी महाशक्तिर्महत्प्रियः।
 दुर्जनध्वंसकोशेष सज्जनोवास्तिसत्फलं।
 पक्षींद्रवाहनोक्षोभ्यः क्षीराब्धिशयनोविधुः।
 30 जनार्दनोजगहेतुर्जितमन्मथविग्रहः।
 चक्रपाणिश्शंखधारी शाीखड्गीगदाधरः।
 एवं विष्णोश्शतंनाम्नामष्टोत्तरमिहेरितम्।
 स्तोत्राणामुत्तमंगुह्यं नामरत्नस्तवाभिधम्।
 
सर्वदासर्वरोगघ्नं चिन्तितार्थफलप्रदम्।
 त्वं तु शीघ्रं महाराज! गच्छ रङ्गस्थलंशुभम्।
 स्नात्वा तुलार्केकावेर्यां माहात्म्यश्रावणं कुरु।
 गवाश्ववस्त्रधान्यान्नभूमि कन्याप्रदोभव।
 द्वादश्यां पायसान्नेन सहस्रं दशभोजय।
 नामरत्नस्तवाख्येनविष्णोस्साष्टशतेनच।
 35 स्तुत्वा श्रीरङ्गनाथंत्वमभीष्ट फलमाप्नुहि।
 नारदःधर्म समुपदिश्येत्थं दौम्यस्स्वाश्रयमाविशत्।
 राजन्! राज्यधुरांन्यस्य सामात्येजाह्नवीसुते।
 प्रतस्तेवाद्यघोषेण दिशोदशविनादयन्।
 सहस्रं हस्तिवर्याणां हयानाम तत्रयम्।
 रथानां च सहस्रद्वे पत्तीनां दशकोटयः।
 तमन्वयुर्महाभागमेकच्छत्रं युधिष्ठिर।
 यथाविधि ततस्स्नात्वा कावेर्यां रङ्गसन्निधौ।
 बहुस्वं विप्रसात्कृत्य रङ्गनाथं प्रणम्यच।
 40 स्तुत्वाचस्तोत्रवर्येण कारयित्वाच मण्डपम्।
 पुनःप्रतस्थेशुद्धात्मा स्वपुराय ससैनिकः।
 माहिष्मतीपुरोद्देशे स्थापयित्वा स्वसैनिकान्।
 शृत्वाम्लेच्छावृतंदेश मरुधन्म्लेच्छपत्तनम्।
 द्वात्रिंशद्युद्धकाङ्खिण्यो ह्यक्षोहिण्यो विनिर्गताः।
 गान्धर्वास्त्रेणसंहृत्य समरेभिमुखान् रिपून्।
 
तुलास्नानप्रभावेन जयलक्ष्मीयुतोनृपः।
 स हस्तिनपुरंप्राप्य सदायोजनगंधिनीम्।
 स्मरन् दाशसुतामेव कृशोभूत् स्मारपीडया।
 गाङ्गेयशृतवृत्तांतः पितरंवीक्ष्य दुःखितम्॥45 कैवर्तेशपुरंप्राप्य तं व्याहरदिदंवचः।
 मत्पित्रेदीयतांकन्या सकृतांजलिरब्रवीत्।
 अहंजघन्यजातीयो दौहित्रश्च तथा भवेत्।
 आसनं मत्सुतासूनोर्यदिस्याद्दीयतेसुता।
 देवव्रतश्च तं प्राह नेच्छांयेव नृपासनम्।
 मत्पित्रे दीयतां पुत्री पृच्छते दैवयोगतः।
 त्वदात्मजोभवेद्भूपो मद्दौहित्रक्षयं व्रजेत्।
 तस्मान्नदीयतेपुत्री त्विति दाशोब्रवीत्तदा।
 निशम्यतद्वदश्श्रीमान् पितृभक्त उदारधीः।
 व्रतीस्यामहमाजीव मित्यघोषमदुच्छकैः।
 50 अपुत्रस्यचमेलोके पितृशुश्रूषया भवेत्।
 तच्छृत्वावचनंदेवा गंभीरं रोमहर्षणम्।
 भीष्मभीष्मेत्यभाषत पुष्पवृष्टि महर्षयन्।
 तदारभ्य च भीष्मेति तं वदन्त्यखिलाजनाः।
 राजायोजन गन्धांतामुद्वाह्य विधिपूर्वकम्।
 रेमेतयाचिरंकालं यं दृष्ट्वा मुमुहुस्सुराः।
 पितृ शुश्रूषयावत्स सर्वेलोकास्त्वयाजिताः।
 स्वच्छंददेहविन्यासो भूयामृत्युभयं न ते।
 
इति दत्वा वरं राजा पाणिनाऽमार्जयत्सुतम्।
 अथयोजनगन्धापि भर्तारं प्राप्य शन्तनुम्।
 कालेद्वौसुषवेपुत्रौ चित्राङ्गं चित्रावीर्यकम्।
 55 नारदःएतत्तेसर्वमाख्यातं यन्मां त्वं परिपृष्टवान्।
 स्नात्वा तुलार्के कावेर्यां यःपठेच्छृणुयादपि।
 पुत्रार्थीलभतेपुत्रं धनार्थी लभते धनम्।
 द्वात्रिंशदब्दं योऽनग्निस्स्नानमज्जनवर्जितः।
 चतुर्विंशति लक्षांक गायत्रीजपमाचरेत्।
 अतः परम् न जानामि प्रायश्चित्तं विधिनृप।
 यःकावेर्यां सकृत्स्नायात्तुलासंस्थे दिवाकरे।
 हरिस्मृतिफलम् प्राप्य हरिभक्तो भवेद्धवम्।
 मध्यंदिने दृडाङ्गोय स्स्नानं त्यक्त्वाऽर्चयेद्धरिम्।
 वैश्वदेवं च यःकुर्यात्सगुल्मव्याधिमान् भवेत्।
 60 अशक्तःपूजनेविष्णोर् बिभृयात्तुलसीदळम्।
 आषाडशुक्लद्वादश्या मभिषिच्यसकृद्धरिम्।
 गोक्षीरेणतुलास्यापि पूजयेत्पुरुषोत्तमम्।
 नभस्य शुद्धद्वादश्यां गोक्षीरैरभिषिंचति।
 एकाययस्तुलस्य्यावा दरिद्रोधनदो भवेत्।
 अमायां च तथाकुर्याद् गयाश्राद्धं कृतं भवेत्।
 मदुत्थाने मच्छयने मदंगपरिवर्तने।
 शाकाहारं प्रकुर्वाणो हृदिशल्यं ममार्पयेत्।
 
अवालुकागतं तोयं सर्वं पर्युषितम् स्मृतम्।
 पुष्पम् पर्युषितम् वय॑म् नवज्यं तुळसीदळम्।
 आर्द्रम् तु तुळसीपत्रं कोमळं प्रोक्तमुत्तमम्।
 खण्डिनीषिणीचुल्ली उदकुम्भ उपस्करः।
 पंचसूनाघशान्त्यर्थं पञ्चयङां समाचरेत्।
 
आहिताग्निस्स्वगृह्यग्नौ तदभावेतु लौकिके।
 सव्यंजनैर्हविश्यैश्च जुहुयात्सघृतौदनैः।
 तदभावेऽम्बसादध्ना पयसावापि हावयेत्।
 हस्तेनान्नाहुतिंकुर्याजलादींस्तु सुवेण हि।
 सर्वाभावे जपेन्मन्त्रं येनकेनापिहावयेत्।
 उत्तानीकृत्यहस्तं तु जुहुयाद्देवतीर्थतः।
 स्वाहाकारस्सुराणांस्यात् स्वधाकारश्च पैतृकः।
 हंतकारोमनुष्याणामितिवेदविदोविदुः।
 70 अन्नमामलकाकारं न्यसेद्भूमै कराग्रतः।
 यथोपजोषं व्याहृत्या परिषेचनमाचरेत्।
 आद्यत्रयं च प्रत्येकं द्वयमेकं ततः परम्।
 ततश्चतुर्णामेकं तु त्रयम् प्रत्येकमेकशः।
 अथोदशानामेकं तु परिषेचनमिष्यते।
 भूमौ तु प्राङ्मुखीमौनी कुक्कुटाण्डप्रमाणतः।
 द्विजोभूतबलिंचैव प्रक्षिपेच्च विधानतः।
 कालद्वये दिवारात्रौ बलिकर्मसमाचरेत्।
 अनावृतेशुचौदेशे वायसश्वबलिं हरेत्।
 
रिक्तो बिक्षात्रयं दद्यादुपपन्नोयथाबलम्।
 75 अघृतं कीटकेशादि दुष्टं भुक्त्वा पतेन्नरः।
 स्त्रीपुत्रपात्रेह्यशुचौ बालेनसहभार्यया।
 हस्तदत्तं चान्यशेषं स्त्रीशूद्रश्राद्धशेषितम्।
 अनुष्टान्नं च येभुङ्क्ते व्रतश्चान्द्रायणंचरेत्।
 नारीप्रथमगर्भान्न मनाचारद्विजान्नकम्।
 अस्नातनारीपक्वान्नमनग्न्यन्नं विटान्नकम्।
 विधवान्नं च शूद्रान्नं श्वमांससदृशं भवेत्।
 वामपार्श्वमदन्नं तु यदन्नं पतितैस्सह।
 संध्ययोरुभयोर्भुक्तं श्वमांससदृशं भवेत्।
 भैक्षमन्नं गृहस्तस्य वर्णिनोभैक्षमेवच।
 80 आत्मार्थेपक्वमन्नम् च श्वमांससदृशं भवेत्।
 वटार्काश्वत्थपत्रेषुनीचानीतेषु चौर्यतः।
 कदळीगर्भपत्राचाप्ययःकांस्योत्थपात्रके।
 नूनेविंशत्पलेभ्यश्च तथैवोद्धृतपात्रके।
 शूद्रचण्डाळनिकटे श्वमांससदृशं भवेत्।
 चण्डालादरवं शृत्वा सद्यश्चांद्रायणं चरेत्।
 अन्नं न लंघयेत्सव्यंपार्श्वमारभ्य सेचयेत्।
 सालग्रामजलं पीत्वाऽमृतोपस्तरणंत्विति।
 काष्ठमौनीस्पृष्टपात्रः कुर्यात्प्राणाग्निहोत्रकम्।
 हृद्गतेतूदकेश्नीयात् पिण्डान्धात्रीप्रमाणतः।
 संसृष्टान्नाहुतिर्यस्तु सद्यशण्डालतांव्रजेत्।
 85
नक्षारलवणैर्युक्तं हुनेदनभिघारितम्।
 पंचप्राणाहुतीर्धेतैर्न कदाचिदुपस्पृशेत्।
 तुलामासेतु कार्तिक्यां येभुङ्क्तेब्रह्मपत्रके।
 अश्वमेधफलंप्राप्य व्रतसाफल्यमश्नुते।
 आस्येचार्धंकरेचाधू योभुलैककबळं नरः।
 वान्ताशीत्युच्यतेलोके वान्ताशीशुनको भवेत्।
 अयःपात्रस्थितं तोयं सप्तमीभानुवासरे।
 धात्रीफलं तिलंचापि रात्रौदधि दिवापयः।
 करेणमधितं तक्रं सुरयासममुच्यते।
 सर्षपं तु सुरातुल्यं रात्रिकालेत्वनापदि।
 90 अष्टम्यां नाळिकेरंच द्वादश्यां तु वटोलिका।
 बृहतीचद्वितीयायां कूश्माण्डं प्रतिपद्यपि।
 अळर्कंचैवपंचम्यां सुरापानं समीरितम्।
 भांशयित्वा यानारी पाकम्कुर्यादनाप्लुता।
 भुक्त्वातदन्नं राजेन्द्र! सद्यश्चण्डालतां व्रजेत्।
 यदन्नं नार्पितं विष्णोः तत्सर्वं सुरयासमम्।
 नित्यनैमित्तिकब्रष्टः केवलंयस्तु काम्यकृत्।
 ब्रह्महन्ता स विज्ञेयस्तस्य ब्रह्महणोव्रतम्।
 व्रतं समाप्ययःकाले नार्चयेद्धर्मबोधकम्।
 यथाबलं च तस्यान्नं योभुकै तस्य रौरवः।
 95 किंचिन्यूनं द्विजोभुक्त्वा भोजनान्ते द्रवौदनम्।
 पिबेत्सलवणं तक्रमातृप्तेस्वस्तिमान् भवेत्।
 
पुरातनं च निस्सारं तप्ततक्रं च वर्जयेत्।
 शुण्ठीजंबीरलवण कृष्णनिम्बदकैर्युतम्।
 तक्रं पर्णपुटेपीत्वा सर्वदारोग्यमश्नुते।
 अतीर्थभोजयित्वैव बालवृद्धातुरानपि।
 स्वयं भुवाद्विराचम्य इन्द्रस्येति मनुं जपेत्।
 आहारपरिणामार्थमगस्त्यं कुम्भकर्णकम्।
 बकं च बडबाग्निंच स्मरेद्भीमं च पंचमम्।
 दातापिर्भक्षितोयेन येनपीतोमहोदधिः।
 100 यन्मयाखादितंसर्वं तदगस्त्यो जरिष्यति।
 एवंभुवाहरेःपादतुळसी भक्षयेच्छुचिः।
 अन्नदोषादिपापौघैर्मुच्यते तुळसीमदन्।
 पठेद्वाशृणुयाद्वापि पुराणं धर्मसंहितम्।
 शैवं वा वैष्णवंवापि श्रोतव्यमनसूयया।
 शिव एव हरिस्साक्षाद्धरिरेव शिवस्स्वयम्।
 तयोस्तु भेदं कुर्वाणः पितृभिन्नरकंव्रजेत्।
 विद्वांसोपि स्वयं सर्वेष्येकं वै धर्म वाचकम्।
 श्रवणार्थं कल्पयेयुर् बहुमानपुरस्सरम्।
 105 आदौ कृत्वातु नैवेद्यं नमस्कृत्यच सत्सभाम्।
 जघन्यासनसंस्थेन श्रोतव्यमनसूयया।
 धर्मवक्तास्वयं शेषो धर्मवक्तास्वयं हरिः।
 तस्मात्संपूजयन् धर्मवक्तारं शृणुयात्कथाम्।
 सत्कथावाचकं यत्र नार्चयन् यदिनिंदति।
 
तत्राग्निदाहोजायेत दुर्मृतिश्चोरभीस्सदा।
 कावेरीकपिलासंगे श्राद्ध पिण्डंगयास्थले।
 रंगधामप्रशंसन्ति पितृणां मुक्तये तथा।
 सर्वतीर्थेषुयस्स्नाया त्तत्स्नानं त्विहलभ्यते।
 कावेर्यां तु तुलामासे सर्वतीर्थाश्रिते जले।
 कावेर्यां स्नानपुण्येन रामायणकथाशृतिः।
 समस्तयज्ञतीर्थाणि पुराणान्तरसत्कथाः।
 रामायणैकश्लोकस्य कलांनाहन्ति षोडशीम्।
 विशेषेणयथाराम चरितां श्रूयतेशुभम् देवर्षिपितरोदृश्या शृणुयुर्वंशवृद्धिदाः।
 नृत्यन्ति पितरःतृप्ताः कृतार्थास्म वयं बत।
 मद्वंश्योरामचरितं पठेदस्माद्विमुक्तये।
 अशुद्धोवाप्यनाचारो मनसापापमाचरन्।
 श्लोकं रामायणं शृत्वा सर्वपापैः प्रमुच्यते।
 हरि विनाकस्यशक्तिः पुराण कथने भवेत्।
 115 अतोसौ नावमन्तव्यःपरिहासेपि धर्मवित्।
 अत्रैवोदाहरंतीममितिहासं पुरातनम्।
 शृण्वतां सर्वपापघ्नं सर्वमङ्गळदायकम्।
 माहिष्मातीपुरोद्देशे ह्यग्रहारोमहानभूत्।
 सुधार्मिकस्सुभिक्षश्च धनदान्यालयोभुवि।
 अग्रभूमिरितिख्यातो विद्वज्जनविभूषितः।
 सहस्रसंख्यैर्विप्रेन्द्र स्सदाचारैस्समाश्रितः।
 
तस्मिन्नासीद्विजवरो वेदवेदान्तपारगः।
 शिवमूर्तिरितिख्यातो धनार्जनपरोभवत्।
 श्रावयसमासविदुषश्श्रीरामचरितंमहत्।
 120 नित्यं दिव्यं तुते शृत्वा पूर्वमेव समाहिताः।
 प्रतिकाण्डसमाप्तौ च निष्कान् दद्युश्शतं शतम्।
 प्रथमारंभसमये दद्युनिष्कसहस्रकम्।
 एवं संभाषयामास बहु मान पुरस्सरम्।
 प्रत्यहं स्वर्णपुष्पैश्च दशभिस्सुरसैःफलैः।
 रम्भादिनाळिकेराद्यैर् धूपदीपाक्षतादिभिः।
 शष्कुल्यापूपातिरस मोदकाचैस्सपानकैः।
 नैवेद्यैर्बहुभिर्नत्वा पुस्तकेराममच्युतम्।
 उपचारैष्योडशभिस्समभ्यर्च्य समाहिताः।
 रामायणमशृण्वंते व्यर्थवाक्यविवर्जिताः।
 125 अभिषेकोत्सवेपुण्ये श्रीरामस्यमहात्मनः।
 वस्त्रैर्निष्कसहस्रैश्च भूषणायैस्समर्चयन्।
 सामुदायिकनिष्काद्यैस्तृप्तैःपौराणिकोस्त्विति।
 श्रोतारोनपृथक्पूजा मकुर्वन् लोपबुद्धयः।
 पुराणवक्ता तान्विप्रानपृच्छत्क्षुभितेंद्रियः।
 किं मे न दक्षिणादत्ता भवद्भिश्श्रोतृभिःकथाम्।
 तच्छ्रुत्वा वचनं तस्य प्राहुरस्माभिरग्रज।
 येनकेनाप्युपायेन भवतः पूजनं कृतम्।
 यथाशक्त्यग्रहारस्थैराशापरिहता तव।
 
त्रिसहस्रं तु निष्काणां प्राप्तं विप्रेन्द्रकिं पुनः।
 130 अत्याशात्मविनाशाय धर्मशास्त्रे च निन्दिता।
 अर्थलुब्धोविकर्मीस्स्या दर्थलुब्धोपमृत्युमान्।
 अर्थलुब्धोदरिद्रस्स्या दर्थलुब्धोह्यधःपतेत्।
 इति वाक्यं सावमानं शृत्वा पौराणिकोधमः।
 भवंत एवलुब्धास्स्युर्यत्पापैर्दूषितोस्म्यहम्।
 इति धिकृत्यतांत्सर्वान् न्निरगात्तक्षणादसौ।
 असौदुराग्रहीयातु किमस्माभिर्न पूजितः।
 इत्यनाहूय तं विप्रं प्रतिगेहमभुंजत।
 सच पौराणिकोदुःखी आमध्याह्न बहिर्वसन्।
 क्षुधातुरश्शपन्कोपात् सकुटुंबेन विनिर्गतः।
 135 तषणेन ते सर्वे नाशितब्रह्मवर्चसः।
 किंचित्कालान्तरेरिक्ता श्चोरक्षामादिपीडिताः।
 श्रीरामगाधाश्रवण प्रभावाद्दुः
खैर्विहीनानरकोद्भवैश्च।
 पौराणिकीस्स्याप्यवमानपापात्तेवै
बकीये बकदैत्यभिक्षाः।

राजा
रामायणस्य श्रवणात् पापिष्ठोपि दिवं व्रजेत्।
 इति संतः प्रशंसन्ति कथमेषां न सद्गतिः।
 नारदःसत्यं पापोपियात्येव सत्कथश्श्रवणाद्दिवम्।
 
धर्मवादीगुरुइँयो नोर्ध्वलोकोगुरुद्रुहम्।
 पुराणवक्ता साक्षात्तु शेष एव न संशयः।
 गुरुर्वा मन्त्रदाता वा तं निन्देन्न कदाचन।
 130 आचार्यो जनको मन्त्रदाता धर्मप्रबोधकः।
 विद्याप्रदोऽन्नदाता च षडैते पितरस्स्मृताः।
 नाकारी निन्दकैर्विप्रैस्सत्क्रिया तं द्विजं प्रति।
 तेनावमानदोषेण नस्वर्गोनाप्यदोगतिः।
 सत्कथाश्श्रवणाज्जाता तत्पापा(zता हि ते।
 बकासुरेण ते सर्वे बाधितादुर्मृतिं गताः।
 नोर्ध्वगाःप्रेतभूताः ते बभूवुस्सगरात्मजाः।
 पूर्वदोषानुसारेण दुर्मृताःकपिलौजसा।
 दग्धांप्रेताःपुनर्गङ्गा जलैः पूता दिवं ययुः।
 शिवमूर्तिस्ततःप्राप्य राजद्वारं दुराग्रही।
 135 ब्रह्मद्रोहेणपापेन क्षीणतेजः पुराणवित्।
 मृतोसुरोबकाख्योभूत्पुण्यशेषाद्दयान्वितः।
 एकचक्रपुरीसंस्थान् पूर्वदोषानुसारतः।
 बकासुरोबाधमानो भीमसेन हतोयुधि।
 तस्मिन् वने मृतस्स्वर्गं प्राप्तश्चाप्सरसांपतिः।
 क्रीडित्वा नन्दनवने रम्भाधररसं पिबन्।
 आकल्पं पुण्यशेषेन परीक्षिदिहसोभवत्।
 सार्वभौमो महायोगी श्रीमान् भागवतोत्तमः।
 नारदः-पर्वणोयश्चतुर्थांश आद्यःप्रतिपदस्त्रयः।
 
योगकालस्सविज्ञेयःश्श्रौतेस्मार्तेच कर्मणि।
 140 मध्याह्नेयदिसंधिस्स्या पर्वप्रतिपदोस्तदा।
 तदहार्यग इष्येत परतश्चेत्परेहनि।
 अपराह्ने यथासंधिः प्रतिपत्पर्वणोर्भवेत्।
 प्रातःपरेद्युःयागस्स्याच्छतुर्थेपि नदुष्यति।
 तेन सर्पिष्मतान्नेन ब्राह्मणान् भोजयेत्ततः।
 नोचेद्धोमो भवे द्व्यर्थो भुलैर्गुल्मीयदिस्स्वयम्।
 अकाल्याद्धोमकालस्स्या दापर्वं पार्वणं तथा।
 आहुतिर्नविहन्येत मध्येदैवीशृतेर्भवेत्।
 प्रतिकूलायथाभर्तु स्त्रीहता विधवायथा।
 निरग्निकं हतम् कर्म तथाऽनर्चितभूसुरम्।
 ।
 145 विप्रार्चनाद्धरिस्तुष्टस्तपश्रेयो विधास्यति।
 अहं स्नातुं गमिष्यामि कावेर्यां तौलिभास्करे।
 सुखीभव महाराज ! कृतकृत्योसि सर्वदा।
 अगस्त्यःइति शृत्वाऽखिलान् धर्मान् वैष्णवान्नारदान् नृपः।
 स्वर्णपुष्पैराभरणैर्दिव्यगन्धांबरैस्तथा।
 दिव्यमाल्यैरनेकैश्च पूजयामास नारदम्।
 दिव्याङ्गुलीयकेयूर कुण्डलादिविभूषणैः।
 अर्चयंत्स महाभाग मानंदाश्रुपरिप्लुतः।
 चतुःप्रदक्षिणं कृत्वा प्रणम्यच पुनःपुनः।
 तत्पार्श्वे शिष्यवत्तास्तौ वेपमानः क्तिाञ्जलिः।
 
पुनर्विश्लेषभीत्या च प्ररुरोदमहामनसः।
 150 राजन् सज्जन विश्लेषादितरदुःखमस्ति किम्।
 नारदोपि नृपेभक्तिं वीक्ष्य विस्मितमानसः।
 उवाच प्रहसन् मन्दमाशीर्वादपुरस्सरम्।
 भवान् कृतार्थोराजेंद्र पाण्डवेयमहामते।
 यत्कृष्णोऽचंचलाभक्तिर्योगिनामपि दुर्लभा।
 इति राजानमामन्त्र्य ऋषिर्गन्तुं प्रचक्रमे।
 तावद्भीमादयस्सर्वे ब्रातरो द्रौपदी सती तत्पादाजं नमस्कृत्य स्तुत्वा प्रांजलयोब्रुवन्।
 
इति शंतनुराजवृत्तान्तः नाम षोडशोऽध्यायः
***************************
अथ सप्तदशोऽध्यायः नारदेनद्रौपदीप्रति मुक्त्युपायकथनम् (स्त्रीधर्माः) द्रौपदीकथं मे भगवन् योगिन् भक्तिस्स्याद्वद केशवे।
 केनोपायेन संसार नरकं संताराम्यहम्।
 अस्माकं कृपयायोगिन् तत्त्वं वेदांतनिश्चितम्।
 वक्तुमर्हस्यशेषेण मुनिवर्यनमोस्तुते।
 नारदःस्वधर्माचरणेनैव स्वस्मिन् भक्तिंप्रयच्छति।
 विष्णुभक्त्यैवसंसारसागरंसंतरेत्सुधीः।
 पतिशुश्रूषणं नार्यास्सर्वधर्मः प्रोच्यतेबुधैः।
 
देवतांतरसेवाच देहसंशोषणं तथा।
 स्त्रीणांभर्तृमतीनां तु न कर्तव्यं कदाचन।
 जीवेभर्तरि या नारी उपोष्यव्रतमाचरेत्।
 5 आयुष्यं हरतेभर्तुस्सानारी नरकं व्रजेत्।
 वर्णिनोगुरुशुश्रूषा पुत्रस्यपितृपूजनम्।
 विष्णुःप्रीणात्यनेनैव स्त्रीणां वत्से न कर्मणा।
 भुलभुक्ते भवे वत्से सुप्तेसुप्ते ततःप्रिये।
 आहूतागच्छननैव पृष्टापदसदुस्तरम्।
 क्रुद्धेमंदस्मितैवस्या यदीच्छसिहरेःपदम्।
 नब्रूयाविप्रियं भर्तुः क्रीडायां वा कदाचन।
 दिव्यालंकारगंधाड्या रथ्यर्थं पतिमाश्रय।
 अग्निकार्येच दैवेच गुरूनामभिवादने।
 गृहकार्येच सर्वत्र जागरूकासदा भव।
 10 अष्टाक्षरपरानित्यं सदाचारपरायणा॥ नामत्रयपराचस्याभवचांतर्बहिश्शुचिः।
 सन्यास्यतिथिसाधूनां सत्कारं कुरु सर्वदा।
 ऋतुस्नाततु यानारी सन्निधिंनोपगच्छति।
 घोरायांबूनहत्यायां युज्यतेनात्रसंशयः।
 यावीजयेच्छसैश्रान्तं भर्तारं भक्तिपूर्वकम्।
 वीज्यमानाप्सरोभिस्सास्वर्गलोके महीयते।
 श्रांतस्याभिमुखेपात्रे पत्युःपानीयमादिशेत्।
 15 स्वर्गमार्गेऽऋतं पीत्वा सुखेन त्रिदिवं व्रजेत्।
 
एकवारं च याभर्तुः पादसंवाहनं चरेत्।
 सा विराजद्विमानस्था सदा स्वर्गेमहीयते।
 भर्तारमेव सेवित्वास्वःप्राप्ता इति शुश्रुम।
 न सत्यं योषितां शौचं नचाचारो नपुण्यधीः।
 असत्याःकेवलं स्वार्था स्त्वादृश्यो न तथा स्त्रियः।
 अत्राद्भुतकथां वच्मि शृणुष्वावहिताशुभा।
 यांशृत्वा मूढबुद्धीनां धर्मे बुद्धिःप्रजायते।
 मुंभोजंतुःपुराकश्चित् दृष्ट्वा काञ्चनसुप्रभाम्।
 पिपीलिकांस्मराविष्टः प्राहरूपवतीं वचः।
 20 कामयेत्वं महाभागे न कस्यापि परिग्रहः।
 स्मरामिनुगृह्णीष्व तव श्रेयोभविष्यति।
 पिपीलिकाकस्त्वं विश्वासनीयो वा पालयिष्यसि मां यदि।
 तथैवास्मि न संदेहो न कस्यास्मि परिग्रहः।
 प्रश्नेनैवसमाख्यातं तव शीलं मयाशुभम्।
 युवतीचास्मकन्याच नान्यं चाभिलषाम्यहम्।
 इति तद्वाक्यमाकर्ण्य रेमेजुंभस्तया सह।
 जनयित्वा प्रजाबही स्तस्यामित्थमरीरमत्।
 Mभोथभार्यांसंस्थाप्य स्वगृहे भक्षणाय सः।
 गतवान् बंधुभिर्दूरं तमदृष्ट्वाशुशोचसा।
 25 एतस्मिन्नंतरेकश्चित् तिर्यगेकांतसंस्थिताम्।
 दृष्ट्वा पिपीलिकांहृष्टः प्रसह्य च वचोब्रवीत्।
 
त्वया संगममिच्छामि बहुकालेदुपस्थितम्।
 रूपमेतादृशं दृष्ट्वा को न मुह्यतिते शुभे।
 तस्मात्त्वं न जहांयद्य त्वद्भर्ता किंकरिष्यति।
 नांगीकरोषि यदि मां त्वामद्यैव तुदाम्यहम्।
 पिपीलिका तस्यवचश्शृत्वोपायैरसांत्वयत्।
 दोषंनिश्चित्य मनसा प्राहसाद्वीनयान्वितम्।
 भवान्प्राज्ञः कथं मोहात्परानारीमपेक्षसे।
 परस्त्रीगमनेसौख्यं क्षणिकं नरकंप्रदम्।
 30 ब्रह्महत्यादिपापानां प्रायश्चित्तंविधीयते।
 परस्त्रीसंगपापस्य न कदा च ननिष्कृतिः।
 योगच्छेत्परनारीतु सकृन्मोहान्नाराधमः।
 सहस्रकृत्वोगायत्री जप्त्वावर्षत्रयं शुचिः।
 योगच्छेत्परनारी तु दुष्टात्माकुलघातकः।
 तेन सद्यः कुलंनश्येत् तुषाग्निनरकंव्रजेत्।
 पतिनिंदतियोनारी कुलटावासकृद्भवेत्।
 स्वर्गस्थेपितृभिस्स्वेस्या नरकानेवगच्छति।
 निष्कृतिस्सर्वपापानां त्रयाणां तु न निष्कृतिः।
 परस्त्रीसंगदोषस्य ऋणस्याप्यनृतस्यच।
 तस्मात्सर्वात्मनाविद्वान् परस्त्रीसंगमं त्यजेत्।
 यदि संगच्छतेमोहाद्रित्क्तोल्पायुर्भवेद्धृवम्।
 तस्मै पतिव्रताभावं न विहंतुम् त्वमर्हसि।
 व्यभिचारेण ते वंशो विनश्यति ममापि भो।
 
इत्युक्तोपि हठात्कारान्मैथुनायोपचक्रमे।
 सापि हुंकृत्यवेगेन पातिव्रत्याग्निनाधतम्।
 शप्त्वाभस्मीभवेत्युच्चै रक्षरक्षदवेत्यगात्।
 निशम्यक्रोशमेतस्यात् जंतुःप्राप जवाद्गृहम्।
 वेपंतीरिदतीदृष्ट्वा किमेतदितिपृष्टवान्।
 तस्मै सर्वं यथापूर्वं सापिवृत्तमवर्णयत्।
 40 तच्छ्रुत्वा विस्मितो भर्ता कंपयित्वा शिरोमुहुः।
 श्लाघयामास भार्यां स्वां पातिव्रत्यान्मुदान्वितः।
 उवाच च तथा भार्यां वेपमानांपतिव्रताम्।
 साध्वीलोकत्वमेवार्ये युवतीरूपवत्यपि।
 असन्निधौ ममत्यक्तो यस्मादन्यपुमांस्त्वया।
 कुलद्वयं समुद्वीक्ष्य त्वया शीलं च रक्षितम्।
 अहमेतादृशस्त्रीकः कृतार्थोस्मि न संशयः।
 इति जंतुद्वयोत्पन्नं शृत्वा वृत्तान्तमद्भुतम्।
 जहास तत्र स्त्रीकः केकयेशोऽतिविस्मितः।
 दृष्ट्वा निष्कारणंहास मपृच्छद्धासकारणम्।
 45 क्रोधना तस्य भार्याथ दन्तान् कटकटाप्यच।
 किमर्थमत्रदेशेपि निर्जनेहासनं कृतम्।
 मय्येवपरिहासस्ते ब्रूहि तत्त्वमशेषतः।
 इति भार्यावचश्शृत्वा राजाभार्यामसांत्वयम्।
 उक्तंचेन्मरणं मे स्यात् तस्माद्यां न कस्यचित्।
 अथ क्रुद्धाच तद्भार्या पातिव्रत्यं निरस्य च।
 
अवश्यमेव वाद्यं स्याज्जीववामरवा शठ।
 नोचेदद्यैवगरळं पीत्वा देहं त्यजाम्यहम्।
 वृधा मां परिहासोक्त्या धिक्करोषिह धुर्मते।
 इति शृत्वा वचोघोरम् पुंश्चलीवाक्यसन्निभम्।
 50 दृष्ट्वा निर्दयचित्तांच निशस्वासाति विस्मितः।
 उवाचच पुनर्मंदम् धर्मबोधेनशिक्षयन्।
 भर्तृशुश्रूषणं भक्त्या नारीणां तप उच्यते।
 मयि जीवति ते पुण्यं घटते स्वर्गसाधनम्।
 करोति विप्रियं भर्तुर्यानारी सकृदेव हि।
 सा तप्ततैलं नरकं भुत्त्वाथ विधवा भवेत्इत्थं स्थिते वदस्वेति हा कथं त्वं ब्रवीषि माम्।
 तेन जानामि दौश्शील्यं दुर्बुद्धेःकुलपांसिनी।
 एतादृशीं त्वामसती कथं भार्येतिमन्महे।
 तस्मान्न विश्वसेत्प्राज्ञः स्त्रींसती वा कदाचन।
 55 धिक्तव्यामसति दुर्बुद्धे ध्वंसवागच्छवामर।
 त्वया न मेस्तिकार्यं हि शीघ्रमुद्भनास्मर।
 इत्युक्त्वा स्वां स्त्रियं मूढां केवलं स्वोदरंभरिम्।
 व्याधियुक्त इव स्नातो राजासंतोषमागतः।
 तस्मादीदृग्विधावत्से नारीणां सहजागुणाः।
 जन्मान्तरकृतैःपुण्यैर्भवेत्काचित्पतिव्रता।
 पातिव्रत्येनपुण्येन नारीवैकुण्ठमश्नुते।
 अन्यथासर्वतीर्थेषु स्नात्वापि नरकम् व्रजेत्।
 
त्वं पुरा पुण्ययोगेन स्त्रीजाता पाण्डुजन्मनाम्।
 अत्रापि पतिशुश्रूषाम् कृत्वास्वर्गम् च लप्स्यसे।
 60 अगस्त्यःइति शृत्वाऽखिलान् धर्मान् द्रौपदी पतिदेवता।
 मूर्धाप्रणाममकरोद् योगीश्वर पदद्वयम्।
 श्रीनारदःन श्रद्धध्यात्कदाचिच्च नास्यागोप्यम् वदेत्तथा।
 मत्स्यवच्चस्त्रियाग्रस्तो नश्येत्स्त्रीवशगो नरः।
 तस्मात्सर्वात्मनाप्राज्ञः पत्न्यां बाह्येन चक्षुषा।
 असक्तो भविताऽजस्रं सौमुख्यम् भविता तदा।
 यूयं च पञ्चभर्तारः पांचालीचाप्सरोधिका।
 वर्तध्वं तु कुरोर्वंशो नापहस्यो यथाभवेत्।
 इत्युक्ताव्रीळिताः पाण्डुपुत्रा धर्मात्माजादयः।
 शपथम् च मितश्चक्रुः पत्न्यार्थे नारदाग्रतः।
 65 एकैकवत्सरं पत्न्यां वसिष्यामो वयं मुने।
 ज्येष्ठःक्रमेण यःपश्येदेकाब्धं सचरेबहिः।
 सत्यमेवोक्तमस्माभि स्त्वत्पादाभ्यां शपामहे।
 स इत्थं सत्यसन्धानां पाण्डवानां महात्मनाम्।
 शृत्वा सत्यमयं वाक्यमिदमाह प्रसन्नधीः।
 महतामपवादश्चेद्भवतामेवदुर्यशः।
 साङ्कर्यमितरेषां स्यात्तदुक्तं बहुशो मया।
 स्वस्तिवोस्तु गमिष्यामि कालक्षेपोमहानभूत्।
 
भवद्भिस्सत्प्रसंगेन कर्मकालो न चिंतितः।
 इत्युक्त्वाशन्तमेनाङ्गं समालभ्य च पाणिना।
 ७0 मन्दहासस्मरन्नेव निर्जगामशनैश्शनैः।
 पाण्डवा धर्मपुत्राद्या साश्रुनेत्रो मुनीश्वरम्।
 अर्चित्वा स्वर्णपुष्पाद्यैर्बजन्तोनुपदं ययुः।
 दृष्ट्वा सुप्रजतोदुःखात् साश्रुनेत्रान् कृतांजस्लीन्।
 एतान् दयान्वितान् भूयो मुनिराह वचोऽर्थवत्।
 जितेंद्रिया जितक्रोधा जितकामाजितारयः।
 सर्वात्मनैकपत्न्यां तु धृढस्नेहा भविष्यथ।
 पुनःपुनरसावेव वक्तव्यार्थार्थसिद्धये।
 कुङ्कमपङ्ककळङ्कितदेहा तुङ्गपयोधरकंपित हारा।
 नूपुरहंस रणत्पदपद्मा कंसवशीकुरु
ते भुवि रामा।
75 यूयं सर्वात्मनावत्सा नित्यं सत्वगुणान्विताः।
 सत्यवाचो भविष्यध्वं वैष्णवास्सात्विकप्रियाः।
 इत्युक्त्वासौ निवत्यैर्नान् जगामाशु विहायसम्।
 पाण्डवादुःखसंतप्ताः कृच्छ्रात्प्राप्यस्वमालयम्।
 यथापूर्वं प्रतिज्ञातं सन्तुष्टाःपर्यपालयन्।
 तथैव कृष्णयासार्धं रामोधर्मात्मजस्सुखम्।
 प्रतिवर्ष तुलामासे कावेरी स्नानतत्परः।
 सत्पात्रदाननिरतस्सदा धर्मरतो भवत्।
 
इति नारदेनद्रौपदींप्रति मुक्त्युपायकथनम् नाम
सप्तदशोऽध्यायः।

***********************
अथ अष्टादशोऽध्यायः
अर्जुन ब्राह्मणयोस्संवादः अथ धर्मात्मजश्श्रीमान् द्रौपती पतिदेवताम्।
 रमयामास धर्मेण राज्यं च परिपालयन्।
 आर्तोथब्राह्मणःकश्चिद्राजद्वारेऽर्जुनाग्रतः।
 चुक्रोश रक्षरक्षेति बृशमुच्छैर्महीपते।
 शृत्वा तथार्जुनोदीन मारादार्तस्यभाषितम्।
 किमर्थं रुद्यतेबृह्मन्नित्यपृच्छत् सविह्वलः।
 ऐश्वर्य गर्वतोराज्ञो मदोन्मत्तस्य दोषतः।
 काश्चिद्गावोहृताबाला मदीयाः कांस्यदोहनाः।
 निस्संगेन मयाक्षान्तं सर्व तत्वविवेकिना।
 ईदृशं हि महदुःखं संसारेस्मिन्निति धृवम्।
 5 ततः कतिपयाहस्सु रसत्रावागत्य तस्कराः।
 धनं गो महिषी मुख्यं हृत्वा स्वच्छंदतोगताः।
 सर्वं मदीयमेवाद्य देयं राज्ञेतिकामिना।
 नोचेच्छपिष्ये राजानं कामीराजा कथं भवेत्।
 अगस्त्यःकिमर्थं कुप्यते बृह्मन्! सर्वं क्षन्तव्यमेव हि।
 द्विजेन लोकगुरुणा यद्बोधैरितरैस्सुखी।
 
किमर्थं न ह्रियेताद्य त्वदोषेणैव गोधनम्।
 शृत्वा तद्वचःप्राह कोपात्प्रस्फुरिताधरम्।
 10 प्रथमाश्रमारभ्य ब्रह्मचर्यव्रतं चरन्।
 वढ्यर्चकश्च भैक्षाशी शुचिर्गुरुकुलेवसन्।
 अधीत्य वेदांश्चतुरो दत्वा च गुरुदक्षिणाम्।
 पित्रोरकरवं नित्यं शुश्रूषां तदनुज्ञया।
 कन्यां कुलद्वये शुद्धां गौरी रूपवतीं शुभाम्।
 आहूय दत्तां पित्राथ सुविभूषणभूषिताम्।
 समानजाती सरीळां समुद्वाह्य मनःप्रियाम्।
 स्वधर्मःक्रियते सम्यगजस्राग्नि पुरस्सरम्।
 नास्नात्वा प्रातरश्नीयां नमहानिशि संध्ययोः।
 नपरान्नं निषिद्धान्नं शूद्रान्नं तु कुतो भवेत्।
 15 न शूद्रात्प्रतिगृह्णीयां न शूद्रस्सेवितो मया।
 तत्तत्कालेऽर्चितोवह्निः स्थालीपाकद्वयं कृतम्।
 नच षण्णवतीश्राद्ध विमुखोहं कदाचन।
 न मयाऽभोज्ययज्ञेना भोजयित्वा नचातिथीन्।
 दद्यां न प्रतिगृह्णीयां सत्यं ब्रूयां नचाऽनृतम्।
 नकदाच न पारुष्य वाक्यमुक्तं परंपरं प्रति।
 मया पर्युषितं नात्तं नास्तीत्युक्तं नयाचके।
 न कदाचिदिवासुप्तं न सायं नारुणोदये।
 न कदाचित्खलासक्तिर् न च सन्तोवमानिताः।
 न नास्तिकोऽहं नो मूल् नचादानादि तत्परः।
 20
माघस्नानं मयाकारी श्रुतं तन्माघवैभवम्।
 निषिद्धव्यञ्जनं नात्तं नानिवेद्यान्नमर्जुन।
 नात्तं कदाचिद्वालैश्चनचबालोऽवमानितः।
 न भुक्तमन्नं स्त्रीशेषं नकदाचित्स्त्रियासह।
 नैकादशीद्वयेभुक्तं नाप्रातः पारणादिने।
 दासीविटागायको वा न कदाचिन्नटोपि वा।
 मद्गृहं प्रापिताःप्रार्थ नापात्रे धनमर्पितम्।
 धनुष्कोट्यां मयास्नातम् दत्तं निष्कशतद्वयम्।
 बहुना किमिहोक्तेन मत्प्रभावं शृणुष्वह।
 25 आत्मप्रशंसयानोक्तं प्रसंगादिदमुच्यते।
 निगृह्ययोनिजं पापं यदिपुण्यं प्रकाशयेत्।
 सद्यो नश्यति तत्पुण्यं पापमेवाभिवर्धते।
 स्नात्वासह्योद्भवात्तीर्थे रङ्गनाथस्यसन्निधौ।
 निरन्तरं तुलामासे कावेरीमहिमागृतः।
 देवर्षिपितरस्सर्वे श्राद्धदानादिकर्मणा।
 श्रद्धयातर्पितास्तत्र रङ्गनाथश्च सेवितः।
 यो रङ्गसन्निधौस्नात्वा कावेर्यां तौलिमासिके।
 शृणुयात्तत्प्रभावं च भुक्तिमुक्ति करेस्थिते।
 किं तस्य दुर्लभं लोके सतु नारायणस्स्मृतः।
 30 जन्मान्तरकृतेपुण्यैः स्नायात्पुण्यनदी जले।
 अनेककल्पसंसिद्धे स्नानं सह्योद्भवेजले।
 तुलामासेतु कावेर्यां स्नायाच्छूिरङ्गसन्निधौ।
 
केवलं तु कृपांतस्मिन् कुर्यात्साक्षाद्धरिस्स्वयम्।
 स्नास्य तस्सह्यजातोये तत्प्रभावं च शृण्वतः।
 ब्रह्मनस्य सुरापस्य सर्वं पापं विनश्यति।
 इति श्रुतं तुलामासे माहात्म्यं पठथतो द्विजात्।
 तदारभ्य च कावेर्यां स्नास्येहं प्रतिवत्सरम्।
 तुलासंक्रमवेळायां सह्यामलक सन्निधौ।
 तत्प्रभावं मयाशृत्वा ह्यर्चितो धर्मवाचकः।
 35 सर्वेपि धर्मवक्तारं कल्पयित्वाद्विजातयः।
 तस्याप्रभावं शृण्वंति धनाद्यैरर्चयंति च।
 अहमेव कृतीलोके यस्स्नायां साजा जले।
 अमुमेव सजानीषे यत्प्रभावाद्धनादिकम्।
 मम पुत्र महाभागा! दशवेदान्तपारगाः।
 चतुर्वेदधरास्सर्वे सर्वच विनयान्विताः।
 पञ्च(काल(यज्ञ) परास्सर्वे प्रातस्स्नानरतास्सदा।
 जितेंद्रियाजितक्रोधास्साधूना मुपकारकाः।
 मयास्नास्यति कावेर्यां तुलामासे स मातृकाः।
 मम पुत्राश्च नप्तारस्तथा सत्कर्मिणोर्जुन।
 40 मम न्यायार्जितं द्रव्यमस्ति विंशत्सहस्रकम्।
 महिष्यश्च तथा गावः सुस्फीताः कांस्यदोहनाः।
 अर्जुनःएतत्सर्वं धनं ब्रह्मन् कुत्र ते विनियुज्यते।
 कथं वा वृद्धिरेतस्य निस्सङ्गत्वं विरुध्यते।
 
द्विजःदेवद्विजार्थे गुर्वर्थे यद्रव्यं प्रतिपाद्यते।
 तत्सर्वं दैवयोगेन स्वयमेवाभिवर्धते।
 धर्म एव परंद्रव्यं धर्म एव परस्सुहृत्।
 धर्म एव परोबंधुनधनं धनमुच्यते।
 लोभावा नास्तिवादश्च ह्यधर्मो धर्मशत्रवः।
 दुराचारश्च पारुष्यं दृष्टिलोभस्श्च ते स्मृताः।
 45 उत्पाटयेत्सुखप्रेप्सुलॊचने अविचारयन्।
 विप्रार्ध एव साद्वर्थे देवार्थे धर्मदर्शनः।
 न्यायार्जितं धनंसर्वं सर्वदाप्रतिपादये।
 पुत्र्यो मे पंचतस्वंग्यस्संति पुष्पशरप्रदाः।
 प्रत्येक क्षेत्रयुग्मेनपरानाहूयदापिताः।
 गव्यं वा महिषीं नैव विक्रीणामिकदाचन।
 तिलं वा तण्डुलं तैलं लवणं व विशेषतः।
 दशरात्रेणशूद्रस्स्यात् तिलतण्डुल विक्रयी।
 गव्यादीनां त्रिरात्रेण सद्योलवण विक्रयी।
 तैल विक्रयपापेन सद्यो नश्येत्कुलं शतम्।
 50 एतेषां तैलयन्त्राख्ये नरके पतनं धृवम्।
 ममस्त्रीवापि पुत्रा वा येचान्ये मद्हालयाः।
 याचके क्वापि नास्तीति शब्दं जानंति नाप्यहम्।
 न सन्निविषान्नाग्नेर् नायुधान्नपिशाचतः।
 अल्पायुर् हि भवेज्जन्तुर् यथालोभेन नास्तिकः।
 
प्रतीकारोस्तिसादेयद्यज्ञानात्प्रवर्तते।
 यदिस्स्याल्लोभमोहार्तः पतत्येव नसंशयः।
 यं निनीषुर्हरिस्स्वर्ग दयया भूतभावनः।
 न तस्यजनयेल्लोभं तत्कुलं चापि वर्धयेत्।
 हरिरेव नियंता हि सर्वेषां सर्वदाप्रभुः।
 55 सत्कर्मैव प्रकुर्वीत सद्भिस्सह सदा वसेत्।
 शृणुयात्सत्कथामेव प्रसीदेत्पद्मलोचनः।
 सज्ञाननिष्ठा गृहिणीपतिव्रता स्वयं चरेत्कर्म
निजं विरक्तिमान्।
 हरिप्रसीदेत्सलोपकारत स्सर्वोपकार
क्षम आश्रमेह्ययम्।
 निष्कामवृत्तेर्गृहएव सत्तपो निवृत्तरागस्य
__गृहं तपोवनम्।
 भार्यापि ध्वीसुधृढात्मबोधिनी
निराश्रमश्चेत्पततीति सन्यसेत्।
 अर्जुनःनमस्ते भूसुरेश्रेष्ठ साधूनामुपकारिणे।
 त्वत्प्रभावं विजिज्ञासुरपृच्छं त्वं क्षमस्व भो।
 सर्वं गवादिकंद्रव्यमानयिष्यामि ते द्विज।
 यथासुखं गृहं गच्छ बहुवन्नित्थं ननामसः।
 60 इत्यर्जुनःप्रसाद्यैनमायुधार्थं गृहं गतः।
 सपार्षदीकं तन्मध्येधर्मपुत्रं ददर्शसः।
 
कृष्णा किरीटिनम् दृष्ट्वा सहसान्तर न्यलीयत।
 न्यवेदयदसौवृत्तम् विप्रस्यातीत लज्जितः।
 कृच्छ्रेणतमनुज्ञाप्य सत्यसन्धोजितेन्द्रियः।
 गृहीत्वा शस्त्रजालानिनगरान्निर्ययौधृतम्।
 विपिनान्तम् ततो गत्वा सूचितेन पथापदैः।
 चोराननुपदं दृष्ट्वा व्यापंचसहस्रकान् ।
 धनुर्विष्पारयन् घोरं नाराचैरधंद्रकैः ।
 निहत्य तान् क्षणेनैव सन्निवर्त्य च गोधनम् ।
 65 द्रव्यं गृहीत्वानिखिलं दत्वा विप्रायपाण्डवः।
 मम सर्वापराधं त्वं क्षमस्वेति ननाम तम्।
 स विप्रोश्रुपरीताक्षस्सन्तुष्टो गोधनैर्युतः।
 प्रायुयोजाशिषोदिप्या अभेद्यार्ब्राह्मणाशिषः।
 प्रणम्य विप्रं पार्थोध सत्वरोतीव विह्वलः।
 पुनस्स्व नगरी प्राप्य पौरश्चाप्यभिनंदितः।
 ननामशिरसापादपङ्कजं धर्मजन्मनः।
 द्रौपदी च नमस्कृत्य कृतांजलिरभाषत।
 आर्य त्वत्तेजसा विप्रस्संयुतोभूद्वादिना।
 सत्यवाक् तीर्थयात्रार्थ ममनुज्ञाप्तुमर्हसि।
 70 त्वं हि धर्मात्मजस्साक्षात् प्रतिज्ञा परिपालय।
 भवादृशमहं तो हि धर्मबीजस्य कारणम्।
 इति ब्रुवन् महाभागः प्रणिपातपुरस्सरम्।
 कथंचित्तमनुज्ञाप्य तीर्थयात्राकृतोद्यमः।
 
सन्यस्यायुधजालानि प्रतस्थे ब्राह्मणैस्सह।
 प्रभासतीर्थमगमत् प्रथमं पाण्डुनन्दनः।
 तत्र स्नात्वा पितृनिष्ट्वा ब्राह्मणेभ्योधनं ददौ।
 ततो व्यासाश्रमं प्राप्य दृष्ट्वा बदरिकाश्रमम्।
 व्यासं तत्र नमश्चक्रे सहस्रादित्यसन्निभम्।
 स तं दृष्ट्वाऽर्जुन श्रीमान्वेदव्यासो महामुनिः।
 75 आतिथ्यं विधिवच्चक्रे स्वागतं परिपृष्टवान्।
 व्यासं प्राह किरीटीच कृतार्थेनांतरात्मना।
 धन्योस्म्यनुगृहीतोस्मि त्वामद्राक्षं महामुने।
 अहं संप्रार्थयन् योगिन् भवदर्शनमागतः।
 कृपां कुरु मयि ब्रह्मन्नभिलाषाभिपूरणात्।
 शृता मया सुभद्रेति भगवद्भगिनीशुभा।
 वसुदेवसुताश्यामा रूपयौवनशालिनी।
 दुर्योधनायदातव्येत्येवं भूताग्रजो हरेः।
 साकन्यैषाममैवस्यात् तथोपायो विचिंत्यताम्।
 80 यदा श्रुता हि सा कन्या सर्वलोकैकसुंदरी।
 तदारभ्य स्मरातॊहमस्वस्थश्च महामुने।
 इति शृत्वाऽर्जुनप्रोक्तं वाक्यं वाक्यविशारदः।
 विस्मयन्नीषदतिस्नेहादिदमाहार्जुनं मुनिः।
 इति अर्जुन ब्राह्मणयोस्संवादः नाम
अष्टादशोऽध्यायः
************************
अथ एकोनविंशोऽध्यायः सुभद्राप्राप्त्यर्थं अर्जुनंप्रति व्यासकृत तुलाकावेरिस्नानोपदेशः, अर्जुनस्य सुभद्राप्राप्त्यर्थं अनेकपुण्यतीर्थ क्षेत्रयात्रा कथनम्, अर्जुनकृत
श्रीकृष्णस्मरणं लक्ष्मीहृदय स्तोत्रं च
व्यासः
अभिलाषो महानेष दुष्पारस्ते धनञ्जय।
 अतीव यत्नस्साध्यं स्यात् सौभद्रं मङ्गलम्महत्।
 इत्युक्त्वा भगवान् व्यासः फल्गुणम् बादरायणः।
 ध्यात्वा मुहूर्तं धर्मात्मा पुनरेवाब्रवीद्वचः।
 शृणु पार्थ! महापुण्यं रहस्यमतिपावनम्।
 सद्यस्सिद्धिकरं दिव्यं सर्वमङ्गलदायकम्।
 सर्वरोगप्रशमनमायुरारोग्यवर्धनम्।
 लक्ष्मीहृदयमित्येतत् स्तोत्रं लक्ष्मीकटाक्षदम्।
 उपदेक्ष्यामिसन्तुष्ट स्तव शुश्रूषयानघ।
 जप त्रिवारं शुद्धात्मा त्रिसंध्यं विजितेन्द्रियः।
 5 सर्वपापप्रशमनं सर्वव्याधि निवारणम्।
 दुष्टमृत्युप्रशमनम् दुष्टदारियनाशनम्।
 ग्रहपीडाप्रशमनं अरिष्ट प्रविभंजनम्।
 पुत्रपौत्रादि जनकं विवाहप्रदमिष्टदम्।
 चोरारिहारि जगतां अखिलेप्सित कल्पकम्।
 सावधानमना भूत्वा शृणु त्वं शुकसत्तम!।
 
अनेकजन्मसंसिद्धि लभ्यं मुक्तिफलप्रदम्।
 धनदान्य महाराज्य सर्वसौभाग्यदायकम्।
 सकृत्पठनमात्रेण महालक्ष्मीःप्रसीदति।
 **क्षीराब्धिमध्येपद्मानां कानने मणिमण्टपे।
 रत्नसिंहासनेदिव्ये तन्मध्ये मणिपङ्कजे।
 तन्मध्ये सुस्थितां देवीं मरीचिजनसेविताम्।
 सुस्नातां पुष्पसुरभी कुटिलालकबन्धनाम्।
 पूर्णेदुबिंबवदनां अर्धचंद्रललाटिकाम्।
 इन्दीवरेक्षणाम्कामां सर्वाण्डभुवनेश्वरीम्।
 तिलप्रसवसुस्निग्धनासिकाऽलङ्कृतांश्रियम्।
 कुन्दावदाधरसनां बन्धूकाधरपल्लवाम्।
 दर्पणाकारविमलां कपोलद्वितयोज्वलाम्।
 माङ्गल्याभरणोपेतां कर्णद्वितयसुंदराम्।
 कमलेशसुभद्राड्ये अभयं ददतीपरम्।
 रोमराजिलताचारु मग्ननाभितलोधरीम्।
 पट्टवस्त्रसमुद्भासी सुनितंबादिलक्ष्णाम्।
 काञ्चनस्तंभविभ्राजद् वरजानूरुशोभिताम्।
 स्मरकाहाळिकागर्वहारिजवां हरिप्रियाम्।
 कमटीपृष्ठसदृश पादाजां चन्द्रवन्नखाम्।
 पङ्कजोदरलावण्यां सुधालभितलाश्रयाम्।
 सर्वाभरणसंयुक्तां सर्वलक्षणलक्षिताम्।
 पितामह महाप्रीतां नित्यतृप्तां हरिप्रियाम्।
 
सर्वमन्त्रमयीं लक्ष्मी शृतिसास्त्रस्वरूपिणीम्।
 परब्रह्ममयीं लक्ष्मी पद्मनाभ कुटुंबिनीम्।
 एवं ध्यात्वा महालक्ष्मी यःपठेत् कवचं परम्।
 महालक्ष्मीशिरःपातु ललाटे ममपङ्कजा।
 कर्णद्वन्द्वं रमापातु नयने नळिनालया।
 नासिकामवतादंबा वाचं वाग्रूपिणी मम।
 दन्तानवतु जिह्वां श्रीः अधरोष्टं हरिप्रिया।
 चिबुकं पातु वरदा कण्ठं गन्धर्वसेविता।
 वक्षः कुक्षिकरौ पायुं पृष्ठमव्याद्रमास्स्वयम्।
 कट्यूरुद्वयकं जानु जङ्के पादद्वयं शिवा।
 सर्वाङ्गमिन्द्रियं प्राणान् पायातायासहारिणी।
 सप्तधातून स्वयंजाता रक्तं शुक्लं मनोस्तिच।
 ज्ञानं बुद्धिर्मनोत्साहान् सर्वं मे पातु पद्मजा।
 मयाकृतं तु यत्तत् वै तत्सर्वं पातु मङ्गळा।
 ममायुरङ्गान् लक्ष्मीः भार्यापुत्रांश्च पुत्रिकाः।
 मित्राणि पातु सततं अखिलं मे वरप्रदा।
 ममारिनाशनार्थाय मायामृत्युंजया फलम्।
 सर्वाभीष्टम् तु मे दद्यात् पातु मां कमलालया।
 सहजा सोदरं चैव शत्रु संहरिणीवधूः।
 बन्धुवर्गं पराशक्तिः पातु मां सर्वमङ्गळा।
 य इदम् कवचं दिव्यं रमायाः प्रयतःपठेत्।
 सर्वसिद्धिमवाप्नोति सर्वरक्षां च शाश्वतीम्।
 "
विशेषेणबृगोर्वारे सद्यस्सिद्धिमवाप्स्यसि।
 स्वयमेवागतं प्रीत्या सुभद्रालप्स्यसेऽचिरात्।
 तव भाग्योदयान्मास स्तौलिकोतीव पावनः।
 कावेरी च महापुण्या दृश्यते रङ्गसन्निधौ।
 तत्र स्नात्वा तुलामासे कावेर्यां रङ्गसन्निधौ।
 श्रीरङ्गशायिनं नत्वाजपतत्सन्निधाविदम्।
 मानसादीनि तीर्थानि गङ्गाद्यास्सरितोखिलाः।
 स्नातुमायांति कावेर्यां तुलासंक्रमणे रवेः।
 ब्रह्महत्यादिपापानां नाशयित्र्योतिपापिना।
 श्रीरङ्गेस्नांति कावेर्यां तौलिकेमासि शुद्धये।
 10 तौलेयस्स्नाति कावेर्यां त्रिदिनं रङ्गसन्निधौ।
 जपंश्च लक्ष्मीहृदयं सर्वपापैः प्रमुच्यते।
 यस्स्नात्वा सह्यजातीर्थे सर्वतीर्थाधिकेशुभे।
 इदं स्तोत्रंजपेल्क्ष्म्यास्स लक्ष्मीशप्रियो भवेत्।
 किंचतेऽतिरहस्यं तु शृणु वक्ष्येऽति सिद्धिदम्।
 इदं जपेद्भूगोवारे पौषमासे धनीभवेत्।
 यः कुर्यात्सर्वयज्ञांश्च सर्वतीर्थावगाहनम्।
 स लभ्यत्सह्यजास्नानं सत्यं सत्यम् वदाम्यहम्।
 इदं स्तोत्रमधीयानः केशवस्याग्रतो जपेत्।
 यं यम् कामयतेकामं तं तमाप्नोत्यसंशयः।
 15 यद्यैश्वर्यं प्रार्थयसे यदि दुर्योधनाज्जयम्।
 यदि सौभद्रकल्याणं जपैतत्सह्यजाजले।
 
तद्गच्छशीघ्रं कावेरी कुरु स्नानमिदं जप।
 पुनश्च कार्तिके स्नाहि कावेर्यां सर्वसिद्धये।
 यस्स्नातिकार्तिकेमासि कावेर्यां पितृदैवते।
 पितरस्तत्क्षणादेव नारके अपि नाकगाः।
 विद्यार्थीवा धनार्थी वा कन्यार्थी वा भृगोर्दिने।
 तौले स्नात्वा तु कावेर्यां जपेल्लक्ष्मीस्तवं हि सः।
 यानारी पुत्रगृनुस्स्यात् स्नात्वोर्जे सह्यजाजले।
 उपलिप्यालयं विष्णोः रङ्गवल्ली पुरो न्यसेत्।
 20 यो दीपरङ्गवल्यायै रलंकुर्याद्धरेःपुरः।
 निरग्निदोषतोमुक्तो ब्रह्मलोके महीयते।
 या साध्वी विधवानारी रङ्गवल्याच्युतालयम्।
 अलंकृत्य हरेस्तोत्रं शृण्वती हरिमश्नुते।
 गच्छोर्जेत्वं सह्यजास्स्नानकारी
श्रीरङ्गेशस्यालयं रंगवल्या।
 आज्योद्भूतैर्लक्षवांख्यदीपैः
___ कन्याहेतोश्श्रद्धयालंकुरुष्व।
 पार्थ! धर्मरहस्यं त्वं शृणुष्वाप्यपरं पुनः।
 द्वियोजने तु पूर्वाब्धेः कावेर्या दक्षिणे तटे।
 शिवक्षेत्रं महापुण्यम् मयूरमिति वर्तते।
 तत्र तौले कुरु स्नानं कावेर्यां शिवसन्निधौ।
 25 सुभद्रां सद्य एव त्वं लप्स्यसे भद्रपूर्वकम्।
 यस्स्नति तौलौ मायूरे कावेर्यां शिवसन्निधौ।
 
अपि पंचमहापापी विशुद्धश्शिवमश्नुते।
 भूमौ मोक्षकरत्वेन कल्पितं द्वयमेव हि।
 श्रीरशैवापि मायूरे स्नानं सह्योद्भवेजले।
 विश्वामित्रो मुनिःपूर्वं क्षत्रियो विप्रनिन्दकः।
 ब्राह्मण्यम् दुर्लभं प्राप मायूरस्नानपुण्यतः।
 कार्तिके वा तुलामासे कावेर्यां स्नाति तत्र यः।
 तस्य पुण्यफलं वक्तुं शक्त एव हरिस्स्वयम्।
 साक्षात्स एव भगवान् परब्रह्मास्वयं हरिः।
 30 श्रीरङ्गेसह्यजास्नायी मोक्षं च लभतेधृवम्।
 त्वं गच्छ शीघ्रं श्रीरङ्गं तुलामासोतिवर्तते।
 अगस्त्यःइत्युपायं समाकर्ण्य सुभद्रोद्वाहकर्मणि।
 सन्तुष्टस्तदनुज्ञातः प्रतस्थे तं प्रणम्यच ।
 ततःप्रयामागंय तत्रस्नानं चकारसः ।
 तत्रांतस्सलिलेमग्नो जजापपरमं मनुम् ।
 तत्रोलूपी नागकन्यां दृष्ट्वा सर्वाङ्गसुंदरीम् ।
 उद्वाह्यविधिना रेमे गंधर्वेण तयार्जुनः ।
 तस्यामिरावान्नाम्नोभू दुलूप्यां तनयोबली।
 अथस्नात्वोत्थिताश्श्रीमान् स्वप्नं दृष्ट्वैव विस्मितः ।
 35 प्रदक्षिणक्रमाद्भूमेर् ब्राह्मणैस्सह पाण्डवः ।
 तत्र तत्र सतीर्थेषु पुण्येषुच सरित्सुच।
 
स्नात्वासंतर्घ्य देवर्षी स्तीर्थश्राद्धपुरस्सरम् ।
 ततस्संगमुखप्रायात् कावेर्यास्सागरस्य च ।
 सस्नातत्रयहापुण्ये कावेरीवार्धिसङ्गमे ।
 यत्र वैवाहिकंभद्रं कावेर्यास्सरितांपतेः ।
 दानंचकार विप्रेभ्यो दशलक्षधनं तदा ।
 तत्र श्वेतवनं नाम भूमौपुण्यांकुरुस्थलम् ।
 यत्र सन्निदधेशंभुः कावेरीसागरोत्सवे ।
 ततो मायूरमगमत् कावेरीस्नानतत्परः ।
 40 तत्र स्नात्वा तुलामासे धनंदत्वा द्विजन्मनाम् ।
 दशलक्षगजान् साश्वान् कन्यागाश्च पुनर्ददौ ।
 प्रणम्य देवदेवेशं सर्वाभीष्टफलप्रदम् ।
 वेगाच्छ्रीरङ्गमगमद्यत्र सन्निहितोहरिः ।
 तत्र श्रीरङ्गमाहात्म्यं शृत्वा पार्थोऽतिविस्मितः ।
 कृतार्थोस्मीति मत्वा च कावेर्यां स्नानमाचरत् ।
 अन्येच बहवोविप्रा नानादिग्भ्यस्समागताः ।
 श्रीरङ्गसन्निधौ स्नात्वा कावेर्यामरुणोदये।
 शुवू रङ्गमाहात्म्यं कावेर्यांश्चैववैभवम् ।
 पार्थोप्यावृश्चिकं तत्र स्नात्वा प्रातर्द्विजैस्सह ।
 45 प्रत्यहंशतनिष्कांश्च दत्वा विप्रेभ्य आदरात् ।
 अर्चयन धर्मवक्तारं स्वर्णपुष्पादिभूषणैः ।
 प्रभावमशृणोद्दिव्यास्सर्व पापघ्नमर्जुनः ।
 कृत्वा षोडशदानानि चंद्रपुष्करिणीतटे।
 
तत्र स्नात्वा च रङ्गेशसन्निधौपितृदेवताः ।
 तर्पयित्वा यथान्यायं रङ्गनाथं प्रणम्य च ।
 पुनश्च चंद्रपद्मिन्यां कार्तिके स्नानमन्वहम् कुर्वन् व्यासोपदिष्टं च सर्वं तत्राकरोत्सुधीः ।
 प्राकार गोपुरादींश्च मण्डपांत्समकारयत्।
 कुर्वन् व्यासोपदिष्टं च सर्वं तत्राकरोत्सुधीः ।
 प्राकार गोपुरादींश्च मण्डपांत्समकारयत्।
 एवं तत्र वसन् धीमान् प्रत्यहं शेषशायिनम् ।
 50 प्रणम्य संस्तुवन् स्तोत्रैरापौषमवसच्च सः ।
 धनुर्मासेऽर्चयन् विष्णुं मुगान्नमरुणोदये।
 निवेदयित्वा बीभत्सुर्भूसुरानप्यपूजयत्।
 ततः प्रतस्थेश्रीरङ्गात्सेतौनातुं कृतत्वरः।
 ब्राह्मणैस्सहधर्मात्मा नत्वा श्रीरङ्गशायिनम्।
 गत्वा सत्यगिरिपुण्यं नत्वासत्यगिरीश्वरम्।
 सत्यपुष्करिणीतीर्थे ब्रह्महत्यापनोदने।
 स्नात्वाश्राद्धक्रियां कृत्वा दानं कृत्वा च भक्तितः।
 गत्वागोष्ठीपुरं प्रादाद् दध्यन्नं हि हरेःप्रियम्।
 यत्र पूर्वं सुरास्सर्वे हिरण्येनप्रपीडिताः।
 तदगम्यस्थलं ज्ञात्वा गोष्ठीभूतास्सुखंगताः।
 55 स्नात्वाच तीर्थेऽमृतपुष्करिण्यां
श्रीपुंडरीकाक्षपदप्रदायाम्।
 इष्ट्वा पितॄश्चद्विजपुङ्गवेभ्यो
दत्वाधनं भूरि नृसिंहमीशम्।
 प्रह्लाददासं परिपालयन्तं ननाम
दैत्येश्वर पीडितं सः।
 ततो जगामधर्मात्मा वृषभादि हरेःप्रियम्।
 यत्र नूपुरनद्याख्या साक्षात्स्रवतिजाह्नवी।
 अदाद्रव्यं चतुर्लक्षं विप्रेभ्यस्स्नानपूर्वकम्।
 पाण्ड्यस्य दानवर्यस्यभगवन्नूपुरच्युता।
 श्रीकान्तप्रेषितागंगा साक्षात्प्रवहतिस्वयम्।
 तीर्थे नूपुरनद्यास्तु स्नात्वादानं चकारसः।
 सा नूपुरनदीनाम गङ्गापततिवैष्णवी।
 तत्र द्वादशरात्रं तु स्थित्वा दत्वाधनंबहु।
 60 प्रणम्य वृषभाद्रीशं ततोऽगान्मोहनंपुरम्।
 तत्र स्नात्वा ब्रह्मतीर्थे ब्रह्महत्यापनोदने।
 मेघश्यामहरिभेजे यं नत्वामृतमश्नुते।
 यत्राच्युतःपरंब्रह्म यो षिद्रूपस्वमायया।
 मोहयित्वाऽखिलान् दैत्यान् सुधांदेवानपाययत्।
 मधुनापहृतेवेदे दुःखार्ते च पितामहे।
 तं हत्वोपादिशद्यत्र तस्मैवेदान् पुरा हरिः।
 ततो ययौ महातेजा मधुरां पाण्ड्यभूपतेः।
 यत्र मत्स्यावतारोभूद् विष्णुर्वैहायसीजले।
 65 तत्र चित्रांगदां नाम कन्यांपाण्ड्यस्य शोभनाम्।
 स्वयंवरसमाहूत उपयेमेऽतिसुंदरीम्।
 
किंचित्कालं तयासार्धं क्रीडासौख्येनतोषितः।
 सेतौस्नातुं महातीर्थे प्रतस्थे पाण्डवस्तदा।
 तत्र गत्वा धनुष्कोटी हरिलीलाध्वजाकृतिम्।
 स्नात्वा कृतार्थोधर्मात्मा त्रिंशल्लक्षमदाद्धनम्।
 ; तत्र नत्वा महालिङ्गं श्रीरामस्थापितम् पुरा।
 स्नानं तत्राकरोन्माघे महापातकनाशने।
 शृत्वा च माघमाहात्म्यं प्रत्यहंचार्चयन् द्विजान्।
 व्रतान्तेभोजयित्वा च मधुरामगमत्पुनः।
 70 पुनश्चित्रांगदां प्राप्य भुञ्जन्वैषयिकंसुखम्।
 आमन्त्र्यपाण्ड्यभूपाल मन्तर्वत्नी तथास्त्रियम्।
 पुनश्श्रीरङ्गमगमत् चैत्रोत्सवकुतूहलात्।
 तत्र चित्राङ्गदापुत्र बभ्रुवाहमसूतसा।
 अथ चैत्रोत्सवेपार्थो रङ्गनाथं प्रणम्य च।
 तत्रागतांश्च द्रव्यायै रन्नस्सन्तर्प्य च द्विजान्।
 उपकुर्वन्प्रतस्थेच प्रतीचींदिश मर्जुनः।
 तंपुण्यंप्राप सह्याद्रिं यत्राधात्रीमयो हरिः।
 दत्तात्रेयो हरिस्साक्षाद्यत्रास्तेयोगि लक्षणः।
 तं दृष्ट्वाशैलसौभाग्य विस्मयोत्फुल्ललोचनः।
 75 तत्र स्नात्वामहातीर्थे सह्यामलकसन्निधौ।
 ततो गोदानसाहस्रं विशेषेणद्विजन्मनाम्।
 लक्षद्वादशसंख्याक गजदानं चकारसः।
 ततो गोकर्णमगमत् सान्निध्यं यत्रधूर्जटेः।
 
तत्रगोकर्णतीर्थेच स्नानंकृत्वा कुटुम्बिने।
 श्रोत्रियाय दरिद्राय ब्राह्मणाय धनं ददौ।
 पुनःप्रभासमगमत् तत्रस्नात्वार्चयन् द्विजान्।
 तान् विसृज्य ततःपार्थो द्वारकापुरमेयिवान्।
 सुभद्रोद्वाहनौत्सुक्यादथ चिंताकुलोर्जुनः।
 एकान्तकुत्रचिद्देशे स्थित्वेत्थं समचिंतयत्।
 80 कथं सुभद्रामुद्वाह्य श्यामां त्रैलोक्यसुंदरीम्।
 कृतार्थोहम् भविष्यामि यथाऽहल्यां सगौतमः।
 नारीरत्नं सुभद्रा मां वृणुयात्स्वयमेववा।
 इति चिंताकुलाःपार्थः कामपाशवशंगतः।
 विरहज्वरसंतापा न्नाद्यगच्छत्तदासुखम्।
 पुनर्दद्याविदंपार्थः प्रत्युत्पन्नमतिस्सुधीः।
 निवासवृक्षस्साधूना मापन्नानांपरागतिः।
 भर्तिभंजनश्श्रीमान् कृष्ण एवगतिर्मम।
 योयं देवोद्वारवत्यांनमस्ते श्रीमाननंतोयदुवंशनाथः।
 भक्तार्तिहंता भगवान् दयाळुर्स
___एव देवश्शरणम् ममास्तु।
85 इत्युस्मरद्वासुदेवं किरीटिमहानुभावं
जगतांशरण्यम्।
 रथाधिरूढो निगमान्तरूढ
स्सचाविरासीत्प्रणतार्तिहन्ता।
 इति एकोनविंशोऽध्यायः
*************************
अथ विंशोऽध्यायः कृष्णेन अर्जुनस्य सन्यासवेषोपदेशः, यतिवेषस्यार्जुनस्य द्वारकाप्रवेशः, रात्रौ श्रीकृष्ण महेंद्रादि सन्निधौ रहसि अर्जुनस्य सुभद्रा विवाहाश्च अथागतं हरि वीक्ष्य हरि दारुकसारथिम्।
 नत्वा नंदाशृभिःपार्थो ह्यभिषिञ्चत्पदांबुजम्।
 आपिबन्निव नेत्राभ्यां बाहुभ्यां परिषस्वजे।
 गोविंदोगौर्वत्समिव लीहनाजिह्वयादरात्।
 तं प्रणम्य महादेवं कृष्णःकृष्णःकृतांजलिः।
 उवाच प्रहसन् मन्दं शरणागतवत्सलम्।
 परिपालय मां कृष्ण शरण्य शरणागतम्।
 श्रीभगवान्महामनोरथेनाद्य भवता चिंतितोऽधुना।
 तथापि भवतो भक्त्याप्रसन्नोहं प्रसादये।
 सन्यासि रूपमाश्रित्य द्वारका त्वं सखे व्रज।
 5 चातुर्मास्यच्छलेनाद्य सर्वं संपद्यते तव।
 अर्जुनःसन्यासयोगनिष्ठास्तु कथं भूयो गृही भवेत्।
 एतदाचक्ष्व देवेश त्वद्भक्ताय नमोस्तु ते।
 श्रीभगवान्नवेषेणयतिर्भूयान्नतु वेदांतकीर्तनात्।
 
मनसो निर्विकल्पत्वे सन्यासो विहितो भवेत्।
 सर्वत्रेश्वर विज्ञानं मनसो निर्विकल्पकम्।
 तावत्कर्माणिकुर्वीत निष्कामो मयि भक्तिमान्।
 मां च पश्यति सर्वत्र मनसा सन्यसेत्ततः।
 वाचा ब्रह्मवदन्मूढो मनसा स विकल्पकः।
 यःकुर्यात् कर्मसन्यासं सोन्धवत् सहसापतेत्।
 10 मां च पश्यति सर्वत्र सर्वं च मयि पश्यति।
 तद्ब्रह्मज्ञानमित्युक्तं विरक्तस्तु व्रजेद्गृहात्।
 सन्यासिवेषमात्रेण कर्मत्यागी पतेदधः।
 यावत्कालं तु गायत्री तावत्कालं न पात्यते।
 विरक्तो या वशी दान्तः क्षान्तो मय्यपि भक्तिमान्।
 सन्न्यस्य सर्वकर्माणि निराशो यदि समाश्रयेत्।
 नटवद्यति वेषोपि द्वापरांते न दुष्यति।
 पतेत् कलौ तु सन्यासवेषमात्रेण वै द्विजः।
 अतो न भीस्त्वयाकार्या यतिवेषम् समाश्रितः।
 व्रज द्वारवतीप्रांते मया सर्वं च साध्यते।
 15 इत्युक्त्वा पाण्डवम् कृष्णो भगवान् रथमास्थितः।
 रात्रावेव त्वरायुक्तो द्वारकामगमद्रहः।
 अथार्जुनोपि हृष्टात्मा विरक्ता इव भावयन्।
 त्रिदंडिद्वारकामेत्य गुहांशैले समाश्रयत्।
 व्रतव्याजेनगोविंदो बलभद्रादि यादवान्।
 आनीय शैलस्याभ्याशं वनभोजनमाश्रयत्।
 
अथ सर्वेव्रतं कृत्वा माङ्गल्यं स हलायुधाः।
 उत्सवांते च ददृशु रर्जुनं यतिरूपिणम्।
 संभ्रमेण नमश्चक्रु रतीव विनयान्विताः।
 अथोग्रसेनो हलिना शास्त्रमालोच्य तत्वतः।
 20 कृतांजलिरुवाचेदं हर्षगद्गदयागिरा।
 भगवन् योगिनां श्रेष्ठ यस्त्वमस्मत्स्थलं श्रितः।
 अहं किं करवाण्यद्य किं करस्तव भो मुने।
 इत्युक्तोप्यभवत्तूष्णीं मीलीताक्षोऽकरोर्जुनम्।
 दृष्ट्वैव यतिमौनस्थं तेनके विससिष्मिरे।
 अथ विस्मयमापन्नाः कृष्णायेत्थं स्व वेदयन्।
 भो भो कृष्णमहायोगी वर्तते गिरिगहरे।
 वृष्णोपि किंचिन्नभूनिस्संगोंतर्मनामुनिः।
 श्रीकृष्णःकिमनेन विचारेति यतीनां विपिनंगतिः।
 योगिनः कतिवर्तते किमस्माकंप्रयोजनम्।
 
यादवाः
इत्थं वत्स न वक्तव्यं त्वयाबालिशबुद्धिना।
 25 यतीन्यस्यगृहेभुङ्क्ते ग्रासमात्रमपि द्विजः।
 तुष्यंति पितरस्सर्वे योलोकास्सहेश्वराः।
 श्रीकृष्णःएवमेव न संदेहस्सन्यासी निगृहंगते।
 पूर्वमामन्त्रणं कुर्यान्नहि भोजं यतिम्।
 
एकभिक्षाशनंभुङ्क्ते यतिःपूर्वं नियन्त्रितः।
 तेभ्यःप्रयश्चेदिंद्रस्थम् दाता च नरकं व्रजेत्।
 अतस्स्वयतिमागच्छेद्यतिरस्मद् गृहंगुरो।
 भुक्त्यैव भोजयिष्यामो नोस्पष्टं समर्थ्यताम्।
 उग्रसेनः- चातुर्मास्यव्रतेमध्य समयोयं समागतः।
 वार्षिकोयमतःकृष्ण! गृहमागमतां यतिः।
 30 बलभद्रःआर्य कृष्णो न जानासि धर्माधर्म विनिश्चयम्।
 बाल्यानुरूपं ब्रूतेसौ पृच्छन्येनं कथं गुरोः।
 इत्युक्त्वा बलबद्राद्यास्सर्वे प्रांजलयोऽब्रुवन्।
 यतयस्समीपमासाद्य श्रद्धाभक्ति समन्विताः।
 चातुर्मास्यविधिनार्यो भवतास्मत्पुरेधुना।
 अथोन्मील्योऽर्जुनान्नेत्रे मौनीमन्दमनामसः।
 अथैनमानीय पुरान्तिके मठे समर्चयामासु स यादवाः।
 भिक्षां गृहीत्वा प्रतिगेहमर्जुनो भुक्त्वा मठे मौनमावहात्।
 एवं स्थितस्य पार्थस्य पंचषादिवसागतः।
 बलभद्रोयदून् प्राह सर्वेषांचोपशृण्वताम्।
 35 कालोयमारब्धश्चातुर्मास्य व्रतं तथा।
 गतागतं कथं योगि स्थितेयम् च जगद्गुरुः।
 तस्मादस्मद्गृहे वासः कल्पनीयो यतश्शुभाः।
 
उद्यानांतगृहेयोगी गुहायां वसतु द्विजः।
 इति वाक्यं समाकर्ण्य बलभद्रसमीरितम्।
 उग्रसेनादयस्सर्वे प्रशशंसुहेलायुधम्।
 एतस्मिन्नंतरेश्रीमान् कंपयित्वा शिरोधराम्।
 ईषद्विस्मयमापन्नो भगवानब्रवीद्वचः।
 श्रीकृष्णःयतिर्यदिवसेद्देहे एकरात्रामनावदि।
 प्रायश्चित्तं च नास्त्यस्य सद्यो भवतिपातकी।
 40 आहूतोपि यतिस्सोयं व्रजेद्वास्मद्गृहं वशी।
 अविरक्तो बुभुक्षुश्च गृहमायात्यसंशयः।
 श्रीरामःवत्स कृष्ण न जानासि धर्माधर्म विनिश्चयम्।
 यतीनां समचित्तानां वसेवासोस्तु वा गृहे।
 तेषां तेजोविशेषेण प्रत्यवायो न विद्यते।
 अभूतपूर्व एतस्य प्रभावश्चोतुलो महान्।
 अस्मद्नेह वसत्यस्मिन् महानभ्युदयो भवेत्।
 वृद्धवाक्यानुसारेण ततो मौनीभवाधुना।
 श्रीकृष्णःयतिस्सर्वात्मना पूज्यस्सवैरेव न संशयः।
 कथं राजगृहेरम्ये वसेदेषोऽति सौख्यदे।
 45 न विश्वासो यतौकार्यः किं पुना राजमंदिरे।
 दास्योयुवत्यःकन्याश्च संचरन्त्यतिशोभनाः।
 
सौभरिश्च महायोगी मांधातृ दुहितर्वशी।
 उद्वाह्य स्मर भाणा” भोगीजात इति शृतम्।
 गौतमप्यंबुधौ माग्नो वशीवर्षसहस्रकम्।
 अहल्यार्थे महातेजाः पश्चाद्रेम तयासह।
 पराशरोपि योगीन्द्रः साक्षादंशो हरेर्मुनिः।
 कैवर्तकन्यां कामार्तस्स्वयमेवाग्रहीजले।
 किमत्रबहुनोक्तेन स्पृष्ट्वा ब्रह्म तिलोत्तमाम्।
 स्वयं भोक्तुं समारेभे कामोवीक्ष्याहसच्चतम्।
 50 सभान्तरे हसंतं तं दृष्ट्वा कृद्धोऽशपच्च सः।
 अर्वे त्वं हरनेत्राग्नि दग्धोसंगोभविष्यसि।
 प्रसादात्पुनरंगी त्वं तस्यलोक सुकंपिनः।
 श्रीरामःपूर्वोक्तानां गृहस्थानां यतीनाम्महदन्तरम्।
 साधुभावं भजस्वाद्य यत्किंचित्क्रियता मया।
 श्रीकृष्णःवनाद्यतिःपुरं नीतः पुरादपिगृहान्तरम्।
 आनीतेन सजानेहं किं किमत्र भविष्यति।
 अथ रामादयस्सर्वे यादवा वृद्धसंमताः।
 वसुदेवे गृहोद्याने यतिमानीयसंनताः।
 अगस्त्यःअथ कालान्तरेराम आलोक्य यति वैभवम्।
 55 शुश्रूषार्थं सुभद्रां तु प्रेरयन् कृष्णमब्रवीत्।

विशिष्ट भर्तृसंपत्तिः कन्यानां तपसःफलम्।
 सुभद्रा प्रेष्यतां यातु सुपतिःप्राप्तये यतेः।
 अन्यो विश्वसनीयः किमस्मिन्नन्तःपुरे भवेत्।
 श्रीकृष्णःयथेष्ठं क्रियतामार्य भवद्भिर्लोकपूजितैः।
 सत्सूग्रसेनमुख्येषु कथं सिध्येन्मतं मम।
 श्रीरामःइत्थं वत्स ! न वक्तव्यं त्वदधीनं यशो हि नः।
 यतेस्स्वरूपमज्ञात्वा बालिशत्वाद्वीषि च।
 यतींद्रोयं महायोगी सदाब्रह्मविभावनः।
 यतिस्संभावितां साधु यतस्सर्वं भविष्यति।
 60 वृथाःसदूषणं कार्यं यतिनिंदा कुलक्षयम्।
 इत्युक्त्वा स्वकराब्जेन कृष्णहस्तं प्रगृह्य सः।
 उग्रसेनादिभिस्सभ्यैरंतिकंचागमद्यतेः।
 एतदागमनं दृष्ट्वा पार्थोऽतिध्यानमास्थितः।
 पद्मासनं समास्थाय निश्चलोभूद्यथाबकः।
 कृष्णःकृतांजलिर्नत्वा मन्दं यतिमुवाच ह।
 भवंतःकृतकृत्या हि परमात्मैकभावनाः।
 संयग्दृष्टं परंब्रह्मा किं ब्रह्मन्नाभिभाषसे।
 नानारूप विभागेन परंब्रह्मैव ते ग्रतः।
 वर्ततेंतर्हृदिस्थं तद्ध्यानं व्यवहरत्यजन्।
 65 इत्युक्तोप्यभवन् मौनी सो?न्मीलित लोचनः।

कृष्णोब्रवीत्तदारामं वचनं स्वार्थसाधनम्।
 अहो ध्यानमहोध्यान महोध्यानम्महात्मनः।
 किं प्रयोजनमस्माभिर् निरपेक्षस्ययोगिनः।
 ध्यानारूढ इवाभाति वशीयोगी न संशयः।
 त्थाप्यश्रद्धधानोहं यतेरन्तःपुराश्रयात्।
 यत्र कुत्रापि वासोस्तु भवेदेतेदलौकिकम्।
 इत्यर्धसंमतिं ज्ञात्वा बलभद्रो महामनाः।
 सेवार्थं प्रेषयामास सुभद्रां यतिसन्निधौ।
 सा सुभद्रा विनीता च शौचाचारसमन्विता।
 ७0 शुश्रूषयामास यति मिंगितज्ञाशुचिस्मिता।
 तत्तत्काले जलं वापि पुष्पं वा यद्यदिच्छति।
 तत्तदेवानयामास भैक्षार्थे च न्यवेदयत्।
 कदाचिदन्तिकं गत्वा लक्षणादर्जुनस्य सा।
 शङ्किता सस्मिता पार्थं प्रणमन्ती कृतांजलिः।
 सुभद्रायोगिन् ! सर्वत्रसंचारी त्वं सदा पुण्यभूमिषु।
 तीर्थस्नायी गतःपार्थो दृष्टो वा यत्र कुत्रचित्।
 इति शृत्वा वचो मंदं विकसद्वदनांबुजः।
 भद्रंरूपंसुभद्रायाः तं निरीक्ष्य शुचिस्मितः।
 सुभद्राअथवा योगिवर्य त्वं स एवाऽर्जुनसंज्ञितः।
 75 तथैव राजसेमेध्य तत्वं ब्रूहि वचो मुने।

अर्जुनःस एवाहं महाभागे! पार्थ इत्यभिशुश्रुतः।
 यतिरूपं समास्थाय त्वत्समागमकाया।
 तद्वाक्यं मधुरं शृत्वा पूर्णचंद्रनिभानना।
 पुळकांगितसर्वाङ्गी तस्थौ मंदस्मिता च सा।
 बभाषे वचनंस्निग्धं सापार्थं लिकुचस्तनी।
 अतीव स्मर भाणार्ता संतप्ताकामवह्निना।
 मत्पाणिग्रहणेऽद्य त्वं कमुपायमुपेक्षसे।
 सुयोधनाय रामाद्या दास्याम इति निश्चिताः।
 यथेतरेन जानंति तथोपायो विचिंत्यताम्।
 80 मनसैव वृणेहं त्वां भवानेव पतिर्मम।
 अन्यं न कामयेजातु क्रियतां गोप्य मे वद।
 अर्जुनःइति तस्य वचश्शृत्वा त्वत्सौंदर्यावलोकनात्।
 त्वदिच्छयैव संप्राप्तस्सुभद्रे लोकसुंदरी।
 मात्वन्यथा मतिंकुर्या स्त्वां विना न च मे रतिः।
 तमेव शरणं गच्छ कृष्णमातिनाशनम्।
 इत्युक्ता सत्वराहृष्टा गत्वा मतुरथांतिकम्।
 किंचित्सा लज्जिता तस्यै स्वाभिप्रायपुरस्सरम्।
 अर्जुनस्याप्यभिप्रायं यथापूर्वमवर्णयत्।
 माता शृत्वा च तद्वाक्यं दुहित्रुस्नेहबंधनम्।
 85 अर्जुनेप्यधिकं दृष्ट्वा यतिवेषं तथार्जुनम्।

विस्मिता च प्रहृष्टा च स्व कन्यायास्स्वयंवरात्।
 कृष्णमेकांतमाहूय यथा वृत्तान्तमब्रवीत्।
 तदा मातृवचश्शृत्वा प्रहसन् मंदमच्युतः।
 उवाच मातरं वाक्यं मोहयन्निवमानसम्।
 
अंबे तं दुर्घटमन्ये सुभद्रायामनोरथम्।
 विघ्नं कुर्वति चार्याद्या न शक्यमिव भाति मे।
 अथापि साधयेसर्वं निर्वृतास्यात्स्वसातुनः।
 येनकेनापि मे मात! स्त्वद्वाक्यं करणं तपः।
 इत्युक्त्वा भगिनींचाह भगवान् भक्तवत्सलः।
 90 अम्ब वत्से न ते भीस्स्या निर्वृतिं साधयेऽखिलाम्।
 मां प्रपन्नोजनःकश्चि न्न भूयोर्हतिशोचितुम्।
 इति कृष्णस्समाभाष्य तीर्थयात्रा प्रसंगतः।
 समुद्रतीरमगमद् रामायैर्यदुभिस्सह।
 तत्र स्नात्वा पुण्यतीर्थे समुपोष्य शुचिव्रताः।
 तत्रैवोपुरहोरात्रं श्रान्तश्च शयितास्तदा।
 निद्रा सक्तेषु सर्वेषु स्वयमाया हि तेष्वपि।
 अर्धरात्रेसमुत्थाय मोहयित्वाखिलान् यदून्।
 एक एवरथारूढो द्वारकां पुनरागतः।
 अथागतं हरिदृष्ट्वा संतुष्टःपाण्डुनंदनः।
 95 सस्मार पितरंगेद्रम् शची सप्तऋषीनपि।
 सप्तर्षयस्स्मृतास्तेन देवेंद्रश्च शचीयुतः।
 आगच्छन् सन्निधौशौरेः पार्थोत्सव चिकीर्षवः।
 
तत्रागतान् मुनीन् दृष्ट्वा ननाम शिरसामुदा।
 प्रणमन्तं तमिंद्रोपि दोर्ध्यामुत्थाप्य हर्षितः।
 आशिषं च प्रयुंजानस्सस्नेहं परिषस्वजे।
 कृष्णस्य च वदांभोजे किरीटेनाग्निवर्चसा।
 संस्पृश्य कल्पवृक्षोत्थैर् दिव्यैःपुष्पैस्समर्चयत्।
 ततस्सप्तऋषीन् कृष्णश्शिरसा संप्रणम्य च।
 मधुपर्कादिकं चक्रे स्वागतंचाप्यपृच्छत।
 100 अद्य मे सफलं जन्म यद्यूयं मद्गृहं गताः।
 
ऋषयः
भगवन् किमिदं वाक्यं भाषसे लोकमिहनम्।
 त्वं ब्रह्मा परमं साक्षान्निर्गुणं प्रकृतेःपरम्।
 इति स्तुत्वा महाभागा हर्षगद्गदया गिरा।
 भगवच्छरणांभोजयुगळं ते ववंदिरे।
 अथ तैस्संयगालोच्य लग्नं परमभास्वरम्।
 पार्थस्य कौतुकंकर्म कारयामास केशवः।
 अथेद्रो भाषणैर्दिव्यैर्भूषयामास फल्गुनम्।
 कल्पवृक्षोद्भवैःपुष्पैः कस्तूरिकुंकुमादिभिः।
 अरुंधती तथेद्राणि वस्त्रैराभरणैश्शुभैः।
 105 लेपनाद्यैर्मुगंधैश्च सुभद्रां तामभूषयत्।
 अलंकृत्य शचीकन्यां साक्षाल्लक्ष्मीमिवस्थिताम्।
 सुनासां सुदतींश्यामां स्वर्णताटंकशोभिताम्।
 दुकूल वस्त्रसंवीता मङ्गळाभिरलङ्कताम्।
 
शुचिस्मितां च सुस्नातां सुभद्रां तां मनोहराम्।
 वरस्य दक्षिणेपार्श्वे स्थापयामास देवकी।
 जुहुवाग्नौ महातेजाः पार्थस्सप्तर्षिसन्निधौ।
 कृष्णस्यानुमते हृष्टो वसुदेवोमहामनाः।
 संस्थाप्यांके निजां कन्यामीषल्लज्जान्वितां शुभाम्।
 मन्त्रपूतेन विधिना प्रादात्पार्थाय धीमते।
 110 अस्मिन् महोत्सवेपुण्ये सर्वलोकसुखावहे।
 देवदुंदुभयोनेदुर्ननृतुश्चाप्सरोगणाः।
 त्रिरग्निं च परिक्रम्य लाजहोमपुरस्सरम्।
 संगृह्य पाण्डवः पाणिं सुभद्रायास्सुपाणिना।
 दक्षिणांधिंचकारास्या अश्माधिष्ठान मङ्गळम्।
 तौ दंपतीमुदायुक्तौ पश्यंता च परस्परम्।
 शुभमा क्षतारोप मन्वभूतां सभांतरे।
 वधूवरौ महाभागौ कृष्णपादांबुजद्वयम्।
 प्रणेमतुर्महात्मानौ सभायामग्नि सन्निधौ।
 प्रकाशयंतौ वपुषा ताम् सभां ते विरेजतुः।
 अथ सप्तर्षयस्ताभ्यां प्रयुज्याशिषमुत्तमाम्।
 कृष्णमामन्त्र्य ते जग्मुस्स्वस्थानं तत्क्षणेन वै।
 शचींद्रो प्रणमंतौ तौ मूर्ध्याघ्राय मुहुर्मुहुः।
 परिष्वज्य स बाहुभ्यामाशीर्वादं प्रयुज्य च।
 त्रिदिवं ययतुस्तुत्वा संस्थितौ सिद्धचारणैः।
 कृष्णोपि रात्रावेवाथ सुभद्रोद्वाह मङ्गलम्।
 
निर्वर्त्य च स्वभक्ताय पाण्डवाय महामुदम्।
 यथापूर्वं समुद्रांते शेतेस्म सुहृदंतरे।
 कृष्णप्रसादात् पार्थोपि सुभद्रां प्राप्य निर्वृतः।
 सुधामिंद्र इव प्राप्य तया रेमे मनस्सुखम्।
 120 अगस्त्यःइदं सुभद्रोपाख्यानं पुण्यमायुष्यवर्धनम्।
 यःपठेच्छृणुयाद्वापि सर्वपापैः प्रमुच्यते।
 यश्शृणोति भृगोवारे स कन्यां विंदतेशुभाम्।
 य इदं पुण्यमाख्यानं पुत्रपौत्रप्रदं शुभम्।
 पुण्यकाले विशेषेण शृणोति च समाहितः।
 ऐश्वर्यमतुलं प्राप्य विष्णुलोकं च विंदति।
 सर्वतीर्थेषु यत्पुण्यं सर्वयज्ञेषु यत्फलम्।
 तत्फलं समवाप्नोति शृत्वा पुण्यकथामिमाम्।
 य इमम् शृणुयाद्गाधां सन्निधौ च हरेर्दिने।
 जाग्रन्निशि स धर्मात्मा शुकवद्वैष्णवो भवेत्।
 125 पूर्वं दाक्षायणीत्यागे पठन्पुण्यकथामिमाम्।
 उपेयेमे पुनश्श्रीमान् पार्वती परमेश्वरः।
 रुद्रनेत्राग्निना दग्धे रतिर्भर्तरिदुःखिता।
 शिवाज्ञया कथामेतां शृत्वा पतिमतीह्यभूत्।
 नानयासदृशी पुण्या कथाशास्त्रेष्वभीष्टदा।
 सकृच्छ्रुत्वा कथामेतां सद्यश्शुद्धिमवाप्नुयात्।
 128 इति अर्जुनस्य सुभद्रा विवाह नाम- विंशोऽध्यायः।
 
********************
अथ एक विंशोऽध्यायः
अर्जुन बलरामयोयुद्धसन्नाहः अथ प्रभातेऽतिरथो धनंजयो रथं
समारोप्य सधन्विनां वरः।
 भार्यां सुभद्रां भगिनीं वरेश्शुभां वीथ्यां
प्रतस्थे कवचीधनुर्धरः।
 सारथ्यमकरोत्तत्र सुभद्रा भर्तृसंमता।
 धनुर्विष्पारयामास निषंगीघोषयन् दिनः।
 निषंयनिनदंघोरं गाण्डीव धनुरुत्थितम्।
 सन्नद्धा स्सहसोत्थस्थुरुद्विग्नाः पुरपालकाः।
 कस्त्वंपलायसे चोरकोजीवेनत्पुरतस्थितः।
 इत्यरुधन्महावेगान्मार्ग कृद्धधनुर्धराः।
 नानायुधधरावीराश्चतुरङ्गबलैस्सह।
 द्वारकापालकानां च पाण्डवस्यारुणोदये।
 5 तद्युद्धमभवद्धोरंतुमलं रोमहर्षणम्।
 धनञ्जयोमहातेजा गाण्डीवंसंस्मरन् हरिम्।
 आकर्णपूर्णमाकृष्य तीक्ष्णान्बाणान् मुमोच ह।
 ततोदृष्ट्वा महाघोरान् बाणानर्जुनलांछनान्।
 तेन भग्नाःपुरस्थातु मशक्तस्ते महाबलाः।
 संकेत भेरीमाजघ्नुर् यद्धोषो योजनत्रयम्।
 तेस्यभाणांश्च तच्चिह्नान् प्रेषयामासुरंजसा।
 
जित्वार्जुनोर्पितान्सर्वान् ययाशङ्ख च पूरयन्।
 मंदस्मितमहाभागं जितवन्तं पुरस्थितान्।
 किंचित्स्मित्वासुभद्रापि पाण्डवं परिषस्वजे।
 10 तत्स्थनालिंगनामोद सुखमन्वभवद्धृदि।
 स्वाराज्यपारमेष्ट्यावा मोक्षेवातीतमर्जुनः।
 भेल्तुहन्यमानायां द्वारकायामथध्वनिः।
 चतुर्योजनमाक्रमत् प्रळयांबुध घोषवत्।
 तत्र भेरीनिनादं तं शृत्वासर्वेपि यादवाः।
 सुप्तास्समुत्थितो वेगात्किमेतदिति संबमात्।
 उग्रसेनादयस्सर्वे बलभद्रपुरोगमाः।
 अन्योन्यमभिसंकुरुद्धाः कृष्णमूचुरिदं वचः।
 नूनंरुंदन्तिरिपवः पुरंनोरंद्रदर्शिनः।
 कथं वेगाद्गमिष्यामः कथंशत्रुपलायनम्।
 15 उदासीनस्तयास्माभिस्सवैरेवागतं पुरात्।
 धिगस्मद्बुद्धिदौर्बल्यं धिगस्मच्छास्त्र दर्शनम्।
 कथमेतादृशंकार्य कोजुगुप्सितमाचरेत्।
 कळंकस्तु कुलेस्माभिस्थापितो बलदर्पितैः।
 कथमेतादृशंसैन्यं नीत्वागच्छेन्महाजवात्।
 आगच्छादाय हे कृष्ण वहत्सैन्यमिदंशनैः।
 इत्युक्त्वाग्रे हली क्रुद्धः प्रतस्थेऽतिरथोरथी।
 इत्युक्तोऽथ हरिःप्राह पुरोयायिनमग्रजम्।
 आर्य! लोकत्रयेस्माकं न विद्यते हि शत्रवः।
 
महेंद्रा वा पुरंशक्तो लोचनेन निरीक्षितुम्।
20 भेरीताडनमद्यार्य न शत्रु विषयं भवेत्।
 अत्याहितमिवाभाति यत्किंचित्कस्यचित्पुरे।
 अस्मदंतःपुरेयोगी वर्ततेच यतीश्वरः।
 यत्किंचित्तस्यवा मन्ये यत्किंचित्तेनवा कृतम्।
 एक एव वसत्यस्मत् गृहेनान्योवलोकितः।
 अस्मत्स्वासा तथा बाला शुश्रूषां कुरुतेयतेः।
 निमित्तं दृश्यते किंचित् पुरस्तादशुभास्पदम्।
 अनंतरं महच्छ्रेयो भवेदेव न संशयः।
 अहमेव पुरोगत्वा तत्कालोचितमारभे।
 वर्तमानेच मय्यार्य त्वच्छिष्ये सर्वसाधके।
 25 भवादृशानांगमनं कथं युद्धेभविष्यति यदिस्याद्यतिहेतोर्वा यत्किंचिदशुभं गृहे।
 तं निहन्मि न संदेहस्संहार्यःपातकीयतिः।
 इत्युक्तवन्तं तं शौरि प्राह वाक्यं बलीहली।
 पूर्वमारभ्यतेकृष्ण यतावेवजितेंद्रिये।
 महद्वैरमभूद्धत मन्येथानान्यथा यतौ।
 अल्पप्रयोजने तात! न भेरीताडनं भवेत्।
 पुरे कलहएवस्यादाक्रान्तं रिपुभिर्बुवम्।
 इत्युक्त्वासैन्यकंसर्वं द्वितीयमिवसागरम्।
 आदायद्वारकांवेगात् दहन्निव रिपून् ययौ।
 30 इति संवदतां तेषां सन्नद्धानां परस्परम्।
 
पार्थाकंशरमादाय कश्चित् पुरुष आगतः।
 बलभद्रादयस्सर्वे तमपृच्छन् भयाकुलाः।
 तं दृष्ट्वा पुरुषंदीनं तीक्ष्णम् नामांकितं शरम्।
 कस्यायं निशितोबाणःकुत्रपार्थस्समागतः।
 अस्माकं पाण्डवानां च बंधूनां कलहः कथम्।
 दूतःनान्येकेचित्पुरी प्राप्ता न कुतश्चिद्भयं भवेत्।
 यतिस्सुभद्रामादाय रथस्थो निर्गतःपुरात्।
 प्रवृत्तं च महद्युद्धं यतिबाणैर्जितावयम्।
 इंद्रोवा त्रिपुरारियं दृष्ट्वा तं मेनिरे जनाः।
 35 तत एतादृशं दृष्ट्वा सर्व एव सुविस्मिताः।
 पराजिताश्चसंविग्नाः प्राप्तोहं प्रेषितोधतैः।
 एतावन्तं हि वृत्तान्तं ज्ञात्वा वेगादिहागतः।
 न्यवेदयं वृत्तमिदं ज्ञात्वायुक्तं विचिन्त्यताम्।
 ततः कोपाग्निनारामं प्रज्वलन्तमिवानलं उवाच वचनं क्रूरं कृष्णस्सर्व इवश्वसन्।
 इतःपरं तु किं कार्यं आर्यास्माभिरिवोचितम्।
 संपादितोमहानर्थस्सर्वैर्वृद्धि पुरस्सरैः।
 देशकालाविरोधेन पूर्वमेव मयोदितम्।
 तत्तथैवाभवत्सर्वं कास्यादत्र प्रतिक्रिया।
 40 युष्माभिश्शास्त्रतत्वज्ञैः वृद्धैश्चवयसौजसा।
 नश्रुतं बालबुद्धित्वाद्वचोपहसितं मम।
 
योगीवास्यात्तपस्वी वा वेदान्तज्ञानवित्तमः।
 पुरीस्तस्सर्वदासंगी विश्वासार्हो यतिःकिमु।
 स्वेच्छागतं वनगतं यतिमानीय पत्तने।
 तत्राप्यंतःपुरे हंत सुकृतं परिपालितम्।
 कोविद्वानंतिकेश्यामां कन्यकां विसृजेद्यतेः।
 विद्वांसोपि विमुह्यति कालोहि दुरितक्रमः।
 प्राणेक्षारमिवाभाति पुनःपुनरुदाहृते।
 गतेजले सेतुबंधः कवयः फण्यतामिह।
 45 अथवा किं विचारेण पाण्डवानामहं पुरम्।
 ध्वंसयिष्यामि नचिराद्वधिष्यामि च पांडवान्।
 एकःपापानि कुरुते फलं भुङ्क्ते महाजनाः।
 एको वंश कुपुत्रस्यात्पितरस्तुपतन्त्यधः।
 अर्जुनेनहतास्सर्वे पाण्डवा यतिरूपिणा।
 इत्युक्त्वा क्रोधतांराक्षः कृष्णोयोगीश्वरेश्वरः।
 उत्थायसारथिं प्राह रथस्संयुज्यतामिति।
 रामः कृष्णवचश्शृत्वा बलभद्रो महाबलः।
 कृष्णं प्रति सहामर्षः क्रोधाद्वचन मब्रवीत्।
 एकस्मिन्नर्जुने पापे यतिरूपो दुरात्मनि।
 50 दुर्वृत्ते कृष्ण किंकार्यमितरैर्धर्मजादिभिः।
 शठस्स एवहंतव्यश्चर्धकः कुलदूषणः।
 रावणेनकृतपापे हतःकिं वा विभीषणः।
 हिरण्येनकृतंपापं प्रह्लादेनमभूत्किमु।
 
इत्युक्त्वा हलमादाय मुसलं च हलायुधः।
 युगांताग्निरिवःक्रुद्धः प्रदहन्निवतेजसा।
 एक एवरथस्थस्तु प्रतस्थे सहसारिपुम्।
 ज्ञात्वा कोपानलं तस्य प्रलयाग्निसमप्रभम्।
 यादवाश्चोग्रसेनाद्या अतिक्रुद्धं हलायुधम्।
 नाशक्नुवन्नपिद्रष्टुं नरसिंहमिवस्थितम्।
 55 कृष्ण एवाग्रजस्याग्रे किंचित्भीतःकृताञ्जलिः।
 स्तंभयन् क्रोधसंरम्भमिदं वचनमब्रवीत्।
 किमप्यार्य न जानाति भवान् लौकिकसंग्रहम्।
 त्वय्येवं कुपितेकोत्र शुश्रूषुस्तवसंभवेत्।
 पूर्वमेव मयाप्रोक्तं भाविकार्यनिरूपकम्।
 न शृतंवचनंश्रीमन् तदेव फलितंबत।
 सर्वैर्विचार्यगंतव्यम् उग्रसेनादिभिःप्रभो।
 कर्तव्यःपार्थसंहार एतेषां संमतोयदि।
 
अनाकर्णैवमवद्वाक्यं सर्वशास्त्रेषु सम्मतम्।
 यथान्यदैवफलितं तथैवाद्यफलिष्यति।
 60 अभियुक्तश्चमन्वायैस्सन्यासो निंदितःकलौ।
 यतिं न विश्वसेत् प्राज्ञस्सर्वादातु कलौ युगे।
 सर्वज्ञेन त्वया सर्वं विस्मृत्यैवं कृतं विभो।
 इदानीं कुप्यसीव्यर्थं भाव्यर्थं दानवेक्ष्यवै।
 यदाऽद्राक्षंगुहासंस्थं मौनीभूत महंयतिम्।
 पुराभ्याशे यदानीतो हृदयं कंपितंतदा।
 
यदाविहाय विपिनं भवद्भिःप्रापितःपुरीम्।
 तदानीमेव सुमहाननरुथ इति निश्चितम्।
 किंच हास्य कथंवक्ष्ये सर्वलोकविगर्हितम्।
 
यतेस्समीपं कोवाद्य बालाम् प्रेषयतिस्त्रियम्।
65 यदादांतःपुरंनीतो भवद्भिस्तत्ववेदिभिः।
 तदैवास्मत्स्व सा बाला यतिशुश्रूषणच्छलात्।
 सन्यास्यासक्तहृदयास्या देवेतिहि विनिश्चयः।
 इदानीं च पुनः क्षोभम् करिष्यतिकुलेभृशम्।
 भवितव्यं भवत्येव कालोहि दुरतिक्रमः।
 तस्मादर्जुनसंहारात्किंश्रेयः प्रतिपद्यते।
 इत्युक्तोपि हलीक्रुद्धः कालाग्निरिव मूर्चितः।
 विजāभेऽनपेक्ष्यैनं पर्वणीवमहार्णवः।
 सर्वेपियादवाःक्रुद्धा हनिष्यामो यतिंशरम्।
 इतिसर्वेपिविक्रांता गच्छन्तं राममन्वयुः।
 ७0 गच्छन्नेवाब्रवीत्कृष्णो यतिं हन्याहमेव तम्।
 हतएव न संदेहो विरुद्धोस्माभिरर्जुनः।
 यदिदमिष्ठपित्रोर्वा क्रियतां पार्थनिग्रहः।
 इत्युक्त्वाथ पुनर्वाक्य मीषत्क्रुद्धो हरिस्तदा।
 अग्रज प्राहवाक्यज्ञः स एवार्हति भाषितुम्।
 यतिवेषोदुरात्मासौ हन्यतेचोच्छठोर्जुनः।
 कीदृशीस्यात्स्वसास्माक मत्रसाधुविचिंत्यताम्।
 पिताहि नानुभद्रेति यंनिन्द्यं श्रूयते वचः।
 
पतिहीनास्वसास्माकं शोचत्यग्रेशलोचना।
 तच्छ्रेयसेस्यादस्माकं मयावाहन्यतेर्जुनः।
 ७5 नैवाद्यविधिनाहृष्ट गांधर्वविधिना स्वयम्।
 पार्थं प्राप्ता न संदेहस्तत्पित्रोरपि संमतम्।
 सुभद्रायाश्चवैधव्यं मन्माता न सहिष्यते।
 तदश्रुमोचनं नोचेत्पार्थ मद्यैवहन्म्यहम्।
 अगस्त्यःइति युक्तियुतंवाक्यं शृत्वा कृष्णस्यमोहनम्।
 मन्त्राहतो यथासर्पस्तब्धदृक्समवैक्षत।
 साभिमानमिदं वाक्यम् पुनःप्राह हली हरिम्।
 वेषेणैव जनोमुह्येत् को वा जानाति हृद्तम्।
 त्वयाकृतमिदंसर्व मिति भाति न संशयः।
 सोर्जुनःस्त्वत्प्रयुक्तोवा स्वयमीदृग्विदोथवा।
 सर्वात्मनावधाच्छ्रेयो न हि स्याद्यदुसंततौ।
 80 श्रीकृष्णःकथमार्योपि मांब्रूषे सर्वज्ञोपि विचक्षणः।
 हन्तहंतात्र किं वक्ष्ये स्यादेवम् मयि मौनिनि।
 साक्षीश्वरस्स एवान विष्णुरंतस्थितो हरिः।
 पिदायच हरिःकर्णा वुच्छैर्हरिमुकीर्तयत्।
 अथैनं सांत्वयन् रामो बभाषे बंधुसंनिधौ।
 मनक्षोभादिदं प्रोक्तं किमर्थं वत्स कुप्यसि।
 इतःपरं वा किं कार्यम् यत्कृत्वा मङ्गलं भवेत्।
 
यथालोकापवादोनो न भवेत्तद्विचिंत्यताम्।
 श्रीकृष्णःमौनीभूतेद्यमय्यार्य संग्रामो वै महानभूत्।
 किमप्युक्ते भवेत् किं वा भवतां यद्धिरोचते।
 नोचेद्यदिष्ठं मत्पित्रोस्तच्छ्रुत्वा क्रियतां तथा।
 85
रामः
तच्छ्रुत्वा वचनं तस्य हली प्राह महात्मनः।
 भवानेव परिज्ञाता त्वद्युक्तिलॊकपूजिता।
 श्रीकृष्णःभवद्भिरेवशास्त्रज्ञः संविचार्योच्यतां गुरो।
 अथवा मन्मतं वच्मि श्रूयतां यदिसंमतम्।
 येनकेनाप्युपायेन पित्रोःकार्यं सदाप्रियम्।
 अथवा स्वयमेवात्र पार्थ प्रीत्यासमावृणोत्।
 सोपि बंधुरयंश्रेयान् गांधर्वोबाहुजन्मनाम्।
 कोपवादोत्र को दोषः कस्यप्रीतिरिवाग्रहे।
 आर्यास्मत्संततावेव देवयानीतुभार्गवी।
 ययातिं सर्वधर्मज्ञमवृणोदिति नशृतम्।
 ९० शकुंतलामहाप्राज्ञा दुष्यंतेन महात्मना।
 गांधर्वेणविवाहेन स्वधर्मेणोररीकृतौ।
 तस्य राज्ञोमहाप्राज्ञो भरतोनाम धार्मिकः।
 संजज्ञेपाण्डवास्तेन भरता इति विश्रुताः।
 शन्तनुश्चमहाराजा जाह्नवीं लोकपावनीम्।
 
उपयेमेस्वयं प्रीत्या तस्यजज्ञेसुतस्तथः।
 भीष्माख्यो वैष्णवश्रेष्ठो वर्ततेद्यापि नशृतम्।
 बहुनाकिमुहोक्तेन दृष्ठांतो त्राहमेव किम्।
 अवमत्यन पित्रार्थी रुक्मिणीमां किमावृणोत्।
 अनेन साधुमार्गेण स्वसास्माकं च पाण्डवम्।
 अंगीचकारसोप्येनां तथैवोक्तं मयाधुना।
 ९5 एवमेव ततस्सर्वे गत्वेंद्रप्रस्थमञ्जसा।
 कल्याणाभिनयं कृत्वा लौकिकंस्यामनिर्वृताः।
 एवं कृतं चेत्संतोषः पित्रोरस्मत्स्वसुस्तथा।
 लोकश्च साधुमन्येत श्रेयोस्माकं भवेन् महत्।
 98 इति अर्जुन बलरामयोर्युद्धसन्नाहनाम
एकविंशोऽध्यायः।

*********************
अथ द्वाविंशोऽध्यायः अर्जुनस्य इन्द्रप्रस्थंप्रतिगमनम् श्रीकृष्णेन पाण्डवार्जुनं प्रति तुलाकावेरिमहिमानुवर्णनम्।
 अगस्त्यःइति शृत्वा हरेर्वाक्यं सर्वतोभद्रमृद्धिमत्।
 प्रशशंसुर्मुदाकृष्ण मुग्रसेनोद्धवादयः।
 अथोद्धवाक्रूरमुखै रुग्रसेनो यदूत्तमैः।
 संयगालोच्य सस्मेरं रामकृष्णावभाषत।
 उग्रसेनः
प्रसिद्धं कुरु तं गत्वा युवामेवसुतोत्सवम्।
 दत्वा च यौतकं संयक्पार्थं संमान्यसंसदि।
 भूषणांबररत्नायै शीघ्रप्रति निवर्ततम्।
 इति शृत्वार्थसंतोषा दुग्रसेनोदितं वचः।
 ततःप्रतस्थे भगवान् बलभद्रोबलान्वितः।
 कृष्णश्चपार्थकल्याणम् निर्वृत्तयितुमादरात्।
 5 उद्दवस्सात्यकिश्चैव ह्यक्रूरश्चार्यचोदितः।
 द्वादशाक्षौहिणीयुक्ता निर्ययुर्बलदर्पिताः।
 प्रस्थाप्यैवं हरिप्रस्थम् हरिहरिहयोपमम्।
 राजाद्वारवतीं प्राप ससैन्यस्समलंकृताम्।
 अथ धर्मात्मजंकृष्णम् कृष्णदासोबलान्वितम्।
 प्रत्युद्धयौ मुदाशंखभेरीनिस्साणनिस्स्वनैः।
 ततो धर्मात्मजं रामकृष्णावत्यन्तहर्षितौ।
 ववंदातौ सुरेशानौ मायामानुषचेष्टितौ।
 भीमाद्याब्रातरस्तस्य बलभद्रम् ववंदिरे।
 कृष्णाभिवादयामास भीमं माद्रीसुतौच तम्।
 10 अर्जुनेन परिष्वक्तस्सस्वजे धर्म नंदनम्।
 राजापि स्वागतं पृष्ट्वा रामकृष्णामुदान्वितः।
 प्रविवेशपुरी ताभ्यां नानालंकारशोभिताम्।
 ततो रामं च गोविंदम् प्रापय्य निजमंदिरम्।
 रत्नैराभरणैर्दिव्यैः लेपनाद्यैरपूजयत्।
 ततो रामश्च कृष्णश्च कुन्तींतामभ्यवंदताम्।
 
ततस्सुभद्रा सव्रीळा मंदं मंदमुपागता।
 नमश्चक्रे मन्दहासा रामकृष्णौ जगत्पती।
 युधिष्ठिरःहे राम हे रमानाथ! प्रपन्नार्तिहरौयुवाम्।
 मद्गृहं स्वयमायातौ तपस्सिद्धाहमेव हि।
 15 अद्य मे सफलं जन्म अद्य मे सफलं कुलम्।
 इत्युक्त्वा पाण्डवश्रेष्ठो जगदीशौ निरीश्वरौ।
 सर्वस्वेनापि राजेंद्र संमनार्हवपूजयत्।
 श्रीकृष्णःइत्थमत्र न वक्तव्यं पितृष्वसृ सुतप्रभो।
 वयं बालाश्च पोष्यास्ते बंधुस्नेहात्प्रशंससि।
 अस्माभिःकाशितं भव्यं यदस्मद्भगिनी शुभौ।
 सुभद्राकृतवद्यद्य तन्नोभद्र मिहाभवत्।
 वसुदेवोग्रसेनाद्या स्सुभद्रोद्वाह हर्षिताः।
 त्वदग्रेचागतादातु मावामर्थम् च यातकम्।
 भवंतो बहुमन्यंते स्वल्पांवास्मत्कृताम् शुभाम्।
 सपर्यामार्यसम्पूर्णां विराजंतोमहत्तया।
 20 उपेयेमे सुभद्रांनो भगिनीं फल्गुनस्स्वयम्।
 इदमेव हि कल्याणं चिरेणाभिमतम्चनः।
 इत्युक्त्वादंपती तौतु रामकृष्णौ सभान्तरे।
 सिंहासनेतु संस्थाप्या नर्चातुर्भूषणोत्तमैः।
 अमोघाभिर्महाशीर्भि भूषयित्वा द्विजन्मनाम्।
 
आरार्तिकं मङ्गलम् च कारयामासतुर्मुदा।
 स्त्रीधनं च सभामध्ये ददतुस्तौ मनस्विनौ।
 दशकोटिस्सुवर्णानां रजतानां चतुर्गुणम्।
 भूषणानिच मुख्यानि कोटिनिष्कृतानिच।
 क्षौमाण्यपि च मुख्यानि तथासर्गांबराणि च।
 सहस्राणि च पंचाश दश्वानां सिंधुजन्मनाम्।
 25 रथानां धर्शनीयानां बालार्कसमतेजसाम्।
 सहस्राणिच चत्वारि गजानामयुतं तथा।
 पत्तिनां मदमत्ताना मरघ्नी कोटिमेवच।
 दासीनां निष्ककण्ठीनां श्यामानां त्रिसहस्रकम्।
 हरमेकमनघु च जाम्बवत्युत्सवेर्पितम्।
 ऋक्षराजेनकृष्णाय योदत्तोहृष्टचेतसा।
 एवं संपूज्य बीभत्सुं भगवात्सु हलायुधः।
 भगिन्यैच सुभद्रायै सुभद्राभरणान्यदात्।
 पंचाशत्कोटिनिष्काहाँ रशनां रत्नमौलिकाम्।
 द्वात्रिंशत्कोटिनिष्काहाँ ताटंकयुगळम् तथा।
 30 नासिकाभरणम् दिव्यं कोटिनिष्काईमौक्तिताम्।
 कम्बळानि च दिव्यानि नाना दिव्यानुलेपनम्।
 पंचाशत्कोटिनिष्काहाँ सुनदीमातृकां भुवम्।
 दंपतीतौ समभ्यर्च्य रामकृष्णौ मुदान्वितौ।
 आपौषमूषतुस्तत्र प्रत्यहं पूजितौच तैः।
 प्रतिपूज्य ततःपार्थ रत्नाभरणवस्त्रकैः।
 
तोषयित्वा तथा कुंतीमन्वहं शुभयागिरा।
 विश्लेषदुःखाद्रुदतीं दृष्ट्वा गन्तुं समुद्यतौ।
 आर्ये नस्साधुमन्यस्य स्वस्मर्तव्या स्सततंत्वया।
 भविष्यतीच कल्याणं सर्वधात्रभजामहे।
 35 अस्मत्स्वसात्वियं बालापरमार्था विचारिणी।
 शिक्षणीयाप्रतीक्ष्याच ब्रातृपुत्री तवप्रिया।
 इति कुन्तींसमाभाष्य सस्नेहवचनैश्शुभैः।
 कृष्णम् जगदतुस्साध्वीं जगन्नाथौशुभांगिरम्।
 साध्वीद्रौपदकल्याणी त्वं सर्वं वेत्सि लौकिकम्।
 बांधवत्वेन यद्भूम श्श्रोतव्यमनसूयया।
 सपत्नीत्वं तु नारीणाम् कण्ठरोगवदुच्यते।
 एतत्सामान्यतोलोके साध्वीनां न तथाकिल।
 इयं सुभद्रा ते वत्सा सदाशुश्रूषणेरता।
 साधुबुद्ध्याप्रतीक्षस्व त्वदाधारमिदं गृहम्।
 40 या नारीकुटिलागेहे तद्नेहं नाभिवर्धते।
 नश्येत्कुयोषिताम् सर्वं गृहिणी गृहमुच्यते।
 सर्वं विचिन्त्यशास्त्रार्थं तत्वबध्यवधारणात्।
 अस्माकं भगिनींबाला संयक्पालयसौहृदात्।
 इत्युक्त्वाभगिनीमिष्टा मापृच्छ्यबलकेशवौ।
 वस्त्रैराभरणैर्दिव्यैः कृष्णां सम्पूज्यचोचतुः।
 सा तद्वाक्यंशुभंशृत्वा द्रौपदी पतिदेवता।
 सस्मितंप्राह तौ मन्द मानंदाच परिप्लुता।
 
लोकेसाधरणम् नार्यः कुटिलास्वार्थ तत्पराः।
 विश्वसेद्वामहासर्प स्त्रियं तम् न कदाचन।
 45 न तथाब्रूहिमामार्य मज्जन्मेनाभिशंसति।
 त्वद्भक्त्या साधयेसर्वं त्वद्भक्ताकुरुमाम् हरे।
 इति प्रसादितःकृष्णः कृष्णयाकृतकृत्यया।
 सुभद्रां भगिनींप्राह भगवान् भूतभावनः।
 कल्याणवंशजेवत्से मद्वाक्यं शृणुसिद्धये।
 पितृमातृकुलव्याधिर् भर्तुश्चेत्प्रतिकूलिका।
 भूषणं पितृवम्शस्य यदिसाद्वी पतिव्रता।
 अनुकूलासदाभर्तुः भवभव्यप्रदासदा।
 प्रत्युत्तरं तु नब्रूयाः पत्यौःकुद्धे कदाचन।
 माकुरुष्व सपत्नीत्वं कृष्णायां चरसौहृदम्।
 50 मातरम् पितरम्वापि मास्मरान्मा कदाचन।
 पतिरेवस्त्रियाबंधुः पतिरेवस्त्रियोधनम्।
 पतिरेवस्त्रियोधर्मः पतिरेवमहत्तपः।
 गुरुनिंदापरश्शिष्यः पितृनिंदा परस्सुतः।
 पतिनिंदापरानारी विण्मूत्रेनरके पतेत्।
 स्त्रिय एव कलौमृत्युर् विधात्रा कल्पितेगृहे।
 नारीसामान्यमेतत्तु क्वचित्साध्वी प्रकल्पते।
 पुरा सीतापिसौमित्री बहूक्त्वादुर्वचस्सती।
 संपाद्यदुःखंभर्तुश्च स्वयं प्रोवाच संकटम्।
 निदर्शयंतीनारीणां दुश्शीलं पापमानसम्।
 55
लोकमाताकरोदेवं लोकरक्षार्थमागता।
 स्त्रैणेजितेंद्रियोलोके पापीस्यादति दुःखितः।
 इतिप्रदर्शयन् रामः प्रालपत्प्राकृतोयथा।
 तस्मात्सर्वात्मना भद्रे माकृदा भर्तुरप्रियम्।
 भूषयंतीकुलं पित्रोः श्रियःपरमवाप्स्यसि।
 इत्युक्त्वा मूर्युपाघ्राय सान्त्वयित्वाऽरुलोचनाम्।
 आमन्त्र्यपाण्डवान् सर्वान् प्रतस्थे स हलीहरिः।
 ततः कृष्णम् नृपोदृष्ट्वा प्रतिष्ठासुं युधिष्ठिरः।
 दुर्योधनःकृतान् क्लेशान् स्मरन् साश्रुरभाषत।
 युधिष्ठिरःमत्तोभाग्यविहीनोस्ति मातुलेयोविभूतले।
 यद्वयं बाल्यमारभ्य क्लेशाननुभवामहे।
 60 पित्राविश्लेषिताबाल्ये पितृव्येणाप्युपेक्षिताः।
 पुरान्निर्वासितास्सर्वे वागणावतपत्तने।
 लाक्षगृहानले हन्तुं वंचिताश्च पुरोचनात्।
 एकचक्राख्यनगरे पुनर्भेक्ष्यैक जीवनाः।
 कृष्णा विवाहमाकर्ण्य दृतराष्ट्रेण पत्तने।
 आहूय बांधवाग्रेतु पुनर्निर्वासितावयम्।
 मातुलेनभवद्भिर्वा कदाचिन्नविचारिताः।
 दरिद्रस्य कुतोबंधुः दरिद्रस्य कुतस्सुखम्।
 दरिद्रस्यकुतो धर्मो मृतप्रायो हि निर्धनः।
 यस्येश्वरःकृपापूर्णा तस्य संपद्भवेद्धवम्।
 65
न यथा शत्रुबाधास्याद्यथैश्वर्यं हरेऽक्षयम्।
 यथावंशोभिवृद्धिस्स्यात् यथाधर्मोभिवर्द्धते।
 सर्वं विचार्यतत्वेन ब्रूहि मे पुरुषोत्तम।
 इत्युक्त्वाविररामाथ साश्रुकण्ठो युधिष्ठिरः।
 श्रीकृष्णःकिमिदं प्राकृतांबुद्धिं कुरुषे नृपसत्तम।
 तत्वज्ञोपि हि संसारे कथं मुह्यसिमूढवत्।
 गतिरेतादृशी राजन् संसारनरकार्णवे।
 अस्माकं कीदृशं दुःखं नाभवत्तद्विचार्यताम्।
 अस्माकं मातुलःकंसः किंकिंवानचकारवै।
 न क्षान्तावा जरासंधकृताः क्लेशामहीपते।
 ७0 वाचामगोचरास्सर्वे वयं किं न सहामहे।
 जन्मान्तरकृतंपापमेवं रूपेण बाधते।
 तस्माद्विचार्य शास्त्राणि कार्योधर्मस्समंजसः।
 माविचारय राजेंद्र! सर्वश्रेयोभवेदितः।
 त्वं हि धर्मरतोनित्यं त्वं हि साधुप्रपूजकः।
 इतःपरं महावृद्धिर्भवेदेव न संशयः।
 शत्रवश्च हतास्स्युस्ते सांराज्यं च भविष्यति।
 धनुर्मासे हरिःप्रीत उषःकालार्चनाद्ध्वम्।
 ददात्यभीष्टमक्षय्यं सत्यमेतद्रवीमि ते।
 धनुर्मास उषःकाले स्नात्वा संयग्जनार्दनम्।
 ७5 समभ्यर्च्य च मुद्ान्नं दध्यन्नं च निवेदय।

भोजयद्विजवर्यांश्च प्रातःकाले यथा बलम्।
 भुक्त्वैवसकलान् भोगान् प्राप्यते वैष्णवं पदम्।
 निवेदयित्वा मुद्दान्नं पुरेंद्राणीसहाईकम्।
 वैष्णवे दुःखितासाद्वी नित्यैश्वर्यसुमंगला।
 मन्त्रं तु लक्ष्मीहृदयं जजापैवं श्रियस्तवम्।
 तौलौस्नात्वा तु कावेर्यां सर्वं कृत्वा तथाशुचिः।
 ततो लक्ष्मीकटाक्षेण सुरराजंसमैहत।
 अगस्त्यःधर्ममेतं महामन्त्रं गोविंदस्तमुपादिशत्।
 श्रीदेवीप्रथमं नाम द्वितीयममृतोद्भवा।
 80 तृतीयं कमलाप्रोक्ता चतुर्थं चंद्रशोभना।
 पञ्चमं विष्णुपत्नीच षष्ठं श्रीवैष्णवी तथा।
 सप्तमं तु वरारोहा ह्यष्टमं हरिवल्लभा।
 नवमं शाङ्गिणीप्रोक्ता दशमं देवदेविका।
 एकादशं महालक्ष्मीर्वादशं लोकसुंदरी।
 त्रयोदशं समाख्याता सर्वाभीष्टफलप्रदा।
 श्रीःपद्माकमला मुकुंदमहिषी लक्ष्मीस्त्रिलोकेश्वरी।
 माक्षीराब्धिसुधाविरिच जननी विद्यासरोजासना।
 सर्वाभीष्टफलप्रदेतिसततं नामानिमेद्वादश।
 प्रातश्शुद्धतराःपठन्त्यभिमतांत्सर्वान् लभंतेशुभान्।
 इत्येतद्गोष्पतिप्रोक्तं लक्ष्मीहृदयमुत्तमम्।
 तौलौस्नात्वा च कावेर्यां जप श्रीवृक्षसन्निधौ।
 85
तौलौ सह्योद्भवातीर्थे श्रीरङ्गेस्नानमाचरन्।
 जपेस्ततःकमलास्तोत्रं सर्वान्कामानवाप्नुयात्।
 कावेरीस्नानमाहात्म्यमवाङ्मनसगोचरम्।
 तत्यासन्नस्तुलामासे नभस्यांतोयमागतः।
 तुलाकावेरीमाहात्म्य श्रवणं च दिनेदिने।
 कुरुराजन् मयाप्रोक्तं तुलामासे विशेषतः।
 तस्मिन् वस्त्रम् सुवर्णम् च ताम्रपात्रादिकं तथा।
 राजतम् कांस्यपात्रं च तांबूलं तण्डुलादिकम्।
 गुग्गुलं घृतसंयुक्तदीपंदेहीतथाकुरु।
 नाळिकेरफलादीनां दानं ब्राह्मणभोजनम्।
 ९० यदिवांछसिकल्याणं भवाब्धिं तर्तुमिच्छसि।
 तौलिके कुरुकावेरी स्नानं चाभीष्टदं शुभम्।
 इतिप्रभोदयन् धर्मं धर्मराजाय केशवः।
 गन्तुं प्रचक्रमेसर्वा नामन्त्र्य स हलायुधः।
 गोविंदं तेजिगमिषू तदा निश्चित्यपाण्डवाः।
 भूषणांबररत्नौवै संपूज्य विरहातुराः।
 तुलाकावेरिमाहात्म्य मपृच्छन् कमलापतिम्।
 ज्ञात्वा तेषामभिप्रायं भगवान् भक्तवत्सलः।
 मेघगंभीरयावाचा प्रहसन् प्राह पाण्डवान्।
 श्रीकृष्णःकावेरीसरितां श्रेष्ठा सर्वलोकैकपावनी।
 तस्याश्च वैभवं वक्ष्ये शृणुताद्य कुरूद्वहः।
 ९5
यत्फलं द्वादशाब्दं तु सप्तसागरमज्जनात्।
 तत्फलं समवाप्नोति तुलाकावेरि मज्जनात्।
 हरस्रग्दक्षिणाशायी मुक्तेस्सोपानसंततिः।
 नृणां साम्राज्यदानकदीक्षिता दक्षिणादिशी।
 सर्वाभीष्टप्रदादेवी सर्वपापप्रणाशिनी।
 सर्वदुःखोपशमनी सर्वदारियनाशिनी।
 स्नानात्सकलरोगनी कावेरीकलिनाशिनी।
 कीर्तनात् सर्वदोषघ्नी ब्रह्माद्रेर्ब्रह्मणस्सुता।
 साक्षाद्ब्रह्ममयीदेवी ब्रह्मलोकप्रदायिनी।
 प्रवृत्तालोकरक्षार्थं दिव्यतोयमनोहरा।
 अस्याप्रभाव को वेत्ति वेदांतेषु प्रकीर्तितम्।
 100 यानीतिर्थानिसर्वत्र पुण्यास्सर्वाश्च निम्नगाः।
 कावेर्याब्रह्मरूपायाः कलांनाहँतिषोडशीम्।
 एतज्जलेन पक्वान्नं केवलंशाकमेववा।
 गयाश्राद्धायुतफलं भजेद्दत्वा न संशयः।
 मद्वंशजास्तुपुत्राद्या दारस्यां तिलोदकम्।
 अस्यां स्नायात्सकृत्पुत्रा भवेन्मुक्तऋणत्रयाम्।
 इतिशंसंतिपितरः परस्परसमागताः।
 वयमेवकृतार्थास्म भजामस्तृप्तिमक्षयाम्।
 इति शंसंतिपितरः परस्परसमागताः।
 अतिपापादुराचारा मानवाभूसुरादयः।
 वेदबाह्यादुरात्मानो गुरुविप्रविदूषकाः।
 
एकादश्यां प्रभातेतु सकृत्स्नात्वाऽमलेजले।
 योगिगम्यं यांति लोकं पुनरावृत्तिवर्जितम्।
 105 यत्रकुत्र तुलामासे कावेर्यामरुणोदये।
 प्रातस्नानं प्रकुर्वती शृत्वामाहात्म्यमुत्तमम्।
 धर्मवक्तारमर्चति पुराणज्ञं सतांमतम्।
 गंधपुष्पाक्षतैर्दिव्यैर् दुराचाराश्चपापिनः।
 पाषण्डाःपतिताश्चैव कितवानास्तिकस्तथा।
 ते सर्वा पितृभिस्सार्धम् मत्कटाक्षेनपाविताः।
 भुक्त्वैहसकलान्भोगान् पुत्रपौत्राभिनंदिताः।
 मम सायुज्यमायांति भवान् मुक्ता न संशयः।
 यस्तुस्नातु तुलामासे पतितोपिसकृन् नृप।
 सर्वपापविनिर्मुक्तो ब्रह्मलोके महीयते।
 110 यो नैमिशे कुरुक्षेत्रे त्रिवेण्याम् वसुधारके।
 तुलापुरुषदानानि कुर्याद|दयेशतम्।
 तस्मादप्यधिकस्सोयं यस्स्नायात्सह्यजाजले।
 तुलारवौ सह्यजायाम् सकृद्दद्यात्तिलोदकम्।
 पितरस्तस्यसंतृप्ता ब्रह्मलोकं प्रयांति च।
 वर्णभ्रष्टोपि वा पार्थ! परान्नादी निरग्निकः।
 सदाचारपरिभ्रष्टो मातापितृविदूषिकाः।
 सद्य एव विशुद्धस्यात् कावेरीस्नानमात्रतः।
 जन्मांतरतपोयोगात् कावेरीस्नानमाचरन्।
 येन्नदानम् सदाकुर्यात् सर्वभूतदयान्वितः।
 
स विष्णुलोकमाप्नोति पुनरावृत्तिवर्जितम्115।
 तुलामासेतु कावेर्यां योनित्यं स्नानमाचरेत्।
 सार्धत्रिकोटि तीर्थेषु स्नातस्स्यात् सतुभूपते।
 तुलामासेतु कावेर्यां स्नात्वा प्रातर्जलेशुभे।
 
अन्नंददद्रासमात्रम् सार्वभौमोभवेन्नरः।
 योदद्याद्वौश्वदेवार्थम् तण्डुलम् ब्राह्मजन्मनाम्।
 न तस्यलोकाःक्षीयंते यावदाचंद्रतारकम्।
 शाकंतिलं च ताम्बूलं गव्यं वा महिषंघृतम्।
 चंदनंवस्त्रमित्येषां दानं शस्तं तुलारवौ।
 तस्मात्सर्वात्मना राजन्! कावेरीस्नानमाचर।
 मदुक्तदाननिरत स्सर्वैश्वर्य मवाप्स्यसि।
 1 20 अगस्त्यःइत्युक्त्वा पाण्डवान् कृष्णः तुलामासस्य वैभवम्।
 शीघ्रम् जिगमिषुःकृष्णो द्वारकां प्राह धर्मजम्।
 अत्रागतस्य मे पार्थ महान्कालोत्यवर्तत।
 शोचंतिचोग्रसेनाद्या ममागमनकाङ्क्षिणः।
 यदायदा स्मरिष्यंति भवंतो मयि सौहृदात्।
 भविष्यति महच्छ्रेय आगमिष्येहमंजसा।
 वयं त्यक्त्वा च मधुरां जरासंधप्रपीडिताः।
 द्वारकाख्यां समुद्रांतःपुरी प्राप्ता अलौकिकीम्।
 यस्यास्तिवित्तं स नरःकुलीनः
स पण्डितस्सशृतवान् गुणज्ञः।

स एव सर्वस्य च दर्शनीय
स्सर्वेजनाः कांचनमाश्रयंति।
 125 अस्मान्स्मरत राज्ञस्था यस्माद्यूयं धनैर्युताः।
 न विस्मर्यावयं रिक्ताः कदाचिदपि पांडवाः।
 इत्युक्तामन्त्र्यतान् सर्वान् कृष्णातीव मुदान्वितः।
 कुंती नत्वा समापृच्छ्य कृष्णां कृष्णोथनिर्ययौ।
 अनुजग्मुर्हरि पार्थः श्चतुरंगबलान्विताः।
 रथाधिरूढोह्यगमद्रथ्यामध्ये यदूत्तमः।
 महतावाद्यघोषेण दिशोदशविनादयन्।
 महाराजो सुवव्राज जगन्नाथं युधिष्ठिरः।
 एतं महोत्सवं द्रष्टुम् सर्वलोकमहोत्सवम्।
 विमानस्थास्सुराव्योम्नि निरंतरमुपागमन्।
 130 ववर्षुःपुष्पवर्षाणि ननृतुश्चाप्सरोगणाः।
 देवदुंदुभयोनेदुः जगुर्गंधर्वयोषितः।
 वेदस्तुतिभिरग्र्याभि रीडिरे तं मुनीश्वराः।
 अनुजग्मुस्तथाकृष्णं पांडवास्तुद्वियोजनम्।
 निवर्तितास्ततस्तेन कृष्णेन नृपपुंगवाः।
 पदात्पदं न चलितु मसमर्थास्तुपांडवाः।
 पथिविश्लेष समये रुरुदुर्वृशदुःखिताः।
 श्रीकृष्णःप्राकृतानामिदं दुःखं न कदाचिद्भवादृशाम्।
 गतागतं करिष्यामो वयं यूयं च पाण्डवाः।
 
गतिरेतादृशीलोके बंधूनामिहदुस्त्यजा।
 इदं सर्वं समालोच्य धीराभवतनित्यदा।
 135
अगस्त्यःकृच्छात्प्रापुःपुरीं दिव्यां कृष्णेन विनिवर्तिताः।
 सहरामः पुरीप्राप कृष्णेन समलंकृताम्।
 प्रत्युद्युतःपौरजनैः द्वारकांद्वारकोत्सवः।
 षत् षष्टिकोटितीर्थानि द्विसप्तभुवनेषु च।
 तानिसर्वाणि राजेंद्र तुलासंप्राप्तेदिवाकरे।
 विष्ण्वाज्ञयासमायांति स्नानार्थं सह्यजांभसि।
 तस्मात्त्वं भूबृतांश्रेष्ठ! दानं कुरु मयोदितम्।
 दानेन हि भवेत् स्वर्गो दानेन हि भवेत् सुखम्।
 सत्पात्रदानहीनस्तु स्वमांसं प्रेत्यखादिति।
 तस्मात् स्नात्वा तुलामासे प्रत्यहं दानमाचर।
 दिनं शून्यमकृत्वा तु व्रतांते च पुनर्नृप।
 140 द्विजानां परमान्नेन सहस्रं भोजयाधिप।
 तन्माहात्म्यप्रवक्तारं संपूजयधनादिभिः।
 राजयानं समारोप्यबहुमानपुरस्सरम्।
 महतावाद्यघोषेण ह्यलं कृत्यपुरं गृहम्।
 ।
 ग्रामं प्रदक्षिणीकृत्य प्रापयित्वा स्ववेश्म च।
 143 सर्वस्वेनापि राजेंद्र! संपूज्याभीष्टमाप्नु हि।
 
इति श्रीकृष्णेन पाण्डवार्जुनप्रति तुलाकावेरिमहिमानुवर्णनम् नाम द्वविंशोऽध्यायः।

**********************
अथ त्रयोविंशोऽध्याय हरिश्चन्द्रप्रति अगस्त्येन कावेर्युत्पत्तिकथनम् दाल्यःइति धर्मान् शुभान् शृत्वा पावनान् कुंभजोदितान्।
 हरिश्चंद्रो प्रहृष्टात्मा पुनःपप्रच्छ सादरम्।
 हरिश्चंद्रःभगवन् योगिनांश्रेष्ठ कुंभयोने महामते।
 कृतकृत्याहमेवाद्य त्वत्पदांभोजसेवनात्।
 नमस्ते योगिवर्याय नमतुभ्यं त्रिमूर्तये।
 नमस्ते मुनिवर्याय नमस्तेदीनबंधवे।
 सर्वेधर्माश्शृताःपुण्या भुक्तिमुक्ति फलप्रदाः।
 विशेषणसमाश्रौषं कावेर्यादिव्यवैभवम्।
 सत्यं प्रसन्नोभगवान्मुकुंदो
___ ममेहविष्णुस्सनकादिवंद्यः।
 नोचेन्ममस्यादिति साधुसंगो
क्तिप्रदो यज्ञजपोपलभ्यः।
।
 कावेरीवैभवंशृत्वा न तृप्तिर्जायते मम।
 अतः पुनस्त्वां पृच्छामि तद्भवान् क्षतुमर्हसि।
 कावेरी सह्यसंभूता लोपामुद्रेति सा कथम्।
 कदा दक्षिणगङ्गेति विशृता लोकपावनी।
 कथं सह्याद्रिसंभूता गङ्गाधिक्यं कथं पुनः।
 
एतत्सर्वं तु विस्तीर्य ब्रूहि मे मुनिपंगव।
 एवं राज्ञा स पृष्टोथ ह्यष्टमूर्त्यांश संभवः।
 मन्दं स्मित्वा प्रशस्त्यैनं व्याजहार मुनिपम्।
 
अगस्त्यः
साधुसाधु महाराज! त्वमेवसुकृतीभुवि।
 धर्मप्रसंगे यच्छ्रद्धा पुनःपुनरभूत्तव।
 10 पुण्यश्लोकाग्रणीस्त्वं हि लोकानुग्रहकाम्यया।
 धर्मान् पृच्छसि राजेंद्र संत एव सतांधनम्।
 संत एव सतांबंधु स्संत एव सतां तपः।
 संतएव सतांमित्रं संतएव सतांव्रतम्।
 तस्माद्ब्रवीमि कावेर्याः प्रभावं पुण्यवर्धनम्।
 इति प्रशस्यतं योगि कावेर्याः पुण्यवैभवम्।
 कुम्भयोनिर्महातेजा व्याहर्तुमुपचक्रमे।
 केशवे द्वारकां यांते धर्मजेन महात्मना।
 अस्मिन्नर्थे पुरापृष्टो दौम्यनाममहामुनिः।
 कावेरी संभवं सर्वं धर्मपुत्रायसोब्रवीत्।
 15 दौम्यःकवेरो नाम राजेंद्र! राजर्षीरमितप्रभः।
 योगिवर्यःप्रसन्नात्मा सर्वविद्या विशारदः।
 जितेंद्रियोजिताहारो निस्संगो निष्परिग्रहः।
 विरक्तस्सर्वधर्मेषु किंचित्कालं तु कर्मठः।
 मुमुक्षुरभवच्छ्रीमान् कर्मकृत्वा सुदुष्करम्।
 
हिमवत्पर्वतेरम्ये तपस्तेपेसुदारुणम्।
 कावेरी योगिनस्तस्य तप्यतस्तप उत्तमम्।
 दिव्यवर्ष सहस्रांते ब्रह्मागम्यतमब्रवीत्।
 
ब्रह्मा
वरं वृणीष्व राजेंद्र! वरदोहमिहागतः।
 राजा तद्वचनं शृत्वा कृतांजलिरभाषत।
 20 कवेरःप्रसन्नो यदि मे देव तपसोस्तिफलं यदि।
 भवतामुक्तिमाकांक्षे किमन्यैर्नश्वरैःफलैः।
 
ब्रह्मा
न वयम् मोक्षदाने तु समर्थास्सकलास्सुराः।
 स एव मुक्तिदस्सत्यं परंब्रह्माऽच्युतस्स्वयम्।
 मम कन्या जगन्माता विष्णुमाया महामुने।
 त्वत्पुत्रीत्वं गता देवी तव मोक्षम्प्रदास्यति।
 इत्युक्त्वा सो स्मरन् मायां विष्णोर्लोकविमोहिनीम्।
 उपतस्थे विशालाक्षी सर्वाभरणभूषिता।
 सा कन्या चिन्मयीसृष्टा देवगंधर्वसंस्तुता।
 पितामहस्तामालोक वाक्यमेतदुवाचह।
 25 बद्रेस्य योगिनोदेवी कन्यात्वं गच्छमुक्तिदा।
 निदीभूत्वाथकावेरी मोक्षमार्गकसाधनी।
 सर्वतीर्थमयीपुण्या लोकांस्त्वं पालयिष्यसि।
 लोपामुद्राख्ययादापि त्वमेकांशेन शोभने।
 
भवभार्याप्यगस्त्यस्य योगींद्रस्यमहात्मनः।
 इत्युक्त्वान्तर्दधे ब्रह्मा हंसारूडोमरैस्सह।
 गतेब्रह्मणि साशक्तिर्विष्णो भगवतोहरेः।
 कमनीयाकृतिःकन्या कवेरस्य मुनेरभूत्।
 तमादाय वरारोहं कन्यां कमललोचनाम्।
 जहर्षभार्ययासार्धं साक्षान्मोक्षाङ्गतोयथा।
 30 कर्मीभूत्वा गृहेकिंचित्कालं विश्रांतिमागतः।
 दृष्ट्वा कर्मचदुष्पारं नैराश्यं परमंगतः।
 कदाचित्प्राह कन्यांस्वां कवेरःकनकाकृतिम्।
 कवेरःपितृत्वं ते मयाप्राप्तं मुक्त एवाहमीश्वरि।
 कावेरीतिसरिद्रूपा यत्प्रसिद्धा भविष्यसि।
 त्वदर्शनेनसंजाता चित्तशुद्धिर्ममाधुना।
 अतोमोक्ष्येऽखिलं कर्म दुष्पारं मोक्षविघ्नकृत्।
 चित्तशुद्धिर्भवेद्यावत् तावन्नित्यंचरेत्सुधीः।
 ततःप्रकृत्याशुद्धात्मा संचरेद्यत्रकुत्रचित्।
 इति शृत्वापितुर्वाक्यं कवेरस्यमहात्मनः।
 देवीभगवतीविष्णोर्माया पितरमब्रवीत्।
 35 सत्यमेवमहाभाग यब्रवीषिह मां प्रति।
 मत्सन्निधानमात्रेण चित्तशुद्धिरभूच्च ते।
 करस्थिता हि मुक्तिस्ते सत्यमेवब्रवीमि ते।
 त्वत्पुत्रीत्वन ते कीर्तिं सर्वलोकेषु पावनीम्।
 
प्रकाशयंती सततं सिंधुरूपाभवाम्यहम्।
 पितृनुद्दिश्य यःकीर्तिं विस्तारयति भूतले।
 ब्रह्मायास्यंतिपितर स्तत्पुत्रश्च व्रजेद्धरिम्।
 इति बृवंती कावेरी सप्रशस्य पुनः पुनः।
 कृतार्थोस्मीति हृष्टात्मा सन्यस्य्यव्यचरन् महीम्।
 सर्वत्र समदृग्योगी समलोष्टाश्मकाञ्चनः।
 40 बहिरंतश्चसर्वत्र समं पश्यन् जनार्दनम्।
 सर्व त्र संदरस्तौनी वायुवद्वीतकल्मषः।
 समदुःखसुखःक्षांतो हस्तप्राप्तं च भक्षयन्।
 निर्विशेषं च संपश्यन् द्विजांत्यजमृगादिषु।
 मानावमानयोस्तुल्यं उन्मत्त इव सर्वदा।
 ‘भावयन् मानसाविष्णुं परमात्मानमीश्वरम्।
 चिन्मयंपरमानंदं ब्रह्मैवांतेऽगमद्धरिम्।
 लोपामुद्रा च कावीरी सच्चिदानंदविग्रहा।
 स्वजले स्नास्यतां तूष्णीं तपस्तेपे विमुक्तये।
 अथाच्युतोहरिस्तत्र तस्याः प्रत्यक्षतांगतः।
 अवदत् पुण्यकावेरी भगवान् भूतभावनः।
 45 श्रीभगवान्वत्से प्रीतोस्मिभद्रं ते तपसानेन ते शुभे।
 अभीष्टद इह प्राप्तो यथेष्ठं प्रीयतां वरः।
 कावेरीनदीभूत्वाऽद्यगच्छेयं लोकपावन हेतुना।
 
इति मे निश्चिताबुद्धि रनुगृह्णीष्व मामिह।
 नास्तिकाश्चकृतघ्नाश्च दुराचारानिरग्नयः।
 एतादृशामहापापा असंभाष्या अमीमयि।
 यथास्स्युस्नानमात्रेण त्वपदस्था हरेनराः।
 विधीयतां महाविष्णोः यदि त्वं भक्तवत्सलः।
 50 श्रीभगवान्दुर्घटोयं परोलोके बहुपुण्यवतामपि।
 एतादृशेषु का चिंता भवेद्भक्त्या तथापि ते।
 किमुपायोस्ति मद्भक्तर्मद्भक्तिर्मुक्तिदायिनी।
 मद्भक्ताया अतोहंते पूरयिष्या मनोरथम्।
 सह्याचले महापुण्ये स्मरणात्पापनाशने।
 दिव्यामलकमूर्तेर्मे पादद्वंद्वं पितामहः।
 अभिषेक्ष्यति शंखेन विरजापुण्यपाथसा।
 तदा तत्पुण्यतीर्थेन प्रवहस्वसु पूरिता।
 भवदक्षिण गङ्गेति दक्षिणां दिशमाश्रिता।
 दाक्षिणात्या यथासर्वे प्यमुष्यात्तव दर्शनात्।
 55 स्वर्गस्थाश्च भविष्यंति तथा पुण्याभविष्यसि।
 अत्यंतदुर्लभं लोके गङ्गाधिक्यं च दद्मि ते।
 महानदीनां मान्यत्वं तत्सांयेका विचारणा।
 दत्तात्रेयावतारोहं त्वत्तीर्थे पुण्यवर्धने।
 शिरस्थाने वसिष्यामि भावयन् मत्पदाश्रयान्।
 किं च कावेरि ते भूयः प्रयच्छामि महा वरम्।
 
मम पादोद्भवागङ्गा नदीनां प्रवरा स्मृता।
 तवोत्संगे वसिष्यामि तस्यास्त्वमधिका भव।
 त्वं पुण्यासर्वतीर्थाना मधिकैव न संशयः।
 शिरस्थाने तवोत्संगे नित्यम् मत्पादसेवनात्।
 60 गङ्गादि सर्वतीर्थेषु नित्यस्नानं चतुर्युगम्।
 सह्यामलकतीर्थस्य कलां नाहँति षोडशीम्।
 तपःकुर्वंति मे घोरम् नैमिशे तु चतुर्युगम्।
 विदंति तत्त्वयिस्नानात् सह्यामलकसन्निधौ।
 सह्यामलकतीर्थं च शंखतीर्थमिति द्विधा।
 तव तीर्थस्य कल्याणी नामधेयं भविष्यति।
 अगस्त्योमुनिरत्रा त्वां तपस्यंती मदंशजः।
 आरोप्यकुंडिकातीर्थे सह्याद्रिं प्रेषयिष्यति।
 लोपामुद्राख्यया त्वं च तस्य भार्या भविष्यसि।
 इत्युक्त्वाऽतर्दधे विष्णु विश्वमूर्तिर्जगन्मयः।
 65 कावेरीच पुनर्विष्णोः ध्यायंतीपदपंकजम्।
 साक्षाद्भगवता माया तेपे सा परमं तपः।
 एतस्मिन्नंतरेगस्त्यो वर्णीनिर्विण्णमानसः।
 संसारंदुस्तरं मत्वा तपस्तेपे सुदारुणम्।
 तत्राऽगस्त्यमुनिं प्राह सर्वलोकपितामहः।
 कृतांजलिपुटैश्श्रीमान् स्तूयमान स्सुरर्षिभिः।
 
ब्रह्मा
सर्वोद्वेगकरंब्रह्मन् किमर्थं क्रियतेतपः।

त्वयाकृतं मयाज्ञातं मद्वाक्यं श्रूयतांमुने।
 नित्यादिकर्मणां मत्वा हेतुत्वं संसृतेकिल।
 मोक्ष एव मनस्तेद्य तस्मात्तुर्याश्रमेमतिः।
 70 कर्मणैव भवेन्मोक्षस्संस्सृतिश्च विनश्यति।
 कांयकर्मविहीनश्चेत् कर्मसन्यासिवद्भवेत्।
 सन्यासोत्युग्रसंचारः किंचिदूनेपतत्यधः।
 ब्रह्मार्पितं कृतंकर्म विकलंचापि मोक्षदम्।
 मध्याह्ने योपिनिष्कामश्शांतं भोजयतेऽतिथिम्।
 गङ्गादि सर्वतीर्थैर्वा किं तस्य व्रतकोटिभिः।
 विरक्त इंद्रियार्थेषु तथा पुत्रगृहादिषु।
 मानावमानयोस्तुल्यः कर्मकृन्मोक्षभाग्भवेत्।
 पंचयज्ञान् यदाकुर्याद्विरक्तो विष्णुभक्तिमान्।
 तत्क्षणेनाखिलं तृप्तं जगत्स्थावर जंगमम्॥75 लोकोपकारकं कर्म निष्कामी कुरुते यदि।
 सर्वभूतदयायुक्तो मुक्तिस्तस्यै करेस्थिता।
 निष्कामकर्मन्यूनं वा मुक्तये केशवार्पितम्।
 भवत्येव न संदेहः सन्यासे विकलःपतेत्।
 ब्रह्मचर्ये तु सन्यासि दुःखस्यादुर्विचिंतनात्।
 भुक्ति भोगस्य सन्यासो मोक्षदस्यादयत्नतः।
 लोकान् गार्हस्त्यधर्मेण वानप्रस्थेन वा पुनः।
 पवित्रीकुरु संचारात्त् कर्मसन्यासिवत्तदा।
 सत्कर्मणा हरेस्थानं प्रापितस्स्यादयत्नतः।
 
किंचित्पापे तु सन्यासे नरकस्स्यादयत्नतः।
 80
अगस्त्यः
कथं मां नरकद्वारे नियोजयसि हे प्रभो।
 ब्रह्मापूर्वकर्मकृतैः पापैः भार्यास्यादप्रिया शठा।
 कुलटा प्रतिकूला वा तां त्यजेत्स्वर्गभाग्सुधीः।
 यो निंदति पतिं नारी प्रतिकुप्यति वा सकृत्।
 शतजन्मशुनी सा स्यात् तादृशाः पूर्वपापिनः।
 तपस्सिद्धस्यते वर्णीन् विशेषान्मदनुग्रहात्।
 भविष्यति शुभाचारा मंदहासो पतिव्रता।
 किंच ब्रवीमि ते वत्स! सुप्रीतस्तपसा तव।
 ब्रह्मचर्यव्रतेनेद्धं मुक्तिदं कर्मवैदिकम्।
 ममकन्या शुभांगी च कावेरीति प्रकीर्तिता।
 85 तपश्चरति सत्कीर्तिर्लोकानुग्रह कांयया।
 विधिना तां समुद्वाह्य तस्यै दत्वाप्यभीप्सितम्।
 दुरात्मनोस्य विंध्यस्य सर्वलोक विरोधिनः।
 तपसा संहरन् गर्वं सह्यंगच्छ नगोत्तमम्।
 कावेर्यांशेन तत्रैनां प्रवहस्वेदिचोदय।
 तत्रस्थां प्राप्य कावेरी लोपामुद्राख्ययायुताम्।
 व्रजस्व लोकरक्षार्थ मलयं च कुलाचलम्।
 इत्युक्त्वांतर्दधे ब्रह्मा मुनर्नोिरमरैस्सह।
 सच विस्मयमाविष्टः पुष्टांगो निरगात्ततः।
 
निरुध्य सूर्यसंचारं वर्धमानं नगोत्तमम्।
 विंध्यंभूमिसमं कृत्वा सांत्वपूर्वं तु याज्ञया।
 90 ममागमनपर्यन्तमेवं वर्तस्व भूधर।

औद्धत्यं मुंच भद्रं ते बहुश्रेयो यदीच्छसि।
 इत्युक्त्वा तं प्रशस्याथ कावेरी दर्शनोत्सुकः।
 संचरन् विपिने तत्र मार्गमाणश्च सह्यजाम्।
 उत्तरे हिमवत्पार्श्वे कावेरी तपसिस्थिताम्।
 दृष्ट्वा रूपवतीं हृष्टः परिपप्रच्छ तां शनैः।
 भद्रे कवेरकल्याणी मामीक्षस्व सुचक्षुषा।
 संतापयसि बाल्ये त्वं किमर्थं कोमळंवपुः।
 सर्वं मनोगतं तेद्य जानामि जगदंबिके।
 ध्यानारूढासि कावेरी रूपयौवन शालिनी।
 95 इति शृत्वा मुनेर्वाक्यं पुरोगस्त्यं ददर्श सा।
 उत्थाय संबमाद्देवी स्वागतेनाभिपूज्य तम्।
 अर्घ्यपाद्यादि सत्कारैः पूजयामास तं मुनिम्।
 इति हरिश्चन्द्रप्रति अगस्त्येन कावेर्युत्पत्तिकथनम्
नाम त्रयोविंशोऽध्यायः
**********************
अथ चतुर्विंशोऽध्यायः।

लोपमुद्राऽगस्त्ययोर्विवाह कथनम् धौम्यःपुनराह सतां देवीं स्मितपूर्वं कृतांजलिम्।

इंदीवराक्षींबिम्बोष्ठी नीलकुंचितमूर्धजाम्।
 अगस्त्यःनिदेशाद्ब्रह्मणोदेवी ह्यागतोस्मितवांतिकम्।
 तवेप्सितं प्रदास्यामि किंच मे पूरयेप्सितम्।
 कावेरीसत्यमेव मुनिश्रेष्ठ! तव वाक्यं मनोरमम्।
 पूरयिष्यसि मेभीष्टं त्वं च देवमनस्थितम्।
 धौम्यःइत्युक्त्वा सापरं ववे कावेरीकुम्भसंभवम्।
 अहं ब्रह्मसुता विष्णोर्माया लोकविमोहिनी।
 लोपामुद्रेति विख्याता लोकानुग्रहकाम्यया।
 कावेरीति सरिद्रूपा भजेयं दक्षिणांदिशम्।
 5 चतुर्दशसु लोकेषु यानितीर्थानि कुंभज।
 मयि सर्वाणि तान्यद्य समागच्छंतु सर्वदा।
 सर्वतीर्थेषु यत्पुण्यम् सर्व यज्ञेषु यत्फलम्।
 सर्वव्रतेषु यत्पुण्यम् तदस्तु स्नानतो मयि।
 मज्जले स्नानमात्रेण यथास्युरनघा नराः।
 उत्तारिता मया सर्वे दाक्षीणात्या अयत्नतः।
 मत्तीरे तु विशेषेण भूमिस्सस्यवती सदा।
 स्थिताश्च धनदान्याड्या स्सदाधीतास्तु भूसुराः।
 सकृच्छ्राद्ध प्रदानेन पितरो मम सन्निधौ।
 भवंतु विष्णुलोकस्थास्तेषां तृप्तिरधाक्षया।
 10
सकृद्ध्यानेन पानेन कावेर्या मम कीर्तनात्।
 दर्शनेनापि पापिष्ठा स्स्वर्गस्थास्स्युरयत्नतः।
 एवं दास्यापि मेऽभीष्टं लोपमुद्राख्ययाप्यहम्।
 भवेयं धर्मपत्नी तु कावेरीत्यंशतस्सरित्।
 अगस्त्यःअस्तु तेऽभीप्सितं भद्रे सर्वमेव न संशयः।
 तत्तेवब्रह्मणाप्रोक्तः पुरैव भवतींप्रति।
 वर्तते सर्वतीर्थानि कुंडिकायां महेश्वरी।
 गङ्गाद्यानिम्नगास्सर्वा स्सांगोपाङ्गास्समागताः।
 ताःप्रविश्यवसै तस्यां कावेर्यंशेन निम्नगा।
 यावद्ब्रह्मगिरिप्राप्ति स्तावन्मत्कुंडिका स्थितः।
 15 तवानुरूपश्शैलोसौ पुण्योब्रह्ममयो महान्।
 सह्योभिधस्त्वं निर्गत्य तस्माद्गच्छ सरिन्मयी।
 दत्तात्रेयो हरिश्श्रीमान् योगी तत्र वसिष्यति।
 तत्पादरजसा पूता भवपुण्या सरिद्वरा।
 पुनीहिचाति पापिष्ठान् सर्वलोकैकपावनी।
 अयमेव वरोदत्तः पूर्वमेवाच्युतेन वै।
 मत्संबंधेन ते देवी प्रभावातिशयोद्यकः।
 अतो परं वरं दद्मि संतुष्टस्तव सुव्रतैः।
 अन्यथा पौनरुक्त्येन मद्दत्तो हि वरोहतः।
 यत्फलं द्वादशाब्दं तु सप्तसागर मज्जनात्।
 20 एकादश्यां सकृत्स्नानात् तवतीर्थेतु तद्भवेत्।

धौम्यःइत्युक्त्वा हर्षपुष्टांगी कमलांगी शुचिस्मिता।
 हारनूपुर केयूर मकुटाद्यारलंकृता।
 बिंबाधरा विशालाक्षी पक्वबिल्व फलैस्तनी।
 दुकूल वसनोपेता स्वर्णताटंक भूषिता।
 विराजज्जघनाकांता रशनादामभूषिता।
 माणिक्यरत्नविलसन् मुद्रिकालसदंगुळीः।
 द्विधाभूताऽथ कावेरी लोपामुद्रेति सा सती।
 सा तत्कमण्डलुजले प्रविवेशपुण्ये
पुण्या कवेरतनयेति सरिद्वरेति।
 तस्थौ पुरोथ मुनिवर्य वरस्य
लोपामुद्रांशतो मदनमोहद मंजुवेषा।
 25 मंदस्मिता मङ्गळरूपवेषा मुनींद्रपादान युगं ननाम।
 दृष्ट्वा सतां हृष्टतनुर्मुनींद्र
उद्वाहकार्येऽस्मरदब्जयोनिम्।
 आरुह्य हंसं भगवान् पितामहस्तत्राजगामाथ
सर्षीन् स संघमैः।
 श्रियाचसाकं गरुडासनो
हरिचंद्रमारुह्य हरश्चसोमः।
 सोभूत्समाजस्सुरनायकानां तेषां
विचित्रो जगदुत्सवाय।
 तथा कवेरोद्भवकुम्भयोन्यो
स्स उत्सवोभूत्सकलोत्सवाय।
 मूर्धा ननामाजज विष्णु शंकर
श्रीपादपद्मान्विनतो मुनीश्वरः।
 तं मूर्युपाघ्राय मुदापि तेन ते
प्रयुंजताशीर्वचनं शुचिस्मिताः।
 लोपामुद्रा लोकमाताम् ववंदे लक्ष्मी
दक्षस्यात्मजां भारती च।
 आलिंग्यैनां मू[पाघ्रायताश्च स्यास्त्वं
भद्रे भर्तृ युक्तेत्यचोचन्।
 30 ततस्सुतां स्वां भुवनस्य मातरं
सा भारती दिव्यविभूषणादिभिः।
 दिव्यांबरैः कुंकुमचंदनादिभि स्तां
भूषयित्वा मुनिपार्श्वमानयत्।
 बृहस्पतिर्ब्रह्महरीशसन्निधौ
सप्तर्षीन् मध्येऽखिलतापसाग्रतः।
 वैवाहिकं गृह्यविधानतोविधि
निवर्तयामास च कुंभजन्मनः।
 चतुर्भुजस्यानुमतेश्चतुर्मुखो
__मुनींद्रवर्याय स कुंभयोनये।
 देवीं स्वकन्यां जलदानपूर्वकं ददौ
___ द्विजेभ्यश्च मुदा धनं बहु।
 वधूसहायः कलशोद्भवोमुनि
नभास्यथा
स्सभांपरीयाय मुनींद्रभूषिताम्।
 भृग्वादि मुख्यामुनयोच्युताग्रता
श्चा शीर्भिरग्राभिरभूषयन् मुनिम्।
 विशुश्रवे मङ्गळतूर्यनिस्स्वनो जगुश्च
गंधर्व वरानभास्यथ।
 मुदा ननर्तुःकुशलाप्सरस्त्रियो
मंदारपुष्पाणि सुराववर्षिरे।
35 अथैष पत्नी सहितोमुदान्वितस्सभां
__ प्रणम्याशु सत्तैश्च नंदितः।
 वरासनरूहश्शुशुभे हिमाचले यथा
भवानीं परिणीयशंकरः।
 ववंद ईशान पदारविंदं मुनींद्रवों द्विजबंदमध्ये।
 आलिंग्यदो• प्रणंअन्तमीश
आम्रायशीघे प्रयुयोज भद्रम्।
 नत्वाथलक्ष्मी सहितं परं हरि प्रक्षाळ्य
पादाब्जयुगं स कुम्भजः।
 तत्तीर्थमेत्याऽखिलतीर्थपावनं
संप्राशयच्छापि सभासदस्तदा।
 लोपामुद्रागस्त्यपत्नीप्राहृष्टा मंदस्मेरा
योगिनां मोहिनी च।
 नत्वादेवी पार्वती विष्णुपत्नीवाण्याः पादं
भोयुग्मं ववंदे।

तां वंदमानां प्रसमीक्ष्य मातरं
लोकत्रयास्याखिल लोकसुंदरीम्।
 निजांकमारोप्य जगत्त्रयोज्ज्वलां
साशिक्षयंती प्रबभाण भारती।
 भद्रेदेवी त्वं हि लोकस्यमाता
साध्वीनामप्यग्रगण्या भवस्व।
 मायास्त्वं विप्रियम् भर्तुरग्रे क्रुद्धपत्यौ
भर्त्तिता वा कदापि।
40 मादुष्टानां गच्छपार्वं कदापि
स्त्रीणां ब्रूया सत्यमेवापदित्वम्।
 वदस्व वाचं मृदुमेवसर्वदा शृणुष्व
विष्णोश्चरितं वृणीष्व।
 विष्णोःपदंमङ्गळमेव सर्वदा गृणीष्व
वाचापति भूषणं सदा।
 भजस्वदाक्ष्यं गृहवस्तु संग्रहे जपस्व
__नामत्रयमिष्टदं हरेः।
 पृच्छस्व धर्मान् पतिमेव सर्वदा ध्यायस्व
भक्त्या पतिपादपंकजं।
 विनिर्गते भर्तरि कार्यहेतुना त्वं जागरूका
भव वह्निरक्षणे।
 भवातिथेयीच भयान्विता
बहिस्सुसावधानाभव विष्णुकर्मणि।

शयिष्वपश्चात्पतिपादपंकजे बुय॑स्व
___ भर्तुस्सभया पुरैव।
 त्रायस्वभक्त्या पतिबांधवान् सदा
नेक्षस्व पित्रोः प्रियबांधवान् वरान्।
 45 ___ एवं प्रवृत्ता गृहिणी पतिप्रिया सुसाधुवृत्ताचटुला
__शुचिस्सदा।
 हृष्टाच भ; सततं स्मरोत्सवे भुक्त्वा च
भोगांस्त्रिदिवं प्रयास्यसि।
 कुयोषितःकुत्सित संगशीलाः कुदृष्टयः
कुंठित धर्मचिंताः।
 उच्छैस्स्वरास्स्वादर
पूरणाश्च द्विषंतिभर्तृन्नरकालयास्स्युः।
 लोपामुद्रा भारतीप्रोक्तधर्मानेत्थं शृत्वा
सस्मितं लोकमाता।
 सर्वान् कुर्वे मातरद्यत्वयोक्तान्
सर्वज्ञाहं शिक्षिताचास्मि भूयः।
 पूर्वपुण्यैस्संचितैरेषभर्ता लब्धोभून्मे
मत्समाका हि लोके।
 अत्र स्वर्गे श्रेय एवाखिलं मे शृत्वेत्थं
तत्सज्जनास्तां ननंदुः।
 धौम्यःविभोह्यभूपं तव शासनाद्गृही
कथं तरेयं भवदुस्तरांबितम्।
 50 निशंयवाक्यं विधिरस्य विस्मितो
___ विहास्य किंचिन्मुखमीक्ष्य वैष्णवम्।
 संचोदितस्तस्य विपावलोकनात् प्रशस्य
तं प्राह मुनिं पितामहः।
 बाल्येविरक्तस्सुक्तिीभवान् मुने
समस्तधर्माश करस्थितास्तव।
 इति ब्रीवन् राममनुं दयान्वितो
विमुक्तयेस्मै समुपादिशद्विधिः।
 व्याचख्यौच पुनस्तस्मै रामभक्ताय योगिने।
 यन्त्रम् मन्त्रस्वरूपं च षडक्षरजपे विधिम्।
 मन्त्रं प्राप्यमहायोगी शुशुभेछ षडक्षरम्।
 बलायतिबलांचैव रघुनाथ इवापरः।
 ततो देवीं सुसम्तुष्टां लोपामुद्राम् हरिप्रियाम्।
 नमस्यंतीमुवाचेदम् मन्दस्मेर मुखाम् विधिः।
 55
ब्रह्मा
त्वज्जलेस्नास्यताम् मुक्ति र्यत्रकुत्रापि शोभने।
 त्वत्तीरे वसताम् लक्ष्मी मत्कटाक्षेण वर्धताम्।
 भगवानपि विश्वात्मा लोपामुद्रा यथाब्रवीत्।
 दक्षिणाशां गमिष्यन्ती नदीरूपप्रवाहिनीम्।
 निज मायांश संभूतां लोकानुग्रहकारिणीम्।
 
विष्णुः
त्रिरात्रं जाह्नवीतीरे पंचरात्रं तु यामुने।
 सद्यःपुनातु कावेरी पापमामरणांतिकम्।
 दक्षिणाशा सुरक्षार्थं लोपामुद्रां समुद्यताम्।
 प्रणमण्ती समुत्थाप्य स्वहस्ताभ्यां हरोब्रवीत्।
 ईश्वरःत्वन्मूलमारभ्यकवेरकन्ये सार्थं
__ मुनींद्रस्तव पुण्यतीरे।
 वसामि नित्यं वसतांजनानामासागरं
___वांछित वस्तुदायी।
60 इत्थं तु देव्यै जगतां विभूत्यै
___ ददुर्वरांस्ते विधिविष्णुरुद्राः।
 अगस्त्यमामन्त्र्य च तस्य पत्नी स्वस्वं
पदं जग्मु रतीवहृष्टाः।
 य इदं पुण्यमाख्यानं पुत्रपौत्रादिवृद्धिदम्।
 पठेद्वा शृणुयाद्वापि सर्वपापैःप्रमुच्यते।
 पुण्यकालेविशेषेण शृणुयाद्यस्समाहितः।
 ऐश्वर्यमतुलं प्राप्य विष्णुलोकं स विंदति।
 सर्वतीर्थेषु यत्पुण्यम् सर्वय यज्ञेषु यत्फलं।
 तत्सर्वं संयगाप्नोति शृत्वा पुण्यकथामिमाम्।
 एतां पुण्यकथां भक्त्या यायोषीच्छृणुयाद्च्छुचिः।
 लोपामुद्राऽगस्त्यमिव मोदते प्राप्य सत्पतिम्।
 65
इति लोपमुद्राऽगस्त्ययोर्विवाह कथनम् नाम
चतुर्विंशोऽध्यायः।

********************
अथ पंचविंशोऽध्यायः।

कावेर्याः कुण्डिकातोनिर्गमनम् धौम्यःगतेषुतेषुदेवेषु ब्रह्मादिषुसुरैस्सह ।
 हिमवत्प्रस्थतागस्त्यः प्रतस्थे स्वस्त्रियासमम् ।
 तत्र स्थितामहात्मानो मुनयोथमुमुक्षवः ।
 कावेरी संपरिष्वज्य स्नेहादूचुस्सुदुःखिताः।
 मुनयःमातस्तेदेवि कल्याणि कल्याणसमये तदा।
 संसारभीतैरस्माभिःमोक्षसंप्रार्थितोच्युतः ।
 अनुशिष्टावयं तेन स दयेन मुमुक्षवः ।
 यत्र स्रवति कावेरी पुण्यमोक्षांकुरस्थली।
 यूयं वसत सर्वेपिसंप्रा.स्मत्पदंत्विति ।
 त्वां विनाद्यवसामोत्र लोकमातःकथंबत ।
 विरक्तानां तथास्माकं तपसा किं प्रयोजनम् ।
 5 धौम्यःइति तेषां वचश्शृत्वा कावेरीसाथै पातनम् ।
 लोपामुद्राग्रतो भर्तुतनींद्रानिद मब्रवीत् ।
 कावेरी
भवद्भिःप्रार्थितंयत्तु तत्तथैव भविष्यति ।
 प्रापयिष्यति मामेष भगवान् ब्रह्मभूपरम् ।
 प्रवाहमियदातस्मात्कावेरी दक्षिणां दिशम् ।
 तदागच्छत सर्वज्ञा मयि स्नेहान्मुनीश्वराः ।
 संसारवारिधे|रा दुद्धृत्य स्नानपानतः ।
 युष्मानहं प्रापयिष्ये वैकुंठं तु न संशयः ।
 येकेचित्प्राणिनस्सर्वे यत्तिरस्थाजडा अपि ।
 यांत्येव दुर्लभांमुक्तिं किमुतश्श्रोत्रियाजनाः ।
 10 अहं भगवतो माया हरेविश्वात्मनो द्विजाः ।
 पितामहस्य कन्या च कवेरादापिता सुता।
 एवं तु लोकरक्षार्थमहमेव समुद्यता।
 स्रवामि दक्षिणाशायां कावेरीति विधारिता ।
 लोपामुद्राख्यया गस्त्यमुनेर्भार्याभवामि च ।
 नदीरूपेण गन्तुं मे न शक्यंशैलकोटिषु ।
 तस्मादनेन भर्ताद्यप्राप्यसेहंब्रह्मभूधरम् ।
 तत्सामनिनदीभीत्वा दक्षिणाशास्वलंकृतिः ।
 तत्रत्यान्पापयांतीचस्नानपानादिदर्शनैः ।
 श्रीरङ्गराजपादाजपूतायास्यामिसागरम् ।
 तदासाद्याचमत्तिरवस्तव्यंद्विजसत्तमाः ।
 अहंवस्तारयिष्यमिदुस्तराद्भववारिधेः ।
 एकादश्यांव्यतीपातेस्नानमात्रेणमज्जले ।
 नरणामपनेष्यामिऋणत्रयविशृंखलाम् ।
 
माभैष्टयूयंविरेंद्राआगमिष्याथमत्तटम् ।
 वैकुण्ठप्रापयिष्यामिकेवलंस्नानमात्रतः ।
 स्नान्मात्रेणमत्तीर्थेमहापापीदिवंव्रजेत् ।
 किंपुनश्श्रद्धस्नात्वामाहात्म्यश्रावकोनरः ।
 मज्जलेस्नास्यतांपुम्सांमत्कीर्तिमपिशृण्वताम् ।
 किंतीथैः किमुवेदैश्चकिंतपोभिकिमध्वरैः।
 20 धौम्यःइत्युक्त्वातान्सांत्वयामालोपामुद्राहरेर्यशः ।
 भर्तुरंतिकमागच्छच्छुरूषार्थंशुचिस्मिता अगस्त्योपिमहातेजास्तामुद्वाह्यविधेस्सुताम् ।
 शयिष्यन्विंध्यशैलंप्रतस्थेदक्षिणांदिशम् ।
 निरुध्यसूर्यसंचारंवर्धमाननगोत्तमम् ।
 विद्यभूमिसमंकृत्वासान्त्वपूर्वतयाचया।
 मदागमनपर्यन्तंएवंवर्तस्वभूधर।
 सत्यमामुंचभद्रंतेयदिश्रेयस्त्वमिच्छसि ।
 इत्युक्त्वातंप्रशस्याथकरेणालंब्यतनगम् ।
 ब्रह्मणोक्तंस्मरन्प्रापसाक्षाद्ब्रह्ममयंगिरिम् ।
 25 ततोगस्त्योमहातेजाःकुण्डिकांब्रह्मनिर्मिताम् ।
 संस्थाप्य ब्रह्मशैलेंद्रे लोपामुद्रां वचोब्रवीत् ।
 
अगस्त्यःशैलेंद्रोब्रह्मनामायं श्रीमान्कावेरिशोभनः ।
 हेतुभूतस्तवोत्पत्तेर्ब्रह्मणा कीर्तितःपुरा ।
 
अत्र कालंप्रतीक्ष्यस्व भद्रे मत्कुंडिकास्थिता।
 ते निर्गमंप्रतीक्ष्येहं नदीरूपेणशोभने ।
 इत्युक्त्वा ब्रह्मणःपुत्रीं कावेरी कुंडिकास्थिताम् ।
 सर्वतीर्थमयींपुण्यां संस्थाप्य ब्रह्मभूधरे ।
 शिष्यानुवाच रक्षध्वमाधानेन पुत्रिकाः ।
 इमंतीर्थमयींदेवीं कावेरी कुण्डिकास्थिताम् ।
 कृत्वा स्नानमयीं शीघ्रमागमिष्यामि पश्यत ।
 30 इति शिष्यांत्समादिश्य योगींद्रःकुम्भसंभवः ।
 स्नातुं स्वर्णमुखीतोये जगाम त्वरितो मुनिः ।
 गतेतस्मिन्मुनिश्रेष्ठे कावेरीकुंडिकास्थिता।
 काझंतीनिर्गमदेवी मनसेदमचिंतयत् ।
 कदाहं निर्गमिष्यामि मुनिसंमतिपूर्वकम् ।
 बहुकालविलंबेपि न कुर्यात्संमतिं मुनिः ।
 निर्गच्छब्रह्मशैलेंद्रादिति पूर्वामुवाच सः ।
 अत्र प्राप्तोपि मे योगीनेच्छत्यद्यापि निर्गमम् ।
 प्रतिज्ञासमयःप्राप्तो मम दोषो न विद्यते ।
 इति ध्यात्वा मुहूर्त तु निर्गमंतं मनसोद्यता ।
 35 एतस्मिंनंतरेब्रह्मा हंसारूडस्सुरैस्सह।
 संचरन्लोकरक्षार्थं दयाळुर्दक्षिणां दिशम् ।
 तीर्थकोटिसमायुक्तं प्रपेदे ब्रह्मभूधरम् ।
 आश्चर्यभूतं शैलेंद्रम्नानावृक्षोपशोभितम् ।
 नानाधातुसमायुक्तं नानारत्नोपशोभितम् ।
 
नानानिर्झरसंयुक्तं नानामृगगणैर्युतम् ।
 दृष्टमात्रेणपापघ्नंमनश्शुद्धिप्रदं शुभम् ।
 परिपूतं च सर्वत्र विष्णुलोकमिवापरम् ।
 तमासाद्य महापुण्ये स्नात्वा च कनकाचले।
 जजाप परममन्त्रं षडक्षरमथाजज ।
 40 दिव्यगंधवहस्तत्र ववौगंधवहं स्तदा।
 स सुगन्धं समाघ्राय सर्वतस्समलोकत ।
 तत्र धात्रीतरुश्श्रीमान् दिव्यःपत्रफलोज्ज्वलः ।
 दृष्टमात्रेणपापघ्नो ददृशे पुरतो विधेः ।
 नाळिकेरफलोपेते तत्र तेजोमयेगिरौ ।
 दिव्यमामलकंदृष्ट्वा हृष्टस्सृष्टाविसिष्मये।
 तत उत्थाय हर्षेण संब्रामा विष्णुमानसः ।
 समीपमपमृत्याथ निद्यातुमुपचक्रमे ।
 
धात्रीतरु ध्यानम् ततश्चान्तर्दधे तत्र तरुरामलकोमहान् ।
 पुनश्चविस्मयाविष्टःकिमेतदिति चिंतयन् ।
 हृदंबुजे ततो ध्यायन्विष्णुमेव ददर्श तम् ।
 45 उपरिष्टाच्चतुर्बाहुं शंखचक्रगदाधरम् ।
 किरीटिनमुदाराङ्गं वनमालोपशोभितम् ।
 
अधस्ताच्छंखचक्राजलक्षणं च पदांबुजम् ।
 ततो ब्रह्ममहातेजा दृष्ट्वा तं वृक्षमीश्वरम् ।
 विनयावनतोभूत्वा ननाम स्वर्णदण्डवत् ।
 
बद्धांजलिपुटोस्तौषीवृक्षं विष्णुमयं द्विधिः ।
 अथोदभूद्वागशरीरिणीशुभा ब्रह्मन्स
__ एवायमधोक्षजो हरिः।
 एनं समभ्यर्च्य भवादृशाःपुरा
ब्रह्मन्हरिप्रापुरनेक कोटयः ।
 ___ त्वं चापि धात्रीतरुमीश्वराकृति
समर्चय ब्रह्मपदप्रसिद्धये।
 उत्तवेति वाणीं विरराम वैष्णवी ब्रह्मापि
दिव्यामलकं ननाम सः ।
50 वैहायसीं गिरि शृत्वा दृष्ट्वामलक वैभवम् ।
 पुलकांगितसर्वाङ्गो हृष्टःकमलसंभवः ।
 दिव्यपुष्पाण्यथाहृत्य तुलसीकुसुमान्यपि ।
 चंदनानि च दिव्यानि माल्यानिसुरभीनि च ।
 स्वकुंडिकास्थलेस्तीर्थे विरजायास्समाहृतिम् ।
 शंखमापूर्यपुष्पाद्यैरभ्यर्च्य त्रिपदाकृतिम् ।
 ध्यायन्हृदंबुजेविष्णुं ब्रामयित्वाच्युतोपरि ।
 पूजावस्तूनि शेषां स्वं संप्रोक्ष्य च तदंबुना।
 अर्चयित्वा पदांभोजं विष्णोरामलकात्मनः ।
 दिव्यगंधांबरस्रग्भि दिव्यैराभरणोत्तमैः ।
 55 कोमलैस्तुळसीपत्रैमल्लिका चंपकादिभिः ।
 नानाफलरसाढ्यैश्च नीवेद्यपरमात्मने ।
 चतुःप्रदक्षिणं कृत्वा प्रणम्य च पुनःपुनः ।
 
स्तुत्वा च विविधै स्तोत्रैह्झौतैस्स्मार्तेश्च वैष्णवैः ।
 दद्यौ तच्चरणांभोजं हृदजेसाशृलोचनः ।
 पुनःप्रादुर्बभूवाथ दिव्यावाणी शुभा तदा।
 चिंतयंत्यां तु कावेर्यां नदीरूपेणनिर्गमम् ।
 भो भो शृणुष्व कावेरी फलितस्ते मनोरथः ।
 शोभनोवासरःप्राप्तः तव प्राप्तुं महोदधिम् ।
 तव दक्षिणगङ्गा त्वं सिद्ध्यत्येव न संशयः ।
 60 कालोयं शारदःप्राप्त स्तुलाविषुव पावनः ।
 ऋतुष्वपि च सर्वेषु शारदोमुक्तिसाधकः ।
 स्वल्पोप्यस्मिन्कृतोधर्मो वर्धतेवटबीजवत् ।
 त्वत्तपःपरिपाकेन तुलासंक्रान्तिरप्यभूत् ।
 पुण्यपुंजे च सोस्मिन्हरेरामलकाकृतैः ।
 अभिषिक्तंपदांभोजं ब्रह्मणा विरजांबुना।
 गायत्रीरूपशंखेन सर्वतीर्थमयेन च ।
 तदेतत्सर्वपापघ्नं सर्वलोकैक पावनम् ।
 सर्वक्रतुफलोपेतं सर्वतीर्थफलप्रदम् ।
 साक्षान्मोक्षप्रदं पुण्यम्सर्वधर्मफलप्रदम् ।
 सर्वाभीष्टप्रदंदिव्य मायुरारोग्यमुक्तिदम् ।
 भगवत्पादसलिलं जगद्रक्षणहेतुना।
 त्वदर्थमेव सर्वत्र स्रोतोरूपेण निर्गतम् ।
 तत्तीर्थपूरिता त्वं च पावयास्वाखिलंजग्त् ।
 अगस्त्यकुण्डिकातीर्थे ब्रह्महत्यापनोदनैः ।
 
सहिता प्रापत क्षिप्रं दक्षिणाशैकपावनी।
 त्वन्मूलेभगवान्विष्णु हरिरामलकाकृतिः ।
 पादांभुजाभ्यां त्वं श्रीमान्पवित्री कुरु ते सदा।
 त्वद्यशःप्रकुटीकुर्वन्नास्यतां त्वयि मुक्तिदः ।
 ब्रह्मैवावलकोत्रासा वास्ते ब्रह्ममयेगिरौ ।
 70 विरजास्नानमेवस्या त्त्वयिस्नानं न संशयः ।
 विरजायांसकृत्स्नात्वा मुक्त एव न संशयः ।
 तव पुण्यतमे तीर्थे सह्यामलकसन्निधौ ।
 तत्फलं लभतांस्नात्वा यत्र कुत्रापि तौलिके।
 इति शृत्वा नभोवाणी कावेर्यत्यंतहर्षिता।
 त्वरया कुण्डिकास्थानि तीर्थान्यपि वचोब्रवीत् ।
 प्रभावो मे तस्सख्योजातंवोत्र निदर्शनम् ।
 शीघ्रमेव मया सार्धमागच्छत न शंकिताः ।
 तां तथैवेत्यवोचं ततानि तीर्थानि सह्यजाम् ।
 ज्ञात्वा तेषामभिप्रायं जहर्ष च कवेरजा।
 75 ततस्सुगंधीपाश्चात्यः पयोदजलशीतलः ।
 सर्वतःकंपयन्वृक्षा नभितो दशयोजनम् ।
 वातिस्म पवनोवेगादाज्ञया ब्रह्मणस्तदा।
 इंद्रेणचोदिता मेघास्तदानीं च दिशोदश।
 आच्छाद्य शैलदेशंच स्वल्पंतु मुमुचुर्जलम् ।
 तस्मिन्नवसरेयुक्ता कावेरीतीर्थकोटिभिः ।
 ज्ञात्वानुकूलंसंप्राप्तं कालंपरमशोभनम् ।
 
स्मृत्वाचागस्त्यसंकेतं वायुवेग विघट्टितात् ।
 जगतामुपकाराय निर्ययौ करका मुखात् ।
 भगवत्पादसलिलं पुण्यतीर्थसमाश्रिता ।
 80 संतुष्ठा पावनीभूता पावयंत्यखिलं जगत् ।
 सर्वदृश्याप्रसुस्राव फेनबुद्भुद पूरिता।
 स्नात्वागस्त्यो पि वेगेन दुर्दिने मन्दमारुते ।
 तत्रागत्य समालोक्य कुंडिकां पतितां भुवि ।
 संबमात्सर्वतःपश्यन्किमेतदिति विस्मितः ।
 इतस्ततःप्रधावंश्च निर्जगमाश्रयाबहिः।
 मध्येदीनमुखान्शिष्यान्वेपमानान् कृतांजलीन् ।
 उवाच दीनयावाचा किंचित्कृद्धो मुनीश्वरः ।
 पुत्रकाःक्वगतादेवी कावेरी कुण्डिकास्थिता।
 हित्वायुष्मद्वशांदेवीं युष्माभिःक्व गतांबत 185 इत्युक्तामुनिनाभीता शिशष्या प्रांजलयोकुवन् ।
 वयं वर्षार्धितावृक्षं भगवन्नाश्रितास्तदा।
 तावदेवाशु निर्याता कावेरीकरका मुखात् ।
 मध्ये मार्गसमाक्रम्य ह्यस्माभिस्सांत्विता बहु ।
 प्रणिपत्य निरुद्धा च ततो स्मानब्रवीच्छ सा।
 ज्ञात्वा मुनेरभिप्रायं निष्क्रांतास्मीह नान्यथा।
 मम ब्रह्मगिरावस्मिन्निर्याणं गुरुणोदितम् ।
 ईश्वरानुग्रहाच्चाहं निर्गता पुण्यसंक्रमे।
 सर्वज्ञस्सगुरुर्वोद्य ज्ञात्वा सर्वं क्षमिष्यति ।
 
मम निष्क्रमणं तस्य सन्निधावागसेभवेत् ।
90 सर्वमालोच्य मे वृत्तमिष्टमेवेति वक्ष्यति ।
 न च कुप्यति युष्मांत्स
इत्युक्त्वानिस्सृता जवात् ।
91 इति कावेर्याः कुण्डिकातोनिर्गमनम् नाम
पंचविंशोऽध्यायः
*********
********
अथ षड्विंशोऽध्यायः।

कावेर्यागस्त्ययोस्संवादः दाल्भ्यःइति शिष्यवचश्शृत्वा ज्ञात्वा सर्वं च पौर्विकम्।
 कंपयित्वा शिरोऽगस्त्यः प्रहस्य च मुनीश्वरः।
 द्रष्ट्रकामोथ कावेरी रिक्तो नष्टधनो यथा।
 दिव्यचक्षुरपि श्रीमान् विष्णुमाया विमोहितः।
 विचिन्वन् परिबभ्राम वृक्षावृक्षं वनाद्वनम्।
 क्वगतासि महाभागे क्वासि कावेरिशोभने।
 कुनारीव कृतार्था त्वं मां केनापहसिष्यसि।
 तत्र तत्र प्रधावंश्च विलपंश्च मुहुर्मुहुः।
 'तत्र तत्र श्रमात्तिष्ठन् प्रहसंश्च पुनःपुनः।
 वदंश्च हन्तहंतेति नासान्यस्तांगुलिमुनिः।
 भर्त्सयंश्च मुहुश्शिष्यान् प्रार्थयन्निष्टदेवताम्।
 5 एवं परिभ्रमन् वेगादुन्मत्त इव तद्वने।

पुनस्तुष्टाव कावेरी लोकानामिष्टदेवताम्।
 कल्याणि कंजासन दिव्यकन्ये
कल्याणि तीर्थे कमलेश माये।
 जगद्धरित्री जगदंबदेवी कावेरि कावेरिकुतोगतासि।
 त्रायस्वरूपेति त्रिगुणाभिरामे
शुचिस्मिते शोभन वाणिशांते।
 श्रियस्सुखश्रीपतिशक्तिरूपे
कावेरि कावेरिकुतोगतासि।
 इति स्तुत्वा मुनित्विा निर्दोषां पूर्वसंविदम्।
 परंब्रह्मेति तां ध्यात्वा पुनस्तुष्टाव सांजलिः।
 ततः क्रोशद्वयेऽतीते कावेरी ब्रह्मनिर्मितम्।
 स्वं रूपं दर्शयामास मंदहासाकृतांजलिः।
 10 ततो दृष्ट्वा महादेवींदेवानामपि मोहिनीम्।
 संतुष्टः प्राह योगींद्रो विस्मितश्च हसन् मुदा।
 अगस्त्यःमामनादृत्य कल्याणी किमर्थं निस्सृता बहिः।
 का हानिस्तव कावेरी निर्गमे मम सन्निधौ।
 कावेरीत्वदाज्ञामेव योगीन्द्र पुरस्कृत्यात्र भूधरे।
 इष्टमेवेति ते मत्वा निर्गतास्मीह नान्यथा।
 सह्याद्रौ निर्गमन्वेति ब्रह्मणा च त्वयोदितम्।
 हरिणापि ततो ज्ञात्वा लोकानुग्रहकाम्यया।
 
‘स्मृत्वा च तव संकेतं निस्सृता ब्रह्मभूधरात्।

सर्वं जानासि सर्वज्ञ त्वमेवात्र तु पौर्विकम्।
15 पुण्यदेशेच कालेच निर्गतेश्वरचोदिता।
 परब्रह्मस्वरूपस्य विष्णोर्भगवतो हरेः।
 सर्वतीर्थमयं तीर्थं प्राप्य पादान निस्सृतम्।
 मत्तपःफलरूपेण स्रोतारूपेण निर्गतम्।
 विशेषात् पुण्यकालेस्मिन् तुलाविषुव संज्ञिते।
 सहस्राोदयसमये मदनुग्रहकारणात्।
 एतत्सर्वं विचार्यैव ब्रह्माख्यम् चाप्यगोत्तमम्।
 ज्ञात्वा तत्तीर्थसहिता निर्गता कुंडिकामुखात्।
 स्मरंतश्च पिबंतश्च दर्शयंतश्च मज्जलम्।
 कीर्तिमंतश्च माहात्म्यं मत्तीर्थस्नानकारिणः।
 20 शृण्वन्तश्श्रावयंतश्च मम माहात्म्यमुत्तमम्।
 अति पापप्रसक्ता वा उपपातकिनो पि वा।
 सर्वे भवंतु मुक्ताश्च प्रयांतु च हरेःपदम्।
 इति मत्वा महायोगिन् पुण्यकालमिमं स्थलम्।
 विष्णोराज्ञां पुरस्कृत्य तदभिजल पाविता।
 सर्वलोकोपकाराय सर्वपापप्रणाशिनी।
 निर्गतास्मि सरिद्रूपा तन्मत्वं क्षतुमर्हसि।
 कावेर्याः कलिनाशिनी इति शृत्वा शुभं वचः।
 25 मंदस्मितप्रहृष्टात्मा प्राह सह्योद्भवं मुनिः।
 
अगस्त्यः
साधु भद्रे शृतं सर्वं सर्वं जानामि पौर्विकम्।
 त्वयि स्नेहाच्छ चापर्यात्वद्दर्शन कुतूहलात्।
 कोपोभून्मम कावेरी जगदंबाद्यनान्यथा।
 गच्छ त्वं लोकरक्षार्थ मनुज्ञाता ममापि च।
 हरिपादाजसंभूतं शङ्खतीर्थमिति शृतम्।
 शंखांबुब्रह्मणायस्माद् भ्रामितं केशवोपरि।
 अस्मिन्तीर्थे सकृत्स्नात्वा मध्यपोपि विशुद्ध्यति।
 इदं च कुण्डिका तीर्थं सर्वबंध विनाशनम्।
 ब्रह्मणा कुण्डिकातीर्थादभिषिक्तो यतो हरिः।
 एतत्तोयम् सकृत्स्पृष्ट्वा सर्वपापैः प्रमुच्यते।
 30 अभिषिक्तस्स्वयं विष्णुरयं धात्रीवपुःपुरा।
 कुण्डिकाविरजातीर्थैर् ब्रह्मणोद्धृतपाणिना।
 इयमाकाशगंगेति विश्रुता मुक्तिदायिनी।
 इदमेव महातीर्थं सह्यामलक संज्ञितम्।
 ऋणमोचनमित्याहुः ऋणत्रय विमोचनात्।
 अस्यास्तीरे सकृत्पिण्डम् पितॄनुद्दिश्य यश्चरेत्।
 ऋणत्रयविनिर्मुक्त श्राद्धकर्ता च निर्मलः।
 सर्वक्रतुफलं प्राप्य विष्णुसारूप्यमश्नुते।
 एतैःपुण्यैर्महातीर्थैर् सर्वतीर्थफलप्रदैः।
 शैलाद्ब्रह्ममयादस्मात् प्रवहस्वाखिलेश्वरी।
 सर्वलोक हितार्थाय स्नास्यतामिष्ट सिद्धये।
 विशेषाद्दक्षिणात्यानां रक्षणाय शुभे व्रज।
 
एतत्तीर्थानि सर्वाणि त्वमेवात्र न संशयः।
 विष्णुपादाब्ज संभूता गङ्गेति प्रथितोत्तरा।
 यथा भागीरथी गङ्गा जाह्नवीसिंधुरित्यपि।
 तथा दक्षिण गङ्गेति सह्यामलकसंभवा।
 कावेरी दिव्यदेवीति सर्वतीर्थमयीति च।
 मरुद्वधेति नामानि भविष्यति तवानघे।
 40 ध्यात्वा दृष्ट्वा च पीत्वा च त्वत्तीर्थं तीर्थमात्रके।
 अयत्नेन भविष्यंति नारायण पदाश्रयाः।
 अहं च वैखानस वालखिल्यै
रन्यैर्मुद्रिस्सनकादिभिश्च।
 सर्वत्र ते पुण्यतटे वसामि
श्रीपार्वतीनाथमुखैस्सुरैश्च।
 त्वत्पुण्य तीर्थे ये स्नात्वा तर्पयेयुः पितॄन् जलैः।
 आकल्पं तृप्तिमापन्नाः पितरस्स्वर्गभागिनः।
 सह्यामलक मूलेत्र श्राद्धं कुर्यात्तटे तव।
 अर्धोदये पितॄणाम् तु गयाश्राद्धफलं भवेत्।
 ऋणत्रय विनिर्मुक्तश्श्राद्धकर्ता च तत्क्षणात्।
 भुक्त्वैह सकलान् भोगानंते विष्णुपदम् व्रजेत्।
 45।
 अत्रैव विरजास्नानं त्वयिस्नानं भवेद्धवम्।
 जीवन्मुक्ता भविष्यति स्नात्वा ते सलिले नराः।
 महारहस्यं किंचात्र शृणुष्वत्यद्भुतं शुभे।
 ब्रवीमि दिव्यदृष्ट्याद्या सर्वशास्त्रेषु निश्चितम्।
 
महापातकिनां पापं मोचयंत्यःक्षणेन हि।
 गङ्गादिका महानद्यस्तीर्थानि च महंति च।
 मलीकृताश्च तत्पापैरशक्तास्तद्विमोचने।
 शोचंत्यश्शरणं देवं ययुर्लोकपिताहम्।
 तासां विशुद्धये ब्रह्मसभायां मुनिभिस्सह।
 संयग्विचार्य शास्त्रेषु वेदेषु च वचोब्रवीत्।
 50 स्नास्यं तु सह्यधात्र्यग्रे कावेर्यां तौलिसंक्रमे।
 श्रीरङ्गसन्निधौ वापि जंभुनाथस्य पार्श्वके।
 यत्रकुत्रापि वा क्षेत्रे सर्वपापापनुत्तये।
 हरिणापि तथा प्रोक्तं प्रायश्चित्तं विधास्यता।
 दत्तात्रेयोपि भगवानत्र पादाजरेणुभिः।
 साक्षाद्विष्णुः परंब्रह्मा त्वच्छीर्षम् पावयेत्सदा।
 तत्पादरजसापूत शीर्षत्वं लोकपावनी।
 कृतकृत्या हि कावेरी सर्वतीर्थाधिका तथा।
 विभूत्यै दाक्षिणत्यानां क्षेमाय विजयाय च।
 मयाचैवाभ्यनुज्ञाता गच्छ देवि यथासुखम्।
 दाल्भ्यःइति शृत्वा मुनेर्वाक्यं हर्षन्नेव पुरस्सरम्।
 परिक्रंय मुनिं नत्वा निश्चक्राम जलांतिकम्।
 ततः प्रकाशमापन्ना लोकरक्षार्थमंजसा।
 कावेरी सरितांश्रेष्ठा सुस्राव सरिदाकृतिः।
 तरङ्गसंघझंकार वेदघोषविभूषणा।
 
मत्स्य कच्छप संपूर्णा शिंशुमारभयंकरी।
 क्वचित्वेनिलपूराढ्या क्वचिच्छेवल संकुला।
 क्वचित्सारस संपूर्णा चक्रवाकोपशोभिता।
 क्वचित्पद्मवनोपेता क्वचित्कल्हारशोभिता।
 क्वचित्कैरवपुष्पाढ्या क्वचिदुत्पलशोभिता।
 60 क्वचित्सिमिततोयाढ्या क्वचिज्झळझळारवा।
 क्वचित्प्रत्यागतावर्ता वेणीभूतजलाक्वचित्।
 क्वचित्क्वचित्संकुचिता विस्तृता च क्वचित्क्वचित्।
 प्लावयंती क्वचिद्देशं जलौ पैराधतांगतैः।
 क्वचिच्छागरसलिला क्वचित्संकुचितस्थला।
 क्वचिदत्यंतवेगाड्या कुत्रचिन्निश्चलाकृतिः।
 क्वचिदुत्तंगतो वेगात् पातंतीभीमनिस्स्वना।
 तीरमुन्मूलयंती च कर्षती गुल्मभूरुहण्।
 लोकाभरणभूता सा भूमेराभरणम् परम्।
 हरस्रग्दक्षिणाशाया मुक्तेस्सोपानपद्धतिः।
 65 नृणाम् साम्राज्यदाने च दीक्षिता सह्यसंभवा।
 सर्वाभीष्टप्रदा देवी सर्वताप प्रणाशिनी।
 स्नानात्कालुष्यदोषघ्नी कावेरी कलिनाशिनी।
 कीर्तनात्सर्वदोषघ्नी ब्रह्माद्रेर्ब्रह्मणस्सुता।
 साक्षाद्रह्ममयी देवी ब्रह्मलोक प्रदायिनी।
 प्रवृत्ता लोकरक्षार्थं दिव्यतोया मनोहरा।
 अगस्त्योपि ततश्शिष्यैस्सप्तभिस्सप्ततंतुवित्।
 
अत्यद्भुताकृतिं दृष्ट्वा कावेरी विस्मयाकुलः।
 शनैश्शनैः प्रतस्थे च हर्षबाष्पविलोचनः।
 70 इति कावेर्यागस्त्ययोस्संवादः नाम षड्विंशोऽध्यायः।

******************
सप्तविंशोऽध्यायः
देवादीनां कावेरि दर्शनम् दाल्ब्यःततो देवर्षिगंधर्वास्सिद्धचारण किन्नराः ।
।
 पितर्श्वमहात्मानो विद्याधरमहोरगाः ।
 ।
 विमानस्सूरुसंकाशैर् द्योतयंतो दिशोदश ।
 ।
 कावेरीमद्भुताकारं दरष्टुमभ्यागतस्तदा।
 ।
 देवदुंदुभयोनेदुर् ननृतुश्चाप्सरोगमाः।
 ।
 बृगुर्गंधर्वसिद्धाश्च येचान्ये बहवस्तदा।
 ।
 हर्षाज्जयजयेत्युच्छैः कावेरी तुष्टुवुस्सुराः ।
 ब्रह्माविष्णुश्चरुद्रश्च सुर्सधैरभिष्टुताः।
 ।
 तत्रागम्न् पुराणैश्च कावेरीदर्शनोत्सुकाः।
 ।
 जगौकीर्तिं च कावेर्यास्सर्वं सम्भूय वै जगत् ।
 ।
 देवाः अहो विचित्ररूपोयं कावेरीदृश्यतेधुना।
 ।
 अहो महान्प्रवाहोस्या अवाङ्मनसगोचरः।
 ।
 अस्याप्रभाव को वेत्ति वेदांतेषूपबृह्मितम् ।
 ।
 स एवात्र महान् देवस्सर्वज्ञो भगवान् हरिः ।
 ।

वयं धन्यातमास्सर्वे सर्वथैव न संशयः ।
।
 तर्पयिष्यंति कावेर्याः तीर्थेनानेननोद्विजाः ।
 ।
 अस्मादुद्दिश्य कावेर्यास्सर्वेचापि जलाप्लुताः ।
 ।
 यज्ञादीन्कुर्वतेसंतः पालिता राजभिस्सदा।
 ।
 इत्यन्योन्यं सुरास्सर्वे हर्षा चुस्सुगद्गदम् ।
 ।
 प्रीताश्चपितरस्सर्वे हर्षगद्गदयागिरा।
 ।
 10 ऊचुःपरस्परं राजं त्समालिंग्य च बाहुभिः ।
 ।
 वयमेवकृतार्थास्मो भजामस्तृप्तिमक्षयाम् ।
 ।
 मद्वंशजास्तु सत्पुत्रा दद्युरस्यांतिलोदकम् ।
 ।
 अहो अत्यद्भुताकारा कावेरीतीर्थकोटिभिः।
 ।
 दृश्यते लोकरक्षार्थं ब्रह्माद्रौब्रह्मनिर्मिता।
 ।
 यानितीर्थानि पुण्यानि पुण्यासार्वाश्च निम्नगाः ।
 ।
 कावेर्याःब्रह्मरूपायाः कलांनाहँति षोडशीम् ।
 ।
 एतज्जलेनपक्वान्नात् केवलाच्छाकतोपि वा।
 ।
 गयाश्राद्धऽयुतफलं भवेत्येव हि नोदृवम् ।
 ।
 अहो न स्संततिस्संयगुपर्युपरि वर्धताम् ।
 ।
15 अस्यां स्नायात् सकृत्पुत्रो यस्यमुक्ता ऋणत्रयात् ।
।

ऋषयः
कावेरीकलिनाशिनीशृतिमयी ब्रह्मस्वरूपाखिलैः ।
।
 पुण्यैस्तीर्थगणायुतैरधिगता संदृश्यते निर्मला।
 ।
 एतत्तीर्थपवित्रिताश्चसकला लोकाभविष्यत्यहो।
 ।
 संतुष्टा यम यातनादि विमुखास्फीताश्च सर्वैश्शुभैः ।
 
वयं च देवाःपितरश्च कृतार्था आब्रह्मसुखिता भवामः।
।
 स्नात्वा यदिस्स्या स्सलिलेतिपुण्ये
कुर्वंति सत्कर्मसु तृप्तये नः।
।
 इति पितृसुरसंधैरसंस्तुता सह्यजाता
निखिलनिगममूर्धक्षीरपूराभिरामा।
 सकलसुकृतफेनातत्वशास्त्रार्थगर्ता
मखततिसिकताड्या बुद्धधाकारतीर्था ।
।
 वैकुण्ठसोपानतरंगसङ्घा पापक्षयेगर्जन दर्पघोषा।
 ।
 कवेर्कन्याकुटिलांबुवेगा
कल्हारकंजोत्पल कैरवाड्या।
।
 भूम्यैकभूषा भुवनस्यमाता
भूम्यैकपुण्यां कुगदिन्यदेशा।
।
 लोकैकरक्षार्थमभिप्रवृत्तां पुत्रीं कवेरस्य नदी समीक्ष्य ।
 ।
 वरं हरोस्त्ये प्रददौ प्रहृष्टो नृणामयत्नादतिपावनाय ।
 यत्स्यात्सुपुण्यम्
सकलार्णवेषु स्नानचा|दयपुन्यकाले।
।
 तत्पुण्यदाता भवदब्धिसंगे
__श्वेताभिधाने विपिनेवसामि ।
।
 पितामहश्च भगवान् संतुष्टश्चतुराननः ।
 ।
 वरंप्रदान् महातेजाः कावेर्ये कीर्तिसिद्धये ।
 ।
 25 त्वत्तीरवासिनो विप्रा विकर्मस्थाश्च नास्तिकाः।
 ।
 त्वज्जलेस्नानमात्रेण निष्पापायांतु मत्पदम् ।
 ।

अथ प्राह महातेजाः कावेरी कमलेक्षणः ।
।
 अनुग्रहायलोकानां भगवान् पुरुषोत्तमः ।
 ।
 वरं वरय भद्रं ते वरदोहमिहागतः ।
 ।
 यथेष्ठं पृच्छतां वत्से यत्ते मनसिवर्तते ।
 ।
 कावेरीअतिपापादुरात्मानो मानवाभूसुरादयः ।
 ।
 वेदबाह्यदुराचारा गुरुविप्रविदूषकः ।
 ।
 एकादश्यां प्रभाते तु सकृत्स्नावाजले मम।
 ।
 योगिगम्यं भवल्लोकं पुनरावृत्तिवर्जितम् ।
 ।
 वैकुण्ठंयांतु देवेश मम त्वं वरदोयदि।
 ।
 30 श्रीभगवान्-।
 तथैवास्तुमहाभागे पुण्यतीर्थे कवेरजे।
 ।
 वरमन्यं प्रदास्यामि मत्सेवा सकलेष्टदा।
 ।
 यत्र कुत्र तुलामासे तव तीर्थेऽतिपावने।
 ।
 प्रातस्स्नानं प्रकुर्वन्तिशृत्वाच तववैभवम् ।
 ।
 धर्मवक्तारमर्चति पुराणज्ञं सतांमतम् ।
 ।
 गंधपुष्पाक्षतैर्दिव्यैर्दुराचारैश्च यो नराः।
 ।
 पाषण्डाःपतिताश्चपि कितवानास्तिकास्तथा।
 ।
 ते सर्वेपितृभिस्सार्धं मत्कटाक्षेण पालिताः ।
 ।
 भुक्त्वैहसकलान् भोगान् पुत्रपौत्राभि नंदिताः ।
 ।
 मम सायुज्यमायांतु मोक्षलक्ष्मी न संशयः ।
 ।
 35 दाल्भ्यः
इति लब्ध वरां देवीं ब्रह्मविष्णुशिवात्मिकाम् ।
।
 सुज्योतिका च कनका कावेरी दिव्य निम्नगाम् ।
 ।
 अभ्यर्चयन्त्यौ पद्मावैर्ववंधा ते सुहर्षिते ।
 ।
 सुज्योतिकांचकनकां पापघ्नौ स्मृतिमात्रतः।
 ।
 स्वागतेन यथान्यायं प्रतिपूज्य कवेरजा।
 ।
 रराजभृशसंतुष्टा चैत्रसूर्यप्रभा यथा।
 ।
 यस्स्नाति संघमे तासां पतितोपि सकृन्नरः।
 सर्वपापविनिर्मुक्तो ब्रह्मलोके महीयते।
 तुलापुरुषदानानि कुर्यादर्दोदयेशतम्।
 यो नैमिशे त्रिवेण्यां तु सकृत्स्नायात्ससंगमे।
 40 योऽर्चयेत्तुळसीपत्रैर्वासुदेवम्शतं समाः।
 सहस्रनामभिर्दिव्यै स्स्नायात्तसां स संगमे ।
 कृपया वासुदेवस्य यस्यसंसारमोचनम् ।
 तस्य तत्संगमेस्नानं जन्मान्तरतपःफलम् ।
 यस्तु तत्संगमे स्नात्वा सकृद्दद्यात्तिलोदकम् ।
 पितरस्तस्य संतृप्ता ब्रह्मलोकं व्रजति च ।
 वर्णभ्रष्टापि पाषण्डः परान्नादी निरग्निकः ।
 सदाचार परिभ्रष्टः पितृमातृ विदूषकः ।
 स्मृत्वा त्रिवेणी यस्स्नाति वापीकूपजलादिके ।
 सद्य एव विशुद्धस्स्यात् तत्र स्नायात्तु किं पुनः ।
 45 योऽन्नदानं प्रकुर्वंति सर्वभूत दयान्वितः ।
 जन्मांतर तपोयोगात् त्रिवेण्यां स्नानमाचरेत् ।
 
सुज्योतिकेति कनकेति कवेरजेति
___ हेमावतीत्यपि सकृत्कपिलेति वापि।
 स्नायाज्जपन्मनुमिमं श्रुतिगोपितं
यस्स्नातो भवेन् मकरमासि स रामसेतौ ।
 सुज्योतिका स्वर्णमुखीसंयुक्ता सह्यकन्यका।
 हैमवत्यामहानद्यां मुक्ताकपिलया तथा।
 तत्र तत्रार्चिता देवैर्मुनीद्वैस्स्नान तत्परैः ।
 आजगाम महावेगात् सागरं सरितांपतिम् ।
 कावेर्याहेमवत्याश्च पुण्ये यस्स्नातिसंगमे।
 तस्य पुण्यफलं वक्तुं शक्त एव हरिस्स्वयम् ।
 50 किं तस्य वेदैःकिंतीर्थैः किं व्रतैः किं मघादिभिः ।
 हेमवत्याश्च कावेर्यास्संगमे स्नानकृद्यदि।
 गृहंति पितरःपिण्डान् गयायां दिवसंस्थिताः ।
 कावेरी कपिलासंगे प्रत्यक्षाःपितृदेवताः।
 कावेरीकपिलासंगे यो माघे स्नानमाचरेत् ।
 सार्धत्रिकोटितीर्थेषु स्नातो भवति भूपते ।
 कावेरी हेमवत्याश्च संगमे स्नाति कार्तिके।
 नरकस्था स्तत्पितरस्सर्वात्युभयवंशजाः ।
 यो हेमवत्याःकावेर्यास्संगमे यस्य कस्यचित् ।
 ग्रासमात्रं ददात्यन्नं सार्वभौमो भवेद्भवम् ।
 55 कावेरीकपिलासंगं संस्मरन्नन्यवारिषु ।
 स्नायात्सतत्स्नानफलं प्राप्यपापैः प्रमुच्यते ।
 
केदारे वा कुरुक्षेत्रे शतजन्मसु योन्नदः ।
 कपिलासह्यजा योगे सलभेत्स्नानमुत्तमम् ।
 कावेरीकपिलायोगे यश्श्राद्धं सकृदाचरेत् ।
 पितॄणां तत्क्षणात्तस्य गयाश्राद्धायुतं भवेत् ।
 दुर्लभं कपिलास्नानं दुर्लभं हरिकीर्तनम् ।
 दुर्लभा विष्णुपूजा च दुर्लभा वैष्णवाःकलौ।
 दुर्लभा भगवद्गीता दुर्लभा सत्कथा हरिः ।
 दुर्लभं रामचरितं दुर्लभं तुळसीदळम् ।
 60 दुर्लभं जाह्नवीस्नानम् तुलामासे विशेषतः ।
 दुर्लभंरङ्गनाथस्यदर्शनम् च कलौयुगे।
 दुर्लभं चंद्रपद्मिन्यास्सलिले पितृतर्पणम् ।
 दुर्लभो धर्मवक्ता च सदाचारश्चदुर्लभः ।
 ।
 दुर्लभो तुळसीपूजा तत्त्वज्ञानं च दुर्लभम् ।
 दया च सर्वभूतेषु दुर्लभा च समामतिः ।
 दुर्लभस्साधुसंगश्च दुर्लभं सेतु दर्शनम् ।
 दुर्लभं कार्तिकस्नानं माघस्नानं च दुर्लभ ।
 65 एतत्सर्वं लभेन्माचे कावेरीकपिलाप्लवात् ।
 स्नायाद्वा मानसे तीर्थे कुर्याद्वा सेतुदर्शनम् ।
 कावेरी हेमवत्योश्च संगे वा स्नानमाचरेत् ।
 रामायणं वा शृणुयात् कुर्याद्वा पितृपूजनम् ।
 कावेरीकपिलायोगे स्नायाद्वा नान्यथागतिः ।
 कावेरीविरजासाक्षाद् वैकुण्ठं रङ्गमंदिरम् ।
 
हेमवत्यमृताब्धिस्स्यात् कपिलाशुद्धवारिभिः ।
 एतेषां पुण्यतीर्थनां कीर्तितं सदृशं भवेत् ।
 यद्यस्ति सदृशं तीर्थं जानात्येव स ईश्वरः ।
 तीर्थवर्ये तु कावेर्या श्शिवलिंगानुकोटिशः ।
 विष्णुक्षेत्राणिपुण्यानि सर्वत्र च मुनीश्वराः ।
 ७0 आसह्याद्रेरासमुद्रात्कावेर्यास्तु तटद्वये ।
 वसंति योगिनो नित्यं ब्रह्मज्ञानाय तापसाः ।
 कावेरीतीरगा विप्रा स्सर्व एव मुनीश्वराः ।
 पश्वाद्या पुण्यभूपलाः चण्डालः पुण्यभूरुहः ।
 कीटाद्याश्शुनकाद्याश्च कावेर्यां ये वसंति च ।
 ते सर्वेयवनाःपूर्वं पश्चात्तप्तास्तथा भवन् ।
 वर्तते प्रणिमात्रं वा यत्किंचित्सह्यजा तटे।
 कावेरीवायुनापूतं विशुद्धं स्वर्गमाप्नुयात् ।
 कावेर्यां स्नाति यो मूडाश्श्रीकण्ठेश्वरसन्निधौ ।
 ब्रह्मनोवा सुरापानौ सर्वपापैःप्रमुच्यते ।
 ७5 योम्रियते महापापी श्रीरङ्गेश्वरसन्निधौ ।
 सप्तजन्मधनाड्य स्स्याद्वेदवेदांतपारगाः।
 कावेरी कपिलासंगे श्रीकण्ठेश्वरसन्निधौ ।
 श्राद्धकृन्माघसप्तम्यां कुलकोटि समुद्धरेत् ।
 दीयतेका च मात्रं वा श्रीरङ्गेशस्य सन्निधौ ।
 तुलापुरुषदानस्य त्रिरावृत्तिफलं लभेत् ।
 अन्यत्र स्नाति कावेर्यां श्रीरङ्गं संस्मरन्हृदि ।
 
लभेत्स्नानफलं सोपि सह्यामलकसन्निधौ ।
 श्रीकण्ठं यस्स्मरन् विप्रान् भोजयेदिंदुवासरे ।
 पुत्रपौत्रधनाद्यास्यु स्तद्वंश्याश्शतजन्मसु ।
 विटोपि विधिहीनोपि शौचाचारविवर्जितः ।
 श्रीरङ्गेति सकृज्जप्त्वा सर्वपापैःप्रमुच्यते ।
 यत्रकुत्रापि पापात्मा श्रीरङ्ग संस्मरेत्सदा।
 तस्य कर्णेजपेदंते तारकब्रह्म शंकरः ।
 श्रीकण्ठेति सकृज्जप्त्वा कावेर्यां स्नाति पापधीः ।
 सर्वपापविनिर्मुक्त श्शिवलोकेमहीयते ।
 कावेरीवाब्धिसंगे तु यस्य स्नानं न संभवेत् ।
 तत्संगं स्नान पुण्यस्यात् संवर्तेश्वरकीर्तनात् ।
 85 संवर्तलिङ्गं योमास्स्नानकाले जपेत्सकृत् ।
 शतजन्मसु यज्वा च संवर्तसदृशोभवेत्।
 स्वनांनास्थापितं लिङ्गम् संवर्तमुनिना नृप।
 तं लिङ्गस्मरणात् सद्यो मध्यपोपि विशुद्ध्यति।
 शृणु प्रभावं कावेर्या स्संक्षेपाद्वर्ण्यतेऽधुना।
 कावेरी वैभवं वक्तुं नालं वर्षशतैरपि।
 तटद्वये च कावेर्या मुक्त्यर्थं मुनिपुङ्गवाः।
 स्थापयामासुरेवं हि शिवलिंगानि कोटिशः।
 यावंलिङ्गानि कावेर्याः दिव्यानि च तटद्वये।
 श्रीरङ्गे सह्यजास्नानात्तत्तत्सेवाफलं भवेत्।
 ९0 पुण्यानामपि तीर्थानां क्षेत्राणामप्यसंभवे।

श्रीरंगसन्निधौ स्नायात् कावेर्यां पापकृत्तमः।
 यस्तुलामासि कावेयाँ श्रीरङ्गे स्नाति वा सकृत्।
 स साक्षाद्विरजा तोये स्नात एव न संशयः।
 यस्स्नात्वा सह्यजा तीर्थे श्रीरङ्गे तौलिके रवौ।
 ग्रासमात्रं द्विजे दद्याद्धनाढ्यश्शतजन्मसु।
 श्रीरङ्गे स्नाति यो माघे कार्तिके सह्यजाजले।
 तस्य पुण्यफलं वक्तुं शक्त एव हरिस्स्वयम्।
 पुत्रपौत्राभिवृद्ध्यर्थ मायुष्यार्थं महीपते।
 श्रीरङ्गे सह्यजास्नायी सर्वान् कामानवाप्नुयात्।
 95 यस्स्नायात् सह्यजातीर्थे श्रीरङ्गे तौलिसंक्रम।
 तर्पयेद्वा पितॄन् भक्त्या तीर्थैः कृष्णेस्तु निर्धनः।
 शृणुपुण्यफलं तस्य माहात्म्यमपि पार्थिव।
 कुलकोटि समुद्धृत्य मातृतः पितृतस्तदा।
 भुक्त्वैह सकलान् भोगान् स्थित्वा मन्वंतरं दिवि।
 जायते चक्रवर्तीह पुत्रपौत्राभिसंवृतः।
 यस्तुलासंक्रमे विद्वान् श्रीरङ्गेश्राद्धमाचरेत्।
 पितरो दिवि मोदंते ह्यक्षयां तृप्तिमागताः।
 पितॄणामति तृप्तानामाशिषा वा शतं कुलम्।
 श्रियदूर्जितमेवस्यात् पुत्रपौत्राभिनंदितम्।
 100 कुरुक्षेत्रेगयायाम् च पुण्यं श्राद्धशतं नयेत्।
 श्रीरङ्गे तौलिके तत्स्यात् पितृणाम् तिलतर्पणम्।
 एवं प्रकारा कावेरी शिवविष्णुस्थलोज्ज्वला।
 
पावयंत्यखिलान् लोकान् प्रयातिपतिमंबुधिम्।
 इति देवादीनां कावेरि दर्शनम् नाम सप्तविंशोऽध्यायः
******************
अथ अष्टाविंशोऽध्यायः
कावेरी समुद्रयोर्विवाहः अथ संचोदितोवायुर्वरुणेन महात्मना।
 प्रत्युजगाम कावेरीमप्सरोभिस्समंततः।
 देवदुंदुभिनिर्घोषश्च स्वस्तिवाचोद्विजन्मनाम्।
 साधुवादस्सुराणां च शुशुवे दशदिक्ष्वपि।
 नमस्ते इति शब्दश्च कृतार्थोस्मीति च ध्वनिः।
 सर्वलोकाश्च तुष्टास्स्यु रितिशब्दोमहानभूत्।
 सनकाद्याश्च योगींद्राश्शक्राद्याश्च सुरास्तथा।
 अभिजग्नुश्च कावेरीं कृतार्थेनांतरात्मना।
 ततश्श्वेतवने श्रीमान् दिव्यो वैवाहिकोत्सवः।
 संबभूवसमुद्रस्य पावनः परमाद्भुतः।
 5 ब्रह्मशर्वश्चभगवान् चक्रपाणिर्जगत्पतिः।
 तत्रागमन् सुरैस्सार्धं प्रशशंसुमहोत्सवम्।
 ततो ब्रह्म महातेजाः पुरस्कृत्य महेश्वरम्।
 कारयित्वा यथान्यायं सर्वं वैवाहिकं शुभम्।
 कावेरी प्रददौ कन्या सागराय माहात्मने।
 पुरेस्वर्गोपमे तत्र नानासौधोपशोभिते।
 जगतां मङ्गळार्थाय सो भवत्सुमहोत्सवः।
 
दिविमङ्गळतूर्यानिस्वना ननृतुश्चाप्सरसांगणास्सुराः।
 ववृषुः कुसुमानिचारणा जगुरुच्छैश्च कवेरजा यशः।
 इत्थं प्रवृत्ते तु महोत्सवे तदा
___ कवेरजायाः क्षितिमण्डले महान्।
 गृहेगृहे भूगृहमेधिनां महो
विश्वंबराचापि यथार्थनाम्न्यभूत्।
 10 पाणिग्रहण काले तु कावेर्यावरुणस्स्वयम्।
 दशकोटिसुवर्णानां रत्नानि विविधानि च।
 अलंकृतेभ्यो विप्रेभ्यो ददौ श्रीमान् महामनाः।
 एवमुद्वाह्यकावेरी सर्वलोकनमस्कृताम्।
 रत्नसिंहासनारूढो रेजेब्धिस्सुरराडिव।
 कावेरी सरितांश्रेष्ठ नमश्चक्रेसुरेश्वरान्।
 तामुत्थाप्यमुदातेमी आशीर्वादं प्रचक्रिरे।
 ततो वरमदादेतं कावेर्यै भगवान् भवः।
 अत्र स्नाने नृणां देवी सप्ताब्धिस्स्नानजं फलम्।
 यत्र कुत्र त्वयि स्नाने तिरश्चामपि तद्भवेत्।
 15 दिशतिस्म वरंब्रह्म कावेर्यैःप्रीतमानसः।
 स्नात्वा तुलार्के ये तीर्थे शृणुयुस्तव वैभवम्।
 तेषां भोगश्चमोक्षश्च संभूयान् मदनुग्रहात्।
 ततः श्रीमान् हरिःप्रादा वरं प्रीतोह्यनुत्तमम्।
 मत्कथाश्रवणेनस्या दशवर्षेषु यत्फलम्।
 
सकृत्स्व महिमशृत्या तत्फलं स्यान्न संशयः।
 इति दत्वा वरं देव्यै लोकरक्षार्थमीश्वराः।
 देवर्षिभिस्तुताप्रीता स्स्वस्वं धामप्रपेदिरे।
 कावेरी च ततो लोकान् पावयंतीस्ववारिणा।
 प्रीताब्धिना च संगम्य मुमुदे च बृशंसुखम्।
 यस्स्नायात्साजांभोधि संगमे तौलिकेरवौ।
 तस्य पुण्यफलं वक्तुं शक्ता एव जनार्दनः।
 20 समुद्र सह्यजासंगे यस्स्नात्वा तर्पयेत्पितॄण्।
 पितॄणां प्रीतये सह्यामलकश्राद्धकृद्भवेत्।
 यःकुर्यात्पंचपापानि शतवर्ष दिनेदिने।
 सोपि स्नात्वा संगमुखे मुच्यते सर्वकिल्बिषैः।
 सूर्यग्रहे तु यस्स्नायात् कावेर्यांबुधि संगमे।
 तेन षोडशदानानि नैमिशे चरितानि वै।
 यःपुत्रार्थी धनार्थीच कन्यार्थ्यायुष्यतत्परः।
 स सकृत्संगमे स्नात्वा सर्वान् कामानवाप्नुयात्।
 स्नात्वा संगमुखेयस्तु नमेच्छेतवनेश्वरम्।
 सर्वाभीष्टं ददातीशस्तस्य किं नाम दुर्लभम्।
 25 सकृत्संगमुखे स्नानाद् ब्रह्मज्ञानं च संभवेत्।
 ब्रह्मज्ञानेन मुक्तिस्स्यात् किमन्यस्तीर्थकोटिभिः।
 स्नात्वा संगमुखे दद्यात् काचं वापि रवेर्ग्रहे।
 शतजन्मसु विप्राग्र्यो धनाढ्यो वेदपारगः।
 कावेर्युत्पत्तिकल्याणवैभवं शृणुयान्नरः।
 
तत्कुलं धनदान्याढ्यं वर्धते शतवार्षिकम्।
 समुद्रसह्यजासंगवैभवं यश्शृणोति वै।
 स सर्वपापविनिर्मुक्तस्स्वर्गलोके महीयते।
 श्रीरङ्गेवाब्धिसंगे वा मयूरे वपि पापकृत्।
 तुलार्के स्नानतश्शुद्धो ब्रह्मलोके महीयते।
 30 दशाश्वमेधान् यःकुर्यात् कुरुक्षेत्रे नृपोत्तम।
 मयूरे स लभेत्स्नानं कावेर्यां तौलिभास्करे।
 मध्यार्जुने तु संकल्प्य यस्स्नायात् सह्यजाजले।
 ब्रह्महा वा गुरुद्रोही सर्वपापैःप्रमुच्यते।
 सर्वाशक्तोपि मूढात्मा कावेर्युत्पत्ति वैभवम्।
 एकादश्यां सकृच्छ्रुत्वा तत्तत्स्नानफलं लभेत्।
 किमत्र बहुनोक्तेन कावेरीस्मृतिमात्रतः।
 सर्वपापानि नश्येयुस्तमस्सूर्योदये यथा।
 यःप्रभाते तु कावेरी समुत्थाय त्रिरुच्चरेत्।
 तुलार्के सह्यजास्नानाद्यत्फलं तत्फलं लभेत्।
 35 संतु वा कोटितीर्थानि दिव्यपुण्यस्थलान्यपि।
 रङ्गेस्नानस्य कावेर्यां कलां नाहँति षोडशीम्।
 चंद्रकान्तापि कुटिला श्रीरङ्गे स्नानमात्रतः।
 सर्वपापविशुद्धा सा ब्रह्मलोकं जगामह।
 सर्वेषामपिमूढानामिदमेव निदर्शनम्।
 हरिश्चंद्रःभगवन् मुनिशार्दूल चंद्रकान्तापि पुंश्चली।
 
भुक्त्वा कल्पं महाघोरान्नरकान् बहुदुःखदान्।
 देहांतरे तु कावेर्यां स्वर्गस्था स्नानमात्रतः।
 सह्यजा वैभवप्राज्ञ दृष्टान्तोयं कथं भवेत्।
 एतद्विस्तरतस्सर्वं ब्रूहि मे मुनिपुङ्गव।
 अगस्त्यःशृणुष्व नृपशार्दूल हरिश्चंद्रमहामते।
 अयमेव हि दृष्टांतः कावेरी वैभवं प्रति।
 40 वर्तते विधवा या स्त्री सकेशाचासितांबरा।
 ज्ञानवैराग्यहीना च दक्षाकुक्षिप्रपूरणे।
 कुटिला च व्रतत्यक्ता पत्युरप्रियकारिणी।
 प्रतिकूलागुरूणां च स्वबंधुजन वल्लभा।
 एताः कुबुद्धये सत्यः पतिकर्मविनिंदकाः।
 स्नात्वापि सर्वतीर्थेषु न शुद्ध्यंति कदाचन।
 सप्तकल्पं महाघोरं नरकानेकविंशतिम्।
 अनुभूयःक्रमेणैव संतप्ता नरकाग्निभिः।
 दास्यो भूत्वा पुनर्भूमौ स्वामिना बृशदुःखिताः।
 तारामाः कृमिकण्ठास्यु रुग्णाः कल्पशतत्रयम्।
 45 कीटयोनिषुजायंते यावत्यः कृमिजातयः।
 न कदाच नमुंचंति नरकानेकविंशतिम्।
 एतादृशानिपापानि कुर्वंता भूप्रदक्षिणम्।
 विना च सह्यजातोयं प्रायश्चित्तम् न विद्यते।
 चंद्रकांता तु धर्मज्ञ भर्तुनित्यंहितेरता।
 
शुश्रूषां कुर्वतीभर्तु स्तौला वापि कवेरजाम्।
 श्रीरङ्गे पुण्यकावेरीजल संस्पर्शमात्रतः।
 कावेरीवैभवश्लोक श्रवणेनच निर्मला।
 या द्वेष्टी सह्यजातीरे पतिं सा कुटिलाभवेत्।
 शतवर्षं तु कावेर्यां स्नात्वा वापि न शुद्ध्यति।
 50 यातना दह्यमाना च भक्षिता मक्षिकादिभिः।
 कल्पेकल्पे च नरकाननुभुङ्क्तेसुदारुणान्।
 कर्मणामनसा वाचा ज्ञानपूर्वं तु किंचन।
 पापं न कुर्यात्कावेर्यां यदिच्छेच्छ्रेय आत्मनः।
 अक्षय्यंसह्यजातीरे पुण्यं वा पापमेववा।
 अन्यस्थलीयपापानां न कुत्रापिच निष्कृतिः।
 तस्मान्न संस्मरेत्प्राज्ञः पापं सह्योद्भवा तटे।
 पुण्यमेव सदा कुर्यात् सत्यंसत्यं वदाम्यहम्।
 राजन् ! दर्शनमात्रेण कावेर्याः पापसंचयः।
 क्षणेन नाशमायाति तमस्सूर्योदया यथा।
 कृते तु स्नानपानादौ कावेर्यां किमुवर्ण्यते।
 55 स्मरणादर्शनात् पानात् कीर्तनात्स्पर्शनादपि।
 सद्यो हरति कावेरी पापमामरणान्तिकम्।
 स्नानं दानं च कावेर्यां श्रीरङ्गे पितृतर्पणम्।
 तपस्यतां कुरुक्षेत्रे शतकल्पं तु संभवेत्।
 कावेरीवार्धियोगेय श्रीरङ्गेस्नानतौलिके।
 तस्य पुण्यफलं वक्तुं शक्त एव हरिस्स्वयम्।
 
यत्रकुत्रापि कावेर्यां यस्स्नायाद्धरिवासरे।
 किमन्यवर्ण्यते पुण्यं स साक्षात् केशवस्स्मृतः।
 पठतःपुण्यकावेर्या माहात्म्यं शृण्वतश्च वा।
 सप्तजन्मार्जितंपापं तत्क्षणादेव नश्यति।
 60 कावेरीदिव्यकल्याणं यःपठेच्छृणुयादपि।
 नामङ्गळं नदारियं तस्य वंशे कदाचन।
 कावेरीवैभवं शृत्वा योऽर्चयेद्धर्मवाचकम्।
 तस्य पुण्यफलं वक्तुं शक्त एव हरिस्स्वयम्।
 तुलासंक्रममारभ्य यस्स्नायात् सह्यजाजले।
 तस्य पुण्यफलं वक्तुं नाहंशक्तोनचाच्युतः।
 प्रभावं शृणुयाद्विद्वान् यदीच्छेच्छ्रेय आत्मनः।
 भक्त्यापौराणिकं गन्धैरर्चयित्वाऽक्षतादिभिः।
 कावेरीपुण्यमाहात्म्यं श्रोतव्यमनसूयया।
 योवमन्येत मूढात्मा लोभात् पौराणिकींशठः।
 अरण्येनिर्जलेघोरे भवतब्रह्मराक्षसः।
 65 कावेरी दिव्यमाहात्म्यं सहस्रवदनैस्स्वयम्।
 उक्त्वा कल्पशतं शेषो विरमेदिति मे मतिः।
 तस्मात्स एव भगवाननंतो धर्मवाचकः।
 इति बुद्ध्याऽर्चयेत्तं वै नपलत्यन्यथाव्रतम्।
 राजसूयाऽश्वमेधा च तपोधर्मव्रतादयः।
 कावेरी कीर्तिपाठस्य कलां नाहँति षोडशीम्।
 पुण्याद्धिसह्यजास्नानानम् मुक्तिरेव न संशयः।
 
सुदूर तीर्थयात्रायां प्रसिद्धिःकेवलं भवेत्।
 कावेरीमहिमालेशात् सूचितोत्र मया तव।
 वक्ता च दुर्लभाश्श्रोता राजन् कल्पायुतेष्वपि।
 70 बहुजन्मसु दुःखार्तो योनुग्राह्यो हरेस्स्वयम्।
 शृणोतिकीर्ति कावेर्या लभतेचार्चनं हरेः।
 इत्थं धर्मात्मजश्श्रीमान् राजा धौम्यमुनीरितम्।
 माहात्म्यं पुण्यकावेर्या शृत्वा हर्षाश्रुलोचनः।
 हार केयूरकटक कटिसूत्राङ्गुळीयकैः।
 अभ्यर्च्य गन्धपुष्पाद्यैर्दिव्यरत्नांबरादिभिः।
 शिभिकायां समारोप्य वादयन् स्वस्तिवाचनम्।
 दिव्यचामरमादाय स्वयं संवीजयन् शनैः।
 सिंहासने समारोप्य ववंदे शिरसा मुनिम्।
 पुनरानर्च हृष्टात्मा धर्मराजोनुजैस्सह।
 75 नाशक्नोदीक्षितुंस्तोतुं भक्त्युत्कण्ठाश्रुलोचनः।
 इति संपूजितस्तेन धौम्यधर्मसुतेन सः।
 अमोघं मङ्गळं तस्मै प्रयुज्य प्रणतंनृपम्।
 कथंचित्तमनुज्ञाप्य जगाम तपसेवनम्।
 
अगस्त्यः
इदं ते वर्णितं राजन् ! यस्मान् त्वं परिपृष्टवान्।
 कावेरीदिव्यमाहात्म्यं सर्वसौभाग्यदं नृणाम्।
 एतावदेव वक्तव्यं कावेरी महिमाशुभः।
 सारोयं सर्वधर्माणां कारणं सर्वसंपदाम्।
 
य इममखिलधर्मश्रेयसां मूलसारं
सकलमुनिसुरेंद्रैस्सर्वदा सेव्यमानम्।
 पठति जलजनेत्रस्याग्रतो विप्रगोष्ठ्यां __ भुवि च विपुलभोगैः पूज्यते स्वर्गलोके।
 80 य इदं पुण्यमाख्यानं वरमायुष्यवर्धनम्।
 पठेद्वा शृणुयास्मास्सर्व पापैः प्रमुच्यते।
 81 इति कावेरी समुद्रयोविवाहः नाम अष्टाविंशोऽध्याः
**********************
अथ एकोन त्रिंशोऽध्यायः हरिश्चंद्रकृत तुलाकावेरिमाहात्म्यप्रशंस
अगस्त्यः
तुलार्के त्वं च कावेर्यां स्नात्वा श्रीरङ्गसन्निधौ।
 विप्रावमानपापाच्च सद्यश्शुद्धो भविष्यसि।
 ततो यजन् हरिदेवं हयमेधेन माधवम्।
 इह सर्वसमृद्धिं च ब्रह्मलोकमवाप्स्यसि।
 विप्रावमानदोषेण पुरेंद्रो नष्टमङ्गळः।
 स्नात्वा तुलार्के कावेर्यां त्रैलोक्यश्रियमाप्तवान्।
 सर्वकर्माण्यवद्यानि प्रायश्चित्ते तु पापिनाम्।
 इदमेकं तु कावेरी स्नानं निर्दोषमुच्यते।
 स्थूलेनिचैवसूक्ष्माणि शुष्काण्यामा॑णि बुद्धितः।
 निर्वर्तितानि पापानि वाङ्मनःकायजानि च।
 याचकादपि कावेर्यां स्नानार्दाङ्गे अरुणोदये।
 
सद्य एव विनश्यति तूल पिण्डा इवाग्निना।
 अन्यतीर्थेषु माहात्म्य शृत्यास्नानं फलप्रदम्।
 तां विना सेतुकावेर्यो स्नानं पुण्यफलप्रदम्।
 कावेरी महिमासेतोस्सम एव न संशयः।
 तस्मान् महाराज! भवान् कृतार्थो यज्ञप्रसंगादि
हसत्सभायाम्।

स्नानं सुलब्धं बहुपूर्वपुण्यैश्श्रीरङ्गसह्याद्रि सुतासुतो ये।
 त्वं च शीघ्रम् महाराज! गच्छेदानी मरुद्धाम्।
 स्नात्वा तत्रैव विप्राणां दत्वा च विपुलं धनम्।
 पुनश्च चंद्रपद्मिन्यां स्नात्वा रङ्गेश वैभवम्।
 शृत्वा च विधिवद्दानं दत्वा नत्वा च भूसुरान्।
 10 प्रणम्य रङ्गनाथं च सर्वाभीष्टप्रदं हरिम्।
 स्तुत्वा च विविधैस्तोत्रैः संप्रार्थ्य च मनोरथान्।
 तत्रोषित्वा त्रिरात्रं च कावेरी वैभवं पठन्।
 अत्रेष्ट्वा हयमेधेन केशवं भक्तवत्सलम्।
 विख्याप्य चक्रवर्ति त्वं जित्वा च सकलान् रिपून्।
 पुण्यश्लोक इतिख्यातस्सतां मध्ये द्विजन्मनाम्।
 भुंजानस्सकलान् भोगान् पुत्रपौत्रादिभिः पुरे।
 धर्मेण पालयन् राज्यं कृतकृत्यो भविष्यसि।
 दाल्यःइत्यगस्त्योदितान् सर्वान् सर्वपापप्रणाशनान्।
 निशम्यशिरसा भक्त्या ननाम नृपतिर्मुनिम्।
 15
पूजयामास धर्मात्मा दिव्यपुष्पाक्षतादिभिः।
 दिव्यैराभरणैर्वस्त्रैः दिव्यगंधानुलेपनैः।
 उत्थायोत्थाय धर्मात्मा प्रणम्य च मुहुर्मुहुः।
 कृतांजलिरुवाचेनं कृतार्थेनांतरात्मना।
 हरिश्चंद्रःअहमेव कृतार्थोस्मि पावितोस्मि महामुने।
 यच्छावितोस्मि तद्धर्मा नत्रागत्य स्वयं त्वया।
 अद्य मे सफलंजन्म सफलं व्रतमद्य मे।
 अद्य मे पितरस्तृप्ता अद्य मे सफलं कुलम्।
 पुराजन्मन्यपि मया पूजिताश्च महत्तमाः।
 सेवितानि च तीर्थानि दानानि सुकृतानि च।
 20 सत्संगो मेन्यथा भूयात्सकथाश्रवणं कुतः।
 केवलं भक्तियोगेन सुलभो भगवान् हरिः।
 नेष्टापूर्तादिभिर्धमः स्त्वत्प्रभाच्छृतो मया।
 अहो कवेरकन्याया माहात्म्यं श्रुतमद्य वै।
 सदा सत्संगतिः कार्या विद्वद्भिः पापभीरुभिः।
 सत्संगेन भवांबोधिस्सुलभस्ततु मंजसा।
 न तपो दानयोगैश्च स्वधर्माचरणेन च।
 परमापगतो वापि कुर्यात्सत्संगतिं बुधः।
 सत्संगेन विनानृणां भवाब्धेनौषधं महत्।
 सामयानि तीर्थानि नदेवामृच्छिलामयाः।
 ते पुनंत्युरुकालेन दर्शनादेव साधवः।
 25
संसारी प्राकृतः क्वाहं ब्रह्ममूर्तिःकुतो भवान्।
 पुरा संचितसद्धर्मेस्सत्संगति रभून्मय।
 सर्वात्मना कृतार्थोह मुत्तीर्णोस्मि भवांबुधिम्।
 अश्रौषं यदहं दिव्यं कावेर्याश्चरितं हरेः।
 अत्रैवा सर्व यज्ञाश्च सर्वधर्माश्च वै गमाः।
 अत्रैव सर्वतीर्थानि श्रुतादिव्या हरेःकथा।
 यथा प्रीणाति गोविंदः कावेरी स्नानदानतः।
 सत्कथाश्रवणेनापि न तथा धर्मसंग्रहैः।
 इत्यश्रुलोचनश्श्रीमान् हरिश्चंद्रो मुनीश्वरम्।
 अयुतैस्स्वर्णपुष्पैश धूपदीपैरपूजयत्।
 30 पूजयित्वा यथा न्यायमाशीर्भिरभिनंदितः।
 प्रणम्य च मुनीन् सर्वाननुज्ञातो मुनीश्वरैः।
 रथाधिरूढोऽतिरथस्सैन्येन महतावृतः।
 वाद्यघोषेण महता दिशोदश विनादयन्।
 प्रतस्थे रङ्गनाथस्य संदर्शनकुतूहलात्।
 स्फीतान् जनपदांपुष्टान् धर्मिष्ठजन संवृतान्।
 पश्यन् नानाविदान् सर्वांश्चातुर्वर्ण्य विभूषितान्।
 तत्र तत्र द्विजश्रेष्ठैस्स्वाश्रमोदित धर्मभिः।
 पूर्णकुम्भफलोपेतैः प्रहर्षेणाभिनंदितः।
 मरुद्धां महाभागां सर्वतीर्थमनोहराम्।
 स्वच्छवालुकशोभाड्यां हंससारससेविताम्।
 35 कुमुदोत्पलकल्हारकमलैरुपशोभिताम्।

पुरुषार्थप्रदां पुण्याम् पुण्यक्षेत्र पवित्रताम्।
 मुनीनामाश्रमैःपुण्यैः निरन्तरसुशोभिताम्।
 अप्सरोगण संबाधां देवर्षीन् गणशोभिताम्।
 नानावृक्षगणोपेतां नाना तीर्थज्वलां शुभाम्।
 रङ्गनाथपदांभोज पवित्रीकृत पाथसम्।
 दिव्याम्रवनसंछन्नां नाळिकेरवनोज्ज्वलाम्।
 महोत्सव जनोपेतां महापूगोपशोभिताम्।
 महाराजस्समासाद्य सर्वलोक नमस्कृताम्।
 तत्र स्नात्वा महातीर्थे महापातक नाशने।
 40 पीत्वा पुण्यजलं स्वच्छं संतर्घ्य पितृदेवताः।
 देवर्षिपितृदेवानामनृण स्नानदानतः।
 कृत्वाऽगस्त्योदितं तत्र सर्वं निरवशेषतः।
 श्रीरङ्गशायिनं विष्णुं प्रणम्य पुरुषोत्तमम्।
 दत्वान्नवासस्वर्णादि द्विजानां पूर्णमानसः।
 गोपुराणि विचित्राणि नानाप्राकार मण्डपान्।
 कारयित्वा महीपालः प्रसन्नेंद्रियमानसः।
 पूजयित्वा यथान्यायं तत्रत्यान् ब्राह्मणोत्तमान्।
 नमस्कृत्य महातेजा आशीर्भिरभिनंदितः।
 दिव्यैश्चर्यमिदं दृष्ट्वा बृशं विस्मितमानसः।
 45 पुनरायान् महापुण्यं कुरुक्षेत्रं मुनीश्वरान्।
 ववंदे तं मुनींद्राश्च दिव्याशीर्भिरयोजयन्।
 ततःकृतार्थां राजेंद्र वाजमेदं महाक्रतुम्।
 
अकारयं महाराजं शौनकादि मुनीश्वराः।
 सपत्नीके कृतश्रीशे दीक्षितेस्मिन् क्षितेश्वरे।
 तत्र प्रावर्ततश्रीमान् हयमेधोद वैष्णवः।
 यथाकाञ्च्यां विरिचस्य प्रयागे च विमुक्तिदे।
 तथा राज्ञोभिसंजज्ञेऽभूतपूर्वस्तु तादृशः।
 अकारयन्महाराजं याजकाः कुम्भजादयः।
 हयमेधं महायज्ञं यथेंद्रं वृत्रनाशने।
 50 एवं कृत्वा महायज्ञं यथाशास्त्रं महीपतिः।
 ददौ च ऋत्विगादिभ्यो यथान्यायं तु दक्षिणाः।
 कृत्वाचाऽवभृतस्नानं पत्नीयुक्तो विशांपतिः।
 कृत्वा बहुविधं दानं राकाचंद्र इवा बभौ।
 ततःकृतार्थो भृगुकुम्भजादीन्
प्रणम्यभक्त्या बहुधा महीशः।
 दिव्याभिराशीर्भिरतिप्रवृद्धो
मुदंसलेभेनुपमां महात्मा।
 पुनःपुनस्तान्विनयाभि भूषितः
प्रदक्षिणीकृत्य मुनीन् प्रणम्य सः।
 संपूज्यरत्नाभरणांबरादिभिर्
__ लब्ध्वाप्यनुज्ञां बहुमानपूर्विकाम्।
 आनंदबाष्पपरिपूरित नेत्रयुग्मो
विश्लेषकातरमतिर् नृपतिर्मुनीनां।
 निस्साण शंखपटहादिक
म
L
वाद्यघोषै स्संपूजयन् दशदिशस्सबलः प्रतस्थे।
55 सकलनृपतिपूज्यस्सार्वभौमो महात्मा
रथगजतुरगाद्यैर् भूमिमाच्छाद्य सर्वाम्।
 कटकमकुटहारैर्भूषितः पूज्यमानः
पथिजनपदसंधैर् मङ्गळाचार हस्तैः।
 ध्वजरथसुपताका तोरणैर् भासमानां
पुरममर सुपूज्यां प्रापसाकेत नाम्नीम्।
 इति हरिश्चंद्रकृत तुलाकावेरिमाहात्म्यप्रशंसो नाम
एकोनत्रिंशोऽध्यायः
***********************
अथ त्रिंशोऽध्यायः
धर्मवर्मकृत दाल्भ्यपूजावर्णनम् य इदं पुण्यकावेर्याः पाणिग्रहणमङ्गळम्।
 शृणुयात् परया भक्त्या सर्वपापैः प्रमुच्यते।
 कावेर्युद्वाहकल्याणं यश्शृणोति भृगोर्दिने।
 पुत्रपौत्रादि संपन्नो भवेत् सर्वसमृद्धिमान्।
 भक्त्याशृणोति या नारी पत्युरारोग्यकाङ्क्षणी।
 कावेर्या उत्सवं शृत्वा सर्वाभीष्टं समश्नुते।
 अंधाश्च पङ्गवोरुग्णा दरिद्राश्चाति पापिनः।
 कावेर्युत्पत्तिकल्याणं शृत्वा शुद्धाभवन्त्यहो।
 अतिपापैश्च पतिता विना चांद्रायणादिकम्।
 एतदध्यायपठनान्मुच्यते सर्वकिल्बिषैः।
 5
विनावेदं विना दानं विना यज्ञं विना तपः।
 कावेरी वैभवं शृत्वा पूज्यंते सततं सुरैः।
 कावेर्युत्पत्तिकल्याणं तुलामासे शृणोति यः।
 पुत्रादिसंपदस्तस्य वर्धते मङ्गळानि वै।
 यस्स्नात्वा सह्यजा तोये तुलार्के सह्यजोत्सवम्।
 शृणोति परयाभक्त्या सर्वपापैःप्रमुच्यते।
 यं यं कामयतेकामं तंतमाप्नोत्यसंशयः।
 तुलार्के पुण्यकावीरी स्पृष्टा दृष्टा च कीर्तिता।
 शृता स्मृता वा पीता वा सर्व पापानि नाशयेत्।
 या नारी गर्भिणी राजन् शृत्वा सह्योद्भवोत्सवम्।
 कीर्तयन्तं सुतं शिष्टं लभतेनात्रसंशयः।
 10 दाल्भ्यःएतत्ते सर्वमाख्यातं यत्पृष्टोहमिह त्वया।
 जन्मान्तरतपःपुण्यैः शृतन्ते सह्यजोत्सवः।
 त्वमेव सुकृती राजन् विना तीर्थादिसेवनम्।
 भुक्तिमुक्तिकरस्थे ते कावेरी कीर्तिकीर्तनात्।
 द्वयमेवमनुष्याणां भुक्तिमुक्त्यैकसाधनम्।
 श्रवणं विष्णुलीलायाः कावेरी वैभवस्य च।
 नास्त्यन्यत्त्रिषुलोकेषु सर्वदुःख विनाशनम्।
 केशवस्यापि कावेर्या श्चरित्रश्रवणं विना।
 यदि वांछसि वैकुण्ठं संपदश्चान्य दुर्लभाः।
 कावेरी नृपकल्याणी हरि वापि स्मरानिशम्।
 15
राजन्! नास्त्येव नास्त्यैव नास्त्यैवान्यत्कलौ तपः।
 अच्युतानंत गोविंद कावेरीति स्मृतिं विना।
 कावेरी च महाभागा भगवान् हरिरच्युतः।
 शृण्वतां निजमाहात्म्यं सर्वाभीष्टं प्रयच्छति सूतःइति दाल्भ्यो महायोगी सद्धर्मान् धर्मवर्मणे।
 उपदिश्य दयामूर्तिस्ततो गन्तुं प्रचक्रमे।
 धर्मवापि राजेंद्रः कृतार्थेनान्तरात्मना।
 शृत्वा च वैभवं विष्णोः कावेर्याश्चाश्रुलोचनः।
 तोषयामास रत्नाड्यैर्दिव्यैराभरणैर्मुदा।
 अर्चयित्वा यथान्यायं गंधपुष्पाक्षतादिभिः।
 पुनःपुनर्नमस्कृत्य दाल्भ्य पादांबुजद्वयम्।
 20 कृतांजलिर्महातेजाः राजा भागवताग्रणीः।
 विनयेन पुनःप्राह मुनिं विश्लेषकातरः।
 धर्मवर्माअद्य मे सफलं जन्म सफलं च कुलं मम।
 अद्य मे पितरस्तृप्ता अद्य मे मुनयस्सुराः।
 यन्नोगृहं समागत्य स्वयमेव महामुने।
 संसारोत्तारकान् धर्मान् वैष्णवांसत्वमुपादिशः।
 अहं धन्यतमश्रेष्ठ स्त्वादृशां संमतस्सताम्।
 यदश्रौषं कथाःपुण्यैर्वैष्णवीर्लोकपावनीः।
 त्वन्मुखांभोजसंभूतां भुक्तिमुक्तिफलप्रदाम्।
 
न तृप्तिर्जायते ब्रह्मन् पिबतो मे कथासुधाम्।
25 भूयो वर्णय माहात्म्यं कावेर्याःकलिनाशनम्।
 यच्छृत्वाहं गमिष्यामि तद्विष्णोःपरमपदम्।
 इति संप्रार्थयन् राजा श्रीमान् धर्मभृतांवरः।
 पादारविंदाम् राजेंद्रो योगींद्रस्याभ्यवंदत।
 नमन्तं तं समुत्थाप्य पाणिना शांतमानसः।
 मंदस्मितो मनोयोगी राजानं वाक्यमब्रवीत्।
 दाल्भ्यःराजन्! धर्मभृतां श्रेष्ठो भवानेव जगत्त्रये।
 तवैश्वर्यवतोप्यद्य मानसं सत्कथोन्मुखम्।
 त्वय्येव भगवान् विष्णु स्स्वयमेव करोत्कृपाम्।
 यज्जातानैष्ठिकीबुद्धिः कथायां माधवस्य भो।
 30 यस्य संसारचक्रेस्मिन् भ्रमणं स्यात्पुनःपुनः।
 हरेरनुग्रहाभावाद् बाह्ये सज्जतितन्मनः।
 दारिद्र्यव्याजितःकेचि दैश्वर्य व्याजितःपरे।
 दूरीकृताः केशवेन निमज्जंति भवार्णवे।
 द्रव्यावान्वा दरिद्रोवा सत्संगे संमुखो भवेत्।
 स्वेच्छयाकरुणाविष्णोर्यस्मिन्नस्ति विमुक्तिदा।
 सत्वं भागवतश्रेष्ठश्श्रेष्ठस्सर्व महीभृताम्।
 कृतकृत्योसि राजेंद्र ज्ञेयंते नास्तिदापरम्।
 शृणु मद्वचनं राजन्! माच शोकेमनःकृधाः।
 पुनरागम्यते पुण्यां कथयामि कथां हरेः।
 35
अर्धोदयो महापुण्यः पंचपातक नाशनः।
 आयाति स्नानदानाद्यैः पौषेभीष्टफलप्रदः।
 धर्मवर्माअर्धोदयस्य माहात्म्यं लक्षणं च महामुने।
 कथयस्व महायोगिन् दासोहं तव संमतःदाल्भ्यःअर्धोदयस्यमाहात्म्यं को वा वर्णयितुं क्षमः।
 ब्रह्मापि वर्णयच्छ्रीमा नाकल्पं विरमेद्धृवम्।
 पातार्कश्रवणैर्युक्ता याह्यमापौष माघयोः।
 ।
 असाव|दयोयोगः कोटिसूर्य ग्रहाधिकः।
 पातस्यान्तः पूर्वभागस्त्वमायाश्श्रोणा
___ मध्यं भास्करस्योदये चेत्।
 पौषेभानोर्वासरेऽ|दयस्या किंचिन्न्यूने
तं महा पूर्वमाहुः।
 40 असाव|दयोनृणां कोटियज्ञफलप्रदः।
 सर्वतीर्थफलावाप्तिस् तादृशो हि महोदयः।
 महानद्यादि तीर्थेषु पंचपातकवानपि।
 तूष्णीं स्नात्वापि राजेंद्र! सर्वपापैःप्रमुच्यते।
 पतितोवापि पाषण्डः पिशुनो नास्तिकःखलः।
 नद्याम|दये स्नात्वा सद्यो मुच्येतपातकैः।
 स्नानं तुलार्के कावेर्यां प्रयागे माघमज्जनम्।
 नेतावोदयेस्नानं कोटिजन्म तपःफलम्।
 
रामसेतौ सकृत्स्नात्वाप्यन्यकाले नराधमः।
 सद्यः पुत्रो व्रजेत्स्वर्गं पुण्यकाले तु किं पुनः।
 45 रामचंद्रकृतारेखा धनुषालवणांबुधौ।
 तद्दर्शनोद्भवेन्मुक्तिर् न जाने स्नानजं फलम्।
 सेतु यात्रां करोमीति यत्रकुत्रापि संचरन्।
 मध्येमृतोदिवंगच्छेच्छुद्धःपापोपि मानवः।
 यःप्रभाते स्मरेन्नित्यं सेतुं श्रीरामकल्पितम्।
 सेतुस्नानफलं प्राप्य शुद्धो याति सुरालयम्।
 मदागमनमीक्षते सेतुस्नानाय तापसाः।
 सेतुं दृष्ट्वा पुनःप्राप्तः कथयामि तवेप्सितम्।
 श्रवणं रामलीलायाः कावेरी सेतुदर्शनम्।
 वाजपेयाश्वमेधादीन्यजद्भिर्लभ्यते नृभिः।
 50 रामसेतुं सकृदृष्ट्वा शृत्वा रामाभिषेचनम्।
 ब्रह्महापि दिवं गच्छेत् किंपुनश्शुद्धमानसः।
 स्नात्वाऽमायां धनुष्कोट्यां कावेर्यां हरिवासरे।
 कुलकोटि समुद्धृत्य विष्णुसायुज्यमाप्नुयात्।
 इत्युत्तमान् बहून् धर्मानुपदिश्य महामुनिः।
 कृतांजलिं महाराज मभिनंद्य पुनःपुनः।
 दिव्याशीर्भिरमोघाभिर् वर्धयित्वा जयाशिषा।
 विप्रानामन्त्र्य तत्रत्यान् राजानं च कथंचन।
 शिष्यैः परिवृतोयोगी तपोवनमवापसः।
 
सूतः
कावेर्यामहिमाश्रेयान् सूचितोत्र मया द्विजाः।
 सहस्रांशेन संक्षेपात् सर्वपापप्रणाशनः।
 कच्चिच्च संशृतोधर्मः कावेरीवैभवोद्भवः।
 55 येत्वे तदिव्यमाहात्म्यं कावेर्याश्शृणुयान्नराः।
 धर्मवक्तारमर्चति व्रतान्ते दक्षिणादिभिः।
 वित्तशाठ्यमकुर्वाणाः प्रशान्तम् धर्मबोधकम्।
 भुक्त्वैह सकलान् भोगान् विष्णुलोकं प्रयांति ते।
 
ऋषयः
सूत सूत महाप्राज्ञ जीवजीवशतम् समाः।
 अहो विचित्रम् कावेर्या माहात्म्यं श्रुतिगोपितम्।
 श्रावितास्म महाभाग भवाब्धेस्तारितावयम्।
 पातिताश्च वयं श्रीमत् कावेरी पुण्यवैभवात्।
 कृतार्थाश्च महाभाग तपो नः किमतःपरम्।
 इति प्रसन्नानन नीरजामुदा ते
शौनकाद्या मुनयो मुमुक्षवः।
 हरिश्चरित्र श्रवणोत्सवोत्सुका
___ गंधाक्षताद्यैः पुनरप्य पूजयन्।
 60 इति श्रीमदाग्नेयपुराणे तुलाकावेरिमाहात्म्ये अगस्त्य हरिश्चंद्रसंवादे कावेरी कल्याणकथने समुद्रसंगम महिमाद्यनुवर्णनं धर्मवर्मकृत दाल्भ्यपूजावर्णनम्
नाम त्रिंशोऽध्यायः।
 हरिरातत्सत्।
 
सर्वं श्रीकृष्णार्पणमस्तु।

*************************
श्रीमतेरामानुजाय नमः अथ स्कान्दपुराणान्तर्गतम्
तुलामास व्रतोद्यापन विधिः उद्यापनमथो वक्ष्ये तुलामासस्य सत्तमे।
 यच्छ्रुत्वा सर्वपापेभ्यो मुच्यते नात्रसंशयः।
 तुलाकावेरिमाहात्म्यं नित्यं शृत्वा द्विजोत्तमः।
 दिनेदिने हरिशान्तस्स्नानपूर्वं समर्चयेत्।
 कावेरी विधिवत्स्नात्वा स्तोत्रपाठैश्चवैदिकैः।
 संप्रार्थ्य सर्वं संकल्प्य स्तोत्रं कृत्वा समेद्भुधः।
 अर्चयित्वा तु विधिवत् नानापुष्पैश्चपाटलैः।
 पङ्कजै ङ्गराजैश्च बिल्वपत्रैश्चकेतकैः।
 जंबूपत्रैश्चूतपत्रैस्तुळसी मञ्जरीभरैः।
 ।
 करवीरैर्नागदहुँर्वांकुर मनोहरैः।
 ।
 मल्लिकामालतीजाती कुसुमैश्च सुगंधिभिः।
 कोविदारैश्च कुसुमैः पारिजाता सुपुष्पकैः।
 रक्तलोधैर्नीललोधैर्दोणपुष्पैश्च पाटलैः।
 ।
 चारुभूषणवस्त्रैश्च दीपैडूंपैश्चवासितैः।
 दिव्यगन्धैश्च हारिद्रैः कुङ्कुमैरुपशोभितैः।
 उपचारैष्षोडशभि रर्घ्यपाद्यादिभिर्नमेत्।
 ततस्स्वगृहमागत्य गोमयेनोपलिप्य तत्।
 नानावितानैश्चित्रैश्च पूगैश्चकदलैर्युतम्।
 पुष्पचूतवितानैश्च शोभितं कारयेत्दृवम्।
 घृतदीपैशुभैद्रव्यै सहस्रैरयुतैरपि।
 10 गृहमध्ये मण्डपंचाप्यलंकृत्य सु पुष्पकैः।
 तत्रैव मध्ये षट्कोणे चतुरश्रेच शोभने।
 धान्यैश्चतण्डुलैस्सूक्ष्म स्तिलैर्माषैस्समुद्कैः।
 प्रत्येकं भारमात्रेण तदर्वातदर्धकैः।
 उपर्युपरि विस्तीर्य द्वात्रिंशद्दळसंयुतम्।
 कमलं विलिखेद्विद्वान् तन्मध्ये कलशं न्यसेत्।
 तद्वहिश्च चतुर्दिक्षु तद्बहिश्चाष्टदिक्षु च।
 अन्यांश्चतीर्थकलशान् प्रागादिस्थापयेत् क्रमात्।
 चूतपल्लव कूर्चाड्यान् प्रत्येकं सूत्रवेष्टितान्।
 स्थापयेत् कलशान् सर्वानाचार्यो मन्त्रवित्तमः।
 15 तन्मध्ये सह्यजातोयं पूरयित्वा च मन्त्रवत्।
 कावेरी च समावाह्य गन्धपुष्पैस्समर्चयेत्।
 गङ्गां च यमुनाबाह्यकलशेषु चतुर्ध्वपि।
 सरस्वतीं वृद्धगङ्गां प्रागादिष्वर्चये द्बुधः।
 प्रादक्षिण्यक्रमेणैव तद्बहिःकलशाष्टके।
 तुङ्गभद्रा नदी रेवां गण्डकी बहुदामपि।
 नर्मदां सिंधुसरितां ताम्रपर्णी नदीपराम्।
 कृष्णवेणी यथापूर्वमर्चयित्वा तु देशिकः।
 पुण्यतीर्थानिचान्यानि ध्यात्वा प्रागादिदिक् क्रमात्।
 
नाळिकेरैरलंकृत्य कुम्भांश्चैव समर्चयेत्।
 20 कुम्भरत्नैर्वस्त्रगन्धैरक्षतैःकुंकुमैरपि।
 कदळीफलताम्बूलैर् धूपैदीपैर्निवेदनैः।
 आवाह्य तत्र तन्मन्त्रै स्तत्तत्तीर्थाधिदेवताः।
 तन्मध्ये प्रतिमां चैव कुर्यात्कावेरि रूपिणीम्।
 शिवं शिवां च सूर्यं वा लक्ष्मीनारायणं परम्।
 भूमिं दुर्गा भारती वा विघ्नेशं स्कन्धमेव वा।
 विष्वक्सेनं सूत्रवतीं गरुडं च यथाक्रमम्।
 समुद्ररूपामिन्द्राणी रामकृष्णस्वरूपिणीम्।
 गायत्रीं वापि सावित्री वाग्देवी प्रतिमां च वा।
 त्रिपुरप्रतिमामेव ह्यन्नपूर्णा स्वरूपिणीम्।
 25 भुवनेश्वरी परांबालाम् परमानन्दरूपिणीम्।
 लोपामुद्रां महाभागा मगस्त्यमहितां शुभाम्।
 कृष्णम् जाम्भवतींचापि सत्यभामाच रुक्मिणीम्।
 सीताम् च लोकविख्यातां यायालोकप्रसिद्धगाः।
 तेषां मध्येरुचिर्यत्र वराभयहस्तकाम्।
 प्रतिमां कारयेत्तां वै सर्वलक्षण संयुताम्।
 सौवर्णी राजतीर्वापि प्रतिमाःकारयेद्बुधः।
 वस्त्रपीठे तु संस्थाप्य ह्यावाहनपुरस्सरम्।
 तत्रैव पूजयेद्विद्वान् गन्धपुष्पाक्षतादिभिः।
 तथा षोडशकन्याश्च प्रतिमामिव पूजयेत्।
 अभ्यङ्गपूर्वकैःपुष्पैः वस्त्रैराभरणैश्शुभैः।
 
देव्याश्चनामभिर्वापि कन्यामन्त्रैर्मनोरमैः।
 षडक्षरैर्वा कावेर्यास्तथा लक्ष्मीषडक्षरैः।
 गन्धपुष्पाक्षतफलैः नद्यांकन्यास्समर्चयेत्।
 अर्चयित्वा पुनर्नत्वा प्रार्थयेच्च पुनः पुनः।
 नैवेद्यं च तथाकुर्यान्नानाविध फलैरपि।
 त्रयस्त्रिंशत्कोटिसंख्या देवतास्तीर्थदेवताः।
 सर्वा आवाह्यविधिवन्नैवेद्यं च तथार्पयेत्।
 त्रयस्त्रिंशदपूपान् वा शतम् वा कारयेत्तथा।
 तावतो मोदकान् वापि सगुडान् तिलपिण्डकान्।
 35 तावतीलड्डकान्वापि दध्यन्नानि च तण्डुलान्।
 त्रयस्त्रिंशद्दिनानां च त्रयस्त्रिंशन्निवेदनम्।
 तावन्मात्रां ब्राह्मणांश्च कन्यकाश्च त्रयोपि च।
 मिधुनानि द्वादशा वा षट् चत्वारि द्वयं च वा।
 भोजयेद्विधिवत् पश्चात् तथा होमं समाचरेत्।
 होमद्रव्याणि लाजांश्च तण्डुलाश्च घृताक्षतैः।
 दधिक्षीरं गुडं तद्वन्नवनीतम् पयोमधु।
 हविस्तिलास्समित्पुष्पं कुशं वा शर्करादिकम्।
 एवमन्यैस्समिद्धाग्नौ हुनेदष्टोत्तरम् शतम्।
 40 राजाशतगुणं नित्य मधनस्तु यथोदितम्।
 वैश्योपि जुहुयाद्धोमं विप्रहस्ताद्यथा विधि।
 यथावित्तं यथाशक्ति यथाश्रद्धं यथामनः।
 इति व्रतं तुलामासे साङ्गम् सगुणकं चरेत्।
 
तद्ब्रह्मणःपदम् गच्छेत् नात्रकार्यविचारणा।
 इत्येवं विधिवत् कृत्वा प्रतिमां गुरवेऽर्पयेत्।
 वित्तशाठ्यामकुर्वाणो दशदानानि तत्र वै।
 45 कुर्याद्विद्वानतिप्रीत्या यथावित्तं तथैव च।
 वस्त्रं तिलमथाज्यम् च तैलं क्षीरं मधुक्रमात्।
 कूश्माण्डं कुङ्कुमं गन्धः कृसरश्शिवलिङ्गकम्।
 हरिर्हरिद्रालवणम् शूर्पकङ्कत कङ्कणैः।
 एतेषाम् तत्र दानं च विशेषेणसमाचरेत्।
 दद्याद्गुरुस्स्वहस्तेन स्त्रीणां मङ्गळसूत्रकम्।
 पौराणिकाय वै दद्याद् यथाविधि निवेदयेत्।
 एवं हि विधिवत् कृत्वा तत उद्यापनं चरेत्।
 तीर्थानांचैव सर्वेषाम् स्थितानां सह्यजाजले।
 मासान्ते गच्छताम् तेषां ताम्बूलानि निवेदयेत्।
 50 तत्रैवाभीप्सितं प्रार्थ्य नमस्कार समाचरेत्।
 नमोनमस्ते कावेरी सह्यशैलसुते शुभे।
 काळिन्दी सखसर्वेशी सर्वमङ्गळ दायिके।
 पूर्वकर्मव्यपोह्याशु ममदुःखं निवारय।
 दारियं नाशय क्षिप्रम् व्यपोह्य मम किल्बिषम्।
 ममापमृत्युदोषं च जन्मदोषं ममाशुभम्।
 सर्वान् दोषान् व्यपोह्याशु रक्षरक्ष हरिप्रिये।
 वत्सरेवत्सरेप्येवं मम जन्मनिजन्मनि।
 तवतीर्थे महापुण्ये श्रद्धां देहि कवेरजे।
 55।

अत्रैव पुत्रैःपौत्रैश्च सप्तभिर्बान्धवैरपि।
 तीर्थस्नानं ददस्वाद्य सर्वतीर्थंश्च सम्युता।
 नमोनमस्ते कावेरी नमस्ते लोकपूजिते।
 नमस्ते पुण्यसलिले नमस्ते सह्यकन्यके।
 नमस्ते पावनजले नमस्ते कलिनाशिनी।
 नमस्ते सर्वदयिते नमस्ते तीर्थपुङ्गवे।
 नमस्ते भर्तृसुखदे नमस्ते करुणानिधे।
 नमस्ते विष्णुमहिते नमस्ते विष्णुपूजके।
 नमस्तेस्तु नमस्तेस्तु नमस्तेस्तु नमोनमः।
 60 महापापयुतानां तु महादोषोत्थजन्मनाम्।
 विनाश्य सर्वपापानि सर्वमङ्गळदायिकाः।
 तुलास्नानरतानां चाप्याशीर्वचनमिच्छताम्।
 तुलामासव्रतस्थानां कुर्वन्त्वाशिषमद्य वै।
 महितानां द्विजातीनां क्षत्रियाणां विशामपि।
 शूद्राणामन्त्यजातीनां स्त्रीणां चैव विशेषतः।
 तां तां भक्तिकृतां पूजां स्वल्पं वा यदि वा बहून्।
 प्रगृह्य सर्वे मुदिताः मे कुर्वन्तुमहाशिषम्।
 तेषां तेषां माहातीर्थभक्तिं गृहंतु तोषिताः।
 मासादौ मासमध्ये वा मासान्ते वा जलेश्वराः।
 एकवारं द्विवारं वा त्रिवारं चा चतुश्च वा।
 पंचवारं सु षड्वारं सप्तवारमथापि वा।
 
नववारं वाऽष्टवारं दशवारं च वा पुनः।
 स्नानं वा दानमात्र वा कुर्युनैवेद्यमेव वा।
 युष्मानुद्दिश्य होमं वा जपं वा नियमं च वा।
 तुलामासपुराणस्य श्रवणंवान्तिमे दिने।
 योवाकेवा महाभागा यावाकावा स्त्रियोपि वा।
 सर्वेषामणुमात्रेण दृप्ता यूयं वरप्रदाः।
 किंचिन्नैवेद्यमात्रेणानन्तां तृप्तिमागताः।
 दहध्वं सर्वपापौघानपवर्गे निरामये।
 आशीर्वादं वदध्वं वै रक्षध्वं करुणादृशा।
 ताम्बूलं चन्दनं वापि तण्डुलान् तिलमिश्रितान्।
 गुडं वा नाळिकेरं वा ऊर्वारुक मपूपकम्।
 पत्रं पुष्पमेकं वा यथाशक्ति यथायथम्।
 यथाफलं यथावित्तं यथाबुद्धि यथागुणम्।
 दत्तंसर्वं स्वल्पमपि महानैवेद्यवत्तदा।
 संतोषात्प्रतिगृहंतु शुद्धामृतमिदं जलम्।
 इदं नैवेद्यममृतं युष्माकं गच्छतां पथि।
 यथाभागं महाभागा स्सर्वगृहंतु तीर्थगाः।
 एवमुद्यापनं ये तु भक्त्याकुर्यान्नरोत्तमाः।
 भुक्ति मुक्ति करस्थे हि तेषां युष्मदनुग्रहात्।
 इत्येवं सह्यपुत्र्या वै पूजिताः प्रतिमानिताः।
 
तदानीं सर्वसंपूज्यास्तथास्त्विति तथास्त्विति।
 नमस्कृत्य ययुस्सर्वे तीर्थदेवाःपुरातनाः।
 सर्वेविमानान्यारुह्य मासांते स्वांदिशं प्रति।
 इत्येवं मुनयश्शृत्वा तुलाकावेरि वैभवम्।
 वक्तारं पूजयामासुस्सर्वे मुदितमानसाः।
 सूतःवक्तारं परमंध्यात्वा हरिमेवागतं स्वयम्।
 अलंकृत्य सु मालाभिः कौशेयैःकुण्डलादिभिः।
 धनं धान्यं तण्डुलांश्च वस्त्राण्याभरणानि च।
 आरोग्य शिभिकां वापि स्वस्ति वाचनपूर्वकम्।
 ग्रामप्रदक्षिणं कृत्वा नानावादित्र निस्स्वनैः।
 गृहं प्रवेश्य सद्भक्त्या पूजयेद्विधिपूर्वकम्।
 सह्यजायाश्च तृप्त्यर्थं कन्यादानं समाचरेत्।
 श्रीकर पुष्टिदशुद्धं सर्वमङ्गळदायकम्।
 तुलाकावेरि माहात्म्यमुक्तं पूर्ण दयाळुना।
 अति रहस्यतमं शुभदं नृणां
परमपावन मान्तरशुद्धिदम्।
।

इति श्रीस्कान्दपुराणे उत्तरखण्डे स्कन्दमहेश्वरसंवादे तुलाकावेरीनानवतोद्यापन नाम 
एकत्रिंशोऽध्यायः। सर्वं श्रीरङ्गनाथार्पणमस्तु 
हरिः ओं कावेरी वर्धता काले कालेवर्षतु वासवः, श्रीजनायोजयतु श्रीश्रीश्च वर्धताम् । 
कायेन वाचामनसेन्द्रियैर्वा सुचात्मनावा प्रकृतेस्स्वभावात् । करोमि यद्यत्सकलं परस्मै नारायणायेति समर्पयामि। 
सर्व श्रीकृष्णार्पणमस्तु 
* 
श्रीतुलाकावेरि माहात्म्यं संपूर्णम् *